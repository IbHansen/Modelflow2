%% Generated by Sphinx.
\def\sphinxdocclass{report}
\documentclass[letterpaper,10pt,english]{sphinxmanual}
\ifdefined\pdfpxdimen
   \let\sphinxpxdimen\pdfpxdimen\else\newdimen\sphinxpxdimen
\fi \sphinxpxdimen=.75bp\relax
\ifdefined\pdfimageresolution
    \pdfimageresolution= \numexpr \dimexpr1in\relax/\sphinxpxdimen\relax
\fi
%% let collapsible pdf bookmarks panel have high depth per default
\PassOptionsToPackage{bookmarksdepth=5}{hyperref}

\PassOptionsToPackage{warn}{textcomp}
\usepackage[utf8]{inputenc}
\ifdefined\DeclareUnicodeCharacter
% support both utf8 and utf8x syntaxes
  \ifdefined\DeclareUnicodeCharacterAsOptional
    \def\sphinxDUC#1{\DeclareUnicodeCharacter{"#1}}
  \else
    \let\sphinxDUC\DeclareUnicodeCharacter
  \fi
  \sphinxDUC{00A0}{\nobreakspace}
  \sphinxDUC{2500}{\sphinxunichar{2500}}
  \sphinxDUC{2502}{\sphinxunichar{2502}}
  \sphinxDUC{2514}{\sphinxunichar{2514}}
  \sphinxDUC{251C}{\sphinxunichar{251C}}
  \sphinxDUC{2572}{\textbackslash}
\fi
\usepackage{cmap}
\usepackage[T1]{fontenc}
\usepackage{amsmath,amssymb,amstext}
\usepackage{babel}



\usepackage{tgtermes}
\usepackage{tgheros}
\renewcommand{\ttdefault}{txtt}



\usepackage[Bjarne]{fncychap}
\usepackage{sphinx}

\fvset{fontsize=auto}
\usepackage{geometry}


% Include hyperref last.
\usepackage{hyperref}
% Fix anchor placement for figures with captions.
\usepackage{hypcap}% it must be loaded after hyperref.
% Set up styles of URL: it should be placed after hyperref.
\urlstyle{same}

\addto\captionsenglish{\renewcommand{\contentsname}{Contents:}}

\usepackage{sphinxmessages}
\setcounter{tocdepth}{3}
\setcounter{secnumdepth}{3}


\title{ModelFlow}
\date{Jul 26, 2022}
\release{}
\author{Ib Hansen}
\newcommand{\sphinxlogo}{\vbox{}}
\renewcommand{\releasename}{}
\makeindex
\begin{document}

\pagestyle{empty}
\sphinxmaketitle
\pagestyle{plain}
\sphinxtableofcontents
\pagestyle{normal}
\phantomsection\label{\detokenize{index::doc}}



\chapter{Indices and tables}
\label{\detokenize{index:indices-and-tables}}\begin{itemize}
\item {} 
\sphinxAtStartPar
\DUrole{xref,std,std-ref}{genindex}

\item {} 
\sphinxAtStartPar
\DUrole{xref,std,std-ref}{modindex}

\item {} 
\sphinxAtStartPar
\DUrole{xref,std,std-ref}{search}

\end{itemize}


\chapter{Introduction}
\label{\detokenize{index:introduction}}
\sphinxAtStartPar
Dette er en prøve
og en ande kllgg


\chapter{Installation}
\label{\detokenize{index:installation}}

\section{Install Miniconda}
\label{\detokenize{index:install-miniconda}}
\sphinxAtStartPar
\sphinxurl{https://docs.conda.io/en/latest/miniconda.html} to download the latest version 3.9
\begin{itemize}
\item {} 
\sphinxAtStartPar
open the file to start instalation

\item {} 
\sphinxAtStartPar
asked to install for: select just me

\item {} 
\sphinxAtStartPar
in the start menu: select anaconda prompt

\end{itemize}


\section{Install Modelflow in the base enviroment}
\label{\detokenize{index:install-modelflow-in-the-base-enviroment}}
\sphinxAtStartPar
kdæoijpoijasdf

\begin{sphinxVerbatim}[commandchars=\\\{\}]
\PYG{n}{conda}  \PYG{n}{install}  \PYG{o}{\PYGZhy{}}\PYG{n}{c} \PYG{n}{ibh} \PYG{o}{\PYGZhy{}}\PYG{n}{c}  \PYG{n}{conda}\PYG{o}{\PYGZhy{}}\PYG{n}{forge} \PYG{n}{modelflow} \PYG{n}{jupyter} \PYG{o}{\PYGZhy{}}\PYG{n}{y}
\PYG{n}{pip} \PYG{n}{install} \PYG{n}{dash\PYGZus{}interactive\PYGZus{}graphviz}
\PYG{n}{jupyter} \PYG{n}{contrib} \PYG{n}{nbextension} \PYG{n}{install} \PYG{o}{\PYGZhy{}}\PYG{o}{\PYGZhy{}}\PYG{n}{user}
\PYG{n}{jupyter} \PYG{n}{nbextension} \PYG{n}{enable} \PYG{n}{hide\PYGZus{}input\PYGZus{}all}\PYG{o}{/}\PYG{n}{main}
\PYG{n}{jupyter} \PYG{n}{nbextension} \PYG{n}{enable} \PYG{n}{splitcell}\PYG{o}{/}\PYG{n}{splitcell}
\PYG{n}{jupyter} \PYG{n}{nbextension} \PYG{n}{enable} \PYG{n}{toc2}\PYG{o}{/}\PYG{n}{main}
\end{sphinxVerbatim}


\section{Install Modelflow in the separate enviroment}
\label{\detokenize{index:install-modelflow-in-the-separate-enviroment}}
\sphinxAtStartPar
In this case we call the enviorement ‘mf’:

\begin{sphinxVerbatim}[commandchars=\\\{\}]
\PYG{n}{conda} \PYG{n}{create} \PYG{o}{\PYGZhy{}}\PYG{n}{n} \PYG{n}{mf} \PYG{o}{\PYGZhy{}}\PYG{n}{c} \PYG{n}{ibh} \PYG{o}{\PYGZhy{}}\PYG{n}{c}  \PYG{n}{conda}\PYG{o}{\PYGZhy{}}\PYG{n}{forge} \PYG{n}{modelflow} \PYG{n}{jupyter} \PYG{o}{\PYGZhy{}}\PYG{n}{y}
\PYG{n}{conda} \PYG{n}{activate} \PYG{n}{mf}
\PYG{n}{pip} \PYG{n}{install} \PYG{n}{dash\PYGZus{}interactive\PYGZus{}graphviz}
\PYG{n}{jupyter} \PYG{n}{contrib} \PYG{n}{nbextension} \PYG{n}{install} \PYG{o}{\PYGZhy{}}\PYG{o}{\PYGZhy{}}\PYG{n}{user}
\PYG{n}{jupyter} \PYG{n}{nbextension} \PYG{n}{enable} \PYG{n}{hide\PYGZus{}input\PYGZus{}all}\PYG{o}{/}\PYG{n}{main}
\PYG{n}{jupyter} \PYG{n}{nbextension} \PYG{n}{enable} \PYG{n}{splitcell}\PYG{o}{/}\PYG{n}{splitcell}
\PYG{n}{jupyter} \PYG{n}{nbextension} \PYG{n}{enable} \PYG{n}{toc2}\PYG{o}{/}\PYG{n}{main}
\end{sphinxVerbatim}


\section{In windows this can be useful}
\label{\detokenize{index:in-windows-this-can-be-useful}}
\begin{sphinxVerbatim}[commandchars=\\\{\}]
\PYG{n}{conda} \PYG{n}{install} \PYG{n}{xlwings}
\end{sphinxVerbatim}


\section{To update ModelFlow}
\label{\detokenize{index:to-update-modelflow}}
\begin{sphinxVerbatim}[commandchars=\\\{\}]
\PYG{n}{conda} \PYG{n}{update} \PYG{n}{modelflow} \PYG{o}{\PYGZhy{}}\PYG{n}{c} \PYG{n}{ibh} \PYG{o}{\PYGZhy{}}\PYG{n}{c} \PYG{n}{conda}\PYG{o}{\PYGZhy{}}\PYG{n}{forge}  \PYG{o}{\PYGZhy{}}\PYG{n}{y}
\end{sphinxVerbatim}


\chapter{Core Modules, creates and solves model instances}
\label{\detokenize{index:core-modules-creates-and-solves-model-instances}}

\section{Modelclass, Defines the model class}
\label{\detokenize{index:module-modelclass}}\label{\detokenize{index:modelclass-defines-the-model-class}}\index{module@\spxentry{module}!modelclass@\spxentry{modelclass}}\index{modelclass@\spxentry{modelclass}!module@\spxentry{module}}
\sphinxAtStartPar
Created on Mon Sep 02 19:41:11 2013

\sphinxAtStartPar
This module creates model class instances.

\sphinxAtStartPar
@author: Ib
\index{node (class in modelclass)@\spxentry{node}\spxextra{class in modelclass}}

\begin{fulllineitems}
\phantomsection\label{\detokenize{index:modelclass.node}}
\pysigstartsignatures
\pysiglinewithargsret{\sphinxbfcode{\sphinxupquote{class\DUrole{w}{  }}}\sphinxcode{\sphinxupquote{modelclass.}}\sphinxbfcode{\sphinxupquote{node}}}{\emph{\DUrole{n}{lev}}, \emph{\DUrole{n}{parent}}, \emph{\DUrole{n}{child}}}{}
\pysigstopsignatures
\sphinxAtStartPar
A named tuple used when to drawing the logical structure. Describes an edge of the dependency graph
\begin{quote}\begin{description}
\item[{Lev}] \leavevmode
\sphinxAtStartPar
Level from start

\item[{Parent}] \leavevmode
\sphinxAtStartPar
The parent

\item[{Child}] \leavevmode
\sphinxAtStartPar
The child

\end{description}\end{quote}
\index{child (modelclass.node attribute)@\spxentry{child}\spxextra{modelclass.node attribute}}

\begin{fulllineitems}
\phantomsection\label{\detokenize{index:modelclass.node.child}}
\pysigstartsignatures
\pysigline{\sphinxbfcode{\sphinxupquote{child}}}
\pysigstopsignatures
\sphinxAtStartPar
Alias for field number 2

\end{fulllineitems}

\index{lev (modelclass.node attribute)@\spxentry{lev}\spxextra{modelclass.node attribute}}

\begin{fulllineitems}
\phantomsection\label{\detokenize{index:modelclass.node.lev}}
\pysigstartsignatures
\pysigline{\sphinxbfcode{\sphinxupquote{lev}}}
\pysigstopsignatures
\sphinxAtStartPar
Alias for field number 0

\end{fulllineitems}

\index{parent (modelclass.node attribute)@\spxentry{parent}\spxextra{modelclass.node attribute}}

\begin{fulllineitems}
\phantomsection\label{\detokenize{index:modelclass.node.parent}}
\pysigstartsignatures
\pysigline{\sphinxbfcode{\sphinxupquote{parent}}}
\pysigstopsignatures
\sphinxAtStartPar
Alias for field number 1

\end{fulllineitems}


\end{fulllineitems}

\index{BaseModel (class in modelclass)@\spxentry{BaseModel}\spxextra{class in modelclass}}

\begin{fulllineitems}
\phantomsection\label{\detokenize{index:modelclass.BaseModel}}
\pysigstartsignatures
\pysiglinewithargsret{\sphinxbfcode{\sphinxupquote{class\DUrole{w}{  }}}\sphinxcode{\sphinxupquote{modelclass.}}\sphinxbfcode{\sphinxupquote{BaseModel}}}{\emph{\DUrole{n}{i\_eq}\DUrole{o}{=}\DUrole{default_value}{\textquotesingle{}\textquotesingle{}}}, \emph{\DUrole{n}{modelname}\DUrole{o}{=}\DUrole{default_value}{\textquotesingle{}testmodel\textquotesingle{}}}, \emph{\DUrole{n}{silent}\DUrole{o}{=}\DUrole{default_value}{False}}, \emph{\DUrole{n}{straight}\DUrole{o}{=}\DUrole{default_value}{False}}, \emph{\DUrole{n}{funks}\DUrole{o}{=}\DUrole{default_value}{{[}{]}}}, \emph{\DUrole{n}{tabcomplete}\DUrole{o}{=}\DUrole{default_value}{True}}, \emph{\DUrole{n}{previousbase}\DUrole{o}{=}\DUrole{default_value}{False}}, \emph{\DUrole{n}{use\_preorder}\DUrole{o}{=}\DUrole{default_value}{True}}, \emph{\DUrole{n}{normalized}\DUrole{o}{=}\DUrole{default_value}{True}}, \emph{\DUrole{n}{safeorder}\DUrole{o}{=}\DUrole{default_value}{False}}, \emph{\DUrole{n}{var\_description}\DUrole{o}{=}\DUrole{default_value}{\{\}}}, \emph{\DUrole{o}{**}\DUrole{n}{kwargs}}}{}
\pysigstopsignatures
\sphinxAtStartPar
Class which defines a model from equations

\sphinxAtStartPar
In itself the BaseModel is of no use.

\sphinxAtStartPar
The \sphinxstylestrong{model} class enriches BaseModel with additional
Mixin classes which has additional methods and properties.

\sphinxAtStartPar
A model instance has a number of properties among which theese can be particular useful:
\begin{quote}
\begin{quote}\begin{description}
\item[{allvar}] \leavevmode
\sphinxAtStartPar
Information regarding all variables

\item[{basedf}] \leavevmode
\sphinxAtStartPar
A dataframe with first result created with this model instance

\item[{lastdf}] \leavevmode
\sphinxAtStartPar
A dataframe with the last result created with this model instance

\end{description}\end{quote}
\end{quote}

\sphinxAtStartPar
The two result dataframes are used for comparision and visualisation. The user can set both basedf and altdf.
\index{from\_eq() (modelclass.BaseModel class method)@\spxentry{from\_eq()}\spxextra{modelclass.BaseModel class method}}

\begin{fulllineitems}
\phantomsection\label{\detokenize{index:modelclass.BaseModel.from_eq}}
\pysigstartsignatures
\pysiglinewithargsret{\sphinxbfcode{\sphinxupquote{classmethod\DUrole{w}{  }}}\sphinxbfcode{\sphinxupquote{from\_eq}}}{\emph{\DUrole{n}{equations}}, \emph{\DUrole{n}{modelname}\DUrole{o}{=}\DUrole{default_value}{\textquotesingle{}testmodel\textquotesingle{}}}, \emph{\DUrole{n}{silent}\DUrole{o}{=}\DUrole{default_value}{False}}, \emph{\DUrole{n}{straight}\DUrole{o}{=}\DUrole{default_value}{False}}, \emph{\DUrole{n}{funks}\DUrole{o}{=}\DUrole{default_value}{{[}{]}}}, \emph{\DUrole{n}{params}\DUrole{o}{=}\DUrole{default_value}{\{\}}}, \emph{\DUrole{n}{tabcomplete}\DUrole{o}{=}\DUrole{default_value}{True}}, \emph{\DUrole{n}{previousbase}\DUrole{o}{=}\DUrole{default_value}{False}}, \emph{\DUrole{n}{normalized}\DUrole{o}{=}\DUrole{default_value}{True}}, \emph{\DUrole{n}{norm}\DUrole{o}{=}\DUrole{default_value}{True}}, \emph{\DUrole{n}{sym}\DUrole{o}{=}\DUrole{default_value}{False}}, \emph{\DUrole{n}{sep}\DUrole{o}{=}\DUrole{default_value}{\textquotesingle{}\textbackslash{}n\textquotesingle{}}}, \emph{\DUrole{o}{**}\DUrole{n}{kwargs}}}{}
\pysigstopsignatures
\sphinxAtStartPar
Creates a model from macro Business logic language.

\sphinxAtStartPar
That is the model specification is first exploded.
\begin{quote}\begin{description}
\item[{Parameters}] \leavevmode\begin{itemize}
\item {} 
\sphinxAtStartPar
\sphinxstyleliteralstrong{\sphinxupquote{equations}} \textendash{} The model

\item {} 
\sphinxAtStartPar
\sphinxstyleliteralstrong{\sphinxupquote{modelname}} \textendash{} Name of the model. Defaults to ‘testmodel’.

\item {} 
\sphinxAtStartPar
\sphinxstyleliteralstrong{\sphinxupquote{silent}} \textendash{} Suppress messages. Defaults to False.

\item {} 
\sphinxAtStartPar
\sphinxstyleliteralstrong{\sphinxupquote{straigth}} \textendash{} Don’t reorder the model. Defaults to False.

\item {} 
\sphinxAtStartPar
\sphinxstyleliteralstrong{\sphinxupquote{funks}} \textendash{} Functions incorporated in the model specification . Defaults to {[}{]}.

\item {} 
\sphinxAtStartPar
\sphinxstyleliteralstrong{\sphinxupquote{params}} \textendash{} For later use. Defaults to \{\}.

\item {} 
\sphinxAtStartPar
\sphinxstyleliteralstrong{\sphinxupquote{tabcomplete}} \textendash{} Allow tab compleetion in editor, for large model time consuming. Defaults to True.

\item {} 
\sphinxAtStartPar
\sphinxstyleliteralstrong{\sphinxupquote{previousbase}} \textendash{} Use previous run as basedf not the first. Defaults to False.

\item {} 
\sphinxAtStartPar
\sphinxstyleliteralstrong{\sphinxupquote{norm}} \textendash{} Normalize the model. Defaults to True.

\item {} 
\sphinxAtStartPar
\sphinxstyleliteralstrong{\sphinxupquote{sym}} \textendash{} If normalize do it symbolic. Defaults to False.

\item {} 
\sphinxAtStartPar
\sphinxstyleliteralstrong{\sphinxupquote{sep}} \textendash{} Seperate the equations. Defaults to newline.

\end{itemize}

\item[{Returns}] \leavevmode
\sphinxAtStartPar
A model instance

\end{description}\end{quote}

\end{fulllineitems}

\index{get\_histmodel() (modelclass.BaseModel method)@\spxentry{get\_histmodel()}\spxextra{modelclass.BaseModel method}}

\begin{fulllineitems}
\phantomsection\label{\detokenize{index:modelclass.BaseModel.get_histmodel}}
\pysigstartsignatures
\pysiglinewithargsret{\sphinxbfcode{\sphinxupquote{get\_histmodel}}}{}{}
\pysigstopsignatures
\sphinxAtStartPar
return a model instance with a model which generates historic values for equations
marked by a frml name I or IDENT

\sphinxAtStartPar
Uses {\hyperref[\detokenize{index:modelmanipulation.find_hist_model}]{\sphinxcrossref{\sphinxcode{\sphinxupquote{find\_hist\_model}}}}}

\end{fulllineitems}

\index{analyzemodelnew() (modelclass.BaseModel method)@\spxentry{analyzemodelnew()}\spxextra{modelclass.BaseModel method}}

\begin{fulllineitems}
\phantomsection\label{\detokenize{index:modelclass.BaseModel.analyzemodelnew}}
\pysigstartsignatures
\pysiglinewithargsret{\sphinxbfcode{\sphinxupquote{analyzemodelnew}}}{\emph{\DUrole{n}{silent}}}{}
\pysigstopsignatures
\sphinxAtStartPar
Analyze a model

\sphinxAtStartPar
The function creats:\sphinxstylestrong{Self.allvar} is a dictory with an entry for every variable in the model
the key is the variable name.
For each endogeneous variable there is a directory with thees keys:
\begin{quote}\begin{description}
\item[{Maxlag}] \leavevmode
\sphinxAtStartPar
The max lag for this variable

\item[{Maxlead}] \leavevmode
\sphinxAtStartPar
The max Lead for this variable

\item[{Endo}] \leavevmode
\sphinxAtStartPar
1 if the variable is endogeneous (ie on the left hand side of =

\item[{Frml}] \leavevmode
\sphinxAtStartPar
String with the formular for this variable

\item[{Frmlnumber}] \leavevmode
\sphinxAtStartPar
The number of the formular

\item[{Varnr}] \leavevmode
\sphinxAtStartPar
Number of this variable

\item[{Terms}] \leavevmode
\sphinxAtStartPar
The frml for this variable translated to terms

\item[{Frmlname}] \leavevmode
\sphinxAtStartPar
The frmlname for this variable

\item[{Startnr}] \leavevmode
\sphinxAtStartPar
Start of this variable in gauss seidel solutio vector :Advanced:

\item[{Matrix}] \leavevmode
\sphinxAtStartPar
This lhs element is a matrix

\item[{Dropfrml}] \leavevmode
\sphinxAtStartPar
If this frml shoud be excluded from the evaluation.

\end{description}\end{quote}

\sphinxAtStartPar
In addition theese properties will be created:
\begin{quote}\begin{description}
\item[{Endogene}] \leavevmode
\sphinxAtStartPar
Set of endogeneous variable in the model

\item[{Exogene}] \leavevmode
\sphinxAtStartPar
Se exogeneous variable in the model

\item[{Maxnavlen}] \leavevmode
\sphinxAtStartPar
The longest variable name

\item[{Blank}] \leavevmode
\sphinxAtStartPar
An emty string which can contain the longest variable name

\item[{Solveorder}] \leavevmode
\sphinxAtStartPar
The order in which the model is solved \sphinxhyphen{} initaly the order of the equations in the model

\item[{Normalized}] \leavevmode
\sphinxAtStartPar
This model is normalized

\item[{Endogene\_true}] \leavevmode
\sphinxAtStartPar
Set of endogeneous variables in model if normalized, else the set of declared endogeneous variables

\end{description}\end{quote}

\end{fulllineitems}

\index{smpl() (modelclass.BaseModel method)@\spxentry{smpl()}\spxextra{modelclass.BaseModel method}}

\begin{fulllineitems}
\phantomsection\label{\detokenize{index:modelclass.BaseModel.smpl}}
\pysigstartsignatures
\pysiglinewithargsret{\sphinxbfcode{\sphinxupquote{smpl}}}{\emph{\DUrole{n}{start}\DUrole{o}{=}\DUrole{default_value}{\textquotesingle{}\textquotesingle{}}}, \emph{\DUrole{n}{slut}\DUrole{o}{=}\DUrole{default_value}{\textquotesingle{}\textquotesingle{}}}, \emph{\DUrole{n}{df}\DUrole{o}{=}\DUrole{default_value}{None}}}{}
\pysigstopsignatures
\sphinxAtStartPar
Defines the model.current\_per which is used for calculation period/index
when no parameters are issues the current current period is returned

\sphinxAtStartPar
Either none or all parameters have to be provided

\end{fulllineitems}

\index{check\_sim\_smpl() (modelclass.BaseModel method)@\spxentry{check\_sim\_smpl()}\spxextra{modelclass.BaseModel method}}

\begin{fulllineitems}
\phantomsection\label{\detokenize{index:modelclass.BaseModel.check_sim_smpl}}
\pysigstartsignatures
\pysiglinewithargsret{\sphinxbfcode{\sphinxupquote{check\_sim\_smpl}}}{\emph{\DUrole{n}{databank}}}{}
\pysigstopsignatures
\sphinxAtStartPar
Checks if the current period (the SMPL) is can contain the lags and the leads

\end{fulllineitems}

\index{set\_smpl() (modelclass.BaseModel method)@\spxentry{set\_smpl()}\spxextra{modelclass.BaseModel method}}

\begin{fulllineitems}
\phantomsection\label{\detokenize{index:modelclass.BaseModel.set_smpl}}
\pysigstartsignatures
\pysiglinewithargsret{\sphinxbfcode{\sphinxupquote{set\_smpl}}}{\emph{\DUrole{n}{start}\DUrole{o}{=}\DUrole{default_value}{\textquotesingle{}\textquotesingle{}}}, \emph{\DUrole{n}{slut}\DUrole{o}{=}\DUrole{default_value}{\textquotesingle{}\textquotesingle{}}}, \emph{\DUrole{n}{df}\DUrole{o}{=}\DUrole{default_value}{None}}}{}
\pysigstopsignatures
\sphinxAtStartPar
Sets the scope for the models time range, and restors it afterward
\begin{quote}\begin{description}
\item[{Parameters}] \leavevmode\begin{itemize}
\item {} 
\sphinxAtStartPar
\sphinxstyleliteralstrong{\sphinxupquote{start}} \textendash{} Start time. Defaults to ‘’.

\item {} 
\sphinxAtStartPar
\sphinxstyleliteralstrong{\sphinxupquote{slut}} \textendash{} End time. Defaults to ‘’.

\item {} 
\sphinxAtStartPar
\sphinxstyleliteralstrong{\sphinxupquote{df}} (\sphinxstyleliteralemphasis{\sphinxupquote{Dataframe}}\sphinxstyleliteralemphasis{\sphinxupquote{, }}\sphinxstyleliteralemphasis{\sphinxupquote{optional}}) \textendash{} Used on a dataframe not self.basedf. Defaults to None.

\end{itemize}

\end{description}\end{quote}

\end{fulllineitems}

\index{set\_smpl\_relative() (modelclass.BaseModel method)@\spxentry{set\_smpl\_relative()}\spxextra{modelclass.BaseModel method}}

\begin{fulllineitems}
\phantomsection\label{\detokenize{index:modelclass.BaseModel.set_smpl_relative}}
\pysigstartsignatures
\pysiglinewithargsret{\sphinxbfcode{\sphinxupquote{set\_smpl\_relative}}}{\emph{\DUrole{n}{start\_ofset}\DUrole{o}{=}\DUrole{default_value}{0}}, \emph{\DUrole{n}{slut\_ofset}\DUrole{o}{=}\DUrole{default_value}{0}}}{}
\pysigstopsignatures
\sphinxAtStartPar
Sets the scope for the models time range relative to the current, and restores it afterward

\end{fulllineitems}

\index{keepswitch() (modelclass.BaseModel method)@\spxentry{keepswitch()}\spxextra{modelclass.BaseModel method}}

\begin{fulllineitems}
\phantomsection\label{\detokenize{index:modelclass.BaseModel.keepswitch}}
\pysigstartsignatures
\pysiglinewithargsret{\sphinxbfcode{\sphinxupquote{keepswitch}}}{\emph{\DUrole{n}{switch}\DUrole{o}{=}\DUrole{default_value}{False}}, \emph{\DUrole{n}{scenarios}\DUrole{o}{=}\DUrole{default_value}{\textquotesingle{}*\textquotesingle{}}}}{}
\pysigstopsignatures
\sphinxAtStartPar
temporary place basedf,lastdf in keep\_solutions
if scenarios contains * or ? they are separated by | else space

\end{fulllineitems}

\index{endograph (modelclass.BaseModel property)@\spxentry{endograph}\spxextra{modelclass.BaseModel property}}

\begin{fulllineitems}
\phantomsection\label{\detokenize{index:modelclass.BaseModel.endograph}}
\pysigstartsignatures
\pysigline{\sphinxbfcode{\sphinxupquote{property\DUrole{w}{  }}}\sphinxbfcode{\sphinxupquote{endograph}}}
\pysigstopsignatures
\sphinxAtStartPar
Dependencygraph for currrent periode endogeneous variable, used for reorder the equations
if self.safeorder is true feedback for all lags are included

\sphinxAtStartPar
safeorder was a fix to handle lags = \sphinxhyphen{}0 which unexpected was used in WB models. Now it is handeled in modelpattern

\end{fulllineitems}

\index{calculate\_freq (modelclass.BaseModel property)@\spxentry{calculate\_freq}\spxextra{modelclass.BaseModel property}}

\begin{fulllineitems}
\phantomsection\label{\detokenize{index:modelclass.BaseModel.calculate_freq}}
\pysigstartsignatures
\pysigline{\sphinxbfcode{\sphinxupquote{property\DUrole{w}{  }}}\sphinxbfcode{\sphinxupquote{calculate\_freq}}}
\pysigstopsignatures
\sphinxAtStartPar
The number of operators in the model

\end{fulllineitems}

\index{flop\_get (modelclass.BaseModel property)@\spxentry{flop\_get}\spxextra{modelclass.BaseModel property}}

\begin{fulllineitems}
\phantomsection\label{\detokenize{index:modelclass.BaseModel.flop_get}}
\pysigstartsignatures
\pysigline{\sphinxbfcode{\sphinxupquote{property\DUrole{w}{  }}}\sphinxbfcode{\sphinxupquote{flop\_get}}}
\pysigstopsignatures
\sphinxAtStartPar
The number of operators in the model prolog,core and epilog

\end{fulllineitems}

\index{get\_columnsnr() (modelclass.BaseModel method)@\spxentry{get\_columnsnr()}\spxextra{modelclass.BaseModel method}}

\begin{fulllineitems}
\phantomsection\label{\detokenize{index:modelclass.BaseModel.get_columnsnr}}
\pysigstartsignatures
\pysiglinewithargsret{\sphinxbfcode{\sphinxupquote{get\_columnsnr}}}{\emph{\DUrole{n}{df}}}{}
\pysigstopsignatures
\sphinxAtStartPar
returns a dict a databanks variables as keys and column number as item
used for fast getting and setting of variable values in the dataframe

\end{fulllineitems}

\index{outeval() (modelclass.BaseModel method)@\spxentry{outeval()}\spxextra{modelclass.BaseModel method}}

\begin{fulllineitems}
\phantomsection\label{\detokenize{index:modelclass.BaseModel.outeval}}
\pysigstartsignatures
\pysiglinewithargsret{\sphinxbfcode{\sphinxupquote{outeval}}}{\emph{\DUrole{n}{databank}}}{}
\pysigstopsignatures
\sphinxAtStartPar
takes a list of terms and translates to a evaluater function called los

\sphinxAtStartPar
The model axcess the data through:Dataframe.value{[}rowindex+lag,coloumnindex{]} which is very efficient

\end{fulllineitems}

\index{eqcolumns() (modelclass.BaseModel method)@\spxentry{eqcolumns()}\spxextra{modelclass.BaseModel method}}

\begin{fulllineitems}
\phantomsection\label{\detokenize{index:modelclass.BaseModel.eqcolumns}}
\pysigstartsignatures
\pysiglinewithargsret{\sphinxbfcode{\sphinxupquote{eqcolumns}}}{\emph{\DUrole{n}{a}}, \emph{\DUrole{n}{b}}}{}
\pysigstopsignatures
\sphinxAtStartPar
compares two lists

\end{fulllineitems}

\index{xgenr() (modelclass.BaseModel method)@\spxentry{xgenr()}\spxextra{modelclass.BaseModel method}}

\begin{fulllineitems}
\phantomsection\label{\detokenize{index:modelclass.BaseModel.xgenr}}
\pysigstartsignatures
\pysiglinewithargsret{\sphinxbfcode{\sphinxupquote{xgenr}}}{\emph{\DUrole{n}{databank}}, \emph{\DUrole{n}{start}\DUrole{o}{=}\DUrole{default_value}{\textquotesingle{}\textquotesingle{}}}, \emph{\DUrole{n}{slut}\DUrole{o}{=}\DUrole{default_value}{\textquotesingle{}\textquotesingle{}}}, \emph{\DUrole{n}{silent}\DUrole{o}{=}\DUrole{default_value}{0}}, \emph{\DUrole{n}{samedata}\DUrole{o}{=}\DUrole{default_value}{1}}, \emph{\DUrole{o}{**}\DUrole{n}{kwargs}}}{}
\pysigstopsignatures
\sphinxAtStartPar
Evaluates this model on a databank from start to slut (means end in Danish).

\sphinxAtStartPar
First it finds the values in the Dataframe, then creates the evaluater function through the \sphinxstyleemphasis{outeval} function
(\sphinxcode{\sphinxupquote{modelclass.model.fouteval()}})
then it evaluates the function and returns the values to a the Dataframe in the databank.

\sphinxAtStartPar
The text for the evaluater function is placed in the model property \sphinxstylestrong{make\_los\_text}
where it can be inspected
in case of problems.

\end{fulllineitems}

\index{findpos() (modelclass.BaseModel method)@\spxentry{findpos()}\spxextra{modelclass.BaseModel method}}

\begin{fulllineitems}
\phantomsection\label{\detokenize{index:modelclass.BaseModel.findpos}}
\pysigstartsignatures
\pysiglinewithargsret{\sphinxbfcode{\sphinxupquote{findpos}}}{}{}
\pysigstopsignatures
\sphinxAtStartPar
find a startposition in the calculation array for a model
places startposition for each variable in model.allvar{[}variable{]}{[}‘startpos’{]}
places the max startposition in model.maxstart

\end{fulllineitems}

\index{make\_gaussline() (modelclass.BaseModel method)@\spxentry{make\_gaussline()}\spxextra{modelclass.BaseModel method}}

\begin{fulllineitems}
\phantomsection\label{\detokenize{index:modelclass.BaseModel.make_gaussline}}
\pysigstartsignatures
\pysiglinewithargsret{\sphinxbfcode{\sphinxupquote{make\_gaussline}}}{\emph{\DUrole{n}{vx}}, \emph{\DUrole{n}{nodamp}\DUrole{o}{=}\DUrole{default_value}{False}}}{}
\pysigstopsignatures
\sphinxAtStartPar
takes a list of terms and translates to a line in a gauss\sphinxhyphen{}seidel solver for
simultanius models
the variables are mapped to position in a vector which has all relevant varaibles lagged
this is in order to provide opertunity to optimise data and solving

\sphinxAtStartPar
New version to take hand of several lhs variables. Dampning is not allowed for
this. But can easely be implemented by makeing a function to multiply tupels

\end{fulllineitems}

\index{make\_resline() (modelclass.BaseModel method)@\spxentry{make\_resline()}\spxextra{modelclass.BaseModel method}}

\begin{fulllineitems}
\phantomsection\label{\detokenize{index:modelclass.BaseModel.make_resline}}
\pysigstartsignatures
\pysiglinewithargsret{\sphinxbfcode{\sphinxupquote{make\_resline}}}{\emph{\DUrole{n}{vx}}}{}
\pysigstopsignatures
\sphinxAtStartPar
takes a list of terms and translates to a line calculating line

\end{fulllineitems}

\index{createstuff3() (modelclass.BaseModel method)@\spxentry{createstuff3()}\spxextra{modelclass.BaseModel method}}

\begin{fulllineitems}
\phantomsection\label{\detokenize{index:modelclass.BaseModel.createstuff3}}
\pysigstartsignatures
\pysiglinewithargsret{\sphinxbfcode{\sphinxupquote{createstuff3}}}{\emph{\DUrole{n}{dfxx}}}{}
\pysigstopsignatures
\sphinxAtStartPar
Connect a dataframe with the solution vector used by the iterative sim2 solver)
return a function to place data in solution vector and to retrieve it again.

\end{fulllineitems}

\index{outsolve() (modelclass.BaseModel method)@\spxentry{outsolve()}\spxextra{modelclass.BaseModel method}}

\begin{fulllineitems}
\phantomsection\label{\detokenize{index:modelclass.BaseModel.outsolve}}
\pysigstartsignatures
\pysiglinewithargsret{\sphinxbfcode{\sphinxupquote{outsolve}}}{\emph{\DUrole{n}{order}\DUrole{o}{=}\DUrole{default_value}{\textquotesingle{}\textquotesingle{}}}, \emph{\DUrole{n}{exclude}\DUrole{o}{=}\DUrole{default_value}{{[}{]}}}}{}
\pysigstopsignatures
\sphinxAtStartPar
returns a string with a function which calculates a
Gauss\sphinxhyphen{}Seidle iteration of a model
exclude is list of endogeneous variables not to be solved
uses:
model.solveorder the order in which the variables is calculated
model.allvar{[}v{]}{[}“gauss”{]} the ccalculation

\end{fulllineitems}

\index{make\_solver() (modelclass.BaseModel method)@\spxentry{make\_solver()}\spxextra{modelclass.BaseModel method}}

\begin{fulllineitems}
\phantomsection\label{\detokenize{index:modelclass.BaseModel.make_solver}}
\pysigstartsignatures
\pysiglinewithargsret{\sphinxbfcode{\sphinxupquote{make\_solver}}}{\emph{\DUrole{n}{ljit}\DUrole{o}{=}\DUrole{default_value}{False}}, \emph{\DUrole{n}{order}\DUrole{o}{=}\DUrole{default_value}{\textquotesingle{}\textquotesingle{}}}, \emph{\DUrole{n}{exclude}\DUrole{o}{=}\DUrole{default_value}{{[}{]}}}, \emph{\DUrole{n}{cache}\DUrole{o}{=}\DUrole{default_value}{False}}}{}
\pysigstopsignatures
\sphinxAtStartPar
makes a function which performs a Gaus\sphinxhyphen{}Seidle iteration
if ljit=True a Jittet function will also be created.
The functions will be placed in:
model.solve
model.solve\_jit

\end{fulllineitems}

\index{base\_sim() (modelclass.BaseModel method)@\spxentry{base\_sim()}\spxextra{modelclass.BaseModel method}}

\begin{fulllineitems}
\phantomsection\label{\detokenize{index:modelclass.BaseModel.base_sim}}
\pysigstartsignatures
\pysiglinewithargsret{\sphinxbfcode{\sphinxupquote{base\_sim}}}{\emph{\DUrole{n}{databank}}, \emph{\DUrole{n}{start}\DUrole{o}{=}\DUrole{default_value}{\textquotesingle{}\textquotesingle{}}}, \emph{\DUrole{n}{slut}\DUrole{o}{=}\DUrole{default_value}{\textquotesingle{}\textquotesingle{}}}, \emph{\DUrole{n}{max\_iterations}\DUrole{o}{=}\DUrole{default_value}{1}}, \emph{\DUrole{n}{first\_test}\DUrole{o}{=}\DUrole{default_value}{1}}, \emph{\DUrole{n}{ljit}\DUrole{o}{=}\DUrole{default_value}{False}}, \emph{\DUrole{n}{exclude}\DUrole{o}{=}\DUrole{default_value}{{[}{]}}}, \emph{\DUrole{n}{silent}\DUrole{o}{=}\DUrole{default_value}{False}}, \emph{\DUrole{n}{new}\DUrole{o}{=}\DUrole{default_value}{False}}, \emph{\DUrole{n}{conv}\DUrole{o}{=}\DUrole{default_value}{{[}{]}}}, \emph{\DUrole{n}{samedata}\DUrole{o}{=}\DUrole{default_value}{True}}, \emph{\DUrole{n}{dumpvar}\DUrole{o}{=}\DUrole{default_value}{{[}{]}}}, \emph{\DUrole{n}{ldumpvar}\DUrole{o}{=}\DUrole{default_value}{False}}, \emph{\DUrole{n}{dumpwith}\DUrole{o}{=}\DUrole{default_value}{15}}, \emph{\DUrole{n}{dumpdecimal}\DUrole{o}{=}\DUrole{default_value}{5}}, \emph{\DUrole{n}{lcython}\DUrole{o}{=}\DUrole{default_value}{False}}, \emph{\DUrole{n}{setbase}\DUrole{o}{=}\DUrole{default_value}{False}}, \emph{\DUrole{n}{setlast}\DUrole{o}{=}\DUrole{default_value}{True}}, \emph{\DUrole{n}{alfa}\DUrole{o}{=}\DUrole{default_value}{0.2}}, \emph{\DUrole{n}{sim}\DUrole{o}{=}\DUrole{default_value}{True}}, \emph{\DUrole{n}{absconv}\DUrole{o}{=}\DUrole{default_value}{0.01}}, \emph{\DUrole{n}{relconv}\DUrole{o}{=}\DUrole{default_value}{1e\sphinxhyphen{}05}}, \emph{\DUrole{n}{debug}\DUrole{o}{=}\DUrole{default_value}{False}}, \emph{\DUrole{n}{stats}\DUrole{o}{=}\DUrole{default_value}{False}}, \emph{\DUrole{o}{**}\DUrole{n}{kwargs}}}{}
\pysigstopsignatures
\sphinxAtStartPar
solves a model with data from a databank if the model has a solve function else it will be created.

\sphinxAtStartPar
The default options are resonable for most use:
\begin{quote}\begin{description}
\item[{Parameters}] \leavevmode\begin{itemize}
\item {} 
\sphinxAtStartPar
\sphinxstyleliteralstrong{\sphinxupquote{start}}\sphinxstyleliteralstrong{\sphinxupquote{,}}\sphinxstyleliteralstrong{\sphinxupquote{slut}} \textendash{} Start and end of simulation, default as much as possible taking max lag into acount

\item {} 
\sphinxAtStartPar
\sphinxstyleliteralstrong{\sphinxupquote{max\_iterations}} \textendash{} Max interations

\item {} 
\sphinxAtStartPar
\sphinxstyleliteralstrong{\sphinxupquote{first\_test}} \textendash{} First iteration where convergence is tested

\item {} 
\sphinxAtStartPar
\sphinxstyleliteralstrong{\sphinxupquote{ljit}} \textendash{} If True Numba is used to compile just in time \sphinxhyphen{} takes time but speeds solving up

\item {} 
\sphinxAtStartPar
\sphinxstyleliteralstrong{\sphinxupquote{new}} \textendash{} Force creation a new version of the solver (for testing)

\item {} 
\sphinxAtStartPar
\sphinxstyleliteralstrong{\sphinxupquote{exclude}} \textendash{} Don’t use use theese foormulas

\item {} 
\sphinxAtStartPar
\sphinxstyleliteralstrong{\sphinxupquote{silent}} \textendash{} Suppres solving informations

\item {} 
\sphinxAtStartPar
\sphinxstyleliteralstrong{\sphinxupquote{conv}} \textendash{} Variables on which to measure if convergence has been achived

\item {} 
\sphinxAtStartPar
\sphinxstyleliteralstrong{\sphinxupquote{samedata}} \textendash{} If False force a remap of datatrframe to solving vector (for testing)

\item {} 
\sphinxAtStartPar
\sphinxstyleliteralstrong{\sphinxupquote{dumpvar}} \textendash{} Variables to dump

\item {} 
\sphinxAtStartPar
\sphinxstyleliteralstrong{\sphinxupquote{ldumpvar}} \textendash{} toggels dumping of dumpvar

\item {} 
\sphinxAtStartPar
\sphinxstyleliteralstrong{\sphinxupquote{dumpwith}} \textendash{} with of dumps

\item {} 
\sphinxAtStartPar
\sphinxstyleliteralstrong{\sphinxupquote{dumpdecimal}} \textendash{} decimals in dumps

\item {} 
\sphinxAtStartPar
\sphinxstyleliteralstrong{\sphinxupquote{lcython}} \textendash{} Use Cython to compile the model (experimental )

\item {} 
\sphinxAtStartPar
\sphinxstyleliteralstrong{\sphinxupquote{alfa}} \textendash{} Dampning of formulas marked for dampning (\textless{}Z\textgreater{} in frml name)

\item {} 
\sphinxAtStartPar
\sphinxstyleliteralstrong{\sphinxupquote{sim}} \textendash{} For later use

\item {} 
\sphinxAtStartPar
\sphinxstyleliteralstrong{\sphinxupquote{absconv}} \textendash{} Treshold for applying relconv to test convergence

\item {} 
\sphinxAtStartPar
\sphinxstyleliteralstrong{\sphinxupquote{relconv}} \textendash{} Test for convergence

\item {} 
\sphinxAtStartPar
\sphinxstyleliteralstrong{\sphinxupquote{debug}} \textendash{} Output debug information

\item {} 
\sphinxAtStartPar
\sphinxstyleliteralstrong{\sphinxupquote{stats}} \textendash{} Output solving statistics

\end{itemize}

\item[{Return outdf}] \leavevmode
\sphinxAtStartPar
A dataframe with the solution

\end{description}\end{quote}

\end{fulllineitems}

\index{outres() (modelclass.BaseModel method)@\spxentry{outres()}\spxextra{modelclass.BaseModel method}}

\begin{fulllineitems}
\phantomsection\label{\detokenize{index:modelclass.BaseModel.outres}}
\pysigstartsignatures
\pysiglinewithargsret{\sphinxbfcode{\sphinxupquote{outres}}}{\emph{\DUrole{n}{order}\DUrole{o}{=}\DUrole{default_value}{\textquotesingle{}\textquotesingle{}}}, \emph{\DUrole{n}{exclude}\DUrole{o}{=}\DUrole{default_value}{{[}{]}}}}{}
\pysigstopsignatures
\sphinxAtStartPar
returns a string with a function which calculates a
calculation for residual check
exclude is list of endogeneous variables not to be solved
uses:
model.solveorder the order in which the variables is calculated

\end{fulllineitems}

\index{make\_res() (modelclass.BaseModel method)@\spxentry{make\_res()}\spxextra{modelclass.BaseModel method}}

\begin{fulllineitems}
\phantomsection\label{\detokenize{index:modelclass.BaseModel.make_res}}
\pysigstartsignatures
\pysiglinewithargsret{\sphinxbfcode{\sphinxupquote{make\_res}}}{\emph{\DUrole{n}{order}\DUrole{o}{=}\DUrole{default_value}{\textquotesingle{}\textquotesingle{}}}, \emph{\DUrole{n}{exclude}\DUrole{o}{=}\DUrole{default_value}{{[}{]}}}}{}
\pysigstopsignatures
\sphinxAtStartPar
makes a function which performs a Gaus\sphinxhyphen{}Seidle iteration
if ljit=True a Jittet function will also be created.
The functions will be placed in:
\begin{itemize}
\item {} 
\sphinxAtStartPar
model.solve

\item {} 
\sphinxAtStartPar
model.solve\_jit

\end{itemize}

\end{fulllineitems}

\index{base\_res() (modelclass.BaseModel method)@\spxentry{base\_res()}\spxextra{modelclass.BaseModel method}}

\begin{fulllineitems}
\phantomsection\label{\detokenize{index:modelclass.BaseModel.base_res}}
\pysigstartsignatures
\pysiglinewithargsret{\sphinxbfcode{\sphinxupquote{base\_res}}}{\emph{\DUrole{n}{databank}}, \emph{\DUrole{n}{start}\DUrole{o}{=}\DUrole{default_value}{\textquotesingle{}\textquotesingle{}}}, \emph{\DUrole{n}{slut}\DUrole{o}{=}\DUrole{default_value}{\textquotesingle{}\textquotesingle{}}}, \emph{\DUrole{n}{silent}\DUrole{o}{=}\DUrole{default_value}{1}}, \emph{\DUrole{o}{**}\DUrole{n}{kwargs}}}{}
\pysigstopsignatures
\sphinxAtStartPar
calculates a model with data from a databank
Used for check wether each equation gives the same result as in the original databank’

\end{fulllineitems}


\end{fulllineitems}

\index{Org\_model\_Mixin (class in modelclass)@\spxentry{Org\_model\_Mixin}\spxextra{class in modelclass}}

\begin{fulllineitems}
\phantomsection\label{\detokenize{index:modelclass.Org_model_Mixin}}
\pysigstartsignatures
\pysigline{\sphinxbfcode{\sphinxupquote{class\DUrole{w}{  }}}\sphinxcode{\sphinxupquote{modelclass.}}\sphinxbfcode{\sphinxupquote{Org\_model\_Mixin}}}
\pysigstopsignatures
\sphinxAtStartPar
The model class, used for calculating models

\sphinxAtStartPar
Compared to BaseModel it allows for simultaneous model and contains a number of properties
and functions to analyze and manipulate models and visualize results.
\index{lister (modelclass.Org\_model\_Mixin property)@\spxentry{lister}\spxextra{modelclass.Org\_model\_Mixin property}}

\begin{fulllineitems}
\phantomsection\label{\detokenize{index:modelclass.Org_model_Mixin.lister}}
\pysigstartsignatures
\pysigline{\sphinxbfcode{\sphinxupquote{property\DUrole{w}{  }}}\sphinxbfcode{\sphinxupquote{lister}}}
\pysigstopsignatures
\sphinxAtStartPar
lists used in the equations
\begin{quote}\begin{description}
\item[{Returns}] \leavevmode
\sphinxAtStartPar
Dictionary of lists defined in the input equations.

\item[{Return type}] \leavevmode
\sphinxAtStartPar
dict

\end{description}\end{quote}

\end{fulllineitems}

\index{listud (modelclass.Org\_model\_Mixin property)@\spxentry{listud}\spxextra{modelclass.Org\_model\_Mixin property}}

\begin{fulllineitems}
\phantomsection\label{\detokenize{index:modelclass.Org_model_Mixin.listud}}
\pysigstartsignatures
\pysigline{\sphinxbfcode{\sphinxupquote{property\DUrole{w}{  }}}\sphinxbfcode{\sphinxupquote{listud}}}
\pysigstopsignatures
\sphinxAtStartPar
returns a string of the models listdefinitions

\sphinxAtStartPar
used when ceating (small) models based on this model

\end{fulllineitems}

\index{vlist() (modelclass.Org\_model\_Mixin method)@\spxentry{vlist()}\spxextra{modelclass.Org\_model\_Mixin method}}

\begin{fulllineitems}
\phantomsection\label{\detokenize{index:modelclass.Org_model_Mixin.vlist}}
\pysigstartsignatures
\pysiglinewithargsret{\sphinxbfcode{\sphinxupquote{vlist}}}{\emph{\DUrole{n}{pat}}}{}
\pysigstopsignatures
\sphinxAtStartPar
Returns a list of variable in the model matching the pattern, the pattern can be a list of patterns
\begin{quote}\begin{description}
\item[{Parameters}] \leavevmode
\sphinxAtStartPar
\sphinxstyleliteralstrong{\sphinxupquote{pat}} (\sphinxstyleliteralemphasis{\sphinxupquote{string}}\sphinxstyleliteralemphasis{\sphinxupquote{ or }}\sphinxstyleliteralemphasis{\sphinxupquote{list of strings}}) \textendash{} One or more pattern seperated by space wildcards * and ?, special pattern: \#ENDO

\item[{Returns}] \leavevmode
\sphinxAtStartPar
list of variable names matching the pat.

\item[{Return type}] \leavevmode
\sphinxAtStartPar
out (list)

\end{description}\end{quote}

\end{fulllineitems}

\index{list\_names() (modelclass.Org\_model\_Mixin static method)@\spxentry{list\_names()}\spxextra{modelclass.Org\_model\_Mixin static method}}

\begin{fulllineitems}
\phantomsection\label{\detokenize{index:modelclass.Org_model_Mixin.list_names}}
\pysigstartsignatures
\pysiglinewithargsret{\sphinxbfcode{\sphinxupquote{static\DUrole{w}{  }}}\sphinxbfcode{\sphinxupquote{list\_names}}}{\emph{\DUrole{n}{input}}, \emph{\DUrole{n}{pat}}, \emph{\DUrole{n}{sort}\DUrole{o}{=}\DUrole{default_value}{True}}}{}
\pysigstopsignatures
\sphinxAtStartPar
returns a list of variable in input  matching the pattern, the pattern can be a list of patterns

\end{fulllineitems}

\index{exodif() (modelclass.Org\_model\_Mixin method)@\spxentry{exodif()}\spxextra{modelclass.Org\_model\_Mixin method}}

\begin{fulllineitems}
\phantomsection\label{\detokenize{index:modelclass.Org_model_Mixin.exodif}}
\pysigstartsignatures
\pysiglinewithargsret{\sphinxbfcode{\sphinxupquote{exodif}}}{\emph{\DUrole{n}{a}\DUrole{o}{=}\DUrole{default_value}{None}}, \emph{\DUrole{n}{b}\DUrole{o}{=}\DUrole{default_value}{None}}}{}
\pysigstopsignatures
\sphinxAtStartPar
Finds the differences between two dataframes in exogeneous variables for the model
Defaults to getting the two dataframes (basedf and lastdf) internal to the model instance

\sphinxAtStartPar
Exogeneous with a name ending in \textless{}endo\textgreater{}\_\_RES are not taken in, as they are part of a un\_normalized model
\begin{quote}\begin{description}
\item[{Parameters}] \leavevmode\begin{itemize}
\item {} 
\sphinxAtStartPar
\sphinxstyleliteralstrong{\sphinxupquote{a}} (\sphinxstyleliteralemphasis{\sphinxupquote{TYPE}}\sphinxstyleliteralemphasis{\sphinxupquote{, }}\sphinxstyleliteralemphasis{\sphinxupquote{optional}}) \textendash{} DESCRIPTION. Defaults to None. If None model.basedf will be used.

\item {} 
\sphinxAtStartPar
\sphinxstyleliteralstrong{\sphinxupquote{b}} (\sphinxstyleliteralemphasis{\sphinxupquote{TYPE}}\sphinxstyleliteralemphasis{\sphinxupquote{, }}\sphinxstyleliteralemphasis{\sphinxupquote{optional}}) \textendash{} DESCRIPTION. Defaults to None. If None model.lastdf will be used.

\end{itemize}

\item[{Returns}] \leavevmode
\sphinxAtStartPar
the difference between the models exogenous variables in a and b.

\item[{Return type}] \leavevmode
\sphinxAtStartPar
DataFrame

\end{description}\end{quote}

\end{fulllineitems}

\index{get\_eq\_values() (modelclass.Org\_model\_Mixin method)@\spxentry{get\_eq\_values()}\spxextra{modelclass.Org\_model\_Mixin method}}

\begin{fulllineitems}
\phantomsection\label{\detokenize{index:modelclass.Org_model_Mixin.get_eq_values}}
\pysigstartsignatures
\pysiglinewithargsret{\sphinxbfcode{\sphinxupquote{get\_eq\_values}}}{\emph{\DUrole{n}{varnavn}}, \emph{\DUrole{n}{last}\DUrole{o}{=}\DUrole{default_value}{True}}, \emph{\DUrole{n}{databank}\DUrole{o}{=}\DUrole{default_value}{None}}, \emph{\DUrole{n}{nolag}\DUrole{o}{=}\DUrole{default_value}{False}}, \emph{\DUrole{n}{per}\DUrole{o}{=}\DUrole{default_value}{None}}, \emph{\DUrole{n}{showvar}\DUrole{o}{=}\DUrole{default_value}{False}}, \emph{\DUrole{n}{alsoendo}\DUrole{o}{=}\DUrole{default_value}{False}}}{}
\pysigstopsignatures
\sphinxAtStartPar
Returns a dataframe with values from a frml determining a variable
\begin{description}
\item[{options:}] \leavevmode\begin{quote}\begin{description}
\item[{last}] \leavevmode
\sphinxAtStartPar
the lastdf is used else baseline dataframe

\item[{nolag}] \leavevmode
\sphinxAtStartPar
only line for each variable

\end{description}\end{quote}

\end{description}

\end{fulllineitems}

\index{get\_eq\_dif() (modelclass.Org\_model\_Mixin method)@\spxentry{get\_eq\_dif()}\spxextra{modelclass.Org\_model\_Mixin method}}

\begin{fulllineitems}
\phantomsection\label{\detokenize{index:modelclass.Org_model_Mixin.get_eq_dif}}
\pysigstartsignatures
\pysiglinewithargsret{\sphinxbfcode{\sphinxupquote{get\_eq\_dif}}}{\emph{\DUrole{n}{varnavn}}, \emph{\DUrole{n}{filter}\DUrole{o}{=}\DUrole{default_value}{False}}, \emph{\DUrole{n}{nolag}\DUrole{o}{=}\DUrole{default_value}{False}}, \emph{\DUrole{n}{showvar}\DUrole{o}{=}\DUrole{default_value}{False}}}{}
\pysigstopsignatures
\sphinxAtStartPar
returns a dataframe with difference of values from formula

\end{fulllineitems}

\index{get\_var\_growth() (modelclass.Org\_model\_Mixin method)@\spxentry{get\_var\_growth()}\spxextra{modelclass.Org\_model\_Mixin method}}

\begin{fulllineitems}
\phantomsection\label{\detokenize{index:modelclass.Org_model_Mixin.get_var_growth}}
\pysigstartsignatures
\pysiglinewithargsret{\sphinxbfcode{\sphinxupquote{get\_var\_growth}}}{\emph{\DUrole{n}{varname}}, \emph{\DUrole{n}{showname}\DUrole{o}{=}\DUrole{default_value}{False}}, \emph{\DUrole{n}{diff}\DUrole{o}{=}\DUrole{default_value}{False}}}{}
\pysigstopsignatures
\sphinxAtStartPar
Returns the  growth rate of this variable in the base and the last dataframe

\end{fulllineitems}

\index{get\_values() (modelclass.Org\_model\_Mixin method)@\spxentry{get\_values()}\spxextra{modelclass.Org\_model\_Mixin method}}

\begin{fulllineitems}
\phantomsection\label{\detokenize{index:modelclass.Org_model_Mixin.get_values}}
\pysigstartsignatures
\pysiglinewithargsret{\sphinxbfcode{\sphinxupquote{get\_values}}}{\emph{\DUrole{n}{v}}, \emph{\DUrole{n}{pct}\DUrole{o}{=}\DUrole{default_value}{False}}}{}
\pysigstopsignatures
\sphinxAtStartPar
returns a dataframe with the data points for a node,  including lags

\end{fulllineitems}

\index{\_\_getitem\_\_() (modelclass.Org\_model\_Mixin method)@\spxentry{\_\_getitem\_\_()}\spxextra{modelclass.Org\_model\_Mixin method}}

\begin{fulllineitems}
\phantomsection\label{\detokenize{index:modelclass.Org_model_Mixin.__getitem__}}
\pysigstartsignatures
\pysiglinewithargsret{\sphinxbfcode{\sphinxupquote{\_\_getitem\_\_}}}{\emph{\DUrole{n}{name}}}{}
\pysigstopsignatures
\sphinxAtStartPar
To execute the index operator {[}{]}

\sphinxAtStartPar
Uses the {\hyperref[\detokenize{index:modelvis.vis}]{\sphinxcrossref{\sphinxcode{\sphinxupquote{vis}}}}} operator

\end{fulllineitems}

\index{todynare() (modelclass.Org\_model\_Mixin method)@\spxentry{todynare()}\spxextra{modelclass.Org\_model\_Mixin method}}

\begin{fulllineitems}
\phantomsection\label{\detokenize{index:modelclass.Org_model_Mixin.todynare}}
\pysigstartsignatures
\pysiglinewithargsret{\sphinxbfcode{\sphinxupquote{todynare}}}{\emph{\DUrole{n}{paravars}\DUrole{o}{=}\DUrole{default_value}{{[}{]}}}, \emph{\DUrole{n}{paravalues}\DUrole{o}{=}\DUrole{default_value}{{[}{]}}}}{}
\pysigstopsignatures
\sphinxAtStartPar
This is a function which converts a Modelflow  model instance to Dynare .mod format

\end{fulllineitems}


\end{fulllineitems}

\index{Model\_help\_Mixin (class in modelclass)@\spxentry{Model\_help\_Mixin}\spxextra{class in modelclass}}

\begin{fulllineitems}
\phantomsection\label{\detokenize{index:modelclass.Model_help_Mixin}}
\pysigstartsignatures
\pysigline{\sphinxbfcode{\sphinxupquote{class\DUrole{w}{  }}}\sphinxcode{\sphinxupquote{modelclass.}}\sphinxbfcode{\sphinxupquote{Model\_help\_Mixin}}}
\pysigstopsignatures
\sphinxAtStartPar
Helpers to model
\index{timer() (modelclass.Model\_help\_Mixin static method)@\spxentry{timer()}\spxextra{modelclass.Model\_help\_Mixin static method}}

\begin{fulllineitems}
\phantomsection\label{\detokenize{index:modelclass.Model_help_Mixin.timer}}
\pysigstartsignatures
\pysiglinewithargsret{\sphinxbfcode{\sphinxupquote{static\DUrole{w}{  }}}\sphinxbfcode{\sphinxupquote{timer}}}{\emph{\DUrole{n}{input}\DUrole{o}{=}\DUrole{default_value}{\textquotesingle{}test\textquotesingle{}}}, \emph{\DUrole{n}{show}\DUrole{o}{=}\DUrole{default_value}{True}}, \emph{\DUrole{n}{short}\DUrole{o}{=}\DUrole{default_value}{True}}}{}
\pysigstopsignatures
\sphinxAtStartPar
A timer context manager, implemented using a
generator function. This one will report time even if an exception occurs
the time in seconds can be retrieved by \textless{}retunr value\textgreater{}.seconds “””
\begin{quote}\begin{description}
\item[{Parameters}] \leavevmode\begin{itemize}
\item {} 
\sphinxAtStartPar
\sphinxstyleliteralstrong{\sphinxupquote{input}} (\sphinxstyleliteralemphasis{\sphinxupquote{string}}\sphinxstyleliteralemphasis{\sphinxupquote{, }}\sphinxstyleliteralemphasis{\sphinxupquote{optional}}) \textendash{} a name. The default is ‘test’.

\item {} 
\sphinxAtStartPar
\sphinxstyleliteralstrong{\sphinxupquote{show}} (\sphinxstyleliteralemphasis{\sphinxupquote{bool}}\sphinxstyleliteralemphasis{\sphinxupquote{, }}\sphinxstyleliteralemphasis{\sphinxupquote{optional}}) \textendash{} show the results. The default is True.

\item {} 
\sphinxAtStartPar
\sphinxstyleliteralstrong{\sphinxupquote{short}} (\sphinxstyleliteralemphasis{\sphinxupquote{bool}}\sphinxstyleliteralemphasis{\sphinxupquote{, }}\sphinxstyleliteralemphasis{\sphinxupquote{optional}}) \textendash{} . The default is False.

\end{itemize}

\item[{Return type}] \leavevmode
\sphinxAtStartPar
None.

\end{description}\end{quote}

\end{fulllineitems}

\index{update\_old() (modelclass.Model\_help\_Mixin static method)@\spxentry{update\_old()}\spxextra{modelclass.Model\_help\_Mixin static method}}

\begin{fulllineitems}
\phantomsection\label{\detokenize{index:modelclass.Model_help_Mixin.update_old}}
\pysigstartsignatures
\pysiglinewithargsret{\sphinxbfcode{\sphinxupquote{static\DUrole{w}{  }}}\sphinxbfcode{\sphinxupquote{update\_old}}}{\emph{\DUrole{n}{indf}}, \emph{\DUrole{n}{updates}}, \emph{\DUrole{n}{lprint}\DUrole{o}{=}\DUrole{default_value}{False}}, \emph{\DUrole{n}{scale}\DUrole{o}{=}\DUrole{default_value}{1.0}}, \emph{\DUrole{n}{create}\DUrole{o}{=}\DUrole{default_value}{True}}, \emph{\DUrole{n}{keep\_growth}\DUrole{o}{=}\DUrole{default_value}{False}}, \emph{\DUrole{n}{start}\DUrole{o}{=}\DUrole{default_value}{\textquotesingle{}\textquotesingle{}}}, \emph{\DUrole{n}{end}\DUrole{o}{=}\DUrole{default_value}{\textquotesingle{}\textquotesingle{}}}}{}
\pysigstopsignatures\begin{quote}

\sphinxAtStartPar
Updates a dataframe and returns a dataframe
\end{quote}
\begin{quote}\begin{description}
\item[{Parameters}] \leavevmode\begin{itemize}
\item {} 
\sphinxAtStartPar
\sphinxstyleliteralstrong{\sphinxupquote{indf}} (\sphinxstyleliteralemphasis{\sphinxupquote{DataFrame}}) \textendash{} input dataframe.

\item {} 
\sphinxAtStartPar
\sphinxstyleliteralstrong{\sphinxupquote{basis}} (\sphinxstyleliteralemphasis{\sphinxupquote{string}}) \textendash{} lines with variable updates look below.

\item {} 
\sphinxAtStartPar
\sphinxstyleliteralstrong{\sphinxupquote{lprint}} (\sphinxstyleliteralemphasis{\sphinxupquote{bool}}\sphinxstyleliteralemphasis{\sphinxupquote{, }}\sphinxstyleliteralemphasis{\sphinxupquote{optional}}) \textendash{} if True each update is printed  Defaults to False.

\item {} 
\sphinxAtStartPar
\sphinxstyleliteralstrong{\sphinxupquote{scale}} (\sphinxstyleliteralemphasis{\sphinxupquote{float}}\sphinxstyleliteralemphasis{\sphinxupquote{, }}\sphinxstyleliteralemphasis{\sphinxupquote{optional}}) \textendash{} A multiplier used on all update input . Defaults to 1.0.

\item {} 
\sphinxAtStartPar
\sphinxstyleliteralstrong{\sphinxupquote{create}} (\sphinxstyleliteralemphasis{\sphinxupquote{bool}}\sphinxstyleliteralemphasis{\sphinxupquote{, }}\sphinxstyleliteralemphasis{\sphinxupquote{optional}}) \textendash{} Creates a variables if not in the dataframe . Defaults to True.

\item {} 
\sphinxAtStartPar
\sphinxstyleliteralstrong{\sphinxupquote{keep\_growth}} (\sphinxstyleliteralemphasis{\sphinxupquote{bool}}\sphinxstyleliteralemphasis{\sphinxupquote{, }}\sphinxstyleliteralemphasis{\sphinxupquote{optional}}) \textendash{} Keep the growth rate after the update time frame. Defaults to False.

\item {} 
\sphinxAtStartPar
\sphinxstyleliteralstrong{\sphinxupquote{start}} (\sphinxstyleliteralemphasis{\sphinxupquote{string}}\sphinxstyleliteralemphasis{\sphinxupquote{, }}\sphinxstyleliteralemphasis{\sphinxupquote{optional}}) \textendash{} Global start

\item {} 
\sphinxAtStartPar
\sphinxstyleliteralstrong{\sphinxupquote{end}} \textendash{} Global end

\end{itemize}

\end{description}\end{quote}

\end{fulllineitems}

\index{update() (modelclass.Model\_help\_Mixin static method)@\spxentry{update()}\spxextra{modelclass.Model\_help\_Mixin static method}}

\begin{fulllineitems}
\phantomsection\label{\detokenize{index:modelclass.Model_help_Mixin.update}}
\pysigstartsignatures
\pysiglinewithargsret{\sphinxbfcode{\sphinxupquote{static\DUrole{w}{  }}}\sphinxbfcode{\sphinxupquote{update}}}{\emph{\DUrole{n}{indf}}, \emph{\DUrole{n}{updates}}, \emph{\DUrole{n}{lprint}\DUrole{o}{=}\DUrole{default_value}{False}}, \emph{\DUrole{n}{scale}\DUrole{o}{=}\DUrole{default_value}{1.0}}, \emph{\DUrole{n}{create}\DUrole{o}{=}\DUrole{default_value}{True}}, \emph{\DUrole{n}{keep\_growth}\DUrole{o}{=}\DUrole{default_value}{False}}}{}
\pysigstopsignatures\begin{quote}

\sphinxAtStartPar
Updates a dataframe and returns a dataframe
\end{quote}
\begin{quote}\begin{description}
\item[{Parameters}] \leavevmode\begin{itemize}
\item {} 
\sphinxAtStartPar
\sphinxstyleliteralstrong{\sphinxupquote{indf}} (\sphinxstyleliteralemphasis{\sphinxupquote{DataFrame}}) \textendash{} input dataframe.

\item {} 
\sphinxAtStartPar
\sphinxstyleliteralstrong{\sphinxupquote{basis}} (\sphinxstyleliteralemphasis{\sphinxupquote{string}}) \textendash{} lines with variable updates look below.

\item {} 
\sphinxAtStartPar
\sphinxstyleliteralstrong{\sphinxupquote{lprint}} (\sphinxstyleliteralemphasis{\sphinxupquote{bool}}\sphinxstyleliteralemphasis{\sphinxupquote{, }}\sphinxstyleliteralemphasis{\sphinxupquote{optional}}) \textendash{} if True each update is printed  Defaults to False.

\item {} 
\sphinxAtStartPar
\sphinxstyleliteralstrong{\sphinxupquote{scale}} (\sphinxstyleliteralemphasis{\sphinxupquote{float}}\sphinxstyleliteralemphasis{\sphinxupquote{, }}\sphinxstyleliteralemphasis{\sphinxupquote{optional}}) \textendash{} A multiplier used on all update input . Defaults to 1.0.

\item {} 
\sphinxAtStartPar
\sphinxstyleliteralstrong{\sphinxupquote{create}} (\sphinxstyleliteralemphasis{\sphinxupquote{bool}}\sphinxstyleliteralemphasis{\sphinxupquote{, }}\sphinxstyleliteralemphasis{\sphinxupquote{optional}}) \textendash{} Creates a variables if not in the dataframe . Defaults to True.

\item {} 
\sphinxAtStartPar
\sphinxstyleliteralstrong{\sphinxupquote{keep\_growth}} (\sphinxstyleliteralemphasis{\sphinxupquote{bool}}\sphinxstyleliteralemphasis{\sphinxupquote{, }}\sphinxstyleliteralemphasis{\sphinxupquote{optional}}) \textendash{} Keep the growth rate after the update time frame. Defaults to False.

\end{itemize}

\item[{Returns}] \leavevmode
\sphinxAtStartPar
the updated dataframe .

\item[{Return type}] \leavevmode
\sphinxAtStartPar
df (TYPE)

\end{description}\end{quote}

\sphinxAtStartPar
A line in updates looks like this:

\sphinxAtStartPar
´\textless{}\sphinxtitleref{{[}{[}start{]} end{]}}\textgreater{}\textasciigrave{} \textless{}var\textgreater{} \textless{}=|+|*|\%|=growth|+growth|=diff\textgreater{} \textless{}value\textgreater{}… {[}\textendash{}keep\_growth\_rate|\textendash{}no\_keep\_growth\_rate{]}

\end{fulllineitems}

\index{insertModelVar() (modelclass.Model\_help\_Mixin method)@\spxentry{insertModelVar()}\spxextra{modelclass.Model\_help\_Mixin method}}

\begin{fulllineitems}
\phantomsection\label{\detokenize{index:modelclass.Model_help_Mixin.insertModelVar}}
\pysigstartsignatures
\pysiglinewithargsret{\sphinxbfcode{\sphinxupquote{insertModelVar}}}{\emph{\DUrole{n}{dataframe}}, \emph{\DUrole{n}{addmodel}\DUrole{o}{=}\DUrole{default_value}{{[}{]}}}}{}
\pysigstopsignatures
\sphinxAtStartPar
Inserts all variables from this model, not already in the dataframe.
If model is specified, the dataframw will contain all variables from this and model models.

\sphinxAtStartPar
also located at the module level for backward compability

\end{fulllineitems}

\index{Model\_help\_Mixin.defsub (class in modelclass)@\spxentry{Model\_help\_Mixin.defsub}\spxextra{class in modelclass}}

\begin{fulllineitems}
\phantomsection\label{\detokenize{index:modelclass.Model_help_Mixin.defsub}}
\pysigstartsignatures
\pysigline{\sphinxbfcode{\sphinxupquote{class\DUrole{w}{  }}}\sphinxbfcode{\sphinxupquote{defsub}}}
\pysigstopsignatures
\sphinxAtStartPar
A subclass of dict.
if a \sphinxstyleemphasis{defsub} is indexed by a nonexisting keyword it just return the keyword

\end{fulllineitems}

\index{test\_model() (modelclass.Model\_help\_Mixin method)@\spxentry{test\_model()}\spxextra{modelclass.Model\_help\_Mixin method}}

\begin{fulllineitems}
\phantomsection\label{\detokenize{index:modelclass.Model_help_Mixin.test_model}}
\pysigstartsignatures
\pysiglinewithargsret{\sphinxbfcode{\sphinxupquote{test\_model}}}{\emph{\DUrole{n}{base\_input}}, \emph{\DUrole{n}{start}\DUrole{o}{=}\DUrole{default_value}{None}}, \emph{\DUrole{n}{end}\DUrole{o}{=}\DUrole{default_value}{None}}, \emph{\DUrole{n}{maxvar}\DUrole{o}{=}\DUrole{default_value}{1000000}}, \emph{\DUrole{n}{maxerr}\DUrole{o}{=}\DUrole{default_value}{100}}, \emph{\DUrole{n}{tol}\DUrole{o}{=}\DUrole{default_value}{0.0001}}, \emph{\DUrole{n}{showall}\DUrole{o}{=}\DUrole{default_value}{False}}, \emph{\DUrole{n}{dec}\DUrole{o}{=}\DUrole{default_value}{8}}, \emph{\DUrole{n}{width}\DUrole{o}{=}\DUrole{default_value}{30}}, \emph{\DUrole{n}{ref\_df}\DUrole{o}{=}\DUrole{default_value}{None}}}{}
\pysigstopsignatures
\sphinxAtStartPar
Compares a straight calculation with the input dataframe.

\sphinxAtStartPar
shows which variables dont have the same value

\sphinxAtStartPar
Very useful when implementing a model where the results are known
\begin{quote}\begin{description}
\item[{Parameters}] \leavevmode\begin{itemize}
\item {} 
\sphinxAtStartPar
\sphinxstyleliteralstrong{\sphinxupquote{df}} (\sphinxstyleliteralemphasis{\sphinxupquote{DataFrame}}) \textendash{} dataframe to run.

\item {} 
\sphinxAtStartPar
\sphinxstyleliteralstrong{\sphinxupquote{start}} (\sphinxstyleliteralemphasis{\sphinxupquote{index}}\sphinxstyleliteralemphasis{\sphinxupquote{, }}\sphinxstyleliteralemphasis{\sphinxupquote{optional}}) \textendash{} start period. Defaults to None.

\item {} 
\sphinxAtStartPar
\sphinxstyleliteralstrong{\sphinxupquote{end}} (\sphinxstyleliteralemphasis{\sphinxupquote{index}}\sphinxstyleliteralemphasis{\sphinxupquote{, }}\sphinxstyleliteralemphasis{\sphinxupquote{optional}}) \textendash{} end period. Defaults to None.

\item {} 
\sphinxAtStartPar
\sphinxstyleliteralstrong{\sphinxupquote{maxvar}} (\sphinxstyleliteralemphasis{\sphinxupquote{int}}\sphinxstyleliteralemphasis{\sphinxupquote{, }}\sphinxstyleliteralemphasis{\sphinxupquote{optional}}) \textendash{} how many variables are to be chekked. Defaults to 1\_000\_000.

\item {} 
\sphinxAtStartPar
\sphinxstyleliteralstrong{\sphinxupquote{maxerr}} (\sphinxstyleliteralemphasis{\sphinxupquote{int}}\sphinxstyleliteralemphasis{\sphinxupquote{, }}\sphinxstyleliteralemphasis{\sphinxupquote{optional}}) \textendash{} how many errors to check Defaults to 100.

\item {} 
\sphinxAtStartPar
\sphinxstyleliteralstrong{\sphinxupquote{tol}} (\sphinxstyleliteralemphasis{\sphinxupquote{float}}\sphinxstyleliteralemphasis{\sphinxupquote{, }}\sphinxstyleliteralemphasis{\sphinxupquote{optional}}) \textendash{} check for absolute value of difference. Defaults to 0.0001.

\item {} 
\sphinxAtStartPar
\sphinxstyleliteralstrong{\sphinxupquote{showall}} (\sphinxstyleliteralemphasis{\sphinxupquote{bolean}}\sphinxstyleliteralemphasis{\sphinxupquote{, }}\sphinxstyleliteralemphasis{\sphinxupquote{optional}}) \textendash{} show more . Defaults to False.

\item {} 
\sphinxAtStartPar
\sphinxstyleliteralstrong{\sphinxupquote{ref\_df}} (\sphinxstyleliteralemphasis{\sphinxupquote{DataFrame}}\sphinxstyleliteralemphasis{\sphinxupquote{, }}\sphinxstyleliteralemphasis{\sphinxupquote{optional}}) \textendash{} this dataframe is used for reference, used if add\_factors has been calculated

\end{itemize}

\item[{Returns}] \leavevmode
\sphinxAtStartPar
None.

\end{description}\end{quote}

\end{fulllineitems}


\end{fulllineitems}

\index{Dekomp\_Mixin (class in modelclass)@\spxentry{Dekomp\_Mixin}\spxextra{class in modelclass}}

\begin{fulllineitems}
\phantomsection\label{\detokenize{index:modelclass.Dekomp_Mixin}}
\pysigstartsignatures
\pysigline{\sphinxbfcode{\sphinxupquote{class\DUrole{w}{  }}}\sphinxcode{\sphinxupquote{modelclass.}}\sphinxbfcode{\sphinxupquote{Dekomp\_Mixin}}}
\pysigstopsignatures
\sphinxAtStartPar
This class defines methods and properties related to equation attribution analyses (dekomp)
\index{dekomp() (modelclass.Dekomp\_Mixin method)@\spxentry{dekomp()}\spxextra{modelclass.Dekomp\_Mixin method}}

\begin{fulllineitems}
\phantomsection\label{\detokenize{index:modelclass.Dekomp_Mixin.dekomp}}
\pysigstartsignatures
\pysiglinewithargsret{\sphinxbfcode{\sphinxupquote{dekomp}}}{\emph{\DUrole{n}{varnavn}}, \emph{\DUrole{n}{start}\DUrole{o}{=}\DUrole{default_value}{\textquotesingle{}\textquotesingle{}}}, \emph{\DUrole{n}{end}\DUrole{o}{=}\DUrole{default_value}{\textquotesingle{}\textquotesingle{}}}, \emph{\DUrole{n}{basedf}\DUrole{o}{=}\DUrole{default_value}{None}}, \emph{\DUrole{n}{altdf}\DUrole{o}{=}\DUrole{default_value}{None}}, \emph{\DUrole{n}{lprint}\DUrole{o}{=}\DUrole{default_value}{True}}, \emph{\DUrole{n}{time\_att}\DUrole{o}{=}\DUrole{default_value}{False}}}{}
\pysigstopsignatures
\sphinxAtStartPar
Print all variables that determines input variable (varnavn)
optional \textendash{} enter period and databank to get var values for chosen period

\end{fulllineitems}

\index{dekomp\_plot\_per() (modelclass.Dekomp\_Mixin method)@\spxentry{dekomp\_plot\_per()}\spxextra{modelclass.Dekomp\_Mixin method}}

\begin{fulllineitems}
\phantomsection\label{\detokenize{index:modelclass.Dekomp_Mixin.dekomp_plot_per}}
\pysigstartsignatures
\pysiglinewithargsret{\sphinxbfcode{\sphinxupquote{dekomp\_plot\_per}}}{\emph{\DUrole{n}{varnavn}}, \emph{\DUrole{n}{sort}\DUrole{o}{=}\DUrole{default_value}{False}}, \emph{\DUrole{n}{pct}\DUrole{o}{=}\DUrole{default_value}{True}}, \emph{\DUrole{n}{per}\DUrole{o}{=}\DUrole{default_value}{\textquotesingle{}\textquotesingle{}}}, \emph{\DUrole{n}{threshold}\DUrole{o}{=}\DUrole{default_value}{0.0}}, \emph{\DUrole{n}{rename}\DUrole{o}{=}\DUrole{default_value}{True}}, \emph{\DUrole{n}{time\_att}\DUrole{o}{=}\DUrole{default_value}{False}}, \emph{\DUrole{n}{ysize}\DUrole{o}{=}\DUrole{default_value}{7}}}{}
\pysigstopsignatures
\sphinxAtStartPar
Returns  a waterfall diagram with attribution for a variable in one time frame
\begin{quote}\begin{description}
\item[{Parameters}] \leavevmode\begin{itemize}
\item {} 
\sphinxAtStartPar
\sphinxstyleliteralstrong{\sphinxupquote{varnavn}} (\sphinxstyleliteralemphasis{\sphinxupquote{TYPE}}) \textendash{} variable name.

\item {} 
\sphinxAtStartPar
\sphinxstyleliteralstrong{\sphinxupquote{sort}} (\sphinxstyleliteralemphasis{\sphinxupquote{TYPE}}\sphinxstyleliteralemphasis{\sphinxupquote{, }}\sphinxstyleliteralemphasis{\sphinxupquote{optional}}) \textendash{} . The default is False.

\item {} 
\sphinxAtStartPar
\sphinxstyleliteralstrong{\sphinxupquote{pct}} (\sphinxstyleliteralemphasis{\sphinxupquote{TYPE}}\sphinxstyleliteralemphasis{\sphinxupquote{, }}\sphinxstyleliteralemphasis{\sphinxupquote{optional}}) \textendash{} display pct contribution . The default is True.

\item {} 
\sphinxAtStartPar
\sphinxstyleliteralstrong{\sphinxupquote{per}} (\sphinxstyleliteralemphasis{\sphinxupquote{TYPE}}\sphinxstyleliteralemphasis{\sphinxupquote{, }}\sphinxstyleliteralemphasis{\sphinxupquote{optional}}) \textendash{} DESCRIPTION. The default is ‘’.

\item {} 
\sphinxAtStartPar
\sphinxstyleliteralstrong{\sphinxupquote{threshold}} (\sphinxstyleliteralemphasis{\sphinxupquote{TYPE}}\sphinxstyleliteralemphasis{\sphinxupquote{, }}\sphinxstyleliteralemphasis{\sphinxupquote{optional}}) \textendash{} cutoff. The default is 0.0.

\item {} 
\sphinxAtStartPar
\sphinxstyleliteralstrong{\sphinxupquote{rename}} (\sphinxstyleliteralemphasis{\sphinxupquote{TYPE}}\sphinxstyleliteralemphasis{\sphinxupquote{, }}\sphinxstyleliteralemphasis{\sphinxupquote{optional}}) \textendash{} Use descriptions instead of variable names. The default is True.

\item {} 
\sphinxAtStartPar
\sphinxstyleliteralstrong{\sphinxupquote{time\_att}} (\sphinxstyleliteralemphasis{\sphinxupquote{TYPE}}\sphinxstyleliteralemphasis{\sphinxupquote{, }}\sphinxstyleliteralemphasis{\sphinxupquote{optional}}) \textendash{} Do time attribution . The default is False.

\end{itemize}

\item[{Return type}] \leavevmode
\sphinxAtStartPar
a matplotlib figure instance .

\end{description}\end{quote}

\end{fulllineitems}

\index{get\_att\_pct() (modelclass.Dekomp\_Mixin method)@\spxentry{get\_att\_pct()}\spxextra{modelclass.Dekomp\_Mixin method}}

\begin{fulllineitems}
\phantomsection\label{\detokenize{index:modelclass.Dekomp_Mixin.get_att_pct}}
\pysigstartsignatures
\pysiglinewithargsret{\sphinxbfcode{\sphinxupquote{get\_att\_pct}}}{\emph{\DUrole{n}{n}}, \emph{\DUrole{n}{filter}\DUrole{o}{=}\DUrole{default_value}{False}}, \emph{\DUrole{n}{lag}\DUrole{o}{=}\DUrole{default_value}{True}}, \emph{\DUrole{n}{start}\DUrole{o}{=}\DUrole{default_value}{\textquotesingle{}\textquotesingle{}}}, \emph{\DUrole{n}{end}\DUrole{o}{=}\DUrole{default_value}{\textquotesingle{}\textquotesingle{}}}, \emph{\DUrole{n}{time\_att}\DUrole{o}{=}\DUrole{default_value}{False}}}{}
\pysigstopsignatures
\sphinxAtStartPar
det attribution pct for a variable.
I little effort to change from multiindex to single node name

\end{fulllineitems}

\index{get\_att\_pct\_to\_from() (modelclass.Dekomp\_Mixin method)@\spxentry{get\_att\_pct\_to\_from()}\spxextra{modelclass.Dekomp\_Mixin method}}

\begin{fulllineitems}
\phantomsection\label{\detokenize{index:modelclass.Dekomp_Mixin.get_att_pct_to_from}}
\pysigstartsignatures
\pysiglinewithargsret{\sphinxbfcode{\sphinxupquote{get\_att\_pct\_to\_from}}}{\emph{\DUrole{n}{to\_var}}, \emph{\DUrole{n}{from\_var}}, \emph{\DUrole{n}{lag}\DUrole{o}{=}\DUrole{default_value}{False}}, \emph{\DUrole{n}{time\_att}\DUrole{o}{=}\DUrole{default_value}{False}}}{}
\pysigstopsignatures
\sphinxAtStartPar
Get the  attribution for a singel variable

\end{fulllineitems}

\index{get\_att\_level() (modelclass.Dekomp\_Mixin method)@\spxentry{get\_att\_level()}\spxextra{modelclass.Dekomp\_Mixin method}}

\begin{fulllineitems}
\phantomsection\label{\detokenize{index:modelclass.Dekomp_Mixin.get_att_level}}
\pysigstartsignatures
\pysiglinewithargsret{\sphinxbfcode{\sphinxupquote{get\_att\_level}}}{\emph{\DUrole{n}{n}}, \emph{\DUrole{n}{filter}\DUrole{o}{=}\DUrole{default_value}{False}}, \emph{\DUrole{n}{lag}\DUrole{o}{=}\DUrole{default_value}{True}}, \emph{\DUrole{n}{start}\DUrole{o}{=}\DUrole{default_value}{\textquotesingle{}\textquotesingle{}}}, \emph{\DUrole{n}{end}\DUrole{o}{=}\DUrole{default_value}{\textquotesingle{}\textquotesingle{}}}, \emph{\DUrole{n}{time\_att}\DUrole{o}{=}\DUrole{default_value}{False}}}{}
\pysigstopsignatures
\sphinxAtStartPar
det attribution pct for a variable.
I little effort to change from multiindex to single node name

\end{fulllineitems}

\index{dekomp\_plot() (modelclass.Dekomp\_Mixin method)@\spxentry{dekomp\_plot()}\spxextra{modelclass.Dekomp\_Mixin method}}

\begin{fulllineitems}
\phantomsection\label{\detokenize{index:modelclass.Dekomp_Mixin.dekomp_plot}}
\pysigstartsignatures
\pysiglinewithargsret{\sphinxbfcode{\sphinxupquote{dekomp\_plot}}}{\emph{\DUrole{n}{varnavn}}, \emph{\DUrole{n}{sort}\DUrole{o}{=}\DUrole{default_value}{True}}, \emph{\DUrole{n}{pct}\DUrole{o}{=}\DUrole{default_value}{True}}, \emph{\DUrole{n}{per}\DUrole{o}{=}\DUrole{default_value}{\textquotesingle{}\textquotesingle{}}}, \emph{\DUrole{n}{top}\DUrole{o}{=}\DUrole{default_value}{0.9}}, \emph{\DUrole{n}{threshold}\DUrole{o}{=}\DUrole{default_value}{0.0}}, \emph{\DUrole{n}{lag}\DUrole{o}{=}\DUrole{default_value}{True}}, \emph{\DUrole{n}{rename}\DUrole{o}{=}\DUrole{default_value}{True}}, \emph{\DUrole{n}{time\_att}\DUrole{o}{=}\DUrole{default_value}{False}}}{}
\pysigstopsignatures
\sphinxAtStartPar
Returns  a chart with attribution for a variable over the smpl
\begin{quote}\begin{description}
\item[{Parameters}] \leavevmode\begin{itemize}
\item {} 
\sphinxAtStartPar
\sphinxstyleliteralstrong{\sphinxupquote{varnavn}} (\sphinxstyleliteralemphasis{\sphinxupquote{TYPE}}) \textendash{} variable name.

\item {} 
\sphinxAtStartPar
\sphinxstyleliteralstrong{\sphinxupquote{sort}} (\sphinxstyleliteralemphasis{\sphinxupquote{TYPE}}\sphinxstyleliteralemphasis{\sphinxupquote{, }}\sphinxstyleliteralemphasis{\sphinxupquote{optional}}) \textendash{} . The default is False.

\item {} 
\sphinxAtStartPar
\sphinxstyleliteralstrong{\sphinxupquote{pct}} (\sphinxstyleliteralemphasis{\sphinxupquote{TYPE}}\sphinxstyleliteralemphasis{\sphinxupquote{, }}\sphinxstyleliteralemphasis{\sphinxupquote{optional}}) \textendash{} display pct contribution . The default is True.

\item {} 
\sphinxAtStartPar
\sphinxstyleliteralstrong{\sphinxupquote{per}} (\sphinxstyleliteralemphasis{\sphinxupquote{TYPE}}\sphinxstyleliteralemphasis{\sphinxupquote{, }}\sphinxstyleliteralemphasis{\sphinxupquote{optional}}) \textendash{} DESCRIPTION. The default is ‘’.

\item {} 
\sphinxAtStartPar
\sphinxstyleliteralstrong{\sphinxupquote{threshold}} (\sphinxstyleliteralemphasis{\sphinxupquote{TYPE}}\sphinxstyleliteralemphasis{\sphinxupquote{, }}\sphinxstyleliteralemphasis{\sphinxupquote{optional}}) \textendash{} cutoff. The default is 0.0.

\item {} 
\sphinxAtStartPar
\sphinxstyleliteralstrong{\sphinxupquote{rename}} (\sphinxstyleliteralemphasis{\sphinxupquote{TYPE}}\sphinxstyleliteralemphasis{\sphinxupquote{, }}\sphinxstyleliteralemphasis{\sphinxupquote{optional}}) \textendash{} Use descriptions instead of variable names. The default is True.

\item {} 
\sphinxAtStartPar
\sphinxstyleliteralstrong{\sphinxupquote{time\_att}} (\sphinxstyleliteralemphasis{\sphinxupquote{TYPE}}\sphinxstyleliteralemphasis{\sphinxupquote{, }}\sphinxstyleliteralemphasis{\sphinxupquote{optional}}) \textendash{} Do time attribution . The default is False.

\item {} 
\sphinxAtStartPar
\sphinxstyleliteralstrong{\sphinxupquote{lag}} (\sphinxstyleliteralemphasis{\sphinxupquote{TYPE}}\sphinxstyleliteralemphasis{\sphinxupquote{, }}\sphinxstyleliteralemphasis{\sphinxupquote{optional}}) \textendash{} separete by lags The default is True.

\item {} 
\sphinxAtStartPar
\sphinxstyleliteralstrong{\sphinxupquote{top}} (\sphinxstyleliteralemphasis{\sphinxupquote{TYPE}}\sphinxstyleliteralemphasis{\sphinxupquote{, }}\sphinxstyleliteralemphasis{\sphinxupquote{optional}}) \textendash{} where to place the title

\end{itemize}

\item[{Return type}] \leavevmode
\sphinxAtStartPar
a matplotlib figure instance .

\end{description}\end{quote}

\end{fulllineitems}

\index{get\_dekom\_gui() (modelclass.Dekomp\_Mixin method)@\spxentry{get\_dekom\_gui()}\spxextra{modelclass.Dekomp\_Mixin method}}

\begin{fulllineitems}
\phantomsection\label{\detokenize{index:modelclass.Dekomp_Mixin.get_dekom_gui}}
\pysigstartsignatures
\pysiglinewithargsret{\sphinxbfcode{\sphinxupquote{get\_dekom\_gui}}}{\emph{\DUrole{n}{var}\DUrole{o}{=}\DUrole{default_value}{\textquotesingle{}\textquotesingle{}}}}{}
\pysigstopsignatures
\sphinxAtStartPar
Interactive wrapper around {\hyperref[\detokenize{index:modelclass.Dekomp_Mixin.dekomp_plot}]{\sphinxcrossref{\sphinxcode{\sphinxupquote{dekomp\_plot}}}}} and {\hyperref[\detokenize{index:modelclass.Dekomp_Mixin.dekomp_plot_per}]{\sphinxcrossref{\sphinxcode{\sphinxupquote{dekomp\_plot\_per}}}}}
\begin{quote}\begin{description}
\item[{Parameters}] \leavevmode
\sphinxAtStartPar
\sphinxstyleliteralstrong{\sphinxupquote{var}} (\sphinxstyleliteralemphasis{\sphinxupquote{TYPE}}\sphinxstyleliteralemphasis{\sphinxupquote{, }}\sphinxstyleliteralemphasis{\sphinxupquote{optional}}) \textendash{} start variable . Defaults to ‘’.

\item[{Returns}] \leavevmode
\sphinxAtStartPar
dict of matplotlib figs .

\item[{Return type}] \leavevmode
\sphinxAtStartPar
show (TYPE)

\end{description}\end{quote}

\end{fulllineitems}

\index{totexplain() (modelclass.Dekomp\_Mixin method)@\spxentry{totexplain()}\spxextra{modelclass.Dekomp\_Mixin method}}

\begin{fulllineitems}
\phantomsection\label{\detokenize{index:modelclass.Dekomp_Mixin.totexplain}}
\pysigstartsignatures
\pysiglinewithargsret{\sphinxbfcode{\sphinxupquote{totexplain}}}{\emph{\DUrole{n}{pat}\DUrole{o}{=}\DUrole{default_value}{\textquotesingle{}*\textquotesingle{}}}, \emph{\DUrole{n}{vtype}\DUrole{o}{=}\DUrole{default_value}{\textquotesingle{}all\textquotesingle{}}}, \emph{\DUrole{n}{stacked}\DUrole{o}{=}\DUrole{default_value}{True}}, \emph{\DUrole{n}{kind}\DUrole{o}{=}\DUrole{default_value}{\textquotesingle{}bar\textquotesingle{}}}, \emph{\DUrole{n}{per}\DUrole{o}{=}\DUrole{default_value}{\textquotesingle{}\textquotesingle{}}}, \emph{\DUrole{n}{top}\DUrole{o}{=}\DUrole{default_value}{0.9}}, \emph{\DUrole{n}{title}\DUrole{o}{=}\DUrole{default_value}{\textquotesingle{}\textquotesingle{}}}, \emph{\DUrole{n}{use}\DUrole{o}{=}\DUrole{default_value}{\textquotesingle{}level\textquotesingle{}}}, \emph{\DUrole{n}{threshold}\DUrole{o}{=}\DUrole{default_value}{0.0}}}{}
\pysigstopsignatures
\sphinxAtStartPar
makes a total explanation for the variables defined by pat
\begin{quote}\begin{description}
\item[{Parameters}] \leavevmode\begin{itemize}
\item {} 
\sphinxAtStartPar
\sphinxstyleliteralstrong{\sphinxupquote{pat}} (\sphinxstyleliteralemphasis{\sphinxupquote{TYPE}}\sphinxstyleliteralemphasis{\sphinxupquote{, }}\sphinxstyleliteralemphasis{\sphinxupquote{optional}}) \textendash{} DESCRIPTION. Defaults to ‘*’.

\item {} 
\sphinxAtStartPar
\sphinxstyleliteralstrong{\sphinxupquote{vtype}} (\sphinxstyleliteralemphasis{\sphinxupquote{TYPE}}\sphinxstyleliteralemphasis{\sphinxupquote{, }}\sphinxstyleliteralemphasis{\sphinxupquote{optional}}) \textendash{} DESCRIPTION. Defaults to ‘all’.

\item {} 
\sphinxAtStartPar
\sphinxstyleliteralstrong{\sphinxupquote{stacked}} (\sphinxstyleliteralemphasis{\sphinxupquote{TYPE}}\sphinxstyleliteralemphasis{\sphinxupquote{, }}\sphinxstyleliteralemphasis{\sphinxupquote{optional}}) \textendash{} DESCRIPTION. Defaults to True.

\item {} 
\sphinxAtStartPar
\sphinxstyleliteralstrong{\sphinxupquote{kind}} (\sphinxstyleliteralemphasis{\sphinxupquote{TYPE}}\sphinxstyleliteralemphasis{\sphinxupquote{, }}\sphinxstyleliteralemphasis{\sphinxupquote{optional}}) \textendash{} DESCRIPTION. Defaults to ‘bar’.

\item {} 
\sphinxAtStartPar
\sphinxstyleliteralstrong{\sphinxupquote{per}} (\sphinxstyleliteralemphasis{\sphinxupquote{TYPE}}\sphinxstyleliteralemphasis{\sphinxupquote{, }}\sphinxstyleliteralemphasis{\sphinxupquote{optional}}) \textendash{} DESCRIPTION. Defaults to ‘’.

\item {} 
\sphinxAtStartPar
\sphinxstyleliteralstrong{\sphinxupquote{top}} (\sphinxstyleliteralemphasis{\sphinxupquote{TYPE}}\sphinxstyleliteralemphasis{\sphinxupquote{, }}\sphinxstyleliteralemphasis{\sphinxupquote{optional}}) \textendash{} DESCRIPTION. Defaults to 0.9.

\item {} 
\sphinxAtStartPar
\sphinxstyleliteralstrong{\sphinxupquote{title}} (\sphinxstyleliteralemphasis{\sphinxupquote{TYPE}}\sphinxstyleliteralemphasis{\sphinxupquote{, }}\sphinxstyleliteralemphasis{\sphinxupquote{optional}}) \textendash{} DESCRIPTION. Defaults to ‘’.

\item {} 
\sphinxAtStartPar
\sphinxstyleliteralstrong{\sphinxupquote{use}} (\sphinxstyleliteralemphasis{\sphinxupquote{TYPE}}\sphinxstyleliteralemphasis{\sphinxupquote{, }}\sphinxstyleliteralemphasis{\sphinxupquote{optional}}) \textendash{} DESCRIPTION. Defaults to ‘level’.

\item {} 
\sphinxAtStartPar
\sphinxstyleliteralstrong{\sphinxupquote{threshold}} (\sphinxstyleliteralemphasis{\sphinxupquote{TYPE}}\sphinxstyleliteralemphasis{\sphinxupquote{, }}\sphinxstyleliteralemphasis{\sphinxupquote{optional}}) \textendash{} DESCRIPTION. Defaults to 0.0.

\end{itemize}

\item[{Returns}] \leavevmode
\sphinxAtStartPar
DESCRIPTION.

\item[{Return type}] \leavevmode
\sphinxAtStartPar
fig (TYPE)

\end{description}\end{quote}

\end{fulllineitems}

\index{get\_att\_gui() (modelclass.Dekomp\_Mixin method)@\spxentry{get\_att\_gui()}\spxextra{modelclass.Dekomp\_Mixin method}}

\begin{fulllineitems}
\phantomsection\label{\detokenize{index:modelclass.Dekomp_Mixin.get_att_gui}}
\pysigstartsignatures
\pysiglinewithargsret{\sphinxbfcode{\sphinxupquote{get\_att\_gui}}}{\emph{\DUrole{n}{var}\DUrole{o}{=}\DUrole{default_value}{\textquotesingle{}FY\textquotesingle{}}}, \emph{\DUrole{n}{spat}\DUrole{o}{=}\DUrole{default_value}{\textquotesingle{}*\textquotesingle{}}}, \emph{\DUrole{n}{desdic}\DUrole{o}{=}\DUrole{default_value}{\{\}}}, \emph{\DUrole{n}{use}\DUrole{o}{=}\DUrole{default_value}{\textquotesingle{}level\textquotesingle{}}}, \emph{\DUrole{n}{ysize}\DUrole{o}{=}\DUrole{default_value}{7}}}{}
\pysigstopsignatures
\sphinxAtStartPar
Creates a jupyter ipywidget to display model level
attributions

\end{fulllineitems}


\end{fulllineitems}

\index{Description\_Mixin (class in modelclass)@\spxentry{Description\_Mixin}\spxextra{class in modelclass}}

\begin{fulllineitems}
\phantomsection\label{\detokenize{index:modelclass.Description_Mixin}}
\pysigstartsignatures
\pysigline{\sphinxbfcode{\sphinxupquote{class\DUrole{w}{  }}}\sphinxcode{\sphinxupquote{modelclass.}}\sphinxbfcode{\sphinxupquote{Description\_Mixin}}}
\pysigstopsignatures
\sphinxAtStartPar
This Class defines description related methods and properties
\index{html\_replace() (modelclass.Description\_Mixin static method)@\spxentry{html\_replace()}\spxextra{modelclass.Description\_Mixin static method}}

\begin{fulllineitems}
\phantomsection\label{\detokenize{index:modelclass.Description_Mixin.html_replace}}
\pysigstartsignatures
\pysiglinewithargsret{\sphinxbfcode{\sphinxupquote{static\DUrole{w}{  }}}\sphinxbfcode{\sphinxupquote{html\_replace}}}{\emph{\DUrole{n}{ind}}}{}
\pysigstopsignatures
\sphinxAtStartPar
Replace special characters in html

\end{fulllineitems}

\index{var\_des() (modelclass.Description\_Mixin method)@\spxentry{var\_des()}\spxextra{modelclass.Description\_Mixin method}}

\begin{fulllineitems}
\phantomsection\label{\detokenize{index:modelclass.Description_Mixin.var_des}}
\pysigstartsignatures
\pysiglinewithargsret{\sphinxbfcode{\sphinxupquote{var\_des}}}{\emph{\DUrole{n}{var}}}{}
\pysigstopsignatures
\sphinxAtStartPar
Returns blank if no description

\end{fulllineitems}

\index{get\_eq\_des() (modelclass.Description\_Mixin method)@\spxentry{get\_eq\_des()}\spxextra{modelclass.Description\_Mixin method}}

\begin{fulllineitems}
\phantomsection\label{\detokenize{index:modelclass.Description_Mixin.get_eq_des}}
\pysigstartsignatures
\pysiglinewithargsret{\sphinxbfcode{\sphinxupquote{get\_eq\_des}}}{\emph{\DUrole{n}{var}}, \emph{\DUrole{n}{show\_all}\DUrole{o}{=}\DUrole{default_value}{False}}}{}
\pysigstopsignatures
\sphinxAtStartPar
Returns a string of descriptions for all variables in an equation:

\end{fulllineitems}

\index{read\_wb\_xml\_var\_des() (modelclass.Description\_Mixin static method)@\spxentry{read\_wb\_xml\_var\_des()}\spxextra{modelclass.Description\_Mixin static method}}

\begin{fulllineitems}
\phantomsection\label{\detokenize{index:modelclass.Description_Mixin.read_wb_xml_var_des}}
\pysigstartsignatures
\pysiglinewithargsret{\sphinxbfcode{\sphinxupquote{static\DUrole{w}{  }}}\sphinxbfcode{\sphinxupquote{read\_wb\_xml\_var\_des}}}{\emph{\DUrole{n}{filename}}}{}
\pysigstopsignatures
\sphinxAtStartPar
Read a xml file with variable description world bank style

\end{fulllineitems}

\index{languages\_wb\_xml\_var\_des() (modelclass.Description\_Mixin static method)@\spxentry{languages\_wb\_xml\_var\_des()}\spxextra{modelclass.Description\_Mixin static method}}

\begin{fulllineitems}
\phantomsection\label{\detokenize{index:modelclass.Description_Mixin.languages_wb_xml_var_des}}
\pysigstartsignatures
\pysiglinewithargsret{\sphinxbfcode{\sphinxupquote{static\DUrole{w}{  }}}\sphinxbfcode{\sphinxupquote{languages\_wb\_xml\_var\_des}}}{\emph{\DUrole{n}{filename}}}{}
\pysigstopsignatures
\sphinxAtStartPar
Find languages in a xml file with variable description world bank style

\end{fulllineitems}

\index{set\_wb\_xml\_var\_description() (modelclass.Description\_Mixin method)@\spxentry{set\_wb\_xml\_var\_description()}\spxextra{modelclass.Description\_Mixin method}}

\begin{fulllineitems}
\phantomsection\label{\detokenize{index:modelclass.Description_Mixin.set_wb_xml_var_description}}
\pysigstartsignatures
\pysiglinewithargsret{\sphinxbfcode{\sphinxupquote{set\_wb\_xml\_var\_description}}}{\emph{\DUrole{n}{filename}}, \emph{\DUrole{n}{language}\DUrole{o}{=}\DUrole{default_value}{\textquotesingle{}English\textquotesingle{}}}}{}
\pysigstopsignatures
\sphinxAtStartPar
set variable descriptions from a xml file with variable description world bank style

\end{fulllineitems}

\index{enrich\_var\_description() (modelclass.Description\_Mixin method)@\spxentry{enrich\_var\_description()}\spxextra{modelclass.Description\_Mixin method}}

\begin{fulllineitems}
\phantomsection\label{\detokenize{index:modelclass.Description_Mixin.enrich_var_description}}
\pysigstartsignatures
\pysiglinewithargsret{\sphinxbfcode{\sphinxupquote{enrich\_var\_description}}}{\emph{\DUrole{n}{var\_description}}}{}
\pysigstopsignatures
\sphinxAtStartPar
Takes a dict of variable descriptions and enhance it for the standard suffixes for generated variables

\end{fulllineitems}


\end{fulllineitems}

\index{Modify\_Mixin (class in modelclass)@\spxentry{Modify\_Mixin}\spxextra{class in modelclass}}

\begin{fulllineitems}
\phantomsection\label{\detokenize{index:modelclass.Modify_Mixin}}
\pysigstartsignatures
\pysigline{\sphinxbfcode{\sphinxupquote{class\DUrole{w}{  }}}\sphinxcode{\sphinxupquote{modelclass.}}\sphinxbfcode{\sphinxupquote{Modify\_Mixin}}}
\pysigstopsignatures
\sphinxAtStartPar
Class to modify a model with new equations, (later alse delete, and new normalization)
\index{eqflip() (modelclass.Modify\_Mixin method)@\spxentry{eqflip()}\spxextra{modelclass.Modify\_Mixin method}}

\begin{fulllineitems}
\phantomsection\label{\detokenize{index:modelclass.Modify_Mixin.eqflip}}
\pysigstartsignatures
\pysiglinewithargsret{\sphinxbfcode{\sphinxupquote{eqflip}}}{\emph{\DUrole{n}{flip}\DUrole{o}{=}\DUrole{default_value}{None}}, \emph{\DUrole{n}{calc\_add}\DUrole{o}{=}\DUrole{default_value}{True}}, \emph{\DUrole{n}{newname}\DUrole{o}{=}\DUrole{default_value}{\textquotesingle{}\textquotesingle{}}}, \emph{\DUrole{n}{sep}\DUrole{o}{=}\DUrole{default_value}{\textquotesingle{}\textbackslash{}n\textquotesingle{}}}}{}
\pysigstopsignatures\begin{quote}\begin{description}
\item[{Parameters}] \leavevmode\begin{itemize}
\item {} 
\sphinxAtStartPar
\sphinxstyleliteralstrong{\sphinxupquote{newnormalisation}} (\sphinxstyleliteralemphasis{\sphinxupquote{TYPE}}\sphinxstyleliteralemphasis{\sphinxupquote{, }}\sphinxstyleliteralemphasis{\sphinxupquote{optional}}) \textendash{} Not implementet yet . The default is None.

\item {} 
\sphinxAtStartPar
\sphinxstyleliteralstrong{\sphinxupquote{newfunks}} (\sphinxstyleliteralemphasis{\sphinxupquote{TYPE}}\sphinxstyleliteralemphasis{\sphinxupquote{, }}\sphinxstyleliteralemphasis{\sphinxupquote{optional}}) \textendash{} Additional userspecified functions. The default is {[}{]}.

\item {} 
\sphinxAtStartPar
\sphinxstyleliteralstrong{\sphinxupquote{calc\_add}} (\sphinxstyleliteralemphasis{\sphinxupquote{bool}}\sphinxstyleliteralemphasis{\sphinxupquote{, }}\sphinxstyleliteralemphasis{\sphinxupquote{optional}}) \textendash{} Additional userspecified functions. The default is {[}{]}.

\end{itemize}

\item[{Returns}] \leavevmode
\sphinxAtStartPar
\begin{itemize}
\item {} 
\sphinxAtStartPar
\sphinxstylestrong{newmodel} (\sphinxstyleemphasis{TYPE}) \textendash{} The new  model with the new and deleted equations .

\item {} 
\sphinxAtStartPar
\sphinxstylestrong{newdf} (\sphinxstyleemphasis{TYPE}) \textendash{} a dataframe with calculated add factors. Origin is the original models lastdf.

\end{itemize}


\end{description}\end{quote}

\end{fulllineitems}

\index{eqdelete() (modelclass.Modify\_Mixin method)@\spxentry{eqdelete()}\spxextra{modelclass.Modify\_Mixin method}}

\begin{fulllineitems}
\phantomsection\label{\detokenize{index:modelclass.Modify_Mixin.eqdelete}}
\pysigstartsignatures
\pysiglinewithargsret{\sphinxbfcode{\sphinxupquote{eqdelete}}}{\emph{\DUrole{n}{deleteeq}\DUrole{o}{=}\DUrole{default_value}{None}}, \emph{\DUrole{n}{newname}\DUrole{o}{=}\DUrole{default_value}{\textquotesingle{}\textquotesingle{}}}}{}
\pysigstopsignatures\begin{quote}\begin{description}
\item[{Parameters}] \leavevmode
\sphinxAtStartPar
\sphinxstyleliteralstrong{\sphinxupquote{deleteeq}} (\sphinxstyleliteralemphasis{\sphinxupquote{TYPE}}\sphinxstyleliteralemphasis{\sphinxupquote{, }}\sphinxstyleliteralemphasis{\sphinxupquote{optional}}) \textendash{} Variables where equations are to be deleted. The default is None.

\item[{Returns}] \leavevmode
\sphinxAtStartPar
\begin{itemize}
\item {} 
\sphinxAtStartPar
\sphinxstylestrong{newmodel} (\sphinxstyleemphasis{TYPE}) \textendash{} The new  model with the new and deleted equations .

\item {} 
\sphinxAtStartPar
\sphinxstylestrong{newdf} (\sphinxstyleemphasis{TYPE}) \textendash{} a dataframe with calculated add factors. Origin is the original models lastdf.

\end{itemize}


\end{description}\end{quote}

\end{fulllineitems}

\index{equpdate() (modelclass.Modify\_Mixin method)@\spxentry{equpdate()}\spxextra{modelclass.Modify\_Mixin method}}

\begin{fulllineitems}
\phantomsection\label{\detokenize{index:modelclass.Modify_Mixin.equpdate}}
\pysigstartsignatures
\pysiglinewithargsret{\sphinxbfcode{\sphinxupquote{equpdate}}}{\emph{\DUrole{n}{updateeq}\DUrole{o}{=}\DUrole{default_value}{None}}, \emph{\DUrole{n}{newfunks}\DUrole{o}{=}\DUrole{default_value}{{[}{]}}}, \emph{\DUrole{n}{add\_add\_factor}\DUrole{o}{=}\DUrole{default_value}{False}}, \emph{\DUrole{n}{calc\_add}\DUrole{o}{=}\DUrole{default_value}{True}}, \emph{\DUrole{n}{do\_preprocess}\DUrole{o}{=}\DUrole{default_value}{True}}, \emph{\DUrole{n}{newname}\DUrole{o}{=}\DUrole{default_value}{\textquotesingle{}\textquotesingle{}}}}{}
\pysigstopsignatures\begin{quote}\begin{description}
\item[{Parameters}] \leavevmode\begin{itemize}
\item {} 
\sphinxAtStartPar
\sphinxstyleliteralstrong{\sphinxupquote{updateeq}} (\sphinxstyleliteralemphasis{\sphinxupquote{TYPE}}) \textendash{} new equations seperated by newline .

\item {} 
\sphinxAtStartPar
\sphinxstyleliteralstrong{\sphinxupquote{newfunks}} (\sphinxstyleliteralemphasis{\sphinxupquote{TYPE}}\sphinxstyleliteralemphasis{\sphinxupquote{, }}\sphinxstyleliteralemphasis{\sphinxupquote{optional}}) \textendash{} Additional userspecified functions. The default is {[}{]}.

\item {} 
\sphinxAtStartPar
\sphinxstyleliteralstrong{\sphinxupquote{calc\_add}} (\sphinxstyleliteralemphasis{\sphinxupquote{bool}}\sphinxstyleliteralemphasis{\sphinxupquote{, }}\sphinxstyleliteralemphasis{\sphinxupquote{optional}}) \textendash{} Additional userspecified functions. The default is {[}{]}.

\end{itemize}

\item[{Returns}] \leavevmode
\sphinxAtStartPar
\begin{itemize}
\item {} 
\sphinxAtStartPar
\sphinxstylestrong{newmodel} (\sphinxstyleemphasis{TYPE}) \textendash{} The new  model with the new and deleted equations .

\item {} 
\sphinxAtStartPar
\sphinxstylestrong{newdf} (\sphinxstyleemphasis{TYPE}) \textendash{} a dataframe with calculated add factors. Origin is the original models lastdf.

\end{itemize}


\end{description}\end{quote}

\end{fulllineitems}

\index{equpdate\_old() (modelclass.Modify\_Mixin method)@\spxentry{equpdate\_old()}\spxextra{modelclass.Modify\_Mixin method}}

\begin{fulllineitems}
\phantomsection\label{\detokenize{index:modelclass.Modify_Mixin.equpdate_old}}
\pysigstartsignatures
\pysiglinewithargsret{\sphinxbfcode{\sphinxupquote{equpdate\_old}}}{\emph{\DUrole{n}{updateeq}\DUrole{o}{=}\DUrole{default_value}{None}}, \emph{\DUrole{n}{newfunks}\DUrole{o}{=}\DUrole{default_value}{{[}{]}}}, \emph{\DUrole{n}{add\_adjust}\DUrole{o}{=}\DUrole{default_value}{False}}, \emph{\DUrole{n}{calc\_add}\DUrole{o}{=}\DUrole{default_value}{True}}, \emph{\DUrole{n}{do\_preprocess}\DUrole{o}{=}\DUrole{default_value}{True}}, \emph{\DUrole{n}{newname}\DUrole{o}{=}\DUrole{default_value}{\textquotesingle{}\textquotesingle{}}}}{}
\pysigstopsignatures\begin{quote}\begin{description}
\item[{Parameters}] \leavevmode\begin{itemize}
\item {} 
\sphinxAtStartPar
\sphinxstyleliteralstrong{\sphinxupquote{updateeq}} (\sphinxstyleliteralemphasis{\sphinxupquote{TYPE}}) \textendash{} new equations seperated by newline .

\item {} 
\sphinxAtStartPar
\sphinxstyleliteralstrong{\sphinxupquote{newfunks}} (\sphinxstyleliteralemphasis{\sphinxupquote{TYPE}}\sphinxstyleliteralemphasis{\sphinxupquote{, }}\sphinxstyleliteralemphasis{\sphinxupquote{optional}}) \textendash{} Additional userspecified functions. The default is {[}{]}.

\item {} 
\sphinxAtStartPar
\sphinxstyleliteralstrong{\sphinxupquote{calc\_add}} (\sphinxstyleliteralemphasis{\sphinxupquote{bool}}\sphinxstyleliteralemphasis{\sphinxupquote{, }}\sphinxstyleliteralemphasis{\sphinxupquote{optional}}) \textendash{} Additional userspecified functions. The default is {[}{]}.

\end{itemize}

\item[{Returns}] \leavevmode
\sphinxAtStartPar
\begin{itemize}
\item {} 
\sphinxAtStartPar
\sphinxstylestrong{newmodel} (\sphinxstyleemphasis{TYPE}) \textendash{} The new  model with the new and deleted equations .

\item {} 
\sphinxAtStartPar
\sphinxstylestrong{newdf} (\sphinxstyleemphasis{TYPE}) \textendash{} a dataframe with calculated add factors. Origin is the original models lastdf.

\end{itemize}


\end{description}\end{quote}

\end{fulllineitems}


\end{fulllineitems}

\index{Graph\_Mixin (class in modelclass)@\spxentry{Graph\_Mixin}\spxextra{class in modelclass}}

\begin{fulllineitems}
\phantomsection\label{\detokenize{index:modelclass.Graph_Mixin}}
\pysigstartsignatures
\pysigline{\sphinxbfcode{\sphinxupquote{class\DUrole{w}{  }}}\sphinxcode{\sphinxupquote{modelclass.}}\sphinxbfcode{\sphinxupquote{Graph\_Mixin}}}
\pysigstopsignatures
\sphinxAtStartPar
This class defines graph related methods and properties
\index{create\_strong\_network() (modelclass.Graph\_Mixin static method)@\spxentry{create\_strong\_network()}\spxextra{modelclass.Graph\_Mixin static method}}

\begin{fulllineitems}
\phantomsection\label{\detokenize{index:modelclass.Graph_Mixin.create_strong_network}}
\pysigstartsignatures
\pysiglinewithargsret{\sphinxbfcode{\sphinxupquote{static\DUrole{w}{  }}}\sphinxbfcode{\sphinxupquote{create\_strong\_network}}}{\emph{\DUrole{n}{g}}, \emph{\DUrole{n}{name}\DUrole{o}{=}\DUrole{default_value}{\textquotesingle{}Network\textquotesingle{}}}, \emph{\DUrole{n}{typeout}\DUrole{o}{=}\DUrole{default_value}{False}}, \emph{\DUrole{n}{show}\DUrole{o}{=}\DUrole{default_value}{False}}}{}
\pysigstopsignatures
\sphinxAtStartPar
create a solveorder and   blockordering of af graph
uses networkx to find the core of the model

\end{fulllineitems}

\index{strongfrml (modelclass.Graph\_Mixin property)@\spxentry{strongfrml}\spxextra{modelclass.Graph\_Mixin property}}

\begin{fulllineitems}
\phantomsection\label{\detokenize{index:modelclass.Graph_Mixin.strongfrml}}
\pysigstartsignatures
\pysigline{\sphinxbfcode{\sphinxupquote{property\DUrole{w}{  }}}\sphinxbfcode{\sphinxupquote{strongfrml}}}
\pysigstopsignatures
\sphinxAtStartPar
To search simultaneity (circularity) in a model
this function returns the equations in each strong block

\end{fulllineitems}

\index{superblock() (modelclass.Graph\_Mixin method)@\spxentry{superblock()}\spxextra{modelclass.Graph\_Mixin method}}

\begin{fulllineitems}
\phantomsection\label{\detokenize{index:modelclass.Graph_Mixin.superblock}}
\pysigstartsignatures
\pysiglinewithargsret{\sphinxbfcode{\sphinxupquote{superblock}}}{}{}
\pysigstopsignatures
\sphinxAtStartPar
finds prolog, core and epilog variables

\end{fulllineitems}

\index{prevar (modelclass.Graph\_Mixin property)@\spxentry{prevar}\spxextra{modelclass.Graph\_Mixin property}}

\begin{fulllineitems}
\phantomsection\label{\detokenize{index:modelclass.Graph_Mixin.prevar}}
\pysigstartsignatures
\pysigline{\sphinxbfcode{\sphinxupquote{property\DUrole{w}{  }}}\sphinxbfcode{\sphinxupquote{prevar}}}
\pysigstopsignatures
\sphinxAtStartPar
returns a set with names of endogenopus variables which do not depend
on current endogenous variables

\end{fulllineitems}

\index{epivar (modelclass.Graph\_Mixin property)@\spxentry{epivar}\spxextra{modelclass.Graph\_Mixin property}}

\begin{fulllineitems}
\phantomsection\label{\detokenize{index:modelclass.Graph_Mixin.epivar}}
\pysigstartsignatures
\pysigline{\sphinxbfcode{\sphinxupquote{property\DUrole{w}{  }}}\sphinxbfcode{\sphinxupquote{epivar}}}
\pysigstopsignatures
\sphinxAtStartPar
returns a set with names of endogenopus variables which do not influence
current endogenous variables

\end{fulllineitems}

\index{preorder (modelclass.Graph\_Mixin property)@\spxentry{preorder}\spxextra{modelclass.Graph\_Mixin property}}

\begin{fulllineitems}
\phantomsection\label{\detokenize{index:modelclass.Graph_Mixin.preorder}}
\pysigstartsignatures
\pysigline{\sphinxbfcode{\sphinxupquote{property\DUrole{w}{  }}}\sphinxbfcode{\sphinxupquote{preorder}}}
\pysigstopsignatures
\sphinxAtStartPar
the endogenous variables which can be calculated in advance

\end{fulllineitems}

\index{epiorder (modelclass.Graph\_Mixin property)@\spxentry{epiorder}\spxextra{modelclass.Graph\_Mixin property}}

\begin{fulllineitems}
\phantomsection\label{\detokenize{index:modelclass.Graph_Mixin.epiorder}}
\pysigstartsignatures
\pysigline{\sphinxbfcode{\sphinxupquote{property\DUrole{w}{  }}}\sphinxbfcode{\sphinxupquote{epiorder}}}
\pysigstopsignatures
\sphinxAtStartPar
the endogenous variables which can be calculated in advance

\end{fulllineitems}

\index{coreorder (modelclass.Graph\_Mixin property)@\spxentry{coreorder}\spxextra{modelclass.Graph\_Mixin property}}

\begin{fulllineitems}
\phantomsection\label{\detokenize{index:modelclass.Graph_Mixin.coreorder}}
\pysigstartsignatures
\pysigline{\sphinxbfcode{\sphinxupquote{property\DUrole{w}{  }}}\sphinxbfcode{\sphinxupquote{coreorder}}}
\pysigstopsignatures
\sphinxAtStartPar
the solution order of the endogenous variables in the simultaneous core of the model

\end{fulllineitems}

\index{coreset (modelclass.Graph\_Mixin property)@\spxentry{coreset}\spxextra{modelclass.Graph\_Mixin property}}

\begin{fulllineitems}
\phantomsection\label{\detokenize{index:modelclass.Graph_Mixin.coreset}}
\pysigstartsignatures
\pysigline{\sphinxbfcode{\sphinxupquote{property\DUrole{w}{  }}}\sphinxbfcode{\sphinxupquote{coreset}}}
\pysigstopsignatures
\sphinxAtStartPar
The set of variables of the endogenous variables in the simultaneous core of the model

\end{fulllineitems}

\index{totgraph\_nolag (modelclass.Graph\_Mixin property)@\spxentry{totgraph\_nolag}\spxextra{modelclass.Graph\_Mixin property}}

\begin{fulllineitems}
\phantomsection\label{\detokenize{index:modelclass.Graph_Mixin.totgraph_nolag}}
\pysigstartsignatures
\pysigline{\sphinxbfcode{\sphinxupquote{property\DUrole{w}{  }}}\sphinxbfcode{\sphinxupquote{totgraph\_nolag}}}
\pysigstopsignatures
\sphinxAtStartPar
The graph of all variables, lagged variables condensed

\end{fulllineitems}

\index{totgraph (modelclass.Graph\_Mixin property)@\spxentry{totgraph}\spxextra{modelclass.Graph\_Mixin property}}

\begin{fulllineitems}
\phantomsection\label{\detokenize{index:modelclass.Graph_Mixin.totgraph}}
\pysigstartsignatures
\pysigline{\sphinxbfcode{\sphinxupquote{property\DUrole{w}{  }}}\sphinxbfcode{\sphinxupquote{totgraph}}}
\pysigstopsignatures
\sphinxAtStartPar
Returns the total graph of the model, including leads and lags

\end{fulllineitems}

\index{endograph\_nolag (modelclass.Graph\_Mixin property)@\spxentry{endograph\_nolag}\spxextra{modelclass.Graph\_Mixin property}}

\begin{fulllineitems}
\phantomsection\label{\detokenize{index:modelclass.Graph_Mixin.endograph_nolag}}
\pysigstartsignatures
\pysigline{\sphinxbfcode{\sphinxupquote{property\DUrole{w}{  }}}\sphinxbfcode{\sphinxupquote{endograph\_nolag}}}
\pysigstopsignatures
\sphinxAtStartPar
Dependencygraph for all periode endogeneous variable, shows total dependencies

\end{fulllineitems}

\index{endograph\_lag\_lead (modelclass.Graph\_Mixin property)@\spxentry{endograph\_lag\_lead}\spxextra{modelclass.Graph\_Mixin property}}

\begin{fulllineitems}
\phantomsection\label{\detokenize{index:modelclass.Graph_Mixin.endograph_lag_lead}}
\pysigstartsignatures
\pysigline{\sphinxbfcode{\sphinxupquote{property\DUrole{w}{  }}}\sphinxbfcode{\sphinxupquote{endograph\_lag\_lead}}}
\pysigstopsignatures
\sphinxAtStartPar
Returns the graph of all endogeneous variables including lags and leads

\end{fulllineitems}

\index{totgraph\_get() (modelclass.Graph\_Mixin method)@\spxentry{totgraph\_get()}\spxextra{modelclass.Graph\_Mixin method}}

\begin{fulllineitems}
\phantomsection\label{\detokenize{index:modelclass.Graph_Mixin.totgraph_get}}
\pysigstartsignatures
\pysiglinewithargsret{\sphinxbfcode{\sphinxupquote{totgraph\_get}}}{\emph{\DUrole{n}{onlyendo}\DUrole{o}{=}\DUrole{default_value}{False}}}{}
\pysigstopsignatures
\sphinxAtStartPar
The graph of all variables including and seperate lagged and leaded variable

\sphinxAtStartPar
onlyendo : only endogenous variables are part of the graph

\end{fulllineitems}

\index{graph\_remove() (modelclass.Graph\_Mixin method)@\spxentry{graph\_remove()}\spxextra{modelclass.Graph\_Mixin method}}

\begin{fulllineitems}
\phantomsection\label{\detokenize{index:modelclass.Graph_Mixin.graph_remove}}
\pysigstartsignatures
\pysiglinewithargsret{\sphinxbfcode{\sphinxupquote{graph\_remove}}}{\emph{\DUrole{n}{paralist}}}{}
\pysigstopsignatures
\sphinxAtStartPar
Removes a list of variables from the totgraph and totgraph\_nolag
mostly used to remove parmeters from the graph, makes it less crowded

\end{fulllineitems}

\index{graph\_restore() (modelclass.Graph\_Mixin method)@\spxentry{graph\_restore()}\spxextra{modelclass.Graph\_Mixin method}}

\begin{fulllineitems}
\phantomsection\label{\detokenize{index:modelclass.Graph_Mixin.graph_restore}}
\pysigstartsignatures
\pysiglinewithargsret{\sphinxbfcode{\sphinxupquote{graph\_restore}}}{}{}
\pysigstopsignatures
\sphinxAtStartPar
If nodes has been removed by the graph\_remove, calling this function will restore them

\end{fulllineitems}


\end{fulllineitems}

\index{Graph\_Draw\_Mixin (class in modelclass)@\spxentry{Graph\_Draw\_Mixin}\spxextra{class in modelclass}}

\begin{fulllineitems}
\phantomsection\label{\detokenize{index:modelclass.Graph_Draw_Mixin}}
\pysigstartsignatures
\pysigline{\sphinxbfcode{\sphinxupquote{class\DUrole{w}{  }}}\sphinxcode{\sphinxupquote{modelclass.}}\sphinxbfcode{\sphinxupquote{Graph\_Draw\_Mixin}}}
\pysigstopsignatures
\sphinxAtStartPar
This class defines methods and properties which draws and vizualize using different
graphs of the model
\index{treewalk() (modelclass.Graph\_Draw\_Mixin method)@\spxentry{treewalk()}\spxextra{modelclass.Graph\_Draw\_Mixin method}}

\begin{fulllineitems}
\phantomsection\label{\detokenize{index:modelclass.Graph_Draw_Mixin.treewalk}}
\pysigstartsignatures
\pysiglinewithargsret{\sphinxbfcode{\sphinxupquote{treewalk}}}{\emph{\DUrole{n}{g}}, \emph{\DUrole{n}{navn}}, \emph{\DUrole{n}{level}\DUrole{o}{=}\DUrole{default_value}{0}}, \emph{\DUrole{n}{parent}\DUrole{o}{=}\DUrole{default_value}{\textquotesingle{}Start\textquotesingle{}}}, \emph{\DUrole{n}{maxlevel}\DUrole{o}{=}\DUrole{default_value}{20}}, \emph{\DUrole{n}{lpre}\DUrole{o}{=}\DUrole{default_value}{True}}}{}
\pysigstopsignatures
\sphinxAtStartPar
Traverse the call tree from name, and returns a generator

\sphinxAtStartPar
to get a list just write: list(treewalk(…))
maxlevel determins the number of generations to back up

\sphinxAtStartPar
lpre=0 we walk the dependent
lpre=1 we walk the precednc nodes

\end{fulllineitems}

\index{drawendo() (modelclass.Graph\_Draw\_Mixin method)@\spxentry{drawendo()}\spxextra{modelclass.Graph\_Draw\_Mixin method}}

\begin{fulllineitems}
\phantomsection\label{\detokenize{index:modelclass.Graph_Draw_Mixin.drawendo}}
\pysigstartsignatures
\pysiglinewithargsret{\sphinxbfcode{\sphinxupquote{drawendo}}}{\emph{\DUrole{o}{**}\DUrole{n}{kwargs}}}{}
\pysigstopsignatures
\sphinxAtStartPar
draws a graph of of the whole model

\end{fulllineitems}

\index{drawendo\_lag\_lead() (modelclass.Graph\_Draw\_Mixin method)@\spxentry{drawendo\_lag\_lead()}\spxextra{modelclass.Graph\_Draw\_Mixin method}}

\begin{fulllineitems}
\phantomsection\label{\detokenize{index:modelclass.Graph_Draw_Mixin.drawendo_lag_lead}}
\pysigstartsignatures
\pysiglinewithargsret{\sphinxbfcode{\sphinxupquote{drawendo\_lag\_lead}}}{\emph{\DUrole{o}{**}\DUrole{n}{kwargs}}}{}
\pysigstopsignatures
\sphinxAtStartPar
draws a graph of of the whole model

\end{fulllineitems}

\index{drawmodel() (modelclass.Graph\_Draw\_Mixin method)@\spxentry{drawmodel()}\spxextra{modelclass.Graph\_Draw\_Mixin method}}

\begin{fulllineitems}
\phantomsection\label{\detokenize{index:modelclass.Graph_Draw_Mixin.drawmodel}}
\pysigstartsignatures
\pysiglinewithargsret{\sphinxbfcode{\sphinxupquote{drawmodel}}}{\emph{\DUrole{n}{lag}\DUrole{o}{=}\DUrole{default_value}{True}}, \emph{\DUrole{o}{**}\DUrole{n}{kwargs}}}{}
\pysigstopsignatures
\sphinxAtStartPar
draws a graph of of the whole model

\end{fulllineitems}

\index{plotadjacency() (modelclass.Graph\_Draw\_Mixin method)@\spxentry{plotadjacency()}\spxextra{modelclass.Graph\_Draw\_Mixin method}}

\begin{fulllineitems}
\phantomsection\label{\detokenize{index:modelclass.Graph_Draw_Mixin.plotadjacency}}
\pysigstartsignatures
\pysiglinewithargsret{\sphinxbfcode{\sphinxupquote{plotadjacency}}}{\emph{\DUrole{n}{size}\DUrole{o}{=}\DUrole{default_value}{(5, 5)}}, \emph{\DUrole{n}{title}\DUrole{o}{=}\DUrole{default_value}{\textquotesingle{}Structure\textquotesingle{}}}, \emph{\DUrole{n}{nolag}\DUrole{o}{=}\DUrole{default_value}{False}}}{}
\pysigstopsignatures
\sphinxAtStartPar
Draws an adjacendy matrix
\begin{quote}\begin{description}
\item[{Parameters}] \leavevmode\begin{itemize}
\item {} 
\sphinxAtStartPar
\sphinxstyleliteralstrong{\sphinxupquote{size}} (\sphinxstyleliteralemphasis{\sphinxupquote{TYPE}}\sphinxstyleliteralemphasis{\sphinxupquote{, }}\sphinxstyleliteralemphasis{\sphinxupquote{optional}}) \textendash{} DESCRIPTION. Defaults to (5, 5).

\item {} 
\sphinxAtStartPar
\sphinxstyleliteralstrong{\sphinxupquote{title}} (\sphinxstyleliteralemphasis{\sphinxupquote{TYPE}}\sphinxstyleliteralemphasis{\sphinxupquote{, }}\sphinxstyleliteralemphasis{\sphinxupquote{optional}}) \textendash{} DESCRIPTION. Defaults to ‘Structure’.

\item {} 
\sphinxAtStartPar
\sphinxstyleliteralstrong{\sphinxupquote{nolag}} (\sphinxstyleliteralemphasis{\sphinxupquote{TYPE}}\sphinxstyleliteralemphasis{\sphinxupquote{, }}\sphinxstyleliteralemphasis{\sphinxupquote{optional}}) \textendash{} DESCRIPTION. Defaults to False.

\end{itemize}

\item[{Returns}] \leavevmode
\sphinxAtStartPar
A adjacency matrix drawing.

\item[{Return type}] \leavevmode
\sphinxAtStartPar
fig (matplotlib figure)

\end{description}\end{quote}

\end{fulllineitems}

\index{draw() (modelclass.Graph\_Draw\_Mixin method)@\spxentry{draw()}\spxextra{modelclass.Graph\_Draw\_Mixin method}}

\begin{fulllineitems}
\phantomsection\label{\detokenize{index:modelclass.Graph_Draw_Mixin.draw}}
\pysigstartsignatures
\pysiglinewithargsret{\sphinxbfcode{\sphinxupquote{draw}}}{\emph{\DUrole{n}{navn}}, \emph{\DUrole{n}{down}\DUrole{o}{=}\DUrole{default_value}{1}}, \emph{\DUrole{n}{up}\DUrole{o}{=}\DUrole{default_value}{1}}, \emph{\DUrole{n}{lag}\DUrole{o}{=}\DUrole{default_value}{False}}, \emph{\DUrole{n}{endo}\DUrole{o}{=}\DUrole{default_value}{False}}, \emph{\DUrole{n}{filter}\DUrole{o}{=}\DUrole{default_value}{0}}, \emph{\DUrole{o}{**}\DUrole{n}{kwargs}}}{}
\pysigstopsignatures
\sphinxAtStartPar
draws a graph of dependensies of navn up to maxlevel
\begin{quote}\begin{description}
\item[{Lag}] \leavevmode
\sphinxAtStartPar
show the complete graph including lagged variables else only variables.

\item[{Endo}] \leavevmode
\sphinxAtStartPar
Show only the graph for current endogenous variables

\item[{Down}] \leavevmode
\sphinxAtStartPar
level downstream

\item[{Up}] \leavevmode
\sphinxAtStartPar
level upstream

\end{description}\end{quote}

\end{fulllineitems}

\index{trans() (modelclass.Graph\_Draw\_Mixin method)@\spxentry{trans()}\spxextra{modelclass.Graph\_Draw\_Mixin method}}

\begin{fulllineitems}
\phantomsection\label{\detokenize{index:modelclass.Graph_Draw_Mixin.trans}}
\pysigstartsignatures
\pysiglinewithargsret{\sphinxbfcode{\sphinxupquote{trans}}}{\emph{\DUrole{n}{ind}}, \emph{\DUrole{n}{root}}, \emph{\DUrole{n}{transdic}\DUrole{o}{=}\DUrole{default_value}{None}}, \emph{\DUrole{n}{debug}\DUrole{o}{=}\DUrole{default_value}{False}}}{}
\pysigstopsignatures
\sphinxAtStartPar
as there are many variable starting with SHOCK, the can renamed to save nodes

\end{fulllineitems}

\index{upwalk() (modelclass.Graph\_Draw\_Mixin method)@\spxentry{upwalk()}\spxextra{modelclass.Graph\_Draw\_Mixin method}}

\begin{fulllineitems}
\phantomsection\label{\detokenize{index:modelclass.Graph_Draw_Mixin.upwalk}}
\pysigstartsignatures
\pysiglinewithargsret{\sphinxbfcode{\sphinxupquote{upwalk}}}{\emph{\DUrole{n}{g}}, \emph{\DUrole{n}{navn}}, \emph{\DUrole{n}{level}\DUrole{o}{=}\DUrole{default_value}{0}}, \emph{\DUrole{n}{parent}\DUrole{o}{=}\DUrole{default_value}{\textquotesingle{}Start\textquotesingle{}}}, \emph{\DUrole{n}{maxlevel}\DUrole{o}{=}\DUrole{default_value}{20}}, \emph{\DUrole{n}{filter}\DUrole{o}{=}\DUrole{default_value}{0.0}}, \emph{\DUrole{n}{lpre}\DUrole{o}{=}\DUrole{default_value}{True}}}{}
\pysigstopsignatures
\sphinxAtStartPar
Traverse the call tree from name, and returns a generator

\sphinxAtStartPar
to get a list just write: list(upwalk(…))
maxlevel determins the number
of generations to back maxlevel

\end{fulllineitems}

\index{upwalk\_old() (modelclass.Graph\_Draw\_Mixin method)@\spxentry{upwalk\_old()}\spxextra{modelclass.Graph\_Draw\_Mixin method}}

\begin{fulllineitems}
\phantomsection\label{\detokenize{index:modelclass.Graph_Draw_Mixin.upwalk_old}}
\pysigstartsignatures
\pysiglinewithargsret{\sphinxbfcode{\sphinxupquote{upwalk\_old}}}{\emph{\DUrole{n}{g}}, \emph{\DUrole{n}{navn}}, \emph{\DUrole{n}{level}\DUrole{o}{=}\DUrole{default_value}{0}}, \emph{\DUrole{n}{parent}\DUrole{o}{=}\DUrole{default_value}{\textquotesingle{}Start\textquotesingle{}}}, \emph{\DUrole{n}{up}\DUrole{o}{=}\DUrole{default_value}{20}}, \emph{\DUrole{n}{select}\DUrole{o}{=}\DUrole{default_value}{0.0}}, \emph{\DUrole{n}{lpre}\DUrole{o}{=}\DUrole{default_value}{True}}}{}
\pysigstopsignatures
\sphinxAtStartPar
Traverse the call tree from name, and returns a generator

\sphinxAtStartPar
to get a list just write: list(upwalk(…))
up determins the number
of generations to back up

\end{fulllineitems}

\index{explain() (modelclass.Graph\_Draw\_Mixin method)@\spxentry{explain()}\spxextra{modelclass.Graph\_Draw\_Mixin method}}

\begin{fulllineitems}
\phantomsection\label{\detokenize{index:modelclass.Graph_Draw_Mixin.explain}}
\pysigstartsignatures
\pysiglinewithargsret{\sphinxbfcode{\sphinxupquote{explain}}}{\emph{\DUrole{n}{var}}, \emph{\DUrole{n}{up}\DUrole{o}{=}\DUrole{default_value}{1}}, \emph{\DUrole{n}{start}\DUrole{o}{=}\DUrole{default_value}{\textquotesingle{}\textquotesingle{}}}, \emph{\DUrole{n}{end}\DUrole{o}{=}\DUrole{default_value}{\textquotesingle{}\textquotesingle{}}}, \emph{\DUrole{n}{filter}\DUrole{o}{=}\DUrole{default_value}{0}}, \emph{\DUrole{n}{showatt}\DUrole{o}{=}\DUrole{default_value}{True}}, \emph{\DUrole{n}{lag}\DUrole{o}{=}\DUrole{default_value}{True}}, \emph{\DUrole{n}{debug}\DUrole{o}{=}\DUrole{default_value}{0}}, \emph{\DUrole{n}{noshow}\DUrole{o}{=}\DUrole{default_value}{False}}, \emph{\DUrole{n}{dot}\DUrole{o}{=}\DUrole{default_value}{False}}, \emph{\DUrole{o}{**}\DUrole{n}{kwargs}}}{}
\pysigstopsignatures
\sphinxAtStartPar
Walks a tree to explain the difference between basedf and lastdf

\sphinxAtStartPar
Parameters:
:var:  the variable we are looking at
:up:  how far up the tree will we climb
:select: Only show the nodes which contributes
:showatt: Show the explanation in pct
:lag:  If true, show all lags, else aggregate lags for each variable.
:HR: if true make horisontal graph
:title: Title
:saveas: Filename
:pdf: open the pdf file
:svg: display the svg file
:browser: if true open the svg file in browser
:noshow: Only return the resulting graph
:dot: Return the dot file only

\end{fulllineitems}

\index{todot() (modelclass.Graph\_Draw\_Mixin method)@\spxentry{todot()}\spxextra{modelclass.Graph\_Draw\_Mixin method}}

\begin{fulllineitems}
\phantomsection\label{\detokenize{index:modelclass.Graph_Draw_Mixin.todot}}
\pysigstartsignatures
\pysiglinewithargsret{\sphinxbfcode{\sphinxupquote{todot}}}{\emph{\DUrole{n}{g}}, \emph{\DUrole{n}{navn}\DUrole{o}{=}\DUrole{default_value}{\textquotesingle{}\textquotesingle{}}}, \emph{\DUrole{n}{browser}\DUrole{o}{=}\DUrole{default_value}{False}}, \emph{\DUrole{o}{**}\DUrole{n}{kwargs}}}{}
\pysigstopsignatures
\sphinxAtStartPar
makes a drawing of subtree originating from navn
all is the edges
attributex can be shown
\begin{quote}\begin{description}
\item[{Sink}] \leavevmode
\sphinxAtStartPar
variale to use as sink

\item[{Svg}] \leavevmode
\sphinxAtStartPar
Display the svg image

\end{description}\end{quote}

\end{fulllineitems}

\index{gdraw() (modelclass.Graph\_Draw\_Mixin method)@\spxentry{gdraw()}\spxextra{modelclass.Graph\_Draw\_Mixin method}}

\begin{fulllineitems}
\phantomsection\label{\detokenize{index:modelclass.Graph_Draw_Mixin.gdraw}}
\pysigstartsignatures
\pysiglinewithargsret{\sphinxbfcode{\sphinxupquote{gdraw}}}{\emph{\DUrole{n}{g}}, \emph{\DUrole{o}{**}\DUrole{n}{kwargs}}}{}
\pysigstopsignatures
\sphinxAtStartPar
draws a graph of of the whole model

\end{fulllineitems}

\index{maketip() (modelclass.Graph\_Draw\_Mixin method)@\spxentry{maketip()}\spxextra{modelclass.Graph\_Draw\_Mixin method}}

\begin{fulllineitems}
\phantomsection\label{\detokenize{index:modelclass.Graph_Draw_Mixin.maketip}}
\pysigstartsignatures
\pysiglinewithargsret{\sphinxbfcode{\sphinxupquote{maketip}}}{\emph{\DUrole{n}{v}}, \emph{\DUrole{n}{html}\DUrole{o}{=}\DUrole{default_value}{False}}}{}
\pysigstopsignatures
\sphinxAtStartPar
Return a tooltip for variable v.

\sphinxAtStartPar
For use when generating .dot files for Graphviz

\sphinxAtStartPar
If html==True it can be incorporated into html string

\end{fulllineitems}

\index{makedotnew() (modelclass.Graph\_Draw\_Mixin method)@\spxentry{makedotnew()}\spxextra{modelclass.Graph\_Draw\_Mixin method}}

\begin{fulllineitems}
\phantomsection\label{\detokenize{index:modelclass.Graph_Draw_Mixin.makedotnew}}
\pysigstartsignatures
\pysiglinewithargsret{\sphinxbfcode{\sphinxupquote{makedotnew}}}{\emph{\DUrole{n}{alledges}}, \emph{\DUrole{n}{navn}\DUrole{o}{=}\DUrole{default_value}{\textquotesingle{}\textquotesingle{}}}, \emph{\DUrole{o}{**}\DUrole{n}{kwargs}}}{}
\pysigstopsignatures
\sphinxAtStartPar
makes a drawing of all edges in list alledges
all is the edges

\sphinxAtStartPar
this can handle both attribution and plain
\begin{quote}\begin{description}
\item[{All}] \leavevmode
\sphinxAtStartPar
show values for .dfbase and .lastdf

\item[{Last}] \leavevmode
\sphinxAtStartPar
show the values for .lastdf

\item[{Growthshow}] \leavevmode
\sphinxAtStartPar
Show growthrates

\item[{Attshow}] \leavevmode
\sphinxAtStartPar
Show attributiuons

\item[{Filter}] \leavevmode
\sphinxAtStartPar
Prune tree branches where all(abs(attribution)\textless{}filter value)

\item[{Sink}] \leavevmode
\sphinxAtStartPar
variale to use as sink

\item[{Source}] \leavevmode
\sphinxAtStartPar
variale to use as ssource

\item[{Svg}] \leavevmode
\sphinxAtStartPar
Display the svg image in browser

\item[{Pdf}] \leavevmode
\sphinxAtStartPar
display the pdf result in acrobat reader

\item[{Saveas}] \leavevmode
\sphinxAtStartPar
Save the drawing as name

\item[{Size}] \leavevmode
\sphinxAtStartPar
figure size default (6,6)

\item[{Warnings}] \leavevmode
\sphinxAtStartPar
warnings displayed in command console, default =False

\item[{Invisible}] \leavevmode
\sphinxAtStartPar
set of invisible nodes

\item[{Labels}] \leavevmode
\sphinxAtStartPar
dict of labels for edges

\item[{Transdic}] \leavevmode
\sphinxAtStartPar
dict of translations for consolidation of nodes \{‘SHOCK{[}\_A\sphinxhyphen{}Z{]}*\_\_J’:’SHOCK\_\_J’,’DEV\_\_{[}\_A\sphinxhyphen{}Z{]}*’:’DEV’\}

\item[{Dec}] \leavevmode
\sphinxAtStartPar
decimal places in numbers

\item[{HR}] \leavevmode
\sphinxAtStartPar
horisontal orientation default = False

\item[{Des}] \leavevmode
\sphinxAtStartPar
inject variable descriptions

\item[{Fokus}] \leavevmode
\sphinxAtStartPar
Variable to get special colour

\item[{Fokus2}] \leavevmode
\sphinxAtStartPar
Variable for which values are shown

\item[{Fokus2all}] \leavevmode
\sphinxAtStartPar
Show values for all variables

\end{description}\end{quote}

\end{fulllineitems}

\index{todot2() (modelclass.Graph\_Draw\_Mixin method)@\spxentry{todot2()}\spxextra{modelclass.Graph\_Draw\_Mixin method}}

\begin{fulllineitems}
\phantomsection\label{\detokenize{index:modelclass.Graph_Draw_Mixin.todot2}}
\pysigstartsignatures
\pysiglinewithargsret{\sphinxbfcode{\sphinxupquote{todot2}}}{\emph{\DUrole{n}{alledges}}, \emph{\DUrole{n}{navn}\DUrole{o}{=}\DUrole{default_value}{\textquotesingle{}\textquotesingle{}}}, \emph{\DUrole{o}{**}\DUrole{n}{kwargs}}}{}
\pysigstopsignatures
\sphinxAtStartPar
makes a drawing of all edges in list alledges
all is the edges
\begin{quote}\begin{description}
\item[{All}] \leavevmode
\sphinxAtStartPar
show values for .dfbase and .dflaste

\item[{Last}] \leavevmode
\sphinxAtStartPar
show the values for .dflast

\item[{Sink}] \leavevmode
\sphinxAtStartPar
variale to use as sink

\item[{Source}] \leavevmode
\sphinxAtStartPar
variale to use as ssource

\item[{Svg}] \leavevmode
\sphinxAtStartPar
Display the svg image in browser

\item[{Pdf}] \leavevmode
\sphinxAtStartPar
display the pdf result in acrobat reader

\item[{Saveas}] \leavevmode
\sphinxAtStartPar
Save the drawing as name

\item[{Size}] \leavevmode
\sphinxAtStartPar
figure size default (6,6)

\item[{Warnings}] \leavevmode
\sphinxAtStartPar
warnings displayed in command console, default =False

\item[{Invisible}] \leavevmode
\sphinxAtStartPar
set of invisible nodes

\item[{Labels}] \leavevmode
\sphinxAtStartPar
dict of labels for edges

\item[{Transdic}] \leavevmode
\sphinxAtStartPar
dict of translations for consolidation of nodes \{‘SHOCK{[}\_A\sphinxhyphen{}Z{]}*\_\_J’:’SHOCK\_\_J’,’DEV\_\_{[}\_A\sphinxhyphen{}Z{]}*’:’DEV’\}

\item[{Dec}] \leavevmode
\sphinxAtStartPar
decimal places in numbers

\item[{HR}] \leavevmode
\sphinxAtStartPar
horisontal orientation default = False

\item[{Des}] \leavevmode
\sphinxAtStartPar
inject variable descriptions

\end{description}\end{quote}

\end{fulllineitems}

\index{display\_graph() (modelclass.Graph\_Draw\_Mixin static method)@\spxentry{display\_graph()}\spxextra{modelclass.Graph\_Draw\_Mixin static method}}

\begin{fulllineitems}
\phantomsection\label{\detokenize{index:modelclass.Graph_Draw_Mixin.display_graph}}
\pysigstartsignatures
\pysiglinewithargsret{\sphinxbfcode{\sphinxupquote{static\DUrole{w}{  }}}\sphinxbfcode{\sphinxupquote{display\_graph}}}{\emph{\DUrole{n}{out}}, \emph{\DUrole{n}{fname}}, \emph{\DUrole{o}{**}\DUrole{n}{kwargs}}}{}
\pysigstopsignatures
\sphinxAtStartPar
Generates a graphviz file from the commands in out.

\sphinxAtStartPar
The file is placed in cwd/graph

\sphinxAtStartPar
A png and a svg file is generated, and the svg file is displayed if possible.

\sphinxAtStartPar
options pdf and eps determins if a pdf and an eps file is genrated.

\sphinxAtStartPar
option fpdf will cause the graph displayed in a seperate pdf window

\sphinxAtStartPar
option browser determins if a seperate browser window is open

\end{fulllineitems}


\end{fulllineitems}

\index{Json\_Mixin (class in modelclass)@\spxentry{Json\_Mixin}\spxextra{class in modelclass}}

\begin{fulllineitems}
\phantomsection\label{\detokenize{index:modelclass.Json_Mixin}}
\pysigstartsignatures
\pysigline{\sphinxbfcode{\sphinxupquote{class\DUrole{w}{  }}}\sphinxcode{\sphinxupquote{modelclass.}}\sphinxbfcode{\sphinxupquote{Json\_Mixin}}}
\pysigstopsignatures
\sphinxAtStartPar
This mixin class can dump a model and solution
as json serialiation to a file.

\sphinxAtStartPar
allows the precooking of a model and solution, so
a user can use a model without specifying it in
a session.
\index{modeldump() (modelclass.Json\_Mixin method)@\spxentry{modeldump()}\spxextra{modelclass.Json\_Mixin method}}

\begin{fulllineitems}
\phantomsection\label{\detokenize{index:modelclass.Json_Mixin.modeldump}}
\pysigstartsignatures
\pysiglinewithargsret{\sphinxbfcode{\sphinxupquote{modeldump}}}{\emph{\DUrole{n}{outfile}\DUrole{o}{=}\DUrole{default_value}{\textquotesingle{}\textquotesingle{}}}, \emph{\DUrole{n}{keep}\DUrole{o}{=}\DUrole{default_value}{False}}}{}
\pysigstopsignatures
\sphinxAtStartPar
Dumps a model and its lastdf to a json file

\sphinxAtStartPar
if keep=True the model.keep\_solutions will alse be dumped

\end{fulllineitems}

\index{modelload() (modelclass.Json\_Mixin class method)@\spxentry{modelload()}\spxextra{modelclass.Json\_Mixin class method}}

\begin{fulllineitems}
\phantomsection\label{\detokenize{index:modelclass.Json_Mixin.modelload}}
\pysigstartsignatures
\pysiglinewithargsret{\sphinxbfcode{\sphinxupquote{classmethod\DUrole{w}{  }}}\sphinxbfcode{\sphinxupquote{modelload}}}{\emph{\DUrole{n}{infile}}, \emph{\DUrole{n}{funks}\DUrole{o}{=}\DUrole{default_value}{{[}{]}}}, \emph{\DUrole{n}{run}\DUrole{o}{=}\DUrole{default_value}{False}}, \emph{\DUrole{n}{keep\_json}\DUrole{o}{=}\DUrole{default_value}{False}}, \emph{\DUrole{o}{**}\DUrole{n}{kwargs}}}{}
\pysigstopsignatures
\sphinxAtStartPar
Loads a model and an solution

\end{fulllineitems}


\end{fulllineitems}

\index{Excel\_Mixin (class in modelclass)@\spxentry{Excel\_Mixin}\spxextra{class in modelclass}}

\begin{fulllineitems}
\phantomsection\label{\detokenize{index:modelclass.Excel_Mixin}}
\pysigstartsignatures
\pysigline{\sphinxbfcode{\sphinxupquote{class\DUrole{w}{  }}}\sphinxcode{\sphinxupquote{modelclass.}}\sphinxbfcode{\sphinxupquote{Excel\_Mixin}}}
\pysigstopsignatures
\sphinxAtStartPar
This Mixin handels dumps and loads models into excel
\index{modeldump\_excel() (modelclass.Excel\_Mixin method)@\spxentry{modeldump\_excel()}\spxextra{modelclass.Excel\_Mixin method}}

\begin{fulllineitems}
\phantomsection\label{\detokenize{index:modelclass.Excel_Mixin.modeldump_excel}}
\pysigstartsignatures
\pysiglinewithargsret{\sphinxbfcode{\sphinxupquote{modeldump\_excel}}}{\emph{\DUrole{n}{file}}, \emph{\DUrole{n}{fromfile}\DUrole{o}{=}\DUrole{default_value}{\textquotesingle{}control.xlsm\textquotesingle{}}}, \emph{\DUrole{n}{keep\_open}\DUrole{o}{=}\DUrole{default_value}{False}}}{}
\pysigstopsignatures
\sphinxAtStartPar
Dump model and dataframe to excel workbook
\begin{quote}\begin{description}
\item[{Parameters}] \leavevmode\begin{itemize}
\item {} 
\sphinxAtStartPar
\sphinxstyleliteralstrong{\sphinxupquote{file}} (\sphinxstyleliteralemphasis{\sphinxupquote{TYPE}}) \textendash{} filename.

\item {} 
\sphinxAtStartPar
\sphinxstyleliteralstrong{\sphinxupquote{keep\_open}} (\sphinxstyleliteralemphasis{\sphinxupquote{TYPE}}\sphinxstyleliteralemphasis{\sphinxupquote{, }}\sphinxstyleliteralemphasis{\sphinxupquote{optional}}) \textendash{} Keep the workbook open in excel after returning,  The default is False.

\end{itemize}

\item[{Returns}] \leavevmode
\sphinxAtStartPar
\sphinxstylestrong{wb} \textendash{} xlwings instance of workbook .

\item[{Return type}] \leavevmode
\sphinxAtStartPar
TYPE

\end{description}\end{quote}

\end{fulllineitems}


\end{fulllineitems}

\index{Zip\_Mixin (class in modelclass)@\spxentry{Zip\_Mixin}\spxextra{class in modelclass}}

\begin{fulllineitems}
\phantomsection\label{\detokenize{index:modelclass.Zip_Mixin}}
\pysigstartsignatures
\pysigline{\sphinxbfcode{\sphinxupquote{class\DUrole{w}{  }}}\sphinxcode{\sphinxupquote{modelclass.}}\sphinxbfcode{\sphinxupquote{Zip\_Mixin}}}
\pysigstopsignatures
\sphinxAtStartPar
This experimental class zips a dumped file

\end{fulllineitems}

\index{Solver\_Mixin (class in modelclass)@\spxentry{Solver\_Mixin}\spxextra{class in modelclass}}

\begin{fulllineitems}
\phantomsection\label{\detokenize{index:modelclass.Solver_Mixin}}
\pysigstartsignatures
\pysigline{\sphinxbfcode{\sphinxupquote{class\DUrole{w}{  }}}\sphinxcode{\sphinxupquote{modelclass.}}\sphinxbfcode{\sphinxupquote{Solver\_Mixin}}}
\pysigstopsignatures
\sphinxAtStartPar
This Mixin handels the solving of models.
\index{\_\_call\_\_() (modelclass.Solver\_Mixin method)@\spxentry{\_\_call\_\_()}\spxextra{modelclass.Solver\_Mixin method}}

\begin{fulllineitems}
\phantomsection\label{\detokenize{index:modelclass.Solver_Mixin.__call__}}
\pysigstartsignatures
\pysiglinewithargsret{\sphinxbfcode{\sphinxupquote{\_\_call\_\_}}}{\emph{\DUrole{o}{*}\DUrole{n}{args}}, \emph{\DUrole{o}{**}\DUrole{n}{kwargs}}}{}
\pysigstopsignatures
\sphinxAtStartPar
Runs a model.

\sphinxAtStartPar
Default a straight model is calculated by \sphinxstyleemphasis{xgenr} a simultaneous model is solved by \sphinxstyleemphasis{sim}
\begin{quote}\begin{description}
\item[{Sim}] \leavevmode
\sphinxAtStartPar
If False forces a  model to be calculated (not solved) if True force simulation

\item[{Setbase}] \leavevmode
\sphinxAtStartPar
If True, place the result in model.basedf

\item[{Setlast}] \leavevmode
\sphinxAtStartPar
if False don’t place the results in model.lastdf

\end{description}\end{quote}

\sphinxAtStartPar
if the modelproperty previousbase is true, the previous run is used as basedf.

\end{fulllineitems}

\index{is\_newdata() (modelclass.Solver\_Mixin method)@\spxentry{is\_newdata()}\spxextra{modelclass.Solver\_Mixin method}}

\begin{fulllineitems}
\phantomsection\label{\detokenize{index:modelclass.Solver_Mixin.is_newdata}}
\pysigstartsignatures
\pysiglinewithargsret{\sphinxbfcode{\sphinxupquote{is\_newdata}}}{\emph{\DUrole{n}{databank}}}{}
\pysigstopsignatures
\sphinxAtStartPar
Determins if thius is the same databank as in the previous solution

\end{fulllineitems}

\index{sim() (modelclass.Solver\_Mixin method)@\spxentry{sim()}\spxextra{modelclass.Solver\_Mixin method}}

\begin{fulllineitems}
\phantomsection\label{\detokenize{index:modelclass.Solver_Mixin.sim}}
\pysigstartsignatures
\pysiglinewithargsret{\sphinxbfcode{\sphinxupquote{sim}}}{\emph{\DUrole{n}{databank}}, \emph{\DUrole{n}{start}\DUrole{o}{=}\DUrole{default_value}{\textquotesingle{}\textquotesingle{}}}, \emph{\DUrole{n}{slut}\DUrole{o}{=}\DUrole{default_value}{\textquotesingle{}\textquotesingle{}}}, \emph{\DUrole{n}{silent}\DUrole{o}{=}\DUrole{default_value}{1}}, \emph{\DUrole{n}{samedata}\DUrole{o}{=}\DUrole{default_value}{0}}, \emph{\DUrole{n}{alfa}\DUrole{o}{=}\DUrole{default_value}{1.0}}, \emph{\DUrole{n}{stats}\DUrole{o}{=}\DUrole{default_value}{False}}, \emph{\DUrole{n}{first\_test}\DUrole{o}{=}\DUrole{default_value}{5}}, \emph{\DUrole{n}{max\_iterations}\DUrole{o}{=}\DUrole{default_value}{200}}, \emph{\DUrole{n}{conv}\DUrole{o}{=}\DUrole{default_value}{\textquotesingle{}*\textquotesingle{}}}, \emph{\DUrole{n}{absconv}\DUrole{o}{=}\DUrole{default_value}{0.01}}, \emph{\DUrole{n}{relconv}\DUrole{o}{=}\DUrole{default_value}{1e\sphinxhyphen{}07}}, \emph{\DUrole{n}{stringjit}\DUrole{o}{=}\DUrole{default_value}{True}}, \emph{\DUrole{n}{transpile\_reset}\DUrole{o}{=}\DUrole{default_value}{False}}, \emph{\DUrole{n}{dumpvar}\DUrole{o}{=}\DUrole{default_value}{\textquotesingle{}*\textquotesingle{}}}, \emph{\DUrole{n}{init}\DUrole{o}{=}\DUrole{default_value}{False}}, \emph{\DUrole{n}{ldumpvar}\DUrole{o}{=}\DUrole{default_value}{False}}, \emph{\DUrole{n}{dumpwith}\DUrole{o}{=}\DUrole{default_value}{15}}, \emph{\DUrole{n}{dumpdecimal}\DUrole{o}{=}\DUrole{default_value}{5}}, \emph{\DUrole{n}{chunk}\DUrole{o}{=}\DUrole{default_value}{30}}, \emph{\DUrole{n}{ljit}\DUrole{o}{=}\DUrole{default_value}{False}}, \emph{\DUrole{n}{timeon}\DUrole{o}{=}\DUrole{default_value}{False}}, \emph{\DUrole{n}{fairopt}\DUrole{o}{=}\DUrole{default_value}{\{\textquotesingle{}fair\_max\_iterations \textquotesingle{}: 1\}}}, \emph{\DUrole{n}{progressbar}\DUrole{o}{=}\DUrole{default_value}{False}}, \emph{\DUrole{o}{**}\DUrole{n}{kwargs}}}{}
\pysigstopsignatures
\sphinxAtStartPar
Evaluates this model on a databank from start to slut (means end in Danish).

\sphinxAtStartPar
First it finds the values in the Dataframe, then creates the evaluater function through the \sphinxstyleemphasis{outeval} function

\sphinxAtStartPar
Then it evaluates the function and returns the values to a the Dataframe in the databank.

\sphinxAtStartPar
The text for the evaluater function is placed in the model property \sphinxstylestrong{make\_los\_text}
where it can be inspected
in case of problems.

\sphinxAtStartPar
Solves using Gauss\sphinxhyphen{}Seidle
\begin{quote}\begin{description}
\item[{Parameters}] \leavevmode\begin{itemize}
\item {} 
\sphinxAtStartPar
\sphinxstyleliteralstrong{\sphinxupquote{databank}} (\sphinxstyleliteralemphasis{\sphinxupquote{dataframe}}) \textendash{} Input dataframe

\item {} 
\sphinxAtStartPar
\sphinxstyleliteralstrong{\sphinxupquote{start}} (\sphinxstyleliteralemphasis{\sphinxupquote{optional}}) \textendash{} start of simulation, defaults to ‘’

\item {} 
\sphinxAtStartPar
\sphinxstyleliteralstrong{\sphinxupquote{slut}} (\sphinxstyleliteralemphasis{\sphinxupquote{optional}}) \textendash{} end of simulation, defaults to ‘’

\item {} 
\sphinxAtStartPar
\sphinxstyleliteralstrong{\sphinxupquote{silent}} (\sphinxstyleliteralemphasis{\sphinxupquote{bool}}\sphinxstyleliteralemphasis{\sphinxupquote{, }}\sphinxstyleliteralemphasis{\sphinxupquote{optional}}) \textendash{} keep simulation silent , defaults to 1

\item {} 
\sphinxAtStartPar
\sphinxstyleliteralstrong{\sphinxupquote{samedata}} (\sphinxstyleliteralemphasis{\sphinxupquote{bool}}\sphinxstyleliteralemphasis{\sphinxupquote{, }}\sphinxstyleliteralemphasis{\sphinxupquote{optional}}) \textendash{} the inputdata has exactly same structure as last simulation , defaults to 0

\item {} 
\sphinxAtStartPar
\sphinxstyleliteralstrong{\sphinxupquote{alfa}} (\sphinxstyleliteralemphasis{\sphinxupquote{float}}\sphinxstyleliteralemphasis{\sphinxupquote{, }}\sphinxstyleliteralemphasis{\sphinxupquote{optional}}) \textendash{} Dampeing factor, defaults to 1.0

\item {} 
\sphinxAtStartPar
\sphinxstyleliteralstrong{\sphinxupquote{stats}} (\sphinxstyleliteralemphasis{\sphinxupquote{bool}}\sphinxstyleliteralemphasis{\sphinxupquote{, }}\sphinxstyleliteralemphasis{\sphinxupquote{optional}}) \textendash{} Show statistic after finish, defaults to False

\item {} 
\sphinxAtStartPar
\sphinxstyleliteralstrong{\sphinxupquote{first\_test}} (\sphinxstyleliteralemphasis{\sphinxupquote{int}}\sphinxstyleliteralemphasis{\sphinxupquote{, }}\sphinxstyleliteralemphasis{\sphinxupquote{optional}}) \textendash{} Start testing af number og simulation, defaults to 5

\item {} 
\sphinxAtStartPar
\sphinxstyleliteralstrong{\sphinxupquote{max\_iterations}} (\sphinxstyleliteralemphasis{\sphinxupquote{int}}\sphinxstyleliteralemphasis{\sphinxupquote{, }}\sphinxstyleliteralemphasis{\sphinxupquote{optional}}) \textendash{} Max iterations, defaults to 200

\item {} 
\sphinxAtStartPar
\sphinxstyleliteralstrong{\sphinxupquote{conv}} (\sphinxstyleliteralemphasis{\sphinxupquote{str}}\sphinxstyleliteralemphasis{\sphinxupquote{, }}\sphinxstyleliteralemphasis{\sphinxupquote{optional}}) \textendash{} variables to test for convergence, defaults to ‘*’

\item {} 
\sphinxAtStartPar
\sphinxstyleliteralstrong{\sphinxupquote{absconv}} (\sphinxstyleliteralemphasis{\sphinxupquote{float}}\sphinxstyleliteralemphasis{\sphinxupquote{, }}\sphinxstyleliteralemphasis{\sphinxupquote{optional}}) \textendash{} Test convergence for values above this, defaults to 0.01

\item {} 
\sphinxAtStartPar
\sphinxstyleliteralstrong{\sphinxupquote{relconv}} (\sphinxstyleliteralemphasis{\sphinxupquote{float}}\sphinxstyleliteralemphasis{\sphinxupquote{, }}\sphinxstyleliteralemphasis{\sphinxupquote{optional}}) \textendash{} If relative movement is less, then convergence , defaults to DEFAULT\_relconv

\item {} 
\sphinxAtStartPar
\sphinxstyleliteralstrong{\sphinxupquote{stringjit}} (\sphinxstyleliteralemphasis{\sphinxupquote{bool}}\sphinxstyleliteralemphasis{\sphinxupquote{, }}\sphinxstyleliteralemphasis{\sphinxupquote{optional}}) \textendash{} If just in time compilation do it on a string not a file to import, defaults to True

\item {} 
\sphinxAtStartPar
\sphinxstyleliteralstrong{\sphinxupquote{transpile\_reset}} (\sphinxstyleliteralemphasis{\sphinxupquote{bool}}\sphinxstyleliteralemphasis{\sphinxupquote{, }}\sphinxstyleliteralemphasis{\sphinxupquote{optional}}) \textendash{} Ingnore previous transpiled model, defaults to False

\item {} 
\sphinxAtStartPar
\sphinxstyleliteralstrong{\sphinxupquote{dumpvar}} (\sphinxstyleliteralemphasis{\sphinxupquote{str}}\sphinxstyleliteralemphasis{\sphinxupquote{, }}\sphinxstyleliteralemphasis{\sphinxupquote{optional}}) \textendash{} Variables for which to dump the iterations, defaults to ‘*’

\item {} 
\sphinxAtStartPar
\sphinxstyleliteralstrong{\sphinxupquote{init}} (\sphinxstyleliteralemphasis{\sphinxupquote{bool}}\sphinxstyleliteralemphasis{\sphinxupquote{, }}\sphinxstyleliteralemphasis{\sphinxupquote{optional}}) \textendash{} If True take previous periods value as starting value, defaults to False

\item {} 
\sphinxAtStartPar
\sphinxstyleliteralstrong{\sphinxupquote{ldumpvar}} (\sphinxstyleliteralemphasis{\sphinxupquote{bool}}\sphinxstyleliteralemphasis{\sphinxupquote{, }}\sphinxstyleliteralemphasis{\sphinxupquote{optional}}) \textendash{} Dump iterations, defaults to False

\item {} 
\sphinxAtStartPar
\sphinxstyleliteralstrong{\sphinxupquote{dumpwith}} (\sphinxstyleliteralemphasis{\sphinxupquote{int}}\sphinxstyleliteralemphasis{\sphinxupquote{, }}\sphinxstyleliteralemphasis{\sphinxupquote{optional}}) \textendash{} DESCRIPTION, defaults to 15

\item {} 
\sphinxAtStartPar
\sphinxstyleliteralstrong{\sphinxupquote{dumpdecimal}} (\sphinxstyleliteralemphasis{\sphinxupquote{int}}\sphinxstyleliteralemphasis{\sphinxupquote{, }}\sphinxstyleliteralemphasis{\sphinxupquote{optional}}) \textendash{} DESCRIPTION, defaults to 5

\item {} 
\sphinxAtStartPar
\sphinxstyleliteralstrong{\sphinxupquote{chunk}} (\sphinxstyleliteralemphasis{\sphinxupquote{int}}\sphinxstyleliteralemphasis{\sphinxupquote{, }}\sphinxstyleliteralemphasis{\sphinxupquote{optional}}) \textendash{} Chunk size of transpiled model, defaults to 30

\item {} 
\sphinxAtStartPar
\sphinxstyleliteralstrong{\sphinxupquote{ljit}} (\sphinxstyleliteralemphasis{\sphinxupquote{bool}}\sphinxstyleliteralemphasis{\sphinxupquote{, }}\sphinxstyleliteralemphasis{\sphinxupquote{optional}}) \textendash{} Use just in time compilation, defaults to False

\item {} 
\sphinxAtStartPar
\sphinxstyleliteralstrong{\sphinxupquote{timeon}} (\sphinxstyleliteralemphasis{\sphinxupquote{bool}}\sphinxstyleliteralemphasis{\sphinxupquote{, }}\sphinxstyleliteralemphasis{\sphinxupquote{optional}}) \textendash{} Time the elements, defaults to False

\item {} 
\sphinxAtStartPar
\sphinxstyleliteralstrong{\sphinxupquote{fairopt}} (\sphinxstyleliteralemphasis{\sphinxupquote{TYPE}}\sphinxstyleliteralemphasis{\sphinxupquote{, }}\sphinxstyleliteralemphasis{\sphinxupquote{optional}}) \textendash{} Fair taylor options, defaults to \{‘fair\_max\_iterations’: 1\}

\item {} 
\sphinxAtStartPar
\sphinxstyleliteralstrong{\sphinxupquote{progressbar}} (\sphinxstyleliteralemphasis{\sphinxupquote{TYPE}}\sphinxstyleliteralemphasis{\sphinxupquote{, }}\sphinxstyleliteralemphasis{\sphinxupquote{optional}}) \textendash{} Show progress bar , defaults to False

\end{itemize}

\item[{Returns}] \leavevmode
\sphinxAtStartPar
A dataframe wilt the results

\end{description}\end{quote}

\end{fulllineitems}

\index{grouper() (modelclass.Solver\_Mixin static method)@\spxentry{grouper()}\spxextra{modelclass.Solver\_Mixin static method}}

\begin{fulllineitems}
\phantomsection\label{\detokenize{index:modelclass.Solver_Mixin.grouper}}
\pysigstartsignatures
\pysiglinewithargsret{\sphinxbfcode{\sphinxupquote{static\DUrole{w}{  }}}\sphinxbfcode{\sphinxupquote{grouper}}}{\emph{\DUrole{n}{iterable}}, \emph{\DUrole{n}{n}}, \emph{\DUrole{n}{fillvalue}\DUrole{o}{=}\DUrole{default_value}{\textquotesingle{}\textquotesingle{}}}}{}
\pysigstopsignatures
\sphinxAtStartPar
Collect data into fixed\sphinxhyphen{}length chunks or blocks

\end{fulllineitems}

\index{outsolve2dcunk() (modelclass.Solver\_Mixin method)@\spxentry{outsolve2dcunk()}\spxextra{modelclass.Solver\_Mixin method}}

\begin{fulllineitems}
\phantomsection\label{\detokenize{index:modelclass.Solver_Mixin.outsolve2dcunk}}
\pysigstartsignatures
\pysiglinewithargsret{\sphinxbfcode{\sphinxupquote{outsolve2dcunk}}}{\emph{\DUrole{n}{databank}}, \emph{\DUrole{n}{debug}\DUrole{o}{=}\DUrole{default_value}{1}}, \emph{\DUrole{n}{chunk}\DUrole{o}{=}\DUrole{default_value}{None}}, \emph{\DUrole{n}{ljit}\DUrole{o}{=}\DUrole{default_value}{False}}, \emph{\DUrole{n}{type}\DUrole{o}{=}\DUrole{default_value}{\textquotesingle{}gauss\textquotesingle{}}}, \emph{\DUrole{n}{cache}\DUrole{o}{=}\DUrole{default_value}{False}}}{}
\pysigstopsignatures
\sphinxAtStartPar
takes a list of terms and translates to a evaluater function called los

\sphinxAtStartPar
The model axcess the data through:Dataframe.value{[}rowindex+lag,coloumnindex{]} which is very efficient

\end{fulllineitems}

\index{sim1d() (modelclass.Solver\_Mixin method)@\spxentry{sim1d()}\spxextra{modelclass.Solver\_Mixin method}}

\begin{fulllineitems}
\phantomsection\label{\detokenize{index:modelclass.Solver_Mixin.sim1d}}
\pysigstartsignatures
\pysiglinewithargsret{\sphinxbfcode{\sphinxupquote{sim1d}}}{\emph{\DUrole{n}{databank}}, \emph{\DUrole{n}{start}\DUrole{o}{=}\DUrole{default_value}{\textquotesingle{}\textquotesingle{}}}, \emph{\DUrole{n}{slut}\DUrole{o}{=}\DUrole{default_value}{\textquotesingle{}\textquotesingle{}}}, \emph{\DUrole{n}{silent}\DUrole{o}{=}\DUrole{default_value}{1}}, \emph{\DUrole{n}{samedata}\DUrole{o}{=}\DUrole{default_value}{0}}, \emph{\DUrole{n}{alfa}\DUrole{o}{=}\DUrole{default_value}{1.0}}, \emph{\DUrole{n}{stats}\DUrole{o}{=}\DUrole{default_value}{False}}, \emph{\DUrole{n}{first\_test}\DUrole{o}{=}\DUrole{default_value}{1}}, \emph{\DUrole{n}{max\_iterations}\DUrole{o}{=}\DUrole{default_value}{100}}, \emph{\DUrole{n}{conv}\DUrole{o}{=}\DUrole{default_value}{\textquotesingle{}*\textquotesingle{}}}, \emph{\DUrole{n}{absconv}\DUrole{o}{=}\DUrole{default_value}{1.0}}, \emph{\DUrole{n}{relconv}\DUrole{o}{=}\DUrole{default_value}{1e\sphinxhyphen{}07}}, \emph{\DUrole{n}{init}\DUrole{o}{=}\DUrole{default_value}{False}}, \emph{\DUrole{n}{dumpvar}\DUrole{o}{=}\DUrole{default_value}{\textquotesingle{}*\textquotesingle{}}}, \emph{\DUrole{n}{ldumpvar}\DUrole{o}{=}\DUrole{default_value}{False}}, \emph{\DUrole{n}{dumpwith}\DUrole{o}{=}\DUrole{default_value}{15}}, \emph{\DUrole{n}{dumpdecimal}\DUrole{o}{=}\DUrole{default_value}{5}}, \emph{\DUrole{n}{chunk}\DUrole{o}{=}\DUrole{default_value}{30}}, \emph{\DUrole{n}{ljit}\DUrole{o}{=}\DUrole{default_value}{False}}, \emph{\DUrole{n}{stringjit}\DUrole{o}{=}\DUrole{default_value}{True}}, \emph{\DUrole{n}{transpile\_reset}\DUrole{o}{=}\DUrole{default_value}{False}}, \emph{\DUrole{n}{fairopt}\DUrole{o}{=}\DUrole{default_value}{\{\textquotesingle{}fair\_max\_iterations \textquotesingle{}: 1\}}}, \emph{\DUrole{n}{timeon}\DUrole{o}{=}\DUrole{default_value}{0}}, \emph{\DUrole{o}{**}\DUrole{n}{kwargs}}}{}
\pysigstopsignatures
\sphinxAtStartPar
Evaluates this model on a databank from start to slut (means end in Danish).

\sphinxAtStartPar
First it finds the values in the Dataframe, then creates the evaluater function through the \sphinxstyleemphasis{outeval} function
(\sphinxcode{\sphinxupquote{modelclass.model.fouteval()}})
then it evaluates the function and returns the values to a the Dataframe in the databank.

\sphinxAtStartPar
The text for the evaluater function is placed in the model property \sphinxstylestrong{make\_los\_text}
where it can be inspected
in case of problems.

\end{fulllineitems}

\index{outsolve1dcunk() (modelclass.Solver\_Mixin method)@\spxentry{outsolve1dcunk()}\spxextra{modelclass.Solver\_Mixin method}}

\begin{fulllineitems}
\phantomsection\label{\detokenize{index:modelclass.Solver_Mixin.outsolve1dcunk}}
\pysigstartsignatures
\pysiglinewithargsret{\sphinxbfcode{\sphinxupquote{outsolve1dcunk}}}{\emph{\DUrole{n}{debug}\DUrole{o}{=}\DUrole{default_value}{0}}, \emph{\DUrole{n}{chunk}\DUrole{o}{=}\DUrole{default_value}{None}}, \emph{\DUrole{n}{ljit}\DUrole{o}{=}\DUrole{default_value}{False}}, \emph{\DUrole{n}{cache}\DUrole{o}{=}\DUrole{default_value}{\textquotesingle{}False\textquotesingle{}}}}{}
\pysigstopsignatures
\sphinxAtStartPar
takes a list of terms and translates to a evaluater function called los

\sphinxAtStartPar
The model axcess the data through:Dataframe.value{[}rowindex+lag,coloumnindex{]} which is very efficient

\end{fulllineitems}

\index{newton() (modelclass.Solver\_Mixin method)@\spxentry{newton()}\spxextra{modelclass.Solver\_Mixin method}}

\begin{fulllineitems}
\phantomsection\label{\detokenize{index:modelclass.Solver_Mixin.newton}}
\pysigstartsignatures
\pysiglinewithargsret{\sphinxbfcode{\sphinxupquote{newton}}}{\emph{\DUrole{n}{databank}}, \emph{\DUrole{n}{start}\DUrole{o}{=}\DUrole{default_value}{\textquotesingle{}\textquotesingle{}}}, \emph{\DUrole{n}{slut}\DUrole{o}{=}\DUrole{default_value}{\textquotesingle{}\textquotesingle{}}}, \emph{\DUrole{n}{silent}\DUrole{o}{=}\DUrole{default_value}{1}}, \emph{\DUrole{n}{samedata}\DUrole{o}{=}\DUrole{default_value}{0}}, \emph{\DUrole{n}{alfa}\DUrole{o}{=}\DUrole{default_value}{1.0}}, \emph{\DUrole{n}{stats}\DUrole{o}{=}\DUrole{default_value}{False}}, \emph{\DUrole{n}{first\_test}\DUrole{o}{=}\DUrole{default_value}{1}}, \emph{\DUrole{n}{newton\_absconv}\DUrole{o}{=}\DUrole{default_value}{0.001}}, \emph{\DUrole{n}{max\_iterations}\DUrole{o}{=}\DUrole{default_value}{20}}, \emph{\DUrole{n}{conv}\DUrole{o}{=}\DUrole{default_value}{\textquotesingle{}*\textquotesingle{}}}, \emph{\DUrole{n}{absconv}\DUrole{o}{=}\DUrole{default_value}{1.0}}, \emph{\DUrole{n}{relconv}\DUrole{o}{=}\DUrole{default_value}{1e\sphinxhyphen{}07}}, \emph{\DUrole{n}{nonlin}\DUrole{o}{=}\DUrole{default_value}{False}}, \emph{\DUrole{n}{timeit}\DUrole{o}{=}\DUrole{default_value}{False}}, \emph{\DUrole{n}{newton\_reset}\DUrole{o}{=}\DUrole{default_value}{1}}, \emph{\DUrole{n}{dumpvar}\DUrole{o}{=}\DUrole{default_value}{\textquotesingle{}*\textquotesingle{}}}, \emph{\DUrole{n}{ldumpvar}\DUrole{o}{=}\DUrole{default_value}{False}}, \emph{\DUrole{n}{dumpwith}\DUrole{o}{=}\DUrole{default_value}{15}}, \emph{\DUrole{n}{dumpdecimal}\DUrole{o}{=}\DUrole{default_value}{5}}, \emph{\DUrole{n}{chunk}\DUrole{o}{=}\DUrole{default_value}{30}}, \emph{\DUrole{n}{ljit}\DUrole{o}{=}\DUrole{default_value}{False}}, \emph{\DUrole{n}{stringjit}\DUrole{o}{=}\DUrole{default_value}{True}}, \emph{\DUrole{n}{transpile\_reset}\DUrole{o}{=}\DUrole{default_value}{False}}, \emph{\DUrole{n}{lnjit}\DUrole{o}{=}\DUrole{default_value}{False}}, \emph{\DUrole{n}{init}\DUrole{o}{=}\DUrole{default_value}{False}}, \emph{\DUrole{n}{newtonalfa}\DUrole{o}{=}\DUrole{default_value}{1.0}}, \emph{\DUrole{n}{newtonnodamp}\DUrole{o}{=}\DUrole{default_value}{0}}, \emph{\DUrole{n}{forcenum}\DUrole{o}{=}\DUrole{default_value}{True}}, \emph{\DUrole{n}{fairopt}\DUrole{o}{=}\DUrole{default_value}{\{\textquotesingle{}fair\_max\_iterations \textquotesingle{}: 1\}}}, \emph{\DUrole{o}{**}\DUrole{n}{kwargs}}}{}
\pysigstopsignatures
\sphinxAtStartPar
Evaluates this model on a databank from start to slut (means end in Danish).

\sphinxAtStartPar
First it finds the values in the Dataframe, then creates the evaluater function through the \sphinxstyleemphasis{outeval} function
(\sphinxcode{\sphinxupquote{modelclass.model.fouteval()}})
then it evaluates the function and returns the values to a the Dataframe in the databank.

\sphinxAtStartPar
The text for the evaluater function is placed in the model property \sphinxstylestrong{make\_los\_text}
where it can be inspected
in case of problems.

\end{fulllineitems}

\index{newtonstack() (modelclass.Solver\_Mixin method)@\spxentry{newtonstack()}\spxextra{modelclass.Solver\_Mixin method}}

\begin{fulllineitems}
\phantomsection\label{\detokenize{index:modelclass.Solver_Mixin.newtonstack}}
\pysigstartsignatures
\pysiglinewithargsret{\sphinxbfcode{\sphinxupquote{newtonstack}}}{\emph{\DUrole{n}{databank}}, \emph{\DUrole{n}{start}\DUrole{o}{=}\DUrole{default_value}{\textquotesingle{}\textquotesingle{}}}, \emph{\DUrole{n}{slut}\DUrole{o}{=}\DUrole{default_value}{\textquotesingle{}\textquotesingle{}}}, \emph{\DUrole{n}{silent}\DUrole{o}{=}\DUrole{default_value}{1}}, \emph{\DUrole{n}{samedata}\DUrole{o}{=}\DUrole{default_value}{0}}, \emph{\DUrole{n}{alfa}\DUrole{o}{=}\DUrole{default_value}{1.0}}, \emph{\DUrole{n}{stats}\DUrole{o}{=}\DUrole{default_value}{False}}, \emph{\DUrole{n}{first\_test}\DUrole{o}{=}\DUrole{default_value}{1}}, \emph{\DUrole{n}{newton\_absconv}\DUrole{o}{=}\DUrole{default_value}{0.001}}, \emph{\DUrole{n}{max\_iterations}\DUrole{o}{=}\DUrole{default_value}{20}}, \emph{\DUrole{n}{conv}\DUrole{o}{=}\DUrole{default_value}{\textquotesingle{}*\textquotesingle{}}}, \emph{\DUrole{n}{absconv}\DUrole{o}{=}\DUrole{default_value}{1.0}}, \emph{\DUrole{n}{relconv}\DUrole{o}{=}\DUrole{default_value}{1e\sphinxhyphen{}07}}, \emph{\DUrole{n}{dumpvar}\DUrole{o}{=}\DUrole{default_value}{\textquotesingle{}*\textquotesingle{}}}, \emph{\DUrole{n}{ldumpvar}\DUrole{o}{=}\DUrole{default_value}{False}}, \emph{\DUrole{n}{dumpwith}\DUrole{o}{=}\DUrole{default_value}{15}}, \emph{\DUrole{n}{dumpdecimal}\DUrole{o}{=}\DUrole{default_value}{5}}, \emph{\DUrole{n}{chunk}\DUrole{o}{=}\DUrole{default_value}{30}}, \emph{\DUrole{n}{nchunk}\DUrole{o}{=}\DUrole{default_value}{30}}, \emph{\DUrole{n}{ljit}\DUrole{o}{=}\DUrole{default_value}{False}}, \emph{\DUrole{n}{stringjit}\DUrole{o}{=}\DUrole{default_value}{True}}, \emph{\DUrole{n}{transpile\_reset}\DUrole{o}{=}\DUrole{default_value}{False}}, \emph{\DUrole{n}{nljit}\DUrole{o}{=}\DUrole{default_value}{0}}, \emph{\DUrole{n}{fairopt}\DUrole{o}{=}\DUrole{default_value}{\{\textquotesingle{}fair\_max\_iterations \textquotesingle{}: 1\}}}, \emph{\DUrole{n}{debug}\DUrole{o}{=}\DUrole{default_value}{False}}, \emph{\DUrole{n}{timeit}\DUrole{o}{=}\DUrole{default_value}{False}}, \emph{\DUrole{n}{nonlin}\DUrole{o}{=}\DUrole{default_value}{False}}, \emph{\DUrole{n}{newtonalfa}\DUrole{o}{=}\DUrole{default_value}{1.0}}, \emph{\DUrole{n}{newtonnodamp}\DUrole{o}{=}\DUrole{default_value}{0}}, \emph{\DUrole{n}{forcenum}\DUrole{o}{=}\DUrole{default_value}{True}}, \emph{\DUrole{n}{newton\_reset}\DUrole{o}{=}\DUrole{default_value}{False}}, \emph{\DUrole{o}{**}\DUrole{n}{kwargs}}}{}
\pysigstopsignatures
\sphinxAtStartPar
Evaluates this model on a databank from start to slut (means end in Danish).

\sphinxAtStartPar
First it finds the values in the Dataframe, then creates the evaluater function through the \sphinxstyleemphasis{outeval} function
(\sphinxcode{\sphinxupquote{modelclass.model.fouteval()}})
then it evaluates the function and returns the values to a the Dataframe in the databank.

\sphinxAtStartPar
The text for the evaluater function is placed in the model property \sphinxstylestrong{make\_los\_text}
where it can be inspected
in case of problems.

\end{fulllineitems}

\index{newton\_un\_normalized() (modelclass.Solver\_Mixin method)@\spxentry{newton\_un\_normalized()}\spxextra{modelclass.Solver\_Mixin method}}

\begin{fulllineitems}
\phantomsection\label{\detokenize{index:modelclass.Solver_Mixin.newton_un_normalized}}
\pysigstartsignatures
\pysiglinewithargsret{\sphinxbfcode{\sphinxupquote{newton\_un\_normalized}}}{\emph{\DUrole{n}{databank}}, \emph{\DUrole{n}{start}\DUrole{o}{=}\DUrole{default_value}{\textquotesingle{}\textquotesingle{}}}, \emph{\DUrole{n}{slut}\DUrole{o}{=}\DUrole{default_value}{\textquotesingle{}\textquotesingle{}}}, \emph{\DUrole{n}{silent}\DUrole{o}{=}\DUrole{default_value}{1}}, \emph{\DUrole{n}{samedata}\DUrole{o}{=}\DUrole{default_value}{0}}, \emph{\DUrole{n}{alfa}\DUrole{o}{=}\DUrole{default_value}{1.0}}, \emph{\DUrole{n}{stats}\DUrole{o}{=}\DUrole{default_value}{False}}, \emph{\DUrole{n}{first\_test}\DUrole{o}{=}\DUrole{default_value}{1}}, \emph{\DUrole{n}{newton\_absconv}\DUrole{o}{=}\DUrole{default_value}{0.001}}, \emph{\DUrole{n}{max\_iterations}\DUrole{o}{=}\DUrole{default_value}{20}}, \emph{\DUrole{n}{conv}\DUrole{o}{=}\DUrole{default_value}{\textquotesingle{}*\textquotesingle{}}}, \emph{\DUrole{n}{absconv}\DUrole{o}{=}\DUrole{default_value}{1.0}}, \emph{\DUrole{n}{relconv}\DUrole{o}{=}\DUrole{default_value}{1e\sphinxhyphen{}07}}, \emph{\DUrole{n}{nonlin}\DUrole{o}{=}\DUrole{default_value}{False}}, \emph{\DUrole{n}{timeit}\DUrole{o}{=}\DUrole{default_value}{False}}, \emph{\DUrole{n}{newton\_reset}\DUrole{o}{=}\DUrole{default_value}{1}}, \emph{\DUrole{n}{dumpvar}\DUrole{o}{=}\DUrole{default_value}{\textquotesingle{}*\textquotesingle{}}}, \emph{\DUrole{n}{ldumpvar}\DUrole{o}{=}\DUrole{default_value}{False}}, \emph{\DUrole{n}{dumpwith}\DUrole{o}{=}\DUrole{default_value}{15}}, \emph{\DUrole{n}{dumpdecimal}\DUrole{o}{=}\DUrole{default_value}{5}}, \emph{\DUrole{n}{chunk}\DUrole{o}{=}\DUrole{default_value}{30}}, \emph{\DUrole{n}{ljit}\DUrole{o}{=}\DUrole{default_value}{False}}, \emph{\DUrole{n}{stringjit}\DUrole{o}{=}\DUrole{default_value}{True}}, \emph{\DUrole{n}{transpile\_reset}\DUrole{o}{=}\DUrole{default_value}{False}}, \emph{\DUrole{n}{lnjit}\DUrole{o}{=}\DUrole{default_value}{False}}, \emph{\DUrole{n}{fairopt}\DUrole{o}{=}\DUrole{default_value}{\{\textquotesingle{}fair\_max\_iterations \textquotesingle{}: 1\}}}, \emph{\DUrole{n}{newtonalfa}\DUrole{o}{=}\DUrole{default_value}{1.0}}, \emph{\DUrole{n}{newtonnodamp}\DUrole{o}{=}\DUrole{default_value}{0}}, \emph{\DUrole{n}{forcenum}\DUrole{o}{=}\DUrole{default_value}{True}}, \emph{\DUrole{o}{**}\DUrole{n}{kwargs}}}{}
\pysigstopsignatures
\sphinxAtStartPar
Evaluates this model on a databank from start to slut (means end in Danish).

\sphinxAtStartPar
First it finds the values in the Dataframe, then creates the evaluater function through the \sphinxstyleemphasis{outeval} function
(\sphinxcode{\sphinxupquote{modelclass.model.fouteval()}})
then it evaluates the function and returns the values to a the Dataframe in the databank.

\sphinxAtStartPar
The text for the evaluater function is placed in the model property \sphinxstylestrong{make\_los\_text}
where it can be inspected
in case of problems.

\end{fulllineitems}

\index{newtonstack\_un\_normalized() (modelclass.Solver\_Mixin method)@\spxentry{newtonstack\_un\_normalized()}\spxextra{modelclass.Solver\_Mixin method}}

\begin{fulllineitems}
\phantomsection\label{\detokenize{index:modelclass.Solver_Mixin.newtonstack_un_normalized}}
\pysigstartsignatures
\pysiglinewithargsret{\sphinxbfcode{\sphinxupquote{newtonstack\_un\_normalized}}}{\emph{\DUrole{n}{databank}}, \emph{\DUrole{n}{start}\DUrole{o}{=}\DUrole{default_value}{\textquotesingle{}\textquotesingle{}}}, \emph{\DUrole{n}{slut}\DUrole{o}{=}\DUrole{default_value}{\textquotesingle{}\textquotesingle{}}}, \emph{\DUrole{n}{silent}\DUrole{o}{=}\DUrole{default_value}{1}}, \emph{\DUrole{n}{samedata}\DUrole{o}{=}\DUrole{default_value}{0}}, \emph{\DUrole{n}{alfa}\DUrole{o}{=}\DUrole{default_value}{1.0}}, \emph{\DUrole{n}{stats}\DUrole{o}{=}\DUrole{default_value}{False}}, \emph{\DUrole{n}{first\_test}\DUrole{o}{=}\DUrole{default_value}{1}}, \emph{\DUrole{n}{newton\_absconv}\DUrole{o}{=}\DUrole{default_value}{0.001}}, \emph{\DUrole{n}{max\_iterations}\DUrole{o}{=}\DUrole{default_value}{20}}, \emph{\DUrole{n}{conv}\DUrole{o}{=}\DUrole{default_value}{\textquotesingle{}*\textquotesingle{}}}, \emph{\DUrole{n}{absconv}\DUrole{o}{=}\DUrole{default_value}{1.0}}, \emph{\DUrole{n}{relconv}\DUrole{o}{=}\DUrole{default_value}{1e\sphinxhyphen{}07}}, \emph{\DUrole{n}{dumpvar}\DUrole{o}{=}\DUrole{default_value}{\textquotesingle{}*\textquotesingle{}}}, \emph{\DUrole{n}{ldumpvar}\DUrole{o}{=}\DUrole{default_value}{False}}, \emph{\DUrole{n}{dumpwith}\DUrole{o}{=}\DUrole{default_value}{15}}, \emph{\DUrole{n}{dumpdecimal}\DUrole{o}{=}\DUrole{default_value}{5}}, \emph{\DUrole{n}{chunk}\DUrole{o}{=}\DUrole{default_value}{30}}, \emph{\DUrole{n}{nchunk}\DUrole{o}{=}\DUrole{default_value}{None}}, \emph{\DUrole{n}{ljit}\DUrole{o}{=}\DUrole{default_value}{False}}, \emph{\DUrole{n}{nljit}\DUrole{o}{=}\DUrole{default_value}{0}}, \emph{\DUrole{n}{stringjit}\DUrole{o}{=}\DUrole{default_value}{True}}, \emph{\DUrole{n}{transpile\_reset}\DUrole{o}{=}\DUrole{default_value}{False}}, \emph{\DUrole{n}{fairopt}\DUrole{o}{=}\DUrole{default_value}{\{\textquotesingle{}fair\_max\_iterations \textquotesingle{}: 1\}}}, \emph{\DUrole{n}{debug}\DUrole{o}{=}\DUrole{default_value}{False}}, \emph{\DUrole{n}{timeit}\DUrole{o}{=}\DUrole{default_value}{False}}, \emph{\DUrole{n}{nonlin}\DUrole{o}{=}\DUrole{default_value}{False}}, \emph{\DUrole{n}{newtonalfa}\DUrole{o}{=}\DUrole{default_value}{1.0}}, \emph{\DUrole{n}{newtonnodamp}\DUrole{o}{=}\DUrole{default_value}{0}}, \emph{\DUrole{n}{forcenum}\DUrole{o}{=}\DUrole{default_value}{True}}, \emph{\DUrole{n}{newton\_reset}\DUrole{o}{=}\DUrole{default_value}{False}}, \emph{\DUrole{o}{**}\DUrole{n}{kwargs}}}{}
\pysigstopsignatures
\sphinxAtStartPar
Evaluates this model on a databank from start to slut (means end in Danish).

\sphinxAtStartPar
First it finds the values in the Dataframe, then creates the evaluater function through the \sphinxstyleemphasis{outeval} function
(\sphinxcode{\sphinxupquote{modelclass.model.fouteval()}})
then it evaluates the function and returns the values to a the Dataframe in the databank.

\sphinxAtStartPar
The text for the evaluater function is placed in the model property \sphinxstylestrong{make\_los\_text}
where it can be inspected
in case of problems.

\end{fulllineitems}

\index{res() (modelclass.Solver\_Mixin method)@\spxentry{res()}\spxextra{modelclass.Solver\_Mixin method}}

\begin{fulllineitems}
\phantomsection\label{\detokenize{index:modelclass.Solver_Mixin.res}}
\pysigstartsignatures
\pysiglinewithargsret{\sphinxbfcode{\sphinxupquote{res}}}{\emph{\DUrole{n}{databank}}, \emph{\DUrole{n}{start}\DUrole{o}{=}\DUrole{default_value}{\textquotesingle{}\textquotesingle{}}}, \emph{\DUrole{n}{slut}\DUrole{o}{=}\DUrole{default_value}{\textquotesingle{}\textquotesingle{}}}, \emph{\DUrole{n}{debug}\DUrole{o}{=}\DUrole{default_value}{False}}, \emph{\DUrole{n}{timeit}\DUrole{o}{=}\DUrole{default_value}{False}}, \emph{\DUrole{n}{silent}\DUrole{o}{=}\DUrole{default_value}{False}}, \emph{\DUrole{n}{chunk}\DUrole{o}{=}\DUrole{default_value}{None}}, \emph{\DUrole{n}{ljit}\DUrole{o}{=}\DUrole{default_value}{0}}, \emph{\DUrole{n}{stringjit}\DUrole{o}{=}\DUrole{default_value}{True}}, \emph{\DUrole{n}{transpile\_reset}\DUrole{o}{=}\DUrole{default_value}{False}}, \emph{\DUrole{n}{alfa}\DUrole{o}{=}\DUrole{default_value}{1}}, \emph{\DUrole{n}{stats}\DUrole{o}{=}\DUrole{default_value}{0}}, \emph{\DUrole{n}{samedata}\DUrole{o}{=}\DUrole{default_value}{False}}, \emph{\DUrole{o}{**}\DUrole{n}{kwargs}}}{}
\pysigstopsignatures
\sphinxAtStartPar
calculates the result of a model, no iteration or interaction
The text for the evaluater function is placed in the model property \sphinxstylestrong{make\_res\_text}
where it can be inspected
in case of problems.

\end{fulllineitems}

\index{invert() (modelclass.Solver\_Mixin method)@\spxentry{invert()}\spxextra{modelclass.Solver\_Mixin method}}

\begin{fulllineitems}
\phantomsection\label{\detokenize{index:modelclass.Solver_Mixin.invert}}
\pysigstartsignatures
\pysiglinewithargsret{\sphinxbfcode{\sphinxupquote{invert}}}{\emph{\DUrole{n}{databank}}, \emph{\DUrole{n}{targets}}, \emph{\DUrole{n}{instruments}}, \emph{\DUrole{n}{silent}\DUrole{o}{=}\DUrole{default_value}{1}}, \emph{\DUrole{n}{DefaultImpuls}\DUrole{o}{=}\DUrole{default_value}{0.01}}, \emph{\DUrole{n}{defaultconv}\DUrole{o}{=}\DUrole{default_value}{0.001}}, \emph{\DUrole{n}{nonlin}\DUrole{o}{=}\DUrole{default_value}{False}}, \emph{\DUrole{n}{maxiter}\DUrole{o}{=}\DUrole{default_value}{30}}, \emph{\DUrole{o}{**}\DUrole{n}{kwargs}}}{}
\pysigstopsignatures
\sphinxAtStartPar
Solves instruments for targets

\end{fulllineitems}

\index{errfunk1d() (modelclass.Solver\_Mixin method)@\spxentry{errfunk1d()}\spxextra{modelclass.Solver\_Mixin method}}

\begin{fulllineitems}
\phantomsection\label{\detokenize{index:modelclass.Solver_Mixin.errfunk1d}}
\pysigstartsignatures
\pysiglinewithargsret{\sphinxbfcode{\sphinxupquote{errfunk1d}}}{\emph{\DUrole{n}{a}}, \emph{\DUrole{n}{linenr}}, \emph{\DUrole{n}{overhead}\DUrole{o}{=}\DUrole{default_value}{4}}, \emph{\DUrole{n}{overeq}\DUrole{o}{=}\DUrole{default_value}{0}}}{}
\pysigstopsignatures
\sphinxAtStartPar
Handle errors in sim1d

\end{fulllineitems}

\index{errfunk() (modelclass.Solver\_Mixin method)@\spxentry{errfunk()}\spxextra{modelclass.Solver\_Mixin method}}

\begin{fulllineitems}
\phantomsection\label{\detokenize{index:modelclass.Solver_Mixin.errfunk}}
\pysigstartsignatures
\pysiglinewithargsret{\sphinxbfcode{\sphinxupquote{errfunk}}}{\emph{\DUrole{n}{values}}, \emph{\DUrole{n}{linenr}}, \emph{\DUrole{n}{overhead}\DUrole{o}{=}\DUrole{default_value}{4}}, \emph{\DUrole{n}{overeq}\DUrole{o}{=}\DUrole{default_value}{0}}}{}
\pysigstopsignatures
\sphinxAtStartPar
developement function

\sphinxAtStartPar
to handle run time errors in model calculations

\end{fulllineitems}

\index{show\_iterations() (modelclass.Solver\_Mixin method)@\spxentry{show\_iterations()}\spxextra{modelclass.Solver\_Mixin method}}

\begin{fulllineitems}
\phantomsection\label{\detokenize{index:modelclass.Solver_Mixin.show_iterations}}
\pysigstartsignatures
\pysiglinewithargsret{\sphinxbfcode{\sphinxupquote{show\_iterations}}}{\emph{\DUrole{n}{pat}\DUrole{o}{=}\DUrole{default_value}{\textquotesingle{}*\textquotesingle{}}}, \emph{\DUrole{n}{per}\DUrole{o}{=}\DUrole{default_value}{\textquotesingle{}\textquotesingle{}}}, \emph{\DUrole{n}{last}\DUrole{o}{=}\DUrole{default_value}{0}}, \emph{\DUrole{n}{change}\DUrole{o}{=}\DUrole{default_value}{False}}, \emph{\DUrole{n}{top}\DUrole{o}{=}\DUrole{default_value}{0.9}}}{}
\pysigstopsignatures
\sphinxAtStartPar
shows the last iterations for the most recent simulation.
iterations are dumped if ldumpvar is set to True
variables can be selceted by: dumpvar = ‘\textless{}pattern\textgreater{}’
\begin{quote}\begin{description}
\item[{Parameters}] \leavevmode\begin{itemize}
\item {} 
\sphinxAtStartPar
\sphinxstyleliteralstrong{\sphinxupquote{pat}} (\sphinxstyleliteralemphasis{\sphinxupquote{TYPE}}\sphinxstyleliteralemphasis{\sphinxupquote{, }}\sphinxstyleliteralemphasis{\sphinxupquote{optional}}) \textendash{} Variables for which to show iterations . Defaults to ‘*’.

\item {} 
\sphinxAtStartPar
\sphinxstyleliteralstrong{\sphinxupquote{per}} (\sphinxstyleliteralemphasis{\sphinxupquote{TYPE}}\sphinxstyleliteralemphasis{\sphinxupquote{, }}\sphinxstyleliteralemphasis{\sphinxupquote{optional}}) \textendash{} The time frame for which to show iterations, Defaults to the last projection .

\item {} 
\sphinxAtStartPar
\sphinxstyleliteralstrong{\sphinxupquote{last}} (\sphinxstyleliteralemphasis{\sphinxupquote{TYPE}}\sphinxstyleliteralemphasis{\sphinxupquote{, }}\sphinxstyleliteralemphasis{\sphinxupquote{optional}}) \textendash{} Only show the last iterations . Defaults to 0.

\item {} 
\sphinxAtStartPar
\sphinxstyleliteralstrong{\sphinxupquote{change}} (\sphinxstyleliteralemphasis{\sphinxupquote{TYPE}}\sphinxstyleliteralemphasis{\sphinxupquote{, }}\sphinxstyleliteralemphasis{\sphinxupquote{optional}}) \textendash{} show the changes from iteration to iterations instead of the levels. Defaults to False.

\item {} 
\sphinxAtStartPar
\sphinxstyleliteralstrong{\sphinxupquote{top}} (\sphinxstyleliteralemphasis{\sphinxupquote{float}}\sphinxstyleliteralemphasis{\sphinxupquote{,}}\sphinxstyleliteralemphasis{\sphinxupquote{optional}}) \textendash{} top of chartss between 1 and 0

\end{itemize}

\item[{Returns}] \leavevmode
\sphinxAtStartPar
DESCRIPTION.

\item[{Return type}] \leavevmode
\sphinxAtStartPar
fig (TYPE)

\end{description}\end{quote}

\end{fulllineitems}


\end{fulllineitems}

\index{Dash\_Mixin (class in modelclass)@\spxentry{Dash\_Mixin}\spxextra{class in modelclass}}

\begin{fulllineitems}
\phantomsection\label{\detokenize{index:modelclass.Dash_Mixin}}
\pysigstartsignatures
\pysigline{\sphinxbfcode{\sphinxupquote{class\DUrole{w}{  }}}\sphinxcode{\sphinxupquote{modelclass.}}\sphinxbfcode{\sphinxupquote{Dash\_Mixin}}}
\pysigstopsignatures
\sphinxAtStartPar
This mixin wraps call the Dash dashboard

\end{fulllineitems}

\index{WB\_Mixin (class in modelclass)@\spxentry{WB\_Mixin}\spxextra{class in modelclass}}

\begin{fulllineitems}
\phantomsection\label{\detokenize{index:modelclass.WB_Mixin}}
\pysigstartsignatures
\pysigline{\sphinxbfcode{\sphinxupquote{class\DUrole{w}{  }}}\sphinxcode{\sphinxupquote{modelclass.}}\sphinxbfcode{\sphinxupquote{WB\_Mixin}}}
\pysigstopsignatures
\sphinxAtStartPar
This mixin handles a number of enhancements
\index{wb\_behavioral (modelclass.WB\_Mixin property)@\spxentry{wb\_behavioral}\spxextra{modelclass.WB\_Mixin property}}

\begin{fulllineitems}
\phantomsection\label{\detokenize{index:modelclass.WB_Mixin.wb_behavioral}}
\pysigstartsignatures
\pysigline{\sphinxbfcode{\sphinxupquote{property\DUrole{w}{  }}}\sphinxbfcode{\sphinxupquote{wb\_behavioral}}}
\pysigstopsignatures
\sphinxAtStartPar
returns endogeneous where the frml name contains a Z which signals a stocastic equation

\end{fulllineitems}

\index{wb\_ident (modelclass.WB\_Mixin property)@\spxentry{wb\_ident}\spxextra{modelclass.WB\_Mixin property}}

\begin{fulllineitems}
\phantomsection\label{\detokenize{index:modelclass.WB_Mixin.wb_ident}}
\pysigstartsignatures
\pysigline{\sphinxbfcode{\sphinxupquote{property\DUrole{w}{  }}}\sphinxbfcode{\sphinxupquote{wb\_ident}}}
\pysigstopsignatures
\sphinxAtStartPar
returns endogeneous variables not in wb\_behavioral

\end{fulllineitems}

\index{fix() (modelclass.WB\_Mixin method)@\spxentry{fix()}\spxextra{modelclass.WB\_Mixin method}}

\begin{fulllineitems}
\phantomsection\label{\detokenize{index:modelclass.WB_Mixin.fix}}
\pysigstartsignatures
\pysiglinewithargsret{\sphinxbfcode{\sphinxupquote{fix}}}{\emph{\DUrole{n}{df}}, \emph{\DUrole{n}{pat}\DUrole{o}{=}\DUrole{default_value}{\textquotesingle{}*\textquotesingle{}}}, \emph{\DUrole{n}{start}\DUrole{o}{=}\DUrole{default_value}{\textquotesingle{}\textquotesingle{}}}, \emph{\DUrole{n}{end}\DUrole{o}{=}\DUrole{default_value}{\textquotesingle{}\textquotesingle{}}}}{}
\pysigstopsignatures
\sphinxAtStartPar
Fixes variables to the current values.

\sphinxAtStartPar
for variables where the equation looks like:

\begin{sphinxVerbatim}[commandchars=\\\{\}]
\PYG{n}{var} \PYG{o}{=} \PYG{p}{(}\PYG{n}{rhs}\PYG{p}{)}\PYG{o}{*}\PYG{p}{(}\PYG{l+m+mi}{1}\PYG{o}{\PYGZhy{}}\PYG{n}{var\PYGZus{}d}\PYG{p}{)}\PYG{o}{+}\PYG{n}{var\PYGZus{}x}\PYG{o}{*}\PYG{n}{var\PYGZus{}d}
\end{sphinxVerbatim}

\sphinxAtStartPar
The values in the smpl set by \sphinxstyleemphasis{start} and \sphinxstyleemphasis{end} will be set to:

\begin{sphinxVerbatim}[commandchars=\\\{\}]
\PYG{n}{var\PYGZus{}x} \PYG{o}{=} \PYG{n}{var}
\PYG{n}{var\PYGZus{}d} \PYG{o}{=} \PYG{l+m+mi}{1}
\end{sphinxVerbatim}

\sphinxAtStartPar
The variables fulfilling this are elements of .wb\_behavioral
\begin{quote}\begin{description}
\item[{Parameters}] \leavevmode\begin{itemize}
\item {} 
\sphinxAtStartPar
\sphinxstyleliteralstrong{\sphinxupquote{df}} (\sphinxstyleliteralemphasis{\sphinxupquote{TYPE}}) \textendash{} Input dataframe should contain a solution and all variables ..

\item {} 
\sphinxAtStartPar
\sphinxstyleliteralstrong{\sphinxupquote{pat}} (\sphinxstyleliteralemphasis{\sphinxupquote{TYPE}}\sphinxstyleliteralemphasis{\sphinxupquote{, }}\sphinxstyleliteralemphasis{\sphinxupquote{optional}}) \textendash{} Select variables to endogenize. Defaults to ‘*’.

\item {} 
\sphinxAtStartPar
\sphinxstyleliteralstrong{\sphinxupquote{start}} (\sphinxstyleliteralemphasis{\sphinxupquote{TYPE}}\sphinxstyleliteralemphasis{\sphinxupquote{, }}\sphinxstyleliteralemphasis{\sphinxupquote{optional}}) \textendash{} start periode. Defaults to ‘’.

\item {} 
\sphinxAtStartPar
\sphinxstyleliteralstrong{\sphinxupquote{end}} (\sphinxstyleliteralemphasis{\sphinxupquote{TYPE}}\sphinxstyleliteralemphasis{\sphinxupquote{, }}\sphinxstyleliteralemphasis{\sphinxupquote{optional}}) \textendash{} end periode. Defaults to ‘’.

\end{itemize}

\item[{Returns}] \leavevmode
\sphinxAtStartPar
the resulting daaframe .

\item[{Return type}] \leavevmode
\sphinxAtStartPar
dataframe (TYPE)

\end{description}\end{quote}

\end{fulllineitems}

\index{unfix() (modelclass.WB\_Mixin method)@\spxentry{unfix()}\spxextra{modelclass.WB\_Mixin method}}

\begin{fulllineitems}
\phantomsection\label{\detokenize{index:modelclass.WB_Mixin.unfix}}
\pysigstartsignatures
\pysiglinewithargsret{\sphinxbfcode{\sphinxupquote{unfix}}}{\emph{\DUrole{n}{df}}, \emph{\DUrole{n}{pat}\DUrole{o}{=}\DUrole{default_value}{\textquotesingle{}*\textquotesingle{}}}, \emph{\DUrole{n}{start}\DUrole{o}{=}\DUrole{default_value}{\textquotesingle{}\textquotesingle{}}}, \emph{\DUrole{n}{end}\DUrole{o}{=}\DUrole{default_value}{\textquotesingle{}\textquotesingle{}}}}{}
\pysigstopsignatures
\sphinxAtStartPar
Unfix (endogenize) variables
\begin{quote}\begin{description}
\item[{Parameters}] \leavevmode\begin{itemize}
\item {} 
\sphinxAtStartPar
\sphinxstyleliteralstrong{\sphinxupquote{df}} (\sphinxstyleliteralemphasis{\sphinxupquote{Dataframe}}) \textendash{} Input dataframe, should contain a solution and all variables .

\item {} 
\sphinxAtStartPar
\sphinxstyleliteralstrong{\sphinxupquote{pat}} (\sphinxstyleliteralemphasis{\sphinxupquote{string}}\sphinxstyleliteralemphasis{\sphinxupquote{, }}\sphinxstyleliteralemphasis{\sphinxupquote{optional}}) \textendash{} Select variables to endogenize. Defaults to ‘*’.

\item {} 
\sphinxAtStartPar
\sphinxstyleliteralstrong{\sphinxupquote{start}} (\sphinxstyleliteralemphasis{\sphinxupquote{TYPE}}\sphinxstyleliteralemphasis{\sphinxupquote{, }}\sphinxstyleliteralemphasis{\sphinxupquote{optional}}) \textendash{} start periode. Defaults to ‘’.

\item {} 
\sphinxAtStartPar
\sphinxstyleliteralstrong{\sphinxupquote{end}} (\sphinxstyleliteralemphasis{\sphinxupquote{TYPE}}\sphinxstyleliteralemphasis{\sphinxupquote{, }}\sphinxstyleliteralemphasis{\sphinxupquote{optional}}) \textendash{} end periode. Defaults to ‘’.

\end{itemize}

\item[{Returns}] \leavevmode
\sphinxAtStartPar
A dratframe with all dummies for the selected variablse set to 0 .

\item[{Return type}] \leavevmode
\sphinxAtStartPar
dataframe (TYPE)

\end{description}\end{quote}

\end{fulllineitems}

\index{find\_fix\_dummy\_fixed() (modelclass.WB\_Mixin method)@\spxentry{find\_fix\_dummy\_fixed()}\spxextra{modelclass.WB\_Mixin method}}

\begin{fulllineitems}
\phantomsection\label{\detokenize{index:modelclass.WB_Mixin.find_fix_dummy_fixed}}
\pysigstartsignatures
\pysiglinewithargsret{\sphinxbfcode{\sphinxupquote{find\_fix\_dummy\_fixed}}}{\emph{\DUrole{n}{df}\DUrole{o}{=}\DUrole{default_value}{None}}}{}
\pysigstopsignatures
\sphinxAtStartPar
returns names of actiove exogenizing dummies

\sphinxAtStartPar
sets the property self. exodummy\_per which defines the time over which the dummies are defined

\end{fulllineitems}

\index{fix\_dummy\_fixed (modelclass.WB\_Mixin property)@\spxentry{fix\_dummy\_fixed}\spxextra{modelclass.WB\_Mixin property}}

\begin{fulllineitems}
\phantomsection\label{\detokenize{index:modelclass.WB_Mixin.fix_dummy_fixed}}
\pysigstartsignatures
\pysigline{\sphinxbfcode{\sphinxupquote{property\DUrole{w}{  }}}\sphinxbfcode{\sphinxupquote{fix\_dummy\_fixed}}}
\pysigstopsignatures
\sphinxAtStartPar
returns names of actiove exogenizing dummies

\sphinxAtStartPar
sets the property self. exodummy\_per which defines the time over which the dummies are defined

\end{fulllineitems}

\index{fix\_dummy\_fixed\_old (modelclass.WB\_Mixin property)@\spxentry{fix\_dummy\_fixed\_old}\spxextra{modelclass.WB\_Mixin property}}

\begin{fulllineitems}
\phantomsection\label{\detokenize{index:modelclass.WB_Mixin.fix_dummy_fixed_old}}
\pysigstartsignatures
\pysigline{\sphinxbfcode{\sphinxupquote{property\DUrole{w}{  }}}\sphinxbfcode{\sphinxupquote{fix\_dummy\_fixed\_old}}}
\pysigstopsignatures
\sphinxAtStartPar
returns names of actiove exogenizing dummies

\sphinxAtStartPar
sets the property self. exodummy\_per which defines the time over which the dummies are defined

\end{fulllineitems}

\index{fix\_add\_factor\_fixed (modelclass.WB\_Mixin property)@\spxentry{fix\_add\_factor\_fixed}\spxextra{modelclass.WB\_Mixin property}}

\begin{fulllineitems}
\phantomsection\label{\detokenize{index:modelclass.WB_Mixin.fix_add_factor_fixed}}
\pysigstartsignatures
\pysigline{\sphinxbfcode{\sphinxupquote{property\DUrole{w}{  }}}\sphinxbfcode{\sphinxupquote{fix\_add\_factor\_fixed}}}
\pysigstopsignatures
\sphinxAtStartPar
Returns the add factors corrosponding to the active exogenizing dummies

\end{fulllineitems}

\index{fix\_value\_fixed (modelclass.WB\_Mixin property)@\spxentry{fix\_value\_fixed}\spxextra{modelclass.WB\_Mixin property}}

\begin{fulllineitems}
\phantomsection\label{\detokenize{index:modelclass.WB_Mixin.fix_value_fixed}}
\pysigstartsignatures
\pysigline{\sphinxbfcode{\sphinxupquote{property\DUrole{w}{  }}}\sphinxbfcode{\sphinxupquote{fix\_value\_fixed}}}
\pysigstopsignatures
\sphinxAtStartPar
Returns the exogenizing values corrosponding to the active exogenizing dummies

\end{fulllineitems}

\index{fix\_endo\_fixed (modelclass.WB\_Mixin property)@\spxentry{fix\_endo\_fixed}\spxextra{modelclass.WB\_Mixin property}}

\begin{fulllineitems}
\phantomsection\label{\detokenize{index:modelclass.WB_Mixin.fix_endo_fixed}}
\pysigstartsignatures
\pysigline{\sphinxbfcode{\sphinxupquote{property\DUrole{w}{  }}}\sphinxbfcode{\sphinxupquote{fix\_endo\_fixed}}}
\pysigstopsignatures
\sphinxAtStartPar
Returns the endogeneous variables corrosponding to the active exogenizing dummies

\end{fulllineitems}

\index{fix\_inf() (modelclass.WB\_Mixin method)@\spxentry{fix\_inf()}\spxextra{modelclass.WB\_Mixin method}}

\begin{fulllineitems}
\phantomsection\label{\detokenize{index:modelclass.WB_Mixin.fix_inf}}
\pysigstartsignatures
\pysiglinewithargsret{\sphinxbfcode{\sphinxupquote{fix\_inf}}}{\emph{\DUrole{n}{df}\DUrole{o}{=}\DUrole{default_value}{None}}}{}
\pysigstopsignatures
\sphinxAtStartPar
Display information regarding exogenizing

\end{fulllineitems}


\end{fulllineitems}

\index{model (class in modelclass)@\spxentry{model}\spxextra{class in modelclass}}

\begin{fulllineitems}
\phantomsection\label{\detokenize{index:modelclass.model}}
\pysigstartsignatures
\pysiglinewithargsret{\sphinxbfcode{\sphinxupquote{class\DUrole{w}{  }}}\sphinxcode{\sphinxupquote{modelclass.}}\sphinxbfcode{\sphinxupquote{model}}}{\emph{\DUrole{n}{i\_eq}\DUrole{o}{=}\DUrole{default_value}{\textquotesingle{}\textquotesingle{}}}, \emph{\DUrole{n}{modelname}\DUrole{o}{=}\DUrole{default_value}{\textquotesingle{}testmodel\textquotesingle{}}}, \emph{\DUrole{n}{silent}\DUrole{o}{=}\DUrole{default_value}{False}}, \emph{\DUrole{n}{straight}\DUrole{o}{=}\DUrole{default_value}{False}}, \emph{\DUrole{n}{funks}\DUrole{o}{=}\DUrole{default_value}{{[}{]}}}, \emph{\DUrole{n}{tabcomplete}\DUrole{o}{=}\DUrole{default_value}{True}}, \emph{\DUrole{n}{previousbase}\DUrole{o}{=}\DUrole{default_value}{False}}, \emph{\DUrole{n}{use\_preorder}\DUrole{o}{=}\DUrole{default_value}{True}}, \emph{\DUrole{n}{normalized}\DUrole{o}{=}\DUrole{default_value}{True}}, \emph{\DUrole{n}{safeorder}\DUrole{o}{=}\DUrole{default_value}{False}}, \emph{\DUrole{n}{var\_description}\DUrole{o}{=}\DUrole{default_value}{\{\}}}, \emph{\DUrole{o}{**}\DUrole{n}{kwargs}}}{}
\pysigstopsignatures
\sphinxAtStartPar
This is the main model definition

\end{fulllineitems}

\index{upd (class in modelclass)@\spxentry{upd}\spxextra{class in modelclass}}

\begin{fulllineitems}
\phantomsection\label{\detokenize{index:modelclass.upd}}
\pysigstartsignatures
\pysiglinewithargsret{\sphinxbfcode{\sphinxupquote{class\DUrole{w}{  }}}\sphinxcode{\sphinxupquote{modelclass.}}\sphinxbfcode{\sphinxupquote{upd}}}{\emph{\DUrole{n}{pandas\_obj}}}{}
\pysigstopsignatures
\sphinxAtStartPar
Extend a dataframe to update variables from string

\sphinxAtStartPar
look at {\hyperref[\detokenize{index:modelclass.Model_help_Mixin.update}]{\sphinxcrossref{\sphinxcode{\sphinxupquote{update}}}}}  for syntax
\index{\_\_call\_\_() (modelclass.upd method)@\spxentry{\_\_call\_\_()}\spxextra{modelclass.upd method}}

\begin{fulllineitems}
\phantomsection\label{\detokenize{index:modelclass.upd.__call__}}
\pysigstartsignatures
\pysiglinewithargsret{\sphinxbfcode{\sphinxupquote{\_\_call\_\_}}}{\emph{\DUrole{n}{updates}}, \emph{\DUrole{n}{lprint}\DUrole{o}{=}\DUrole{default_value}{False}}, \emph{\DUrole{n}{scale}\DUrole{o}{=}\DUrole{default_value}{1.0}}, \emph{\DUrole{n}{create}\DUrole{o}{=}\DUrole{default_value}{True}}, \emph{\DUrole{n}{keep\_growth}\DUrole{o}{=}\DUrole{default_value}{False}}}{}
\pysigstopsignatures
\sphinxAtStartPar
Call self as a function.

\end{fulllineitems}


\end{fulllineitems}

\index{create\_model() (in module modelclass)@\spxentry{create\_model()}\spxextra{in module modelclass}}

\begin{fulllineitems}
\phantomsection\label{\detokenize{index:modelclass.create_model}}
\pysigstartsignatures
\pysiglinewithargsret{\sphinxcode{\sphinxupquote{modelclass.}}\sphinxbfcode{\sphinxupquote{create\_model}}}{\emph{\DUrole{n}{navn}}, \emph{\DUrole{n}{hist=0}}, \emph{\DUrole{n}{name=\textquotesingle{}\textquotesingle{}}}, \emph{\DUrole{n}{new=True}}, \emph{\DUrole{n}{finished=False}}, \emph{\DUrole{n}{xmodel=\textless{}class \textquotesingle{}modelclass.model\textquotesingle{}\textgreater{}}}, \emph{\DUrole{n}{straight=False}}, \emph{\DUrole{n}{funks={[}{]}}}}{}
\pysigstopsignatures
\sphinxAtStartPar
Creates either a model instance or a model and a historic model from formulars.

\sphinxAtStartPar
The formulars can be in a string or in af file withe the extension .txt

\sphinxAtStartPar
if:
\begin{quote}\begin{description}
\item[{Navn}] \leavevmode
\sphinxAtStartPar
The model as text or as a file with extension .txt

\item[{Name}] \leavevmode
\sphinxAtStartPar
Name of the model

\item[{New}] \leavevmode
\sphinxAtStartPar
If True, ! used for comments, else () can also be used. False should be avoided, only for old PCIM models.

\item[{Hist}] \leavevmode
\sphinxAtStartPar
If True, a model with calculations of historic value is also created

\item[{Xmodel}] \leavevmode
\sphinxAtStartPar
The model class used for creating model the model instance. Can be used to create models with model subclasses

\item[{Finished}] \leavevmode
\sphinxAtStartPar
If True, the model exploder is not used.

\item[{Straight}] \leavevmode
\sphinxAtStartPar
If True, the formula sequence in the model will be used.

\item[{Funks}] \leavevmode
\sphinxAtStartPar
A list of user defined funktions used in the model

\end{description}\end{quote}

\end{fulllineitems}

\index{get\_a\_value() (in module modelclass)@\spxentry{get\_a\_value()}\spxextra{in module modelclass}}

\begin{fulllineitems}
\phantomsection\label{\detokenize{index:modelclass.get_a_value}}
\pysigstartsignatures
\pysiglinewithargsret{\sphinxcode{\sphinxupquote{modelclass.}}\sphinxbfcode{\sphinxupquote{get\_a\_value}}}{\emph{\DUrole{n}{df}}, \emph{\DUrole{n}{per}}, \emph{\DUrole{n}{var}}, \emph{\DUrole{n}{lag}\DUrole{o}{=}\DUrole{default_value}{0}}}{}
\pysigstopsignatures
\sphinxAtStartPar
returns a value for row=p+lag, column = var

\sphinxAtStartPar
to take care of non additive row index

\end{fulllineitems}

\index{set\_a\_value() (in module modelclass)@\spxentry{set\_a\_value()}\spxextra{in module modelclass}}

\begin{fulllineitems}
\phantomsection\label{\detokenize{index:modelclass.set_a_value}}
\pysigstartsignatures
\pysiglinewithargsret{\sphinxcode{\sphinxupquote{modelclass.}}\sphinxbfcode{\sphinxupquote{set\_a\_value}}}{\emph{\DUrole{n}{df}}, \emph{\DUrole{n}{per}}, \emph{\DUrole{n}{var}}, \emph{\DUrole{n}{lag}\DUrole{o}{=}\DUrole{default_value}{0}}, \emph{\DUrole{n}{value}\DUrole{o}{=}\DUrole{default_value}{nan}}}{}
\pysigstopsignatures
\sphinxAtStartPar
Sets a value for row=p+lag, column = var

\sphinxAtStartPar
to take care of non additive row index

\end{fulllineitems}

\index{insertModelVar() (in module modelclass)@\spxentry{insertModelVar()}\spxextra{in module modelclass}}

\begin{fulllineitems}
\phantomsection\label{\detokenize{index:modelclass.insertModelVar}}
\pysigstartsignatures
\pysiglinewithargsret{\sphinxcode{\sphinxupquote{modelclass.}}\sphinxbfcode{\sphinxupquote{insertModelVar}}}{\emph{\DUrole{n}{dataframe}}, \emph{\DUrole{n}{model}\DUrole{o}{=}\DUrole{default_value}{None}}}{}
\pysigstopsignatures
\sphinxAtStartPar
Inserts all variables from model, not already in the dataframe.
Model can be a list of models

\end{fulllineitems}

\index{lineout() (in module modelclass)@\spxentry{lineout()}\spxextra{in module modelclass}}

\begin{fulllineitems}
\phantomsection\label{\detokenize{index:modelclass.lineout}}
\pysigstartsignatures
\pysiglinewithargsret{\sphinxcode{\sphinxupquote{modelclass.}}\sphinxbfcode{\sphinxupquote{lineout}}}{\emph{\DUrole{n}{vek}}, \emph{\DUrole{n}{pre}\DUrole{o}{=}\DUrole{default_value}{\textquotesingle{}\textquotesingle{}}}, \emph{\DUrole{n}{w}\DUrole{o}{=}\DUrole{default_value}{20}}, \emph{\DUrole{n}{d}\DUrole{o}{=}\DUrole{default_value}{0}}, \emph{\DUrole{n}{pw}\DUrole{o}{=}\DUrole{default_value}{20}}, \emph{\DUrole{n}{endline}\DUrole{o}{=}\DUrole{default_value}{\textquotesingle{}\textbackslash{}n\textquotesingle{}}}}{}
\pysigstopsignatures
\sphinxAtStartPar
Utility to return formated string of vector

\end{fulllineitems}

\index{upddfold() (in module modelclass)@\spxentry{upddfold()}\spxextra{in module modelclass}}

\begin{fulllineitems}
\phantomsection\label{\detokenize{index:modelclass.upddfold}}
\pysigstartsignatures
\pysiglinewithargsret{\sphinxcode{\sphinxupquote{modelclass.}}\sphinxbfcode{\sphinxupquote{upddfold}}}{\emph{\DUrole{n}{base}}, \emph{\DUrole{n}{upd}}}{}
\pysigstopsignatures
\sphinxAtStartPar
takes two dataframes. The values from upd is inserted into base

\end{fulllineitems}

\index{upddf() (in module modelclass)@\spxentry{upddf()}\spxextra{in module modelclass}}

\begin{fulllineitems}
\phantomsection\label{\detokenize{index:modelclass.upddf}}
\pysigstartsignatures
\pysiglinewithargsret{\sphinxcode{\sphinxupquote{modelclass.}}\sphinxbfcode{\sphinxupquote{upddf}}}{\emph{\DUrole{n}{base}}, \emph{\DUrole{n}{upd}}}{}
\pysigstopsignatures
\sphinxAtStartPar
takes two dataframes. The values from upd is inserted into base

\end{fulllineitems}

\index{randomdf() (in module modelclass)@\spxentry{randomdf()}\spxextra{in module modelclass}}

\begin{fulllineitems}
\phantomsection\label{\detokenize{index:modelclass.randomdf}}
\pysigstartsignatures
\pysiglinewithargsret{\sphinxcode{\sphinxupquote{modelclass.}}\sphinxbfcode{\sphinxupquote{randomdf}}}{\emph{\DUrole{n}{df}}, \emph{\DUrole{n}{row}\DUrole{o}{=}\DUrole{default_value}{False}}, \emph{\DUrole{n}{col}\DUrole{o}{=}\DUrole{default_value}{False}}, \emph{\DUrole{n}{same}\DUrole{o}{=}\DUrole{default_value}{False}}, \emph{\DUrole{n}{ran}\DUrole{o}{=}\DUrole{default_value}{False}}, \emph{\DUrole{n}{cpre}\DUrole{o}{=}\DUrole{default_value}{\textquotesingle{}C\textquotesingle{}}}, \emph{\DUrole{n}{rpre}\DUrole{o}{=}\DUrole{default_value}{\textquotesingle{}R\textquotesingle{}}}}{}
\pysigstopsignatures
\sphinxAtStartPar
Randomize and rename the rows and columns of a dataframe, keep the values right:
\begin{quote}\begin{description}
\item[{Ran}] \leavevmode
\sphinxAtStartPar
If True randomize, if False don’t randomize

\item[{Col}] \leavevmode
\sphinxAtStartPar
The columns are renamed and randdomized

\item[{Row}] \leavevmode
\sphinxAtStartPar
The rows are renamed and randdomized

\item[{Same}] \leavevmode
\sphinxAtStartPar
The row and column index are renamed and randdomized the same way

\item[{Cpre}] \leavevmode
\sphinxAtStartPar
Column name prefix

\item[{Rpre}] \leavevmode
\sphinxAtStartPar
Row name prefix

\end{description}\end{quote}

\end{fulllineitems}

\index{join\_name\_lag() (in module modelclass)@\spxentry{join\_name\_lag()}\spxextra{in module modelclass}}

\begin{fulllineitems}
\phantomsection\label{\detokenize{index:modelclass.join_name_lag}}
\pysigstartsignatures
\pysiglinewithargsret{\sphinxcode{\sphinxupquote{modelclass.}}\sphinxbfcode{\sphinxupquote{join\_name\_lag}}}{\emph{\DUrole{n}{df}}}{}
\pysigstopsignatures
\sphinxAtStartPar
creates a new dataframe where  the name and lag from multiindex is joined
as input a dataframe where name and lag are two levels in multiindex

\end{fulllineitems}

\index{timer\_old() (in module modelclass)@\spxentry{timer\_old()}\spxextra{in module modelclass}}

\begin{fulllineitems}
\phantomsection\label{\detokenize{index:modelclass.timer_old}}
\pysigstartsignatures
\pysiglinewithargsret{\sphinxcode{\sphinxupquote{modelclass.}}\sphinxbfcode{\sphinxupquote{timer\_old}}}{\emph{\DUrole{n}{input}\DUrole{o}{=}\DUrole{default_value}{\textquotesingle{}test\textquotesingle{}}}, \emph{\DUrole{n}{show}\DUrole{o}{=}\DUrole{default_value}{True}}, \emph{\DUrole{n}{short}\DUrole{o}{=}\DUrole{default_value}{False}}}{}
\pysigstopsignatures
\sphinxAtStartPar
does not catch exceptrions use model.timer

\sphinxAtStartPar
A timer context manager, implemented using a
generator function. This one will report time even if an exception occurs”””
\begin{quote}\begin{description}
\item[{Parameters}] \leavevmode\begin{itemize}
\item {} 
\sphinxAtStartPar
\sphinxstyleliteralstrong{\sphinxupquote{input}} (\sphinxstyleliteralemphasis{\sphinxupquote{string}}\sphinxstyleliteralemphasis{\sphinxupquote{, }}\sphinxstyleliteralemphasis{\sphinxupquote{optional}}) \textendash{} a name. The default is ‘test’.

\item {} 
\sphinxAtStartPar
\sphinxstyleliteralstrong{\sphinxupquote{show}} (\sphinxstyleliteralemphasis{\sphinxupquote{bool}}\sphinxstyleliteralemphasis{\sphinxupquote{, }}\sphinxstyleliteralemphasis{\sphinxupquote{optional}}) \textendash{} show the results. The default is True.

\item {} 
\sphinxAtStartPar
\sphinxstyleliteralstrong{\sphinxupquote{short}} (\sphinxstyleliteralemphasis{\sphinxupquote{bool}}\sphinxstyleliteralemphasis{\sphinxupquote{, }}\sphinxstyleliteralemphasis{\sphinxupquote{optional}}) \textendash{} . The default is False.

\end{itemize}

\item[{Return type}] \leavevmode
\sphinxAtStartPar
None.

\end{description}\end{quote}

\end{fulllineitems}



\section{Modelpattern, regular expressions to parse equations and models}
\label{\detokenize{index:module-modelpattern}}\label{\detokenize{index:modelpattern-regular-expressions-to-parse-equations-and-models}}\index{module@\spxentry{module}!modelpattern@\spxentry{modelpattern}}\index{modelpattern@\spxentry{modelpattern}!module@\spxentry{module}}
\sphinxAtStartPar
Created on Mon Sep 02 19:32:22 2013

\sphinxAtStartPar
This module defines a number of pattern used in PYFS.
If a new function is intruduced in the model definition language it should added to the
function names in funkname

\sphinxAtStartPar
All functions in the module modeluserfunk will be added to the language and incorporated in the Business
Logic language

\sphinxAtStartPar
@author: Ib
\index{nterm (class in modelpattern)@\spxentry{nterm}\spxextra{class in modelpattern}}

\begin{fulllineitems}
\phantomsection\label{\detokenize{index:modelpattern.nterm}}
\pysigstartsignatures
\pysiglinewithargsret{\sphinxbfcode{\sphinxupquote{class\DUrole{w}{  }}}\sphinxcode{\sphinxupquote{modelpattern.}}\sphinxbfcode{\sphinxupquote{nterm}}}{\emph{\DUrole{n}{number}}, \emph{\DUrole{n}{op}}, \emph{\DUrole{n}{var}}, \emph{\DUrole{n}{lag}}}{}
\pysigstopsignatures\index{lag (modelpattern.nterm attribute)@\spxentry{lag}\spxextra{modelpattern.nterm attribute}}

\begin{fulllineitems}
\phantomsection\label{\detokenize{index:modelpattern.nterm.lag}}
\pysigstartsignatures
\pysigline{\sphinxbfcode{\sphinxupquote{lag}}}
\pysigstopsignatures
\sphinxAtStartPar
Alias for field number 3

\end{fulllineitems}

\index{number (modelpattern.nterm attribute)@\spxentry{number}\spxextra{modelpattern.nterm attribute}}

\begin{fulllineitems}
\phantomsection\label{\detokenize{index:modelpattern.nterm.number}}
\pysigstartsignatures
\pysigline{\sphinxbfcode{\sphinxupquote{number}}}
\pysigstopsignatures
\sphinxAtStartPar
Alias for field number 0

\end{fulllineitems}

\index{op (modelpattern.nterm attribute)@\spxentry{op}\spxextra{modelpattern.nterm attribute}}

\begin{fulllineitems}
\phantomsection\label{\detokenize{index:modelpattern.nterm.op}}
\pysigstartsignatures
\pysigline{\sphinxbfcode{\sphinxupquote{op}}}
\pysigstopsignatures
\sphinxAtStartPar
Alias for field number 1

\end{fulllineitems}

\index{var (modelpattern.nterm attribute)@\spxentry{var}\spxextra{modelpattern.nterm attribute}}

\begin{fulllineitems}
\phantomsection\label{\detokenize{index:modelpattern.nterm.var}}
\pysigstartsignatures
\pysigline{\sphinxbfcode{\sphinxupquote{var}}}
\pysigstopsignatures
\sphinxAtStartPar
Alias for field number 2

\end{fulllineitems}


\end{fulllineitems}

\index{find\_frml() (in module modelpattern)@\spxentry{find\_frml()}\spxextra{in module modelpattern}}

\begin{fulllineitems}
\phantomsection\label{\detokenize{index:modelpattern.find_frml}}
\pysigstartsignatures
\pysiglinewithargsret{\sphinxcode{\sphinxupquote{modelpattern.}}\sphinxbfcode{\sphinxupquote{find\_frml}}}{\emph{\DUrole{n}{equations}}}{}
\pysigstopsignatures
\sphinxAtStartPar
Takes at modeltext and returns a list with where each element is
a string starting  with FRML and ending with \$
It do not check if it is a valid FRML statement

\end{fulllineitems}

\index{split\_frml() (in module modelpattern)@\spxentry{split\_frml()}\spxextra{in module modelpattern}}

\begin{fulllineitems}
\phantomsection\label{\detokenize{index:modelpattern.split_frml}}
\pysigstartsignatures
\pysiglinewithargsret{\sphinxcode{\sphinxupquote{modelpattern.}}\sphinxbfcode{\sphinxupquote{split\_frml}}}{\emph{\DUrole{n}{frml}}}{}
\pysigstopsignatures
\sphinxAtStartPar
Splits a string with a frml into a tuple with 4 parts:
\begin{enumerate}
\sphinxsetlistlabels{\arabic}{enumi}{enumii}{}{.}%
\setcounter{enumi}{-1}
\item {} 
\sphinxAtStartPar
The unsplit frml statement

\item {} 
\sphinxAtStartPar
FRML

\item {} 
\sphinxAtStartPar
\textless{}Frml name\textgreater{}

\item {} 
\sphinxAtStartPar
\textless{}the frml expression\textgreater{}

\end{enumerate}

\end{fulllineitems}

\index{find\_statements() (in module modelpattern)@\spxentry{find\_statements()}\spxextra{in module modelpattern}}

\begin{fulllineitems}
\phantomsection\label{\detokenize{index:modelpattern.find_statements}}
\pysigstartsignatures
\pysiglinewithargsret{\sphinxcode{\sphinxupquote{modelpattern.}}\sphinxbfcode{\sphinxupquote{find\_statements}}}{\emph{\DUrole{n}{a\_model}}}{}
\pysigstopsignatures
\sphinxAtStartPar
splits a modeltest into comments and statements
\begin{itemize}
\item {} 
\sphinxAtStartPar
a \sphinxstyleemphasis{comment} starts with ! and ends at lineend

\item {} 
\sphinxAtStartPar
a \sphinxstyleemphasis{statement} starts with a name and ends with a \$ all characters between are considerd part of the statement

\end{itemize}

\sphinxAtStartPar
The statement is not chekked for meaningfulness
returns a list of tuppels (comment,command,\textless{}rest of statement\textgreater{})

\end{fulllineitems}

\index{model\_parse\_old() (in module modelpattern)@\spxentry{model\_parse\_old()}\spxextra{in module modelpattern}}

\begin{fulllineitems}
\phantomsection\label{\detokenize{index:modelpattern.model_parse_old}}
\pysigstartsignatures
\pysiglinewithargsret{\sphinxcode{\sphinxupquote{modelpattern.}}\sphinxbfcode{\sphinxupquote{model\_parse\_old}}}{\emph{\DUrole{n}{equations}}, \emph{\DUrole{n}{funks}\DUrole{o}{=}\DUrole{default_value}{{[}{]}}}}{}
\pysigstopsignatures\begin{description}
\item[{Takes a model returns a list of tupels. Each tupel contains:}] \leavevmode\begin{quote}\begin{description}
\item[{the compleete formular}] \leavevmode
\item[{FRML}] \leavevmode
\item[{formular name}] \leavevmode
\item[{the expression}] \leavevmode
\item[{list of terms from the expression}] \leavevmode
\end{description}\end{quote}

\end{description}

\sphinxAtStartPar
The purpose of this function is to make model analysis faster. this is 20 times faster than looping over espressions in a model

\end{fulllineitems}

\index{model\_parse() (in module modelpattern)@\spxentry{model\_parse()}\spxextra{in module modelpattern}}

\begin{fulllineitems}
\phantomsection\label{\detokenize{index:modelpattern.model_parse}}
\pysigstartsignatures
\pysiglinewithargsret{\sphinxcode{\sphinxupquote{modelpattern.}}\sphinxbfcode{\sphinxupquote{model\_parse}}}{\emph{\DUrole{n}{equations}}, \emph{\DUrole{n}{funks}\DUrole{o}{=}\DUrole{default_value}{{[}{]}}}}{}
\pysigstopsignatures\begin{description}
\item[{Takes a model returns a list of tupels. Each tupel contains:}] \leavevmode\begin{quote}\begin{description}
\item[{the compleete formular}] \leavevmode
\item[{FRML}] \leavevmode
\item[{formular name}] \leavevmode
\item[{the expression}] \leavevmode
\item[{list of terms from the expression}] \leavevmode
\end{description}\end{quote}

\end{description}

\sphinxAtStartPar
The purpose of this function is to make model analysis faster. this is 20 times faster than looping over espressions in a model

\sphinxAtStartPar
This new model\_parse handels lags of \sphinxhyphen{}0 or +0 which ocours in some models from world bank.

\end{fulllineitems}

\index{list\_extract() (in module modelpattern)@\spxentry{list\_extract()}\spxextra{in module modelpattern}}

\begin{fulllineitems}
\phantomsection\label{\detokenize{index:modelpattern.list_extract}}
\pysigstartsignatures
\pysiglinewithargsret{\sphinxcode{\sphinxupquote{modelpattern.}}\sphinxbfcode{\sphinxupquote{list\_extract}}}{\emph{\DUrole{n}{equations}}, \emph{\DUrole{n}{silent}\DUrole{o}{=}\DUrole{default_value}{True}}}{}
\pysigstopsignatures
\sphinxAtStartPar
creates lists used in a model

\sphinxAtStartPar
returns a dictonary with the lists
if a list is defined several times, the first definition is used

\end{fulllineitems}

\index{check\_syntax\_model() (in module modelpattern)@\spxentry{check\_syntax\_model()}\spxextra{in module modelpattern}}

\begin{fulllineitems}
\phantomsection\label{\detokenize{index:modelpattern.check_syntax_model}}
\pysigstartsignatures
\pysiglinewithargsret{\sphinxcode{\sphinxupquote{modelpattern.}}\sphinxbfcode{\sphinxupquote{check\_syntax\_model}}}{\emph{\DUrole{n}{equations}}, \emph{\DUrole{n}{test}\DUrole{o}{=}\DUrole{default_value}{True}}}{}
\pysigstopsignatures
\sphinxAtStartPar
cheks if equations have syntax errors by calling the python compile.parse

\end{fulllineitems}

\index{udtryk\_parse() (in module modelpattern)@\spxentry{udtryk\_parse()}\spxextra{in module modelpattern}}

\begin{fulllineitems}
\phantomsection\label{\detokenize{index:modelpattern.udtryk_parse}}
\pysigstartsignatures
\pysiglinewithargsret{\sphinxcode{\sphinxupquote{modelpattern.}}\sphinxbfcode{\sphinxupquote{udtryk\_parse}}}{\emph{\DUrole{n}{udtryk}}, \emph{\DUrole{n}{funks}\DUrole{o}{=}\DUrole{default_value}{{[}{]}}}}{}
\pysigstopsignatures
\sphinxAtStartPar
returns a list of terms from an expression ie: lhs=rhs \$
or just an expression like x+b

\end{fulllineitems}

\index{kw\_frml\_name() (in module modelpattern)@\spxentry{kw\_frml\_name()}\spxextra{in module modelpattern}}

\begin{fulllineitems}
\phantomsection\label{\detokenize{index:modelpattern.kw_frml_name}}
\pysigstartsignatures
\pysiglinewithargsret{\sphinxcode{\sphinxupquote{modelpattern.}}\sphinxbfcode{\sphinxupquote{kw\_frml\_name}}}{\emph{\DUrole{n}{frml\_name0}}, \emph{\DUrole{n}{kw}}, \emph{\DUrole{n}{default}\DUrole{o}{=}\DUrole{default_value}{None}}}{}
\pysigstopsignatures
\sphinxAtStartPar
find keywords and associated value from string ‘\textless{}kw=xxx,res=kdkdk\textgreater{}’

\end{fulllineitems}



\section{Modelnewton, Newton solution and calculate derivatives}
\label{\detokenize{index:module-modelnewton}}\label{\detokenize{index:modelnewton-newton-solution-and-calculate-derivatives}}\index{module@\spxentry{module}!modelnewton@\spxentry{modelnewton}}\index{modelnewton@\spxentry{modelnewton}!module@\spxentry{module}}
\sphinxAtStartPar
Created on Fri Jun 19 19:49:50 2020

\sphinxAtStartPar
@author: IBH

\sphinxAtStartPar
Module which handles model differentiation, construction of jacobi matrizex,
and creates dense and sparse solving functions.
\index{diff\_value\_base (class in modelnewton)@\spxentry{diff\_value\_base}\spxextra{class in modelnewton}}

\begin{fulllineitems}
\phantomsection\label{\detokenize{index:modelnewton.diff_value_base}}
\pysigstartsignatures
\pysiglinewithargsret{\sphinxbfcode{\sphinxupquote{class\DUrole{w}{  }}}\sphinxcode{\sphinxupquote{modelnewton.}}\sphinxbfcode{\sphinxupquote{diff\_value\_base}}}{\emph{\DUrole{n}{var}\DUrole{p}{:}\DUrole{w}{  }\DUrole{n}{str}}, \emph{\DUrole{n}{pvar}\DUrole{p}{:}\DUrole{w}{  }\DUrole{n}{str}}, \emph{\DUrole{n}{lag}\DUrole{p}{:}\DUrole{w}{  }\DUrole{n}{int}}, \emph{\DUrole{n}{var\_plac}\DUrole{p}{:}\DUrole{w}{  }\DUrole{n}{int}}, \emph{\DUrole{n}{pvar\_plac}\DUrole{p}{:}\DUrole{w}{  }\DUrole{n}{int}}, \emph{\DUrole{n}{pvar\_endo}\DUrole{p}{:}\DUrole{w}{  }\DUrole{n}{bool}}, \emph{\DUrole{n}{pvar\_exo\_plac}\DUrole{p}{:}\DUrole{w}{  }\DUrole{n}{int}}}{}
\pysigstopsignatures
\sphinxAtStartPar
class define columns in database with values from differentiation

\end{fulllineitems}

\index{diff\_value\_col (class in modelnewton)@\spxentry{diff\_value\_col}\spxextra{class in modelnewton}}

\begin{fulllineitems}
\phantomsection\label{\detokenize{index:modelnewton.diff_value_col}}
\pysigstartsignatures
\pysiglinewithargsret{\sphinxbfcode{\sphinxupquote{class\DUrole{w}{  }}}\sphinxcode{\sphinxupquote{modelnewton.}}\sphinxbfcode{\sphinxupquote{diff\_value\_col}}}{\emph{\DUrole{n}{var}\DUrole{p}{:}\DUrole{w}{  }\DUrole{n}{str}}, \emph{\DUrole{n}{pvar}\DUrole{p}{:}\DUrole{w}{  }\DUrole{n}{str}}, \emph{\DUrole{n}{lag}\DUrole{p}{:}\DUrole{w}{  }\DUrole{n}{int}}, \emph{\DUrole{n}{var\_plac}\DUrole{p}{:}\DUrole{w}{  }\DUrole{n}{int}}, \emph{\DUrole{n}{pvar\_plac}\DUrole{p}{:}\DUrole{w}{  }\DUrole{n}{int}}, \emph{\DUrole{n}{pvar\_endo}\DUrole{p}{:}\DUrole{w}{  }\DUrole{n}{bool}}, \emph{\DUrole{n}{pvar\_exo\_plac}\DUrole{p}{:}\DUrole{w}{  }\DUrole{n}{int}}}{}
\pysigstopsignatures
\sphinxAtStartPar
The hash able class which can be used as pandas columns

\end{fulllineitems}

\index{diff\_value (class in modelnewton)@\spxentry{diff\_value}\spxextra{class in modelnewton}}

\begin{fulllineitems}
\phantomsection\label{\detokenize{index:modelnewton.diff_value}}
\pysigstartsignatures
\pysiglinewithargsret{\sphinxbfcode{\sphinxupquote{class\DUrole{w}{  }}}\sphinxcode{\sphinxupquote{modelnewton.}}\sphinxbfcode{\sphinxupquote{diff\_value}}}{\emph{\DUrole{n}{var}\DUrole{p}{:}\DUrole{w}{  }\DUrole{n}{str}}, \emph{\DUrole{n}{pvar}\DUrole{p}{:}\DUrole{w}{  }\DUrole{n}{str}}, \emph{\DUrole{n}{lag}\DUrole{p}{:}\DUrole{w}{  }\DUrole{n}{int}}, \emph{\DUrole{n}{var\_plac}\DUrole{p}{:}\DUrole{w}{  }\DUrole{n}{int}}, \emph{\DUrole{n}{pvar\_plac}\DUrole{p}{:}\DUrole{w}{  }\DUrole{n}{int}}, \emph{\DUrole{n}{pvar\_endo}\DUrole{p}{:}\DUrole{w}{  }\DUrole{n}{bool}}, \emph{\DUrole{n}{pvar\_exo\_plac}\DUrole{p}{:}\DUrole{w}{  }\DUrole{n}{int}}, \emph{\DUrole{n}{number}\DUrole{p}{:}\DUrole{w}{  }\DUrole{n}{int}\DUrole{w}{  }\DUrole{o}{=}\DUrole{w}{  }\DUrole{default_value}{0}}, \emph{\DUrole{n}{date}\DUrole{p}{:}\DUrole{w}{  }\DUrole{n}{any}\DUrole{w}{  }\DUrole{o}{=}\DUrole{w}{  }\DUrole{default_value}{0}}}{}
\pysigstopsignatures
\sphinxAtStartPar
class to contain values from differentiation

\end{fulllineitems}

\index{newton\_diff (class in modelnewton)@\spxentry{newton\_diff}\spxextra{class in modelnewton}}

\begin{fulllineitems}
\phantomsection\label{\detokenize{index:modelnewton.newton_diff}}
\pysigstartsignatures
\pysiglinewithargsret{\sphinxbfcode{\sphinxupquote{class\DUrole{w}{  }}}\sphinxcode{\sphinxupquote{modelnewton.}}\sphinxbfcode{\sphinxupquote{newton\_diff}}}{\emph{\DUrole{n}{mmodel}}, \emph{\DUrole{n}{df}\DUrole{o}{=}\DUrole{default_value}{None}}, \emph{\DUrole{n}{endovar}\DUrole{o}{=}\DUrole{default_value}{None}}, \emph{\DUrole{n}{onlyendocur}\DUrole{o}{=}\DUrole{default_value}{False}}, \emph{\DUrole{n}{timeit}\DUrole{o}{=}\DUrole{default_value}{False}}, \emph{\DUrole{n}{silent}\DUrole{o}{=}\DUrole{default_value}{True}}, \emph{\DUrole{n}{forcenum}\DUrole{o}{=}\DUrole{default_value}{False}}, \emph{\DUrole{n}{per}\DUrole{o}{=}\DUrole{default_value}{\textquotesingle{}\textquotesingle{}}}, \emph{\DUrole{n}{ljit}\DUrole{o}{=}\DUrole{default_value}{0}}, \emph{\DUrole{n}{nchunk}\DUrole{o}{=}\DUrole{default_value}{None}}, \emph{\DUrole{n}{endoandexo}\DUrole{o}{=}\DUrole{default_value}{False}}}{}
\pysigstopsignatures
\sphinxAtStartPar
Class to handle newron solving
this is for un\sphinxhyphen{}nomalized or normalized models ie models of the forrm

\sphinxAtStartPar
0 = G(y,x)
y = F(y,x)
\index{modeldiff() (modelnewton.newton\_diff method)@\spxentry{modeldiff()}\spxextra{modelnewton.newton\_diff method}}

\begin{fulllineitems}
\phantomsection\label{\detokenize{index:modelnewton.newton_diff.modeldiff}}
\pysigstartsignatures
\pysiglinewithargsret{\sphinxbfcode{\sphinxupquote{modeldiff}}}{}{}
\pysigstopsignatures
\sphinxAtStartPar
Differentiate relations for self.enovar with respect to endogeneous variable
The result is placed in a dictory in the model instanse: model.diffendocur

\end{fulllineitems}

\index{show\_diff() (modelnewton.newton\_diff method)@\spxentry{show\_diff()}\spxextra{modelnewton.newton\_diff method}}

\begin{fulllineitems}
\phantomsection\label{\detokenize{index:modelnewton.newton_diff.show_diff}}
\pysigstartsignatures
\pysiglinewithargsret{\sphinxbfcode{\sphinxupquote{show\_diff}}}{\emph{\DUrole{n}{pat}\DUrole{o}{=}\DUrole{default_value}{\textquotesingle{}*\textquotesingle{}}}}{}
\pysigstopsignatures
\sphinxAtStartPar
Displays espressions for differential koifficients for a variable
if var ends with * all matchning variables are displayes

\end{fulllineitems}

\index{show\_stacked\_diff() (modelnewton.newton\_diff method)@\spxentry{show\_stacked\_diff()}\spxextra{modelnewton.newton\_diff method}}

\begin{fulllineitems}
\phantomsection\label{\detokenize{index:modelnewton.newton_diff.show_stacked_diff}}
\pysigstartsignatures
\pysiglinewithargsret{\sphinxbfcode{\sphinxupquote{show\_stacked\_diff}}}{\emph{\DUrole{n}{time}\DUrole{o}{=}\DUrole{default_value}{None}}, \emph{\DUrole{n}{lhs}\DUrole{o}{=}\DUrole{default_value}{\textquotesingle{}\textquotesingle{}}}, \emph{\DUrole{n}{rhs}\DUrole{o}{=}\DUrole{default_value}{\textquotesingle{}\textquotesingle{}}}, \emph{\DUrole{n}{dec}\DUrole{o}{=}\DUrole{default_value}{2}}, \emph{\DUrole{n}{show}\DUrole{o}{=}\DUrole{default_value}{True}}}{}
\pysigstopsignatures\begin{quote}\begin{description}
\item[{Parameters}] \leavevmode\begin{itemize}
\item {} 
\sphinxAtStartPar
\sphinxstyleliteralstrong{\sphinxupquote{time}} (\sphinxstyleliteralemphasis{\sphinxupquote{list}}\sphinxstyleliteralemphasis{\sphinxupquote{, }}\sphinxstyleliteralemphasis{\sphinxupquote{optional}}) \textendash{} DESCRIPTION. The default is None. Time for which to retrieve stacked jacobi

\item {} 
\sphinxAtStartPar
\sphinxstyleliteralstrong{\sphinxupquote{lhs}} (\sphinxstyleliteralemphasis{\sphinxupquote{string}}\sphinxstyleliteralemphasis{\sphinxupquote{, }}\sphinxstyleliteralemphasis{\sphinxupquote{optional}}) \textendash{} DESCRIPTION. The default is ‘’. Left hand side variables

\item {} 
\sphinxAtStartPar
\sphinxstyleliteralstrong{\sphinxupquote{rhs}} (\sphinxstyleliteralemphasis{\sphinxupquote{TYPE}}\sphinxstyleliteralemphasis{\sphinxupquote{, }}\sphinxstyleliteralemphasis{\sphinxupquote{optional}}) \textendash{} DESCRIPTION. The default is ‘’. Right hand side variabnles

\item {} 
\sphinxAtStartPar
\sphinxstyleliteralstrong{\sphinxupquote{dec}} (\sphinxstyleliteralemphasis{\sphinxupquote{TYPE}}\sphinxstyleliteralemphasis{\sphinxupquote{, }}\sphinxstyleliteralemphasis{\sphinxupquote{optional}}) \textendash{} DESCRIPTION. The default is 2.

\item {} 
\sphinxAtStartPar
\sphinxstyleliteralstrong{\sphinxupquote{show}} (\sphinxstyleliteralemphasis{\sphinxupquote{TYPE}}\sphinxstyleliteralemphasis{\sphinxupquote{, }}\sphinxstyleliteralemphasis{\sphinxupquote{optional}}) \textendash{} DESCRIPTION. The default is True.

\end{itemize}

\item[{Return type}] \leavevmode
\sphinxAtStartPar
selected rows and columns of stacked jacobi as dataframe .

\end{description}\end{quote}

\end{fulllineitems}

\index{get\_diffmodel() (modelnewton.newton\_diff method)@\spxentry{get\_diffmodel()}\spxextra{modelnewton.newton\_diff method}}

\begin{fulllineitems}
\phantomsection\label{\detokenize{index:modelnewton.newton_diff.get_diffmodel}}
\pysigstartsignatures
\pysiglinewithargsret{\sphinxbfcode{\sphinxupquote{get\_diffmodel}}}{}{}
\pysigstopsignatures
\sphinxAtStartPar
Returns a model which calculates the partial derivatives of a model

\end{fulllineitems}

\index{get\_diff\_melted() (modelnewton.newton\_diff method)@\spxentry{get\_diff\_melted()}\spxextra{modelnewton.newton\_diff method}}

\begin{fulllineitems}
\phantomsection\label{\detokenize{index:modelnewton.newton_diff.get_diff_melted}}
\pysigstartsignatures
\pysiglinewithargsret{\sphinxbfcode{\sphinxupquote{get\_diff\_melted}}}{\emph{\DUrole{n}{periode}\DUrole{o}{=}\DUrole{default_value}{None}}, \emph{\DUrole{n}{df}\DUrole{o}{=}\DUrole{default_value}{None}}}{}
\pysigstopsignatures
\sphinxAtStartPar
returns a tall matrix with all values to construct jacobimatrix(es)

\end{fulllineitems}

\index{get\_diff\_mat\_tot() (modelnewton.newton\_diff method)@\spxentry{get\_diff\_mat\_tot()}\spxextra{modelnewton.newton\_diff method}}

\begin{fulllineitems}
\phantomsection\label{\detokenize{index:modelnewton.newton_diff.get_diff_mat_tot}}
\pysigstartsignatures
\pysiglinewithargsret{\sphinxbfcode{\sphinxupquote{get\_diff\_mat\_tot}}}{\emph{\DUrole{n}{df}\DUrole{o}{=}\DUrole{default_value}{None}}}{}
\pysigstopsignatures
\sphinxAtStartPar
Fetch a stacked jacobimatrix for the whole model.current\_per

\sphinxAtStartPar
Returns a sparse matrix.

\end{fulllineitems}

\index{get\_diff\_mat\_1per() (modelnewton.newton\_diff method)@\spxentry{get\_diff\_mat\_1per()}\spxextra{modelnewton.newton\_diff method}}

\begin{fulllineitems}
\phantomsection\label{\detokenize{index:modelnewton.newton_diff.get_diff_mat_1per}}
\pysigstartsignatures
\pysiglinewithargsret{\sphinxbfcode{\sphinxupquote{get\_diff\_mat\_1per}}}{\emph{\DUrole{n}{periode}\DUrole{o}{=}\DUrole{default_value}{None}}, \emph{\DUrole{n}{df}\DUrole{o}{=}\DUrole{default_value}{None}}}{}
\pysigstopsignatures
\sphinxAtStartPar
fetch a dict of one periode sparse jacobimatrices

\end{fulllineitems}

\index{get\_diff\_melted\_var() (modelnewton.newton\_diff method)@\spxentry{get\_diff\_melted\_var()}\spxextra{modelnewton.newton\_diff method}}

\begin{fulllineitems}
\phantomsection\label{\detokenize{index:modelnewton.newton_diff.get_diff_melted_var}}
\pysigstartsignatures
\pysiglinewithargsret{\sphinxbfcode{\sphinxupquote{get\_diff\_melted\_var}}}{\emph{\DUrole{n}{periode}\DUrole{o}{=}\DUrole{default_value}{None}}, \emph{\DUrole{n}{df}\DUrole{o}{=}\DUrole{default_value}{None}}}{}
\pysigstopsignatures
\sphinxAtStartPar
makes dict with all  derivative matrices for all lags

\end{fulllineitems}

\index{get\_diff\_values\_all() (modelnewton.newton\_diff method)@\spxentry{get\_diff\_values\_all()}\spxextra{modelnewton.newton\_diff method}}

\begin{fulllineitems}
\phantomsection\label{\detokenize{index:modelnewton.newton_diff.get_diff_values_all}}
\pysigstartsignatures
\pysiglinewithargsret{\sphinxbfcode{\sphinxupquote{get\_diff\_values\_all}}}{\emph{\DUrole{n}{periode}\DUrole{o}{=}\DUrole{default_value}{None}}, \emph{\DUrole{n}{df}\DUrole{o}{=}\DUrole{default_value}{None}}, \emph{\DUrole{n}{asdf}\DUrole{o}{=}\DUrole{default_value}{False}}}{}
\pysigstopsignatures
\sphinxAtStartPar
stuff the values of derivatives into nested dic

\end{fulllineitems}

\index{get\_feedback() (modelnewton.newton\_diff static method)@\spxentry{get\_feedback()}\spxextra{modelnewton.newton\_diff static method}}

\begin{fulllineitems}
\phantomsection\label{\detokenize{index:modelnewton.newton_diff.get_feedback}}
\pysigstartsignatures
\pysiglinewithargsret{\sphinxbfcode{\sphinxupquote{static\DUrole{w}{  }}}\sphinxbfcode{\sphinxupquote{get\_feedback}}}{\emph{\DUrole{n}{eig\_dic}}, \emph{\DUrole{n}{per}\DUrole{o}{=}\DUrole{default_value}{None}}}{}
\pysigstopsignatures
\sphinxAtStartPar
Returns a dict of max abs eigenvector and the sign

\end{fulllineitems}


\end{fulllineitems}



\chapter{Processing model specification}
\label{\detokenize{index:processing-model-specification}}
\sphinxAtStartPar
The purpose is to process models specified in different ways
\sphinxhyphen{} Macro business language
\sphinxhyphen{} Eviews
\sphinxhyphen{} Excel
\sphinxhyphen{} Latex

\sphinxAtStartPar
and make them into the modelflows business Logic language.


\section{Modelmanipulation, Text processing of of models before they become model class instances}
\label{\detokenize{index:module-modelmanipulation}}\label{\detokenize{index:modelmanipulation-text-processing-of-of-models-before-they-become-model-class-instances}}\index{module@\spxentry{module}!modelmanipulation@\spxentry{modelmanipulation}}\index{modelmanipulation@\spxentry{modelmanipulation}!module@\spxentry{module}}
\sphinxAtStartPar
Created on Mon Sep 02 19:41:11 2013

\sphinxAtStartPar
This module is a textprocessing module which is used to transforms a \sphinxstyleemphasis{template model} for a generic bank into into a unrolled and
expande model which covers all banks \sphinxhyphen{} under  control of a list feature.

\sphinxAtStartPar
The resulting model can be solved after beeing proccesed in the \sphinxstyleemphasis{modelclass} module.

\sphinxAtStartPar
In addition to creating a model for forecasting, the module can also create a model which calculates residulas and and variables
in historic periods.  This model can also be solved by the \sphinxstyleemphasis{modelclass} module.

\sphinxAtStartPar
@author: Ib
\index{safesub (class in modelmanipulation)@\spxentry{safesub}\spxextra{class in modelmanipulation}}

\begin{fulllineitems}
\phantomsection\label{\detokenize{index:modelmanipulation.safesub}}
\pysigstartsignatures
\pysigline{\sphinxbfcode{\sphinxupquote{class\DUrole{w}{  }}}\sphinxcode{\sphinxupquote{modelmanipulation.}}\sphinxbfcode{\sphinxupquote{safesub}}}
\pysigstopsignatures
\sphinxAtStartPar
A subclass of dict.
if a \sphinxstyleemphasis{safesub} is indexed by a nonexisting keyword it just return the keyword
this alows missing keywords when substitution text inspired by Python cookbook

\end{fulllineitems}

\index{sub() (in module modelmanipulation)@\spxentry{sub()}\spxextra{in module modelmanipulation}}

\begin{fulllineitems}
\phantomsection\label{\detokenize{index:modelmanipulation.sub}}
\pysigstartsignatures
\pysiglinewithargsret{\sphinxcode{\sphinxupquote{modelmanipulation.}}\sphinxbfcode{\sphinxupquote{sub}}}{\emph{\DUrole{n}{text}}, \emph{\DUrole{n}{katalog}}}{}
\pysigstopsignatures
\sphinxAtStartPar
Substitutes keywords from dictionary by returning
text.format\_map(safesub(katalog))
Allows missing keywords by using safesub subclass

\end{fulllineitems}

\index{oldsub\_frml() (in module modelmanipulation)@\spxentry{oldsub\_frml()}\spxextra{in module modelmanipulation}}

\begin{fulllineitems}
\phantomsection\label{\detokenize{index:modelmanipulation.oldsub_frml}}
\pysigstartsignatures
\pysiglinewithargsret{\sphinxcode{\sphinxupquote{modelmanipulation.}}\sphinxbfcode{\sphinxupquote{oldsub\_frml}}}{\emph{\DUrole{n}{ibdic}}, \emph{\DUrole{n}{text}}, \emph{\DUrole{n}{plus}\DUrole{o}{=}\DUrole{default_value}{\textquotesingle{}\textquotesingle{}}}, \emph{\DUrole{n}{var}\DUrole{o}{=}\DUrole{default_value}{\textquotesingle{}\textquotesingle{}}}, \emph{\DUrole{n}{lig}\DUrole{o}{=}\DUrole{default_value}{\textquotesingle{}\textquotesingle{}}}, \emph{\DUrole{n}{sep}\DUrole{o}{=}\DUrole{default_value}{\textquotesingle{}\textbackslash{}n\textquotesingle{}}}}{}
\pysigstopsignatures
\sphinxAtStartPar
to repeat substitution from list
\begin{itemize}
\item {} 
\sphinxAtStartPar
\sphinxstyleemphasis{plus} is a seperator, used for creating sums

\item {} 
\sphinxAtStartPar
\sphinxstyleemphasis{var} and \sphinxstyleemphasis{lig} Determins for which items the substitution should take place
by var=abe, lig=’ko’ substitution is only performed for entries where var==’ko’

\end{itemize}

\end{fulllineitems}

\index{sub\_frml() (in module modelmanipulation)@\spxentry{sub\_frml()}\spxextra{in module modelmanipulation}}

\begin{fulllineitems}
\phantomsection\label{\detokenize{index:modelmanipulation.sub_frml}}
\pysigstartsignatures
\pysiglinewithargsret{\sphinxcode{\sphinxupquote{modelmanipulation.}}\sphinxbfcode{\sphinxupquote{sub\_frml}}}{\emph{\DUrole{n}{ibdic}}, \emph{\DUrole{n}{text}}, \emph{\DUrole{n}{plus}\DUrole{o}{=}\DUrole{default_value}{\textquotesingle{}\textquotesingle{}}}, \emph{\DUrole{n}{xvar}\DUrole{o}{=}\DUrole{default_value}{\textquotesingle{}\textquotesingle{}}}, \emph{\DUrole{n}{lig}\DUrole{o}{=}\DUrole{default_value}{\textquotesingle{}\textquotesingle{}}}, \emph{\DUrole{n}{sep}\DUrole{o}{=}\DUrole{default_value}{\textquotesingle{}\textbackslash{}n\textquotesingle{}}}}{}
\pysigstopsignatures
\sphinxAtStartPar
to repeat substitution from list
\begin{itemize}
\item {} 
\sphinxAtStartPar
\sphinxstyleemphasis{plus} is a seperator, used for creating sums

\item {} 
\sphinxAtStartPar
\sphinxstyleemphasis{xvar} and \sphinxstyleemphasis{lig} Determins for which items the substitution should take place
by var=abe, lig=’ko’ substitution is only performed for entries where var==’ko’

\item {} 
\sphinxAtStartPar
\sphinxstyleemphasis{xvar} is the variable to chek against selected in list

\item {} 
\sphinxAtStartPar
\sphinxstyleemphasis{select list} is a list of elements in the xvar list to be included

\item {} 
\sphinxAtStartPar
\sphinxstyleemphasis{matc} is the entry in \sphinxstyleemphasis{select list} from which to select from xvar

\end{itemize}

\end{fulllineitems}

\index{find\_res() (in module modelmanipulation)@\spxentry{find\_res()}\spxextra{in module modelmanipulation}}

\begin{fulllineitems}
\phantomsection\label{\detokenize{index:modelmanipulation.find_res}}
\pysigstartsignatures
\pysiglinewithargsret{\sphinxcode{\sphinxupquote{modelmanipulation.}}\sphinxbfcode{\sphinxupquote{find\_res}}}{\emph{\DUrole{n}{f}}}{}
\pysigstopsignatures
\sphinxAtStartPar
Finds the expression which calculates the residual in a formel. FRML \textless{}res=a,endo=b\textgreater{} x=a*b+c \$

\end{fulllineitems}

\index{find\_res\_dynare() (in module modelmanipulation)@\spxentry{find\_res\_dynare()}\spxextra{in module modelmanipulation}}

\begin{fulllineitems}
\phantomsection\label{\detokenize{index:modelmanipulation.find_res_dynare}}
\pysigstartsignatures
\pysiglinewithargsret{\sphinxcode{\sphinxupquote{modelmanipulation.}}\sphinxbfcode{\sphinxupquote{find\_res\_dynare}}}{\emph{\DUrole{n}{equations}}}{}
\pysigstopsignatures
\sphinxAtStartPar
equations to calculat \_res formulas
FRML \textless{}\textgreater{} x=a*b+c +x\_RES  \$ \sphinxhyphen{}\textgreater{} FRML \textless{}\textgreater{} x\_res =x\sphinxhyphen{}a*b+c  \$

\end{fulllineitems}

\index{find\_res\_dynare\_new() (in module modelmanipulation)@\spxentry{find\_res\_dynare\_new()}\spxextra{in module modelmanipulation}}

\begin{fulllineitems}
\phantomsection\label{\detokenize{index:modelmanipulation.find_res_dynare_new}}
\pysigstartsignatures
\pysiglinewithargsret{\sphinxcode{\sphinxupquote{modelmanipulation.}}\sphinxbfcode{\sphinxupquote{find\_res\_dynare\_new}}}{\emph{\DUrole{n}{equations}}}{}
\pysigstopsignatures
\sphinxAtStartPar
equations to calculat \_res formulas
FRML \textless{}\textgreater{} x=a*b+c +x\_RES  \$ \sphinxhyphen{}\textgreater{} FRML \textless{}\textgreater{} x\_res =x\sphinxhyphen{}a*b+c  \$
not finished to speed time up

\end{fulllineitems}

\index{find\_hist\_model() (in module modelmanipulation)@\spxentry{find\_hist\_model()}\spxextra{in module modelmanipulation}}

\begin{fulllineitems}
\phantomsection\label{\detokenize{index:modelmanipulation.find_hist_model}}
\pysigstartsignatures
\pysiglinewithargsret{\sphinxcode{\sphinxupquote{modelmanipulation.}}\sphinxbfcode{\sphinxupquote{find\_hist\_model}}}{\emph{\DUrole{n}{equations}}}{}
\pysigstopsignatures
\sphinxAtStartPar
takes a unrolled model and create a model which can be run for historic periode

\sphinxAtStartPar
and the identities are also calculeted

\end{fulllineitems}

\index{exounroll() (in module modelmanipulation)@\spxentry{exounroll()}\spxextra{in module modelmanipulation}}

\begin{fulllineitems}
\phantomsection\label{\detokenize{index:modelmanipulation.exounroll}}
\pysigstartsignatures
\pysiglinewithargsret{\sphinxcode{\sphinxupquote{modelmanipulation.}}\sphinxbfcode{\sphinxupquote{exounroll}}}{\emph{\DUrole{n}{in\_equations}}}{}
\pysigstopsignatures
\sphinxAtStartPar
takes a model and makes a new model by enhancing frml’s with \textless{}exo=,j=,jr=\textgreater{}
in their frml name.
\begin{quote}\begin{description}
\item[{Exo}] \leavevmode
\sphinxAtStartPar
the value can be fixed in to a value valuename\_x by setting valuename\_d=1

\item[{Jled}] \leavevmode
\sphinxAtStartPar
a additiv adjustment element is added to the frml

\item[{Jrled}] \leavevmode
\sphinxAtStartPar
a multiplicativ adjustment element is added to the frml

\end{description}\end{quote}

\end{fulllineitems}

\index{tofrml() (in module modelmanipulation)@\spxentry{tofrml()}\spxextra{in module modelmanipulation}}

\begin{fulllineitems}
\phantomsection\label{\detokenize{index:modelmanipulation.tofrml}}
\pysigstartsignatures
\pysiglinewithargsret{\sphinxcode{\sphinxupquote{modelmanipulation.}}\sphinxbfcode{\sphinxupquote{tofrml}}}{\emph{\DUrole{n}{expressions}}, \emph{\DUrole{n}{sep}\DUrole{o}{=}\DUrole{default_value}{\textquotesingle{}\textbackslash{}n\textquotesingle{}}}}{}
\pysigstopsignatures
\sphinxAtStartPar
a function, wich adds FRML to all expressions seperated by \textless{}sep\textgreater{}
if no start is specified the max lag will be used

\end{fulllineitems}

\index{dounloop() (in module modelmanipulation)@\spxentry{dounloop()}\spxextra{in module modelmanipulation}}

\begin{fulllineitems}
\phantomsection\label{\detokenize{index:modelmanipulation.dounloop}}
\pysigstartsignatures
\pysiglinewithargsret{\sphinxcode{\sphinxupquote{modelmanipulation.}}\sphinxbfcode{\sphinxupquote{dounloop}}}{\emph{\DUrole{n}{in\_equations}}, \emph{\DUrole{n}{listin}\DUrole{o}{=}\DUrole{default_value}{False}}}{}
\pysigstopsignatures
\sphinxAtStartPar
Expands (unrolls  do loops in a model template
goes trough a model template until there is no more nested do loops

\end{fulllineitems}

\index{find\_arg() (in module modelmanipulation)@\spxentry{find\_arg()}\spxextra{in module modelmanipulation}}

\begin{fulllineitems}
\phantomsection\label{\detokenize{index:modelmanipulation.find_arg}}
\pysigstartsignatures
\pysiglinewithargsret{\sphinxcode{\sphinxupquote{modelmanipulation.}}\sphinxbfcode{\sphinxupquote{find\_arg}}}{\emph{\DUrole{n}{funk}}, \emph{\DUrole{n}{streng}}}{}
\pysigstopsignatures
\sphinxAtStartPar
chops a string in 3 parts
\begin{enumerate}
\sphinxsetlistlabels{\arabic}{enumi}{enumii}{}{.}%
\item {} 
\sphinxAtStartPar
before ‘funk(‘

\item {} 
\sphinxAtStartPar
in the matching parantesis

\item {} 
\sphinxAtStartPar
after the last matching parenthesis

\end{enumerate}

\end{fulllineitems}

\index{sumunroll\_old() (in module modelmanipulation)@\spxentry{sumunroll\_old()}\spxextra{in module modelmanipulation}}

\begin{fulllineitems}
\phantomsection\label{\detokenize{index:modelmanipulation.sumunroll_old}}
\pysigstartsignatures
\pysiglinewithargsret{\sphinxcode{\sphinxupquote{modelmanipulation.}}\sphinxbfcode{\sphinxupquote{sumunroll\_old}}}{\emph{\DUrole{n}{in\_equations}}, \emph{\DUrole{n}{listin}\DUrole{o}{=}\DUrole{default_value}{False}}}{}
\pysigstopsignatures
\sphinxAtStartPar
expands all sum(list,’expression’) in a model
returns a new model

\end{fulllineitems}

\index{lagarray\_unroll() (in module modelmanipulation)@\spxentry{lagarray\_unroll()}\spxextra{in module modelmanipulation}}

\begin{fulllineitems}
\phantomsection\label{\detokenize{index:modelmanipulation.lagarray_unroll}}
\pysigstartsignatures
\pysiglinewithargsret{\sphinxcode{\sphinxupquote{modelmanipulation.}}\sphinxbfcode{\sphinxupquote{lagarray\_unroll}}}{\emph{\DUrole{n}{in\_equations}}, \emph{\DUrole{n}{funks}\DUrole{o}{=}\DUrole{default_value}{{[}{]}}}}{}
\pysigstopsignatures
\sphinxAtStartPar
expands all sum(list,’expression’) in a model
returns a new model

\end{fulllineitems}

\index{sumunroll() (in module modelmanipulation)@\spxentry{sumunroll()}\spxextra{in module modelmanipulation}}

\begin{fulllineitems}
\phantomsection\label{\detokenize{index:modelmanipulation.sumunroll}}
\pysigstartsignatures
\pysiglinewithargsret{\sphinxcode{\sphinxupquote{modelmanipulation.}}\sphinxbfcode{\sphinxupquote{sumunroll}}}{\emph{\DUrole{n}{in\_equations}}, \emph{\DUrole{n}{listin}\DUrole{o}{=}\DUrole{default_value}{False}}}{}
\pysigstopsignatures
\sphinxAtStartPar
expands all sum(list,’expression’) in a model
if sum(list xvar=lig,’expression’) only list elements where the condition is
satisfied wil be summed

\sphinxAtStartPar
returns a new model

\end{fulllineitems}

\index{argunroll() (in module modelmanipulation)@\spxentry{argunroll()}\spxextra{in module modelmanipulation}}

\begin{fulllineitems}
\phantomsection\label{\detokenize{index:modelmanipulation.argunroll}}
\pysigstartsignatures
\pysiglinewithargsret{\sphinxcode{\sphinxupquote{modelmanipulation.}}\sphinxbfcode{\sphinxupquote{argunroll}}}{\emph{\DUrole{n}{in\_equations}}, \emph{\DUrole{n}{listin}\DUrole{o}{=}\DUrole{default_value}{False}}}{}
\pysigstopsignatures
\sphinxAtStartPar
expands all ARGEXPAND(list,’expression’) in a model
returns a new model

\end{fulllineitems}

\index{creatematrix() (in module modelmanipulation)@\spxentry{creatematrix()}\spxextra{in module modelmanipulation}}

\begin{fulllineitems}
\phantomsection\label{\detokenize{index:modelmanipulation.creatematrix}}
\pysigstartsignatures
\pysiglinewithargsret{\sphinxcode{\sphinxupquote{modelmanipulation.}}\sphinxbfcode{\sphinxupquote{creatematrix}}}{\emph{\DUrole{n}{in\_equations}}, \emph{\DUrole{n}{listin}\DUrole{o}{=}\DUrole{default_value}{False}}}{}
\pysigstopsignatures
\sphinxAtStartPar
expands all ARGEXPAND(list,’expression’) in a model
returns a new model

\end{fulllineitems}

\index{createarray() (in module modelmanipulation)@\spxentry{createarray()}\spxextra{in module modelmanipulation}}

\begin{fulllineitems}
\phantomsection\label{\detokenize{index:modelmanipulation.createarray}}
\pysigstartsignatures
\pysiglinewithargsret{\sphinxcode{\sphinxupquote{modelmanipulation.}}\sphinxbfcode{\sphinxupquote{createarray}}}{\emph{\DUrole{n}{in\_equations}}, \emph{\DUrole{n}{listin}\DUrole{o}{=}\DUrole{default_value}{False}}}{}
\pysigstopsignatures
\sphinxAtStartPar
expands all to\_array(list) in a model
returns a new model

\end{fulllineitems}

\index{kaedeunroll() (in module modelmanipulation)@\spxentry{kaedeunroll()}\spxextra{in module modelmanipulation}}

\begin{fulllineitems}
\phantomsection\label{\detokenize{index:modelmanipulation.kaedeunroll}}
\pysigstartsignatures
\pysiglinewithargsret{\sphinxcode{\sphinxupquote{modelmanipulation.}}\sphinxbfcode{\sphinxupquote{kaedeunroll}}}{\emph{\DUrole{n}{in\_equations}}, \emph{\DUrole{n}{funks}\DUrole{o}{=}\DUrole{default_value}{{[}{]}}}}{}
\pysigstopsignatures
\sphinxAtStartPar
unrolls a chain (kaede) expression \sphinxhyphen{} used in the SMEC moedel

\end{fulllineitems}

\index{check\_syntax\_frml() (in module modelmanipulation)@\spxentry{check\_syntax\_frml()}\spxextra{in module modelmanipulation}}

\begin{fulllineitems}
\phantomsection\label{\detokenize{index:modelmanipulation.check_syntax_frml}}
\pysigstartsignatures
\pysiglinewithargsret{\sphinxcode{\sphinxupquote{modelmanipulation.}}\sphinxbfcode{\sphinxupquote{check\_syntax\_frml}}}{\emph{\DUrole{n}{frml}}}{}
\pysigstopsignatures
\sphinxAtStartPar
check syntax of frml

\end{fulllineitems}

\index{normalize\_a\_frml() (in module modelmanipulation)@\spxentry{normalize\_a\_frml()}\spxextra{in module modelmanipulation}}

\begin{fulllineitems}
\phantomsection\label{\detokenize{index:modelmanipulation.normalize_a_frml}}
\pysigstartsignatures
\pysiglinewithargsret{\sphinxcode{\sphinxupquote{modelmanipulation.}}\sphinxbfcode{\sphinxupquote{normalize\_a\_frml}}}{\emph{\DUrole{n}{frml}}, \emph{\DUrole{n}{show}\DUrole{o}{=}\DUrole{default_value}{False}}}{}
\pysigstopsignatures
\sphinxAtStartPar
Normalize and show a frml

\end{fulllineitems}

\index{nomalize\_a\_model() (in module modelmanipulation)@\spxentry{nomalize\_a\_model()}\spxextra{in module modelmanipulation}}

\begin{fulllineitems}
\phantomsection\label{\detokenize{index:modelmanipulation.nomalize_a_model}}
\pysigstartsignatures
\pysiglinewithargsret{\sphinxcode{\sphinxupquote{modelmanipulation.}}\sphinxbfcode{\sphinxupquote{nomalize\_a\_model}}}{\emph{\DUrole{n}{equations}}}{}
\pysigstopsignatures
\sphinxAtStartPar
a symbolic normalization is performed if there is a syntaxerror

\end{fulllineitems}

\index{normalize() (in module modelmanipulation)@\spxentry{normalize()}\spxextra{in module modelmanipulation}}

\begin{fulllineitems}
\phantomsection\label{\detokenize{index:modelmanipulation.normalize}}
\pysigstartsignatures
\pysiglinewithargsret{\sphinxcode{\sphinxupquote{modelmanipulation.}}\sphinxbfcode{\sphinxupquote{normalize}}}{\emph{\DUrole{n}{in\_equations}}, \emph{\DUrole{n}{sym}\DUrole{o}{=}\DUrole{default_value}{False}}, \emph{\DUrole{n}{funks}\DUrole{o}{=}\DUrole{default_value}{{[}{]}}}}{}
\pysigstopsignatures
\sphinxAtStartPar
Normalize an equation with log or several variables at the left hand side, the first variable is considerd the endogeneeus

\end{fulllineitems}

\index{explode() (in module modelmanipulation)@\spxentry{explode()}\spxextra{in module modelmanipulation}}

\begin{fulllineitems}
\phantomsection\label{\detokenize{index:modelmanipulation.explode}}
\pysigstartsignatures
\pysiglinewithargsret{\sphinxcode{\sphinxupquote{modelmanipulation.}}\sphinxbfcode{\sphinxupquote{explode}}}{\emph{\DUrole{n}{model}}, \emph{\DUrole{n}{norm}\DUrole{o}{=}\DUrole{default_value}{True}}, \emph{\DUrole{n}{sym}\DUrole{o}{=}\DUrole{default_value}{False}}, \emph{\DUrole{n}{funks}\DUrole{o}{=}\DUrole{default_value}{{[}{]}}}, \emph{\DUrole{n}{sep}\DUrole{o}{=}\DUrole{default_value}{\textquotesingle{}\textbackslash{}n\textquotesingle{}}}}{}
\pysigstopsignatures
\sphinxAtStartPar
prepares a model from a model template.

\sphinxAtStartPar
Returns a expanded model which is ready to solve

\sphinxAtStartPar
Eksempel: model = udrul\_model(MinModel.txt)

\end{fulllineitems}

\index{modelprint() (in module modelmanipulation)@\spxentry{modelprint()}\spxextra{in module modelmanipulation}}

\begin{fulllineitems}
\phantomsection\label{\detokenize{index:modelmanipulation.modelprint}}
\pysigstartsignatures
\pysiglinewithargsret{\sphinxcode{\sphinxupquote{modelmanipulation.}}\sphinxbfcode{\sphinxupquote{modelprint}}}{\emph{\DUrole{n}{ind}}, \emph{\DUrole{n}{title}\DUrole{o}{=}\DUrole{default_value}{\textquotesingle{} A model\textquotesingle{}}}, \emph{\DUrole{n}{udfil}\DUrole{o}{=}\DUrole{default_value}{\textquotesingle{}\textquotesingle{}}}, \emph{\DUrole{n}{short}\DUrole{o}{=}\DUrole{default_value}{0}}}{}
\pysigstopsignatures
\sphinxAtStartPar
prettyprinter for a a model.
:udfil: if present is output file
:short: if present condences the model
Can handle both model templates and models

\end{fulllineitems}

\index{lagone() (in module modelmanipulation)@\spxentry{lagone()}\spxextra{in module modelmanipulation}}

\begin{fulllineitems}
\phantomsection\label{\detokenize{index:modelmanipulation.lagone}}
\pysigstartsignatures
\pysiglinewithargsret{\sphinxcode{\sphinxupquote{modelmanipulation.}}\sphinxbfcode{\sphinxupquote{lagone}}}{\emph{\DUrole{n}{ind}}, \emph{\DUrole{n}{funks}\DUrole{o}{=}\DUrole{default_value}{{[}{]}}}, \emph{\DUrole{n}{laglead}\DUrole{o}{=}\DUrole{default_value}{\sphinxhyphen{} 1}}}{}
\pysigstopsignatures
\sphinxAtStartPar
All variables in a string i
s lagged one more time

\end{fulllineitems}

\index{lag\_n\_tup() (in module modelmanipulation)@\spxentry{lag\_n\_tup()}\spxextra{in module modelmanipulation}}

\begin{fulllineitems}
\phantomsection\label{\detokenize{index:modelmanipulation.lag_n_tup}}
\pysigstartsignatures
\pysiglinewithargsret{\sphinxcode{\sphinxupquote{modelmanipulation.}}\sphinxbfcode{\sphinxupquote{lag\_n\_tup}}}{\emph{\DUrole{n}{udtryk}}, \emph{\DUrole{n}{n}\DUrole{o}{=}\DUrole{default_value}{\sphinxhyphen{} 1}}, \emph{\DUrole{n}{funks}\DUrole{o}{=}\DUrole{default_value}{{[}{]}}}}{}
\pysigstopsignatures
\sphinxAtStartPar
return a tuppel og lagged expressions from lag = 0 to lag = n)

\end{fulllineitems}

\index{pastestring() (in module modelmanipulation)@\spxentry{pastestring()}\spxextra{in module modelmanipulation}}

\begin{fulllineitems}
\phantomsection\label{\detokenize{index:modelmanipulation.pastestring}}
\pysigstartsignatures
\pysiglinewithargsret{\sphinxcode{\sphinxupquote{modelmanipulation.}}\sphinxbfcode{\sphinxupquote{pastestring}}}{\emph{\DUrole{n}{ind}}, \emph{\DUrole{n}{post}}, \emph{\DUrole{n}{funks}\DUrole{o}{=}\DUrole{default_value}{{[}{]}}}, \emph{\DUrole{n}{onlylags}\DUrole{o}{=}\DUrole{default_value}{False}}}{}
\pysigstopsignatures
\sphinxAtStartPar
All variable names in a  in a string \sphinxstylestrong{ind} is pasted with the string \sphinxstylestrong{post}

\sphinxAtStartPar
This function can be used to awoid variable name conflict with the internal variable names in sympy.

\sphinxAtStartPar
an advanced function

\end{fulllineitems}

\index{stripstring() (in module modelmanipulation)@\spxentry{stripstring()}\spxextra{in module modelmanipulation}}

\begin{fulllineitems}
\phantomsection\label{\detokenize{index:modelmanipulation.stripstring}}
\pysigstartsignatures
\pysiglinewithargsret{\sphinxcode{\sphinxupquote{modelmanipulation.}}\sphinxbfcode{\sphinxupquote{stripstring}}}{\emph{\DUrole{n}{ind}}, \emph{\DUrole{n}{post}}, \emph{\DUrole{n}{funks}\DUrole{o}{=}\DUrole{default_value}{{[}{]}}}}{}
\pysigstopsignatures
\sphinxAtStartPar
All variable names in a  in a string is \sphinxstylestrong{ind} is stripped of the string \sphinxstylestrong{post}.

\sphinxAtStartPar
This function reverses the pastestring process

\end{fulllineitems}

\index{findindex() (in module modelmanipulation)@\spxentry{findindex()}\spxextra{in module modelmanipulation}}

\begin{fulllineitems}
\phantomsection\label{\detokenize{index:modelmanipulation.findindex}}
\pysigstartsignatures
\pysiglinewithargsret{\sphinxcode{\sphinxupquote{modelmanipulation.}}\sphinxbfcode{\sphinxupquote{findindex}}}{\emph{\DUrole{n}{ind00}}}{}
\pysigstopsignatures
\sphinxAtStartPar
find the index variables meaning variables on the left hand side of = braced by \{\}

\end{fulllineitems}

\index{doablelist() (in module modelmanipulation)@\spxentry{doablelist()}\spxextra{in module modelmanipulation}}

\begin{fulllineitems}
\phantomsection\label{\detokenize{index:modelmanipulation.doablelist}}
\pysigstartsignatures
\pysiglinewithargsret{\sphinxcode{\sphinxupquote{modelmanipulation.}}\sphinxbfcode{\sphinxupquote{doablelist}}}{\emph{\DUrole{n}{expressions}}, \emph{\DUrole{n}{sep}\DUrole{o}{=}\DUrole{default_value}{\textquotesingle{}\textbackslash{}n\textquotesingle{}}}}{}
\pysigstopsignatures
\sphinxAtStartPar
create a list of tupels from expressions seperated by sep,
each element in the list is a tupel (index, number og expression, the expression)

\sphinxAtStartPar
we want make group the expressions acording to index
index is elements on the left of = braced by \{\}

\end{fulllineitems}

\index{doablekeep() (in module modelmanipulation)@\spxentry{doablekeep()}\spxextra{in module modelmanipulation}}

\begin{fulllineitems}
\phantomsection\label{\detokenize{index:modelmanipulation.doablekeep}}
\pysigstartsignatures
\pysiglinewithargsret{\sphinxcode{\sphinxupquote{modelmanipulation.}}\sphinxbfcode{\sphinxupquote{doablekeep}}}{\emph{\DUrole{n}{formulars}}}{}
\pysigstopsignatures
\sphinxAtStartPar
takes index in the lhs and creates a do loop around the lines with same indexes
on the right side you can use \%0\_, \%1\_ an so on to indicate the index, just to awoid typing to much

\sphinxAtStartPar
Also \%i\_ will be changed to all the indexes

\end{fulllineitems}

\index{doable() (in module modelmanipulation)@\spxentry{doable()}\spxextra{in module modelmanipulation}}

\begin{fulllineitems}
\phantomsection\label{\detokenize{index:modelmanipulation.doable}}
\pysigstartsignatures
\pysiglinewithargsret{\sphinxcode{\sphinxupquote{modelmanipulation.}}\sphinxbfcode{\sphinxupquote{doable}}}{\emph{\DUrole{n}{formulars}}, \emph{\DUrole{n}{funks}\DUrole{o}{=}\DUrole{default_value}{{[}{]}}}}{}
\pysigstopsignatures
\sphinxAtStartPar
takes index in the lhs and creates a do loop around the line
on the right side you can use \%0\_, \%1\_ an so on to indicate the index, just to awoid typing to much

\sphinxAtStartPar
Also \%i\_ will be changed to all the indexes

\end{fulllineitems}

\index{findindex\_gams() (in module modelmanipulation)@\spxentry{findindex\_gams()}\spxextra{in module modelmanipulation}}

\begin{fulllineitems}
\phantomsection\label{\detokenize{index:modelmanipulation.findindex_gams}}
\pysigstartsignatures
\pysiglinewithargsret{\sphinxcode{\sphinxupquote{modelmanipulation.}}\sphinxbfcode{\sphinxupquote{findindex\_gams}}}{\emph{\DUrole{n}{ind00}}}{}
\pysigstopsignatures\begin{itemize}
\item {} 
\sphinxAtStartPar
an equation looks like this

\item {} 
\sphinxAtStartPar
\textless{}frmlname\textgreater{} {[}index{]} lhs = rhs

\end{itemize}

\sphinxAtStartPar
this function find frmlname and index variables on the left hand side. meaning variables braced by \{\}

\end{fulllineitems}

\index{un\_normalize\_expression() (in module modelmanipulation)@\spxentry{un\_normalize\_expression()}\spxextra{in module modelmanipulation}}

\begin{fulllineitems}
\phantomsection\label{\detokenize{index:modelmanipulation.un_normalize_expression}}
\pysigstartsignatures
\pysiglinewithargsret{\sphinxcode{\sphinxupquote{modelmanipulation.}}\sphinxbfcode{\sphinxupquote{un\_normalize\_expression}}}{\emph{\DUrole{n}{frml}}}{}
\pysigstopsignatures
\sphinxAtStartPar
This function makes sure that all formulas are unnormalized.
if the formula is already decorated with \textless{}endo=name\textgreater{} this is kept
else the lhs\_varriable is used in \textless{}endo=\textgreater{}

\end{fulllineitems}

\index{un\_normalize\_model() (in module modelmanipulation)@\spxentry{un\_normalize\_model()}\spxextra{in module modelmanipulation}}

\begin{fulllineitems}
\phantomsection\label{\detokenize{index:modelmanipulation.un_normalize_model}}
\pysigstartsignatures
\pysiglinewithargsret{\sphinxcode{\sphinxupquote{modelmanipulation.}}\sphinxbfcode{\sphinxupquote{un\_normalize\_model}}}{\emph{\DUrole{n}{in\_equations}}, \emph{\DUrole{n}{funks}\DUrole{o}{=}\DUrole{default_value}{{[}{]}}}}{}
\pysigstopsignatures
\sphinxAtStartPar
un normalize a model

\end{fulllineitems}

\index{un\_normalize\_simpel() (in module modelmanipulation)@\spxentry{un\_normalize\_simpel()}\spxextra{in module modelmanipulation}}

\begin{fulllineitems}
\phantomsection\label{\detokenize{index:modelmanipulation.un_normalize_simpel}}
\pysigstartsignatures
\pysiglinewithargsret{\sphinxcode{\sphinxupquote{modelmanipulation.}}\sphinxbfcode{\sphinxupquote{un\_normalize\_simpel}}}{\emph{\DUrole{n}{in\_equations}}, \emph{\DUrole{n}{funks}\DUrole{o}{=}\DUrole{default_value}{{[}{]}}}}{}
\pysigstopsignatures
\sphinxAtStartPar
un\sphinxhyphen{}normalize expressions delimeted by linebreaks

\end{fulllineitems}

\index{eksempel() (in module modelmanipulation)@\spxentry{eksempel()}\spxextra{in module modelmanipulation}}

\begin{fulllineitems}
\phantomsection\label{\detokenize{index:modelmanipulation.eksempel}}
\pysigstartsignatures
\pysiglinewithargsret{\sphinxcode{\sphinxupquote{modelmanipulation.}}\sphinxbfcode{\sphinxupquote{eksempel}}}{\emph{\DUrole{n}{ind}}}{}
\pysigstopsignatures
\sphinxAtStartPar
takes a template model as input, creates a model and a histmodel and prints the models

\end{fulllineitems}



\section{Modelnormalize, Transforms and normalizes single equations}
\label{\detokenize{index:module-modelnormalize}}\label{\detokenize{index:modelnormalize-transforms-and-normalizes-single-equations}}\index{module@\spxentry{module}!modelnormalize@\spxentry{modelnormalize}}\index{modelnormalize@\spxentry{modelnormalize}!module@\spxentry{module}}
\sphinxAtStartPar
Created on Sat Nov 28 13:32:47 2020

\sphinxAtStartPar
This Module is used transforming model specifications to modelflow business language.
\begin{itemize}
\item {} \begin{description}
\item[{preprocessing expressions to resolve functions like}] \leavevmode
\sphinxAtStartPar
dlog, log, pct, movavg

\end{description}

\item {} 
\sphinxAtStartPar
replace function names

\item {} 
\sphinxAtStartPar
normalize formulas

\end{itemize}

\sphinxAtStartPar
@author: bruger
\index{Normalized\_frml (class in modelnormalize)@\spxentry{Normalized\_frml}\spxextra{class in modelnormalize}}

\begin{fulllineitems}
\phantomsection\label{\detokenize{index:modelnormalize.Normalized_frml}}
\pysigstartsignatures
\pysiglinewithargsret{\sphinxbfcode{\sphinxupquote{class\DUrole{w}{  }}}\sphinxcode{\sphinxupquote{modelnormalize.}}\sphinxbfcode{\sphinxupquote{Normalized\_frml}}}{\emph{\DUrole{n}{endo\_var}\DUrole{p}{:}\DUrole{w}{  }\DUrole{n}{str}\DUrole{w}{  }\DUrole{o}{=}\DUrole{w}{  }\DUrole{default_value}{\textquotesingle{}\textquotesingle{}}}, \emph{\DUrole{n}{original}\DUrole{p}{:}\DUrole{w}{  }\DUrole{n}{str}\DUrole{w}{  }\DUrole{o}{=}\DUrole{w}{  }\DUrole{default_value}{\textquotesingle{}\textquotesingle{}}}, \emph{\DUrole{n}{preprocessed}\DUrole{p}{:}\DUrole{w}{  }\DUrole{n}{str}\DUrole{w}{  }\DUrole{o}{=}\DUrole{w}{  }\DUrole{default_value}{\textquotesingle{}\textquotesingle{}}}, \emph{\DUrole{n}{normalized}\DUrole{p}{:}\DUrole{w}{  }\DUrole{n}{str}\DUrole{w}{  }\DUrole{o}{=}\DUrole{w}{  }\DUrole{default_value}{\textquotesingle{}\textquotesingle{}}}, \emph{\DUrole{n}{calc\_add\_factor}\DUrole{p}{:}\DUrole{w}{  }\DUrole{n}{str}\DUrole{w}{  }\DUrole{o}{=}\DUrole{w}{  }\DUrole{default_value}{\textquotesingle{}\textquotesingle{}}}, \emph{\DUrole{n}{un\_normalized}\DUrole{p}{:}\DUrole{w}{  }\DUrole{n}{str}\DUrole{w}{  }\DUrole{o}{=}\DUrole{w}{  }\DUrole{default_value}{\textquotesingle{}\textquotesingle{}}}, \emph{\DUrole{n}{fitted}\DUrole{p}{:}\DUrole{w}{  }\DUrole{n}{str}\DUrole{w}{  }\DUrole{o}{=}\DUrole{w}{  }\DUrole{default_value}{\textquotesingle{}\textquotesingle{}}}, \emph{\DUrole{n}{eviews}\DUrole{p}{:}\DUrole{w}{  }\DUrole{n}{str}\DUrole{w}{  }\DUrole{o}{=}\DUrole{w}{  }\DUrole{default_value}{\textquotesingle{}\textquotesingle{}}}}{}
\pysigstopsignatures
\sphinxAtStartPar
class defining result from normalization of expression

\end{fulllineitems}

\index{endovar() (in module modelnormalize)@\spxentry{endovar()}\spxextra{in module modelnormalize}}

\begin{fulllineitems}
\phantomsection\label{\detokenize{index:modelnormalize.endovar}}
\pysigstartsignatures
\pysiglinewithargsret{\sphinxcode{\sphinxupquote{modelnormalize.}}\sphinxbfcode{\sphinxupquote{endovar}}}{\emph{\DUrole{n}{f}}}{}
\pysigstopsignatures
\sphinxAtStartPar
Finds the first variable in a expression

\end{fulllineitems}

\index{funk\_in() (in module modelnormalize)@\spxentry{funk\_in()}\spxextra{in module modelnormalize}}

\begin{fulllineitems}
\phantomsection\label{\detokenize{index:modelnormalize.funk_in}}
\pysigstartsignatures
\pysiglinewithargsret{\sphinxcode{\sphinxupquote{modelnormalize.}}\sphinxbfcode{\sphinxupquote{funk\_in}}}{\emph{\DUrole{n}{funk}}, \emph{\DUrole{n}{a\_string}}}{}
\pysigstopsignatures
\sphinxAtStartPar
Find the first location of a function in a string

\sphinxAtStartPar
if found returns a match object where the group 2 is the interesting stuff used in funk\_find\_arg

\end{fulllineitems}

\index{funk\_replace() (in module modelnormalize)@\spxentry{funk\_replace()}\spxextra{in module modelnormalize}}

\begin{fulllineitems}
\phantomsection\label{\detokenize{index:modelnormalize.funk_replace}}
\pysigstartsignatures
\pysiglinewithargsret{\sphinxcode{\sphinxupquote{modelnormalize.}}\sphinxbfcode{\sphinxupquote{funk\_replace}}}{\emph{\DUrole{n}{funk1}}, \emph{\DUrole{n}{funk2}}, \emph{\DUrole{n}{a\_string}}}{}
\pysigstopsignatures
\sphinxAtStartPar
replace funk1( with funk2(

\sphinxAtStartPar
takes care that funk1 embedded in variable name is not replaced

\end{fulllineitems}

\index{funk\_replace\_list() (in module modelnormalize)@\spxentry{funk\_replace\_list()}\spxextra{in module modelnormalize}}

\begin{fulllineitems}
\phantomsection\label{\detokenize{index:modelnormalize.funk_replace_list}}
\pysigstartsignatures
\pysiglinewithargsret{\sphinxcode{\sphinxupquote{modelnormalize.}}\sphinxbfcode{\sphinxupquote{funk\_replace\_list}}}{\emph{\DUrole{n}{replacelist}}, \emph{\DUrole{n}{a\_string}}}{}
\pysigstopsignatures
\sphinxAtStartPar
Replaces a list of funk1( , funk2(

\end{fulllineitems}

\index{funk\_find\_arg() (in module modelnormalize)@\spxentry{funk\_find\_arg()}\spxextra{in module modelnormalize}}

\begin{fulllineitems}
\phantomsection\label{\detokenize{index:modelnormalize.funk_find_arg}}
\pysigstartsignatures
\pysiglinewithargsret{\sphinxcode{\sphinxupquote{modelnormalize.}}\sphinxbfcode{\sphinxupquote{funk\_find\_arg}}}{\emph{\DUrole{n}{funk\_match}}, \emph{\DUrole{n}{streng}}}{}
\pysigstopsignatures
\sphinxAtStartPar
chops a string in 3 parts
\begin{enumerate}
\sphinxsetlistlabels{\arabic}{enumi}{enumii}{}{.}%
\item {} 
\sphinxAtStartPar
before ‘funk(‘

\item {} 
\sphinxAtStartPar
in the matching parantesis

\item {} 
\sphinxAtStartPar
after the last matching parenthesis

\end{enumerate}

\end{fulllineitems}

\index{preprocess() (in module modelnormalize)@\spxentry{preprocess()}\spxextra{in module modelnormalize}}

\begin{fulllineitems}
\phantomsection\label{\detokenize{index:modelnormalize.preprocess}}
\pysigstartsignatures
\pysiglinewithargsret{\sphinxcode{\sphinxupquote{modelnormalize.}}\sphinxbfcode{\sphinxupquote{preprocess}}}{\emph{\DUrole{n}{udtryk}}, \emph{\DUrole{n}{funks}\DUrole{o}{=}\DUrole{default_value}{{[}{]}}}}{}
\pysigstopsignatures
\sphinxAtStartPar
test processing expanding dlog,diff,movavg,pct,logit functions
\begin{quote}\begin{description}
\item[{Parameters}] \leavevmode\begin{itemize}
\item {} 
\sphinxAtStartPar
\sphinxstyleliteralstrong{\sphinxupquote{udtryk}} (\sphinxstyleliteralemphasis{\sphinxupquote{str}}) \textendash{} model we want to do template expansion on

\item {} 
\sphinxAtStartPar
\sphinxstyleliteralstrong{\sphinxupquote{funks}} (\sphinxstyleliteralemphasis{\sphinxupquote{list}}\sphinxstyleliteralemphasis{\sphinxupquote{, }}\sphinxstyleliteralemphasis{\sphinxupquote{optional}}) \textendash{} list of user defined functions . Defaults to {[}{]}.

\end{itemize}

\item[{Returns}] \leavevmode
\sphinxAtStartPar
None.

\end{description}\end{quote}

\sphinxAtStartPar
has to be changed to (= for when the transition to 3.8 is finished.

\end{fulllineitems}

\index{normal() (in module modelnormalize)@\spxentry{normal()}\spxextra{in module modelnormalize}}

\begin{fulllineitems}
\phantomsection\label{\detokenize{index:modelnormalize.normal}}
\pysigstartsignatures
\pysiglinewithargsret{\sphinxcode{\sphinxupquote{modelnormalize.}}\sphinxbfcode{\sphinxupquote{normal}}}{\emph{\DUrole{n}{ind\_o}}, \emph{\DUrole{n}{the\_endo}\DUrole{o}{=}\DUrole{default_value}{\textquotesingle{}\textquotesingle{}}}, \emph{\DUrole{n}{add\_add\_factor}\DUrole{o}{=}\DUrole{default_value}{True}}, \emph{\DUrole{n}{do\_preprocess}\DUrole{o}{=}\DUrole{default_value}{True}}, \emph{\DUrole{n}{add\_suffix}\DUrole{o}{=}\DUrole{default_value}{\textquotesingle{}\_A\textquotesingle{}}}, \emph{\DUrole{n}{endo\_lhs}\DUrole{o}{=}\DUrole{default_value}{True}}, \emph{\DUrole{n}{make\_fixable}\DUrole{o}{=}\DUrole{default_value}{False}}, \emph{\DUrole{n}{make\_fitted}\DUrole{o}{=}\DUrole{default_value}{False}}, \emph{\DUrole{n}{eviews}\DUrole{o}{=}\DUrole{default_value}{\textquotesingle{}\textquotesingle{}}}}{}
\pysigstopsignatures
\sphinxAtStartPar
normalize an expression g(y,x) = f(y,x) ==\textgreater{} y = F(x,z)

\sphinxAtStartPar
Default find the expression for the first variable on the left hand side (lhs)

\sphinxAtStartPar
The variable \sphinxhyphen{} without lags\sphinxhyphen{}  should not be on rhs.
\begin{quote}\begin{description}
\item[{Parameters}] \leavevmode\begin{itemize}
\item {} 
\sphinxAtStartPar
\sphinxstyleliteralstrong{\sphinxupquote{ind\_o}} (\sphinxstyleliteralemphasis{\sphinxupquote{str}}) \textendash{} input expression, no \$ and no frml name just lhs=rhs

\item {} 
\sphinxAtStartPar
\sphinxstyleliteralstrong{\sphinxupquote{the\_endo}} (\sphinxstyleliteralemphasis{\sphinxupquote{str}}\sphinxstyleliteralemphasis{\sphinxupquote{, }}\sphinxstyleliteralemphasis{\sphinxupquote{optional}}) \textendash{} the endogeneous to isolate on the left hans side. if the first variable in the lhs.
It shoud be on the left hand side.

\item {} 
\sphinxAtStartPar
\sphinxstyleliteralstrong{\sphinxupquote{add\_add\_factor}} (\sphinxstyleliteralemphasis{\sphinxupquote{bool}}\sphinxstyleliteralemphasis{\sphinxupquote{, }}\sphinxstyleliteralemphasis{\sphinxupquote{optional}}) \textendash{} force introduction aof adjustment term, and an expression to calculate it

\item {} 
\sphinxAtStartPar
\sphinxstyleliteralstrong{\sphinxupquote{do\_preprocess}} (\sphinxstyleliteralemphasis{\sphinxupquote{bool}}\sphinxstyleliteralemphasis{\sphinxupquote{, }}\sphinxstyleliteralemphasis{\sphinxupquote{optional}}) \textendash{} DESCRIPTION. preprocess the expression

\item {} 
\sphinxAtStartPar
\sphinxstyleliteralstrong{\sphinxupquote{endo\_lhs}} (\sphinxstyleliteralemphasis{\sphinxupquote{bool}}\sphinxstyleliteralemphasis{\sphinxupquote{, }}\sphinxstyleliteralemphasis{\sphinxupquote{optional}}) \textendash{} If false, accept to normalize for a rhs endogeneous variable

\item {} 
\sphinxAtStartPar
\sphinxstyleliteralstrong{\sphinxupquote{make\_fixable}} (\sphinxstyleliteralemphasis{\sphinxupquote{bool}}\sphinxstyleliteralemphasis{\sphinxupquote{, }}\sphinxstyleliteralemphasis{\sphinxupquote{optional}}) \textendash{} also make this equation exogenizable

\item {} 
\sphinxAtStartPar
\sphinxstyleliteralstrong{\sphinxupquote{fitted}} (\sphinxstyleliteralemphasis{\sphinxupquote{bool}}\sphinxstyleliteralemphasis{\sphinxupquote{,}}\sphinxstyleliteralemphasis{\sphinxupquote{optional}}) \textendash{} create a fitted equations, without exo and adjustment

\end{itemize}

\end{description}\end{quote}

\sphinxAtStartPar
preprocessing handels
\begin{quote}\begin{description}
\item[{Returns}] \leavevmode
\sphinxAtStartPar
Normalized\_frml which will contain the different relevant expressions

\item[{Return type}] \leavevmode
\sphinxAtStartPar
An instance of the class

\end{description}\end{quote}

\end{fulllineitems}

\index{elem\_trans() (in module modelnormalize)@\spxentry{elem\_trans()}\spxextra{in module modelnormalize}}

\begin{fulllineitems}
\phantomsection\label{\detokenize{index:modelnormalize.elem_trans}}
\pysigstartsignatures
\pysiglinewithargsret{\sphinxcode{\sphinxupquote{modelnormalize.}}\sphinxbfcode{\sphinxupquote{elem\_trans}}}{\emph{\DUrole{n}{udtryk}}, \emph{\DUrole{n}{df}\DUrole{o}{=}\DUrole{default_value}{None}}}{}
\pysigstopsignatures
\sphinxAtStartPar
Handeles expression with @elem

\end{fulllineitems}



\section{Onboarding models}
\label{\detokenize{index:onboarding-models}}
\sphinxAtStartPar
Modules to onboard models from different sources.

\sphinxAtStartPar
The process of onboarding involves transforming the original specification
to \sphinxstylestrong{Modelflow Business Logic Language} using what ever tools needed. As Python has very
powerfull string and datatools it is possible to onboard many models \sphinxhyphen{} but by all means not all models.

\sphinxAtStartPar
Be aware, that the functions presented here are made for specific model(families) following specific conventions. If these
conventions are not followed, another model can’t be onboarded


\subsection{Eviews}
\label{\detokenize{index:eviews}}

\subsubsection{From wf1 file}
\label{\detokenize{index:module-modelgrabwf2}}\label{\detokenize{index:from-wf1-file}}\index{module@\spxentry{module}!modelgrabwf2@\spxentry{modelgrabwf2}}\index{modelgrabwf2@\spxentry{modelgrabwf2}!module@\spxentry{module}}
\sphinxAtStartPar
Created on Wed Mar 30 10:06:26 2022

\sphinxAtStartPar
@author: ibhan

\sphinxAtStartPar
Module to handle models in wf1 files
\begin{enumerate}
\sphinxsetlistlabels{\arabic}{enumi}{enumii}{}{.}%
\item {} 
\sphinxAtStartPar
Eviews is started and the wf1 file is loaded.
\begin{enumerate}
\sphinxsetlistlabels{\arabic}{enumii}{enumiii}{}{.}%
\item {} 
\sphinxAtStartPar
Some transformations are performed on data.

\item {} 
\sphinxAtStartPar
The model is unlinked.

\item {} 
\sphinxAtStartPar
The workspace is saved as a wf2 file. Same name with \_modelflow appended.

\end{enumerate}

\item {} 
\sphinxAtStartPar
Eviews is closed

\item {} 
\sphinxAtStartPar
The wf2 file is read as a json file.

\item {} 
\sphinxAtStartPar
Relevant objects are extracted.

\item {} 
\sphinxAtStartPar
The MFMSA variable is  extracted, to be saved in the dumpfile.

\item {} 
\sphinxAtStartPar
The equations are transformed and normalized to modelflow format and classified into identities and stochastic

\item {} 
\sphinxAtStartPar
Stochastic equations are enriched by add\_factor and fixing terms (dummy + fixing value)

\item {} 
\sphinxAtStartPar
For Stochastic equations new fitted variables are generated \sphinxhyphen{} without add add\_factors and dummies.

\item {} 
\sphinxAtStartPar
A model to generate fitted variables is created

\item {} 
\sphinxAtStartPar
A model to generate add\_factors is created.

\item {} 
\sphinxAtStartPar
A model encompassing the original equations, the model for fitted variables and for add\_factors is created.

\item {} 
\sphinxAtStartPar
The data series and scalars are shoveled into a Pandas dataframe
\begin{enumerate}
\sphinxsetlistlabels{\arabic}{enumii}{enumiii}{}{.}%
\item {} 
\sphinxAtStartPar
Some special series are generated as the expression can not be incorporated into modelflow model specifications

\item {} 
\sphinxAtStartPar
The model for fitted values is simulated in the specified timespan

\item {} 
\sphinxAtStartPar
The model for add\_factors is simulated in the timespan set in MFMSA

\end{enumerate}

\item {} 
\sphinxAtStartPar
The data descriptions are extracted into a dictionary.

\item {} 
\sphinxAtStartPar
Data descriptions for dummies, fixed values, fitted values and add\_factors are derived.

\item {} 
\sphinxAtStartPar
Now we have a model and a dataframe with all variables which are needed.

\end{enumerate}
\index{wf1\_to\_wf2() (in module modelgrabwf2)@\spxentry{wf1\_to\_wf2()}\spxextra{in module modelgrabwf2}}

\begin{fulllineitems}
\phantomsection\label{\detokenize{index:modelgrabwf2.wf1_to_wf2}}
\pysigstartsignatures
\pysiglinewithargsret{\sphinxcode{\sphinxupquote{modelgrabwf2.}}\sphinxbfcode{\sphinxupquote{wf1\_to\_wf2}}}{\emph{\DUrole{n}{filename}}, \emph{\DUrole{n}{modelname}\DUrole{o}{=}\DUrole{default_value}{\textquotesingle{}\textquotesingle{}}}, \emph{\DUrole{n}{eviews\_run\_lines}\DUrole{o}{=}\DUrole{default_value}{{[}{]}}}}{}
\pysigstopsignatures\begin{itemize}
\item {} 
\sphinxAtStartPar
Opens a eviews workfile in wf1 format

\item {} 
\sphinxAtStartPar
calculates the eviews\_trend

\item {} 
\sphinxAtStartPar
unlink a model and

\item {} 
\sphinxAtStartPar
writes the workspace back to a wf2 file

\end{itemize}
\begin{quote}\begin{description}
\item[{Parameters}] \leavevmode\begin{itemize}
\item {} 
\sphinxAtStartPar
\sphinxstyleliteralstrong{\sphinxupquote{filename}} (\sphinxstyleliteralemphasis{\sphinxupquote{TYPE}}) \textendash{} DESCRIPTION.

\item {} 
\sphinxAtStartPar
\sphinxstyleliteralstrong{\sphinxupquote{modelname}} (\sphinxstyleliteralemphasis{\sphinxupquote{TYPE}}) \textendash{} default ‘’ then the three first letters of the filenames stem are asumed to be the modelname.

\end{itemize}

\item[{Returns}] \leavevmode
\sphinxAtStartPar
None.

\end{description}\end{quote}

\end{fulllineitems}

\index{wf2\_to\_clean() (in module modelgrabwf2)@\spxentry{wf2\_to\_clean()}\spxextra{in module modelgrabwf2}}

\begin{fulllineitems}
\phantomsection\label{\detokenize{index:modelgrabwf2.wf2_to_clean}}
\pysigstartsignatures
\pysiglinewithargsret{\sphinxcode{\sphinxupquote{modelgrabwf2.}}\sphinxbfcode{\sphinxupquote{wf2\_to\_clean}}}{\emph{\DUrole{n}{wf2name}}, \emph{\DUrole{n}{modelname}\DUrole{o}{=}\DUrole{default_value}{\textquotesingle{}\textquotesingle{}}}, \emph{\DUrole{n}{save\_file}\DUrole{o}{=}\DUrole{default_value}{False}}}{}
\pysigstopsignatures
\sphinxAtStartPar
Takes a eviews .wf2 file \sphinxhyphen{} which is in JSON format \sphinxhyphen{} and place a
dictionary
\begin{quote}\begin{description}
\item[{Parameters}] \leavevmode\begin{itemize}
\item {} 
\sphinxAtStartPar
\sphinxstyleliteralstrong{\sphinxupquote{wf2name}} (\sphinxstyleliteralemphasis{\sphinxupquote{TYPE}}) \textendash{} name of wf2 file .

\item {} 
\sphinxAtStartPar
\sphinxstyleliteralstrong{\sphinxupquote{modelname}} (\sphinxstyleliteralemphasis{\sphinxupquote{TYPE}}\sphinxstyleliteralemphasis{\sphinxupquote{, }}\sphinxstyleliteralemphasis{\sphinxupquote{optional}}) \textendash{} Name og model. Defaults to ‘’.

\item {} 
\sphinxAtStartPar
\sphinxstyleliteralstrong{\sphinxupquote{save\_file}} (\sphinxstyleliteralemphasis{\sphinxupquote{TYPE}}\sphinxstyleliteralemphasis{\sphinxupquote{, }}\sphinxstyleliteralemphasis{\sphinxupquote{optional}}) \textendash{} save the specification, data and description in a dictionary. Defaults to False.

\end{itemize}

\item[{Returns}] \leavevmode
\sphinxAtStartPar
the content of the wf2 file as a dict .

\item[{Return type}] \leavevmode
\sphinxAtStartPar
model\_all\_about (dict)

\end{description}\end{quote}

\end{fulllineitems}

\index{GrabWfModel (class in modelgrabwf2)@\spxentry{GrabWfModel}\spxextra{class in modelgrabwf2}}

\begin{fulllineitems}
\phantomsection\label{\detokenize{index:modelgrabwf2.GrabWfModel}}
\pysigstartsignatures
\pysiglinewithargsret{\sphinxbfcode{\sphinxupquote{class\DUrole{w}{  }}}\sphinxcode{\sphinxupquote{modelgrabwf2.}}\sphinxbfcode{\sphinxupquote{GrabWfModel}}}{\emph{\DUrole{n}{filename: any = \textquotesingle{}\textquotesingle{}}}, \emph{\DUrole{n}{modelname: any = \textquotesingle{}\textquotesingle{}}}, \emph{\DUrole{n}{eviews\_run\_lines: list = \textless{}factory\textgreater{}}}, \emph{\DUrole{n}{model\_all\_about: dict = \textless{}factory\textgreater{}}}, \emph{\DUrole{n}{start: typing.Optional{[}any{]} = None}}, \emph{\DUrole{n}{end: typing.Optional{[}any{]} = None}}, \emph{\DUrole{n}{country\_trans: any = \textless{}function GrabWfModel.\textless{}lambda\textgreater{}\textgreater{}}}, \emph{\DUrole{n}{country\_df\_trans: any = \textless{}function GrabWfModel.\textless{}lambda\textgreater{}\textgreater{}}}, \emph{\DUrole{n}{make\_fitted: bool = False}}, \emph{\DUrole{n}{fit\_start: any = 2000}}, \emph{\DUrole{n}{fit\_end: typing.Optional{[}any{]} = None}}, \emph{\DUrole{n}{do\_add\_factor\_calc: bool = True}}, \emph{\DUrole{n}{test\_frml: str = \textquotesingle{}\textquotesingle{}}}, \emph{\DUrole{n}{disable\_progress: bool = False}}}{}
\pysigstopsignatures
\sphinxAtStartPar
This class takes a world bank model specification, variable data and variable description
and transform it to ModelFlow business language
\begin{quote}\begin{description}
\item[{Parameters}] \leavevmode\begin{itemize}
\item {} 
\sphinxAtStartPar
\sphinxstyleliteralstrong{\sphinxupquote{filename}} \textendash{} any = ‘’  \#wf1 name

\item {} 
\sphinxAtStartPar
\sphinxstyleliteralstrong{\sphinxupquote{modelname}} \textendash{} any = ‘’

\item {} 
\sphinxAtStartPar
\sphinxstyleliteralstrong{\sphinxupquote{eviews\_run\_lines}} \textendash{} list =field(default\_factory=list)

\item {} 
\sphinxAtStartPar
\sphinxstyleliteralstrong{\sphinxupquote{model\_all\_about}} \textendash{} dict = field(default\_factory=dict)

\item {} 
\sphinxAtStartPar
\sphinxstyleliteralstrong{\sphinxupquote{start}} \textendash{} any = None    \# start of testing if not overruled by mfmsa

\item {} 
\sphinxAtStartPar
\sphinxstyleliteralstrong{\sphinxupquote{end}} \textendash{} any = None    \# end of testing if not overruled by mfmsa

\item {} 
\sphinxAtStartPar
\sphinxstyleliteralstrong{\sphinxupquote{country\_trans}} \textendash{} any = lambda x:x{[}:{]}    \# function which transform model specification

\item {} 
\sphinxAtStartPar
\sphinxstyleliteralstrong{\sphinxupquote{country\_df\_trans}} \textendash{} any = lambda x:x     \# function which transforms initial dataframe

\item {} 
\sphinxAtStartPar
\sphinxstyleliteralstrong{\sphinxupquote{make\_fitted}} \textendash{} bool = False \# if True, a clean equation for fittet variables is created

\item {} 
\sphinxAtStartPar
\sphinxstyleliteralstrong{\sphinxupquote{fit\_start}} \textendash{} any = 2000   \# start of fittet model

\item {} 
\sphinxAtStartPar
\sphinxstyleliteralstrong{\sphinxupquote{fit\_end}} \textendash{} any = None  \# end of fittet model unless overruled by mfmsa

\item {} 
\sphinxAtStartPar
\sphinxstyleliteralstrong{\sphinxupquote{do\_add\_factor\_calc}} \textendash{} bool = True  \# calculate the add factors

\item {} 
\sphinxAtStartPar
\sphinxstyleliteralstrong{\sphinxupquote{test\_frml}} \textendash{} str =’’    \# a testmodel as string if used no wf processing

\item {} 
\sphinxAtStartPar
\sphinxstyleliteralstrong{\sphinxupquote{disable\_progress}} \textendash{} bool = False \# Disable progress bar

\end{itemize}

\end{description}\end{quote}
\index{\_\_post\_init\_\_() (modelgrabwf2.GrabWfModel method)@\spxentry{\_\_post\_init\_\_()}\spxextra{modelgrabwf2.GrabWfModel method}}

\begin{fulllineitems}
\phantomsection\label{\detokenize{index:modelgrabwf2.GrabWfModel.__post_init__}}
\pysigstartsignatures
\pysiglinewithargsret{\sphinxbfcode{\sphinxupquote{\_\_post\_init\_\_}}}{}{}
\pysigstopsignatures
\sphinxAtStartPar
Process the model

\end{fulllineitems}

\index{trans\_eviews() (modelgrabwf2.GrabWfModel static method)@\spxentry{trans\_eviews()}\spxextra{modelgrabwf2.GrabWfModel static method}}

\begin{fulllineitems}
\phantomsection\label{\detokenize{index:modelgrabwf2.GrabWfModel.trans_eviews}}
\pysigstartsignatures
\pysiglinewithargsret{\sphinxbfcode{\sphinxupquote{static\DUrole{w}{  }}}\sphinxbfcode{\sphinxupquote{trans\_eviews}}}{\emph{\DUrole{n}{rawmodel}}}{}
\pysigstopsignatures
\sphinxAtStartPar
Takes Eviews specifications and wrangle them into modelflow specifications
\begin{quote}\begin{description}
\item[{Parameters}] \leavevmode
\sphinxAtStartPar
\sphinxstyleliteralstrong{\sphinxupquote{rawmodel}} (\sphinxstyleliteralemphasis{\sphinxupquote{TYPE}}) \textendash{} a raw model .

\item[{Returns}] \leavevmode
\sphinxAtStartPar
a model with the appropiate eviews transformations.

\item[{Return type}] \leavevmode
\sphinxAtStartPar
rawmodel6 (TYPE)

\end{description}\end{quote}

\end{fulllineitems}

\index{var\_description (modelgrabwf2.GrabWfModel property)@\spxentry{var\_description}\spxextra{modelgrabwf2.GrabWfModel property}}

\begin{fulllineitems}
\phantomsection\label{\detokenize{index:modelgrabwf2.GrabWfModel.var_description}}
\pysigstartsignatures
\pysigline{\sphinxbfcode{\sphinxupquote{property\DUrole{w}{  }}}\sphinxbfcode{\sphinxupquote{var\_description}}}
\pysigstopsignatures
\sphinxAtStartPar
Adds var descriptions for add factors, exogenizing dummies and exoggenizing values

\end{fulllineitems}

\index{mfmsa\_options (modelgrabwf2.GrabWfModel property)@\spxentry{mfmsa\_options}\spxextra{modelgrabwf2.GrabWfModel property}}

\begin{fulllineitems}
\phantomsection\label{\detokenize{index:modelgrabwf2.GrabWfModel.mfmsa_options}}
\pysigstartsignatures
\pysigline{\sphinxbfcode{\sphinxupquote{property\DUrole{w}{  }}}\sphinxbfcode{\sphinxupquote{mfmsa\_options}}}
\pysigstopsignatures
\sphinxAtStartPar
Grab the mfmsa options, a world bank speciality

\end{fulllineitems}

\index{mfmsa\_start\_end (modelgrabwf2.GrabWfModel property)@\spxentry{mfmsa\_start\_end}\spxextra{modelgrabwf2.GrabWfModel property}}

\begin{fulllineitems}
\phantomsection\label{\detokenize{index:modelgrabwf2.GrabWfModel.mfmsa_start_end}}
\pysigstartsignatures
\pysigline{\sphinxbfcode{\sphinxupquote{property\DUrole{w}{  }}}\sphinxbfcode{\sphinxupquote{mfmsa\_start\_end}}}
\pysigstopsignatures
\sphinxAtStartPar
Finds the start and end from the MFMSA entry

\end{fulllineitems}

\index{dfmodel (modelgrabwf2.GrabWfModel property)@\spxentry{dfmodel}\spxextra{modelgrabwf2.GrabWfModel property}}

\begin{fulllineitems}
\phantomsection\label{\detokenize{index:modelgrabwf2.GrabWfModel.dfmodel}}
\pysigstartsignatures
\pysigline{\sphinxbfcode{\sphinxupquote{property\DUrole{w}{  }}}\sphinxbfcode{\sphinxupquote{dfmodel}}}
\pysigstopsignatures
\sphinxAtStartPar
The original input data enriched with during variablees, variables containing
values for specific historic years and model specific transformation

\end{fulllineitems}

\index{test\_model() (modelgrabwf2.GrabWfModel method)@\spxentry{test\_model()}\spxextra{modelgrabwf2.GrabWfModel method}}

\begin{fulllineitems}
\phantomsection\label{\detokenize{index:modelgrabwf2.GrabWfModel.test_model}}
\pysigstartsignatures
\pysiglinewithargsret{\sphinxbfcode{\sphinxupquote{test\_model}}}{\emph{\DUrole{n}{start}\DUrole{o}{=}\DUrole{default_value}{None}}, \emph{\DUrole{n}{end}\DUrole{o}{=}\DUrole{default_value}{None}}, \emph{\DUrole{n}{maxvar}\DUrole{o}{=}\DUrole{default_value}{1000000}}, \emph{\DUrole{n}{maxerr}\DUrole{o}{=}\DUrole{default_value}{100}}, \emph{\DUrole{n}{tol}\DUrole{o}{=}\DUrole{default_value}{0.0001}}, \emph{\DUrole{n}{showall}\DUrole{o}{=}\DUrole{default_value}{False}}, \emph{\DUrole{n}{showinput}\DUrole{o}{=}\DUrole{default_value}{False}}}{}
\pysigstopsignatures
\sphinxAtStartPar
Compares a straight calculation with the input dataframe.

\sphinxAtStartPar
shows which variables dont have the same value
\begin{quote}\begin{description}
\item[{Parameters}] \leavevmode\begin{itemize}
\item {} 
\sphinxAtStartPar
\sphinxstyleliteralstrong{\sphinxupquote{df}} (\sphinxstyleliteralemphasis{\sphinxupquote{TYPE}}) \textendash{} dataframe to run.

\item {} 
\sphinxAtStartPar
\sphinxstyleliteralstrong{\sphinxupquote{start}} (\sphinxstyleliteralemphasis{\sphinxupquote{TYPE}}\sphinxstyleliteralemphasis{\sphinxupquote{, }}\sphinxstyleliteralemphasis{\sphinxupquote{optional}}) \textendash{} start period. Defaults to None.

\item {} 
\sphinxAtStartPar
\sphinxstyleliteralstrong{\sphinxupquote{end}} (\sphinxstyleliteralemphasis{\sphinxupquote{TYPE}}\sphinxstyleliteralemphasis{\sphinxupquote{, }}\sphinxstyleliteralemphasis{\sphinxupquote{optional}}) \textendash{} end period. Defaults to None.

\item {} 
\sphinxAtStartPar
\sphinxstyleliteralstrong{\sphinxupquote{maxvar}} (\sphinxstyleliteralemphasis{\sphinxupquote{TYPE}}\sphinxstyleliteralemphasis{\sphinxupquote{, }}\sphinxstyleliteralemphasis{\sphinxupquote{optional}}) \textendash{} how many variables are to be chekked. Defaults to 1\_000\_000.

\item {} 
\sphinxAtStartPar
\sphinxstyleliteralstrong{\sphinxupquote{maxerr}} (\sphinxstyleliteralemphasis{\sphinxupquote{TYPE}}\sphinxstyleliteralemphasis{\sphinxupquote{, }}\sphinxstyleliteralemphasis{\sphinxupquote{optional}}) \textendash{} how many errors to check Defaults to 100.

\item {} 
\sphinxAtStartPar
\sphinxstyleliteralstrong{\sphinxupquote{tol}} (\sphinxstyleliteralemphasis{\sphinxupquote{TYPE}}\sphinxstyleliteralemphasis{\sphinxupquote{, }}\sphinxstyleliteralemphasis{\sphinxupquote{optional}}) \textendash{} check for absolute value of difference. Defaults to 0.0001.

\item {} 
\sphinxAtStartPar
\sphinxstyleliteralstrong{\sphinxupquote{showall}} (\sphinxstyleliteralemphasis{\sphinxupquote{TYPE}}\sphinxstyleliteralemphasis{\sphinxupquote{, }}\sphinxstyleliteralemphasis{\sphinxupquote{optional}}) \textendash{} show more . Defaults to False.

\item {} 
\sphinxAtStartPar
\sphinxstyleliteralstrong{\sphinxupquote{showinput}} (\sphinxstyleliteralemphasis{\sphinxupquote{TYPE}}\sphinxstyleliteralemphasis{\sphinxupquote{, }}\sphinxstyleliteralemphasis{\sphinxupquote{optional}}) \textendash{} show the input values Defaults to False.

\end{itemize}

\item[{Returns}] \leavevmode
\sphinxAtStartPar
None.

\end{description}\end{quote}

\end{fulllineitems}


\end{fulllineitems}



\chapter{Processing results}
\label{\detokenize{index:processing-results}}

\section{Modelvis, Display and vizualize variables}
\label{\detokenize{index:module-modelvis}}\label{\detokenize{index:modelvis-display-and-vizualize-variables}}\index{module@\spxentry{module}!modelvis@\spxentry{modelvis}}\index{modelvis@\spxentry{modelvis}!module@\spxentry{module}}
\sphinxAtStartPar
Created on Fri May 12 11:07:02 2017

\sphinxAtStartPar
@author: hanseni

\sphinxAtStartPar
This module creates functions and classes for visualizing results.
\index{meltdim() (in module modelvis)@\spxentry{meltdim()}\spxextra{in module modelvis}}

\begin{fulllineitems}
\phantomsection\label{\detokenize{index:modelvis.meltdim}}
\pysigstartsignatures
\pysiglinewithargsret{\sphinxcode{\sphinxupquote{modelvis.}}\sphinxbfcode{\sphinxupquote{meltdim}}}{\emph{\DUrole{n}{df}}, \emph{\DUrole{n}{dims}\DUrole{o}{=}\DUrole{default_value}{{[}\textquotesingle{}dima\textquotesingle{}, \textquotesingle{}dimb\textquotesingle{}{]}}}, \emph{\DUrole{n}{source}\DUrole{o}{=}\DUrole{default_value}{\textquotesingle{}Latest\textquotesingle{}}}}{}
\pysigstopsignatures
\sphinxAtStartPar
Melts a wide dataframe the variable names are split to dimensions acording
to the list of texts in dims. in variablenames
the tall dataframe have a variable name for each dimensions
also values and source are introduced ac column names in the dataframe

\end{fulllineitems}

\index{vis (class in modelvis)@\spxentry{vis}\spxextra{class in modelvis}}

\begin{fulllineitems}
\phantomsection\label{\detokenize{index:modelvis.vis}}
\pysigstartsignatures
\pysiglinewithargsret{\sphinxbfcode{\sphinxupquote{class\DUrole{w}{  }}}\sphinxcode{\sphinxupquote{modelvis.}}\sphinxbfcode{\sphinxupquote{vis}}}{\emph{\DUrole{n}{model}\DUrole{o}{=}\DUrole{default_value}{None}}, \emph{\DUrole{n}{pat}\DUrole{o}{=}\DUrole{default_value}{\textquotesingle{}\textquotesingle{}}}, \emph{\DUrole{n}{names}\DUrole{o}{=}\DUrole{default_value}{None}}, \emph{\DUrole{n}{df}\DUrole{o}{=}\DUrole{default_value}{None}}}{}
\pysigstopsignatures
\sphinxAtStartPar
Visualization class. used as a method on a model instance.

\sphinxAtStartPar
The purpose is to select variables acording to a pattern, potential with wildcards
\index{heat() (modelvis.vis method)@\spxentry{heat()}\spxextra{modelvis.vis method}}

\begin{fulllineitems}
\phantomsection\label{\detokenize{index:modelvis.vis.heat}}
\pysigstartsignatures
\pysiglinewithargsret{\sphinxbfcode{\sphinxupquote{heat}}}{\emph{\DUrole{o}{*}\DUrole{n}{args}}, \emph{\DUrole{o}{**}\DUrole{n}{kwargs}}}{}
\pysigstopsignatures
\sphinxAtStartPar
Displays a heatmap of the resulting dataframe

\end{fulllineitems}

\index{plot() (modelvis.vis method)@\spxentry{plot()}\spxextra{modelvis.vis method}}

\begin{fulllineitems}
\phantomsection\label{\detokenize{index:modelvis.vis.plot}}
\pysigstartsignatures
\pysiglinewithargsret{\sphinxbfcode{\sphinxupquote{plot}}}{\emph{\DUrole{o}{*}\DUrole{n}{args}}, \emph{\DUrole{o}{**}\DUrole{n}{kwargs}}}{}
\pysigstopsignatures
\sphinxAtStartPar
Displays a plot for each of the columns in the resulting dataframe

\end{fulllineitems}

\index{plot\_alt() (modelvis.vis method)@\spxentry{plot\_alt()}\spxextra{modelvis.vis method}}

\begin{fulllineitems}
\phantomsection\label{\detokenize{index:modelvis.vis.plot_alt}}
\pysigstartsignatures
\pysiglinewithargsret{\sphinxbfcode{\sphinxupquote{plot\_alt}}}{\emph{\DUrole{n}{title}\DUrole{o}{=}\DUrole{default_value}{\textquotesingle{}Title\textquotesingle{}}}, \emph{\DUrole{o}{*}\DUrole{n}{args}}, \emph{\DUrole{o}{**}\DUrole{n}{kwargs}}}{}
\pysigstopsignatures
\sphinxAtStartPar
Displays a plot for each of the columns in the resulting dataframe

\end{fulllineitems}

\index{box() (modelvis.vis method)@\spxentry{box()}\spxextra{modelvis.vis method}}

\begin{fulllineitems}
\phantomsection\label{\detokenize{index:modelvis.vis.box}}
\pysigstartsignatures
\pysiglinewithargsret{\sphinxbfcode{\sphinxupquote{box}}}{}{}
\pysigstopsignatures
\sphinxAtStartPar
Displays a boxplot comparing basedf and lastdf

\end{fulllineitems}

\index{violin() (modelvis.vis method)@\spxentry{violin()}\spxextra{modelvis.vis method}}

\begin{fulllineitems}
\phantomsection\label{\detokenize{index:modelvis.vis.violin}}
\pysigstartsignatures
\pysiglinewithargsret{\sphinxbfcode{\sphinxupquote{violin}}}{}{}
\pysigstopsignatures
\sphinxAtStartPar
Displays a violinplot comparing basedf and lastdf

\end{fulllineitems}

\index{swarm() (modelvis.vis method)@\spxentry{swarm()}\spxextra{modelvis.vis method}}

\begin{fulllineitems}
\phantomsection\label{\detokenize{index:modelvis.vis.swarm}}
\pysigstartsignatures
\pysiglinewithargsret{\sphinxbfcode{\sphinxupquote{swarm}}}{}{}
\pysigstopsignatures
\sphinxAtStartPar
Displays a swarmlot comparing basedf and lastdf

\end{fulllineitems}

\index{df (modelvis.vis property)@\spxentry{df}\spxextra{modelvis.vis property}}

\begin{fulllineitems}
\phantomsection\label{\detokenize{index:modelvis.vis.df}}
\pysigstartsignatures
\pysigline{\sphinxbfcode{\sphinxupquote{property\DUrole{w}{  }}}\sphinxbfcode{\sphinxupquote{df}}}
\pysigstopsignatures
\sphinxAtStartPar
Returns the result of this instance as a dataframe

\end{fulllineitems}

\index{base (modelvis.vis property)@\spxentry{base}\spxextra{modelvis.vis property}}

\begin{fulllineitems}
\phantomsection\label{\detokenize{index:modelvis.vis.base}}
\pysigstartsignatures
\pysigline{\sphinxbfcode{\sphinxupquote{property\DUrole{w}{  }}}\sphinxbfcode{\sphinxupquote{base}}}
\pysigstopsignatures
\sphinxAtStartPar
Returns basedf

\end{fulllineitems}

\index{pct (modelvis.vis property)@\spxentry{pct}\spxextra{modelvis.vis property}}

\begin{fulllineitems}
\phantomsection\label{\detokenize{index:modelvis.vis.pct}}
\pysigstartsignatures
\pysigline{\sphinxbfcode{\sphinxupquote{property\DUrole{w}{  }}}\sphinxbfcode{\sphinxupquote{pct}}}
\pysigstopsignatures
\sphinxAtStartPar
Returns the pct change

\end{fulllineitems}

\index{year\_pct (modelvis.vis property)@\spxentry{year\_pct}\spxextra{modelvis.vis property}}

\begin{fulllineitems}
\phantomsection\label{\detokenize{index:modelvis.vis.year_pct}}
\pysigstartsignatures
\pysigline{\sphinxbfcode{\sphinxupquote{property\DUrole{w}{  }}}\sphinxbfcode{\sphinxupquote{year\_pct}}}
\pysigstopsignatures
\sphinxAtStartPar
Returns the pct change over 4 periods (used for quarterly data)

\end{fulllineitems}

\index{frml (modelvis.vis property)@\spxentry{frml}\spxextra{modelvis.vis property}}

\begin{fulllineitems}
\phantomsection\label{\detokenize{index:modelvis.vis.frml}}
\pysigstartsignatures
\pysigline{\sphinxbfcode{\sphinxupquote{property\DUrole{w}{  }}}\sphinxbfcode{\sphinxupquote{frml}}}
\pysigstopsignatures
\sphinxAtStartPar
Returns formulas

\end{fulllineitems}

\index{des (modelvis.vis property)@\spxentry{des}\spxextra{modelvis.vis property}}

\begin{fulllineitems}
\phantomsection\label{\detokenize{index:modelvis.vis.des}}
\pysigstartsignatures
\pysigline{\sphinxbfcode{\sphinxupquote{property\DUrole{w}{  }}}\sphinxbfcode{\sphinxupquote{des}}}
\pysigstopsignatures
\sphinxAtStartPar
Returns variable descriptions

\end{fulllineitems}

\index{dif (modelvis.vis property)@\spxentry{dif}\spxextra{modelvis.vis property}}

\begin{fulllineitems}
\phantomsection\label{\detokenize{index:modelvis.vis.dif}}
\pysigstartsignatures
\pysigline{\sphinxbfcode{\sphinxupquote{property\DUrole{w}{  }}}\sphinxbfcode{\sphinxupquote{dif}}}
\pysigstopsignatures
\sphinxAtStartPar
Returns the differens between the basedf and lastdf

\end{fulllineitems}

\index{difpctlevel (modelvis.vis property)@\spxentry{difpctlevel}\spxextra{modelvis.vis property}}

\begin{fulllineitems}
\phantomsection\label{\detokenize{index:modelvis.vis.difpctlevel}}
\pysigstartsignatures
\pysigline{\sphinxbfcode{\sphinxupquote{property\DUrole{w}{  }}}\sphinxbfcode{\sphinxupquote{difpctlevel}}}
\pysigstopsignatures
\sphinxAtStartPar
Returns the differens between the basedf and lastdf

\end{fulllineitems}

\index{difpct (modelvis.vis property)@\spxentry{difpct}\spxextra{modelvis.vis property}}

\begin{fulllineitems}
\phantomsection\label{\detokenize{index:modelvis.vis.difpct}}
\pysigstartsignatures
\pysigline{\sphinxbfcode{\sphinxupquote{property\DUrole{w}{  }}}\sphinxbfcode{\sphinxupquote{difpct}}}
\pysigstopsignatures
\sphinxAtStartPar
Returns the differens between the pct changes in basedf and lastdf

\end{fulllineitems}

\index{print (modelvis.vis property)@\spxentry{print}\spxextra{modelvis.vis property}}

\begin{fulllineitems}
\phantomsection\label{\detokenize{index:modelvis.vis.print}}
\pysigstartsignatures
\pysigline{\sphinxbfcode{\sphinxupquote{property\DUrole{w}{  }}}\sphinxbfcode{\sphinxupquote{print}}}
\pysigstopsignatures
\sphinxAtStartPar
prints the current result

\end{fulllineitems}

\index{\_\_repr\_\_() (modelvis.vis method)@\spxentry{\_\_repr\_\_()}\spxextra{modelvis.vis method}}

\begin{fulllineitems}
\phantomsection\label{\detokenize{index:modelvis.vis.__repr__}}
\pysigstartsignatures
\pysiglinewithargsret{\sphinxbfcode{\sphinxupquote{\_\_repr\_\_}}}{}{}
\pysigstopsignatures
\sphinxAtStartPar
Return repr(self).

\end{fulllineitems}

\index{\_repr\_html\_() (modelvis.vis method)@\spxentry{\_repr\_html\_()}\spxextra{modelvis.vis method}}

\begin{fulllineitems}
\phantomsection\label{\detokenize{index:modelvis.vis._repr_html_}}
\pysigstartsignatures
\pysiglinewithargsret{\sphinxbfcode{\sphinxupquote{\_repr\_html\_}}}{}{}
\pysigstopsignatures
\sphinxAtStartPar
Displays a nice summary of the results when called in a Jupyter enviorement

\end{fulllineitems}

\index{rename() (modelvis.vis method)@\spxentry{rename()}\spxextra{modelvis.vis method}}

\begin{fulllineitems}
\phantomsection\label{\detokenize{index:modelvis.vis.rename}}
\pysigstartsignatures
\pysiglinewithargsret{\sphinxbfcode{\sphinxupquote{rename}}}{\emph{\DUrole{n}{other}\DUrole{o}{=}\DUrole{default_value}{None}}}{}
\pysigstopsignatures
\sphinxAtStartPar
rename columns

\end{fulllineitems}

\index{mul() (modelvis.vis method)@\spxentry{mul()}\spxextra{modelvis.vis method}}

\begin{fulllineitems}
\phantomsection\label{\detokenize{index:modelvis.vis.mul}}
\pysigstartsignatures
\pysiglinewithargsret{\sphinxbfcode{\sphinxupquote{mul}}}{\emph{\DUrole{n}{other}}}{}
\pysigstopsignatures
\sphinxAtStartPar
Multiply the curent result with other

\end{fulllineitems}

\index{mul100 (modelvis.vis property)@\spxentry{mul100}\spxextra{modelvis.vis property}}

\begin{fulllineitems}
\phantomsection\label{\detokenize{index:modelvis.vis.mul100}}
\pysigstartsignatures
\pysigline{\sphinxbfcode{\sphinxupquote{property\DUrole{w}{  }}}\sphinxbfcode{\sphinxupquote{mul100}}}
\pysigstopsignatures
\sphinxAtStartPar
Multiply the current result with 100

\end{fulllineitems}


\end{fulllineitems}

\index{compvis (class in modelvis)@\spxentry{compvis}\spxextra{class in modelvis}}

\begin{fulllineitems}
\phantomsection\label{\detokenize{index:modelvis.compvis}}
\pysigstartsignatures
\pysiglinewithargsret{\sphinxbfcode{\sphinxupquote{class\DUrole{w}{  }}}\sphinxcode{\sphinxupquote{modelvis.}}\sphinxbfcode{\sphinxupquote{compvis}}}{\emph{\DUrole{n}{model}\DUrole{o}{=}\DUrole{default_value}{None}}, \emph{\DUrole{n}{pat}\DUrole{o}{=}\DUrole{default_value}{None}}}{}
\pysigstopsignatures
\sphinxAtStartPar
Class to compare to runs in boxplots
\index{box() (modelvis.compvis method)@\spxentry{box()}\spxextra{modelvis.compvis method}}

\begin{fulllineitems}
\phantomsection\label{\detokenize{index:modelvis.compvis.box}}
\pysigstartsignatures
\pysiglinewithargsret{\sphinxbfcode{\sphinxupquote{box}}}{\emph{\DUrole{o}{*}\DUrole{n}{args}}, \emph{\DUrole{o}{**}\DUrole{n}{kwargs}}}{}
\pysigstopsignatures
\sphinxAtStartPar
Displays a boxplot

\end{fulllineitems}

\index{swarm() (modelvis.compvis method)@\spxentry{swarm()}\spxextra{modelvis.compvis method}}

\begin{fulllineitems}
\phantomsection\label{\detokenize{index:modelvis.compvis.swarm}}
\pysigstartsignatures
\pysiglinewithargsret{\sphinxbfcode{\sphinxupquote{swarm}}}{\emph{\DUrole{o}{*}\DUrole{n}{args}}, \emph{\DUrole{o}{**}\DUrole{n}{kwargs}}}{}
\pysigstopsignatures
\sphinxAtStartPar
Displays a swarmplot

\end{fulllineitems}

\index{violin() (modelvis.compvis method)@\spxentry{violin()}\spxextra{modelvis.compvis method}}

\begin{fulllineitems}
\phantomsection\label{\detokenize{index:modelvis.compvis.violin}}
\pysigstartsignatures
\pysiglinewithargsret{\sphinxbfcode{\sphinxupquote{violin}}}{\emph{\DUrole{o}{*}\DUrole{n}{args}}, \emph{\DUrole{o}{**}\DUrole{n}{kwargs}}}{}
\pysigstopsignatures
\sphinxAtStartPar
Displays a violinplot

\end{fulllineitems}


\end{fulllineitems}

\index{container (class in modelvis)@\spxentry{container}\spxextra{class in modelvis}}

\begin{fulllineitems}
\phantomsection\label{\detokenize{index:modelvis.container}}
\pysigstartsignatures
\pysiglinewithargsret{\sphinxbfcode{\sphinxupquote{class\DUrole{w}{  }}}\sphinxcode{\sphinxupquote{modelvis.}}\sphinxbfcode{\sphinxupquote{container}}}{\emph{\DUrole{n}{lastdf}}, \emph{\DUrole{n}{basedf}}}{}
\pysigstopsignatures
\sphinxAtStartPar
A container, used if to izualize dataframes without a model
\index{smpl() (modelvis.container method)@\spxentry{smpl()}\spxextra{modelvis.container method}}

\begin{fulllineitems}
\phantomsection\label{\detokenize{index:modelvis.container.smpl}}
\pysigstartsignatures
\pysiglinewithargsret{\sphinxbfcode{\sphinxupquote{smpl}}}{\emph{\DUrole{n}{start}\DUrole{o}{=}\DUrole{default_value}{\textquotesingle{}\textquotesingle{}}}, \emph{\DUrole{n}{slut}\DUrole{o}{=}\DUrole{default_value}{\textquotesingle{}\textquotesingle{}}}, \emph{\DUrole{n}{df}\DUrole{o}{=}\DUrole{default_value}{None}}}{}
\pysigstopsignatures
\sphinxAtStartPar
Defines the model.current\_per which is used for calculation period/index
when no parameters are issues the current current period is returned

\sphinxAtStartPar
Either none or all parameters have to be provided

\end{fulllineitems}

\index{vlist() (modelvis.container method)@\spxentry{vlist()}\spxextra{modelvis.container method}}

\begin{fulllineitems}
\phantomsection\label{\detokenize{index:modelvis.container.vlist}}
\pysigstartsignatures
\pysiglinewithargsret{\sphinxbfcode{\sphinxupquote{vlist}}}{\emph{\DUrole{n}{pat}}}{}
\pysigstopsignatures
\sphinxAtStartPar
returns a list of variable matching the pattern

\end{fulllineitems}


\end{fulllineitems}

\index{varvis (class in modelvis)@\spxentry{varvis}\spxextra{class in modelvis}}

\begin{fulllineitems}
\phantomsection\label{\detokenize{index:modelvis.varvis}}
\pysigstartsignatures
\pysiglinewithargsret{\sphinxbfcode{\sphinxupquote{class\DUrole{w}{  }}}\sphinxcode{\sphinxupquote{modelvis.}}\sphinxbfcode{\sphinxupquote{varvis}}}{\emph{\DUrole{n}{model}\DUrole{o}{=}\DUrole{default_value}{None}}, \emph{\DUrole{n}{var}\DUrole{o}{=}\DUrole{default_value}{\textquotesingle{}\textquotesingle{}}}}{}
\pysigstopsignatures
\sphinxAtStartPar
Visualization class. used as a method on a model instance.

\sphinxAtStartPar
The purpose is to select variables acording to a pattern, potential with wildcards
\index{tracedep() (modelvis.varvis method)@\spxentry{tracedep()}\spxextra{modelvis.varvis method}}

\begin{fulllineitems}
\phantomsection\label{\detokenize{index:modelvis.varvis.tracedep}}
\pysigstartsignatures
\pysiglinewithargsret{\sphinxbfcode{\sphinxupquote{tracedep}}}{\emph{\DUrole{n}{down}\DUrole{o}{=}\DUrole{default_value}{1}}, \emph{\DUrole{o}{**}\DUrole{n}{kwargs}}}{}
\pysigstopsignatures
\sphinxAtStartPar
Trace dependensies of name down to level down

\end{fulllineitems}

\index{tracepre() (modelvis.varvis method)@\spxentry{tracepre()}\spxextra{modelvis.varvis method}}

\begin{fulllineitems}
\phantomsection\label{\detokenize{index:modelvis.varvis.tracepre}}
\pysigstartsignatures
\pysiglinewithargsret{\sphinxbfcode{\sphinxupquote{tracepre}}}{\emph{\DUrole{n}{up}\DUrole{o}{=}\DUrole{default_value}{1}}, \emph{\DUrole{o}{**}\DUrole{n}{kwargs}}}{}
\pysigstopsignatures
\sphinxAtStartPar
Trace dependensies of name down to level down

\end{fulllineitems}

\index{\_\_repr\_\_() (modelvis.varvis method)@\spxentry{\_\_repr\_\_()}\spxextra{modelvis.varvis method}}

\begin{fulllineitems}
\phantomsection\label{\detokenize{index:modelvis.varvis.__repr__}}
\pysigstartsignatures
\pysiglinewithargsret{\sphinxbfcode{\sphinxupquote{\_\_repr\_\_}}}{}{}
\pysigstopsignatures
\sphinxAtStartPar
Return repr(self).

\end{fulllineitems}


\end{fulllineitems}

\index{vis\_alt() (in module modelvis)@\spxentry{vis\_alt()}\spxextra{in module modelvis}}

\begin{fulllineitems}
\phantomsection\label{\detokenize{index:modelvis.vis_alt}}
\pysigstartsignatures
\pysiglinewithargsret{\sphinxcode{\sphinxupquote{modelvis.}}\sphinxbfcode{\sphinxupquote{vis\_alt}}}{\emph{\DUrole{n}{grund}}, \emph{\DUrole{n}{mul}}, \emph{\DUrole{n}{title}\DUrole{o}{=}\DUrole{default_value}{\textquotesingle{}Show variables\textquotesingle{}}}, \emph{\DUrole{n}{top}\DUrole{o}{=}\DUrole{default_value}{0.9}}}{}
\pysigstopsignatures
\sphinxAtStartPar
Graph of one of more variables each variable is displayed for 3 banks

\end{fulllineitems}

\index{plotshow() (in module modelvis)@\spxentry{plotshow()}\spxextra{in module modelvis}}

\begin{fulllineitems}
\phantomsection\label{\detokenize{index:modelvis.plotshow}}
\pysigstartsignatures
\pysiglinewithargsret{\sphinxcode{\sphinxupquote{modelvis.}}\sphinxbfcode{\sphinxupquote{plotshow}}}{\emph{\DUrole{n}{df}}, \emph{\DUrole{n}{name}\DUrole{o}{=}\DUrole{default_value}{\textquotesingle{}\textquotesingle{}}}, \emph{\DUrole{n}{ppos}\DUrole{o}{=}\DUrole{default_value}{\sphinxhyphen{} 1}}, \emph{\DUrole{n}{kind}\DUrole{o}{=}\DUrole{default_value}{\textquotesingle{}line\textquotesingle{}}}, \emph{\DUrole{n}{colrow}\DUrole{o}{=}\DUrole{default_value}{2}}, \emph{\DUrole{n}{sharey}\DUrole{o}{=}\DUrole{default_value}{False}}, \emph{\DUrole{n}{top}\DUrole{o}{=}\DUrole{default_value}{0.9}}, \emph{\DUrole{n}{splitchar}\DUrole{o}{=}\DUrole{default_value}{\textquotesingle{}\_\_\textquotesingle{}}}, \emph{\DUrole{n}{savefig}\DUrole{o}{=}\DUrole{default_value}{\textquotesingle{}\textquotesingle{}}}, \emph{\DUrole{o}{*}\DUrole{n}{args}}, \emph{\DUrole{o}{**}\DUrole{n}{kwargs}}}{}
\pysigstopsignatures\begin{quote}\begin{description}
\item[{Parameters}] \leavevmode\begin{itemize}
\item {} 
\sphinxAtStartPar
\sphinxstyleliteralstrong{\sphinxupquote{df}} (\sphinxstyleliteralemphasis{\sphinxupquote{TYPE}}) \textendash{} Dataframe .

\item {} 
\sphinxAtStartPar
\sphinxstyleliteralstrong{\sphinxupquote{name}} (\sphinxstyleliteralemphasis{\sphinxupquote{TYPE}}\sphinxstyleliteralemphasis{\sphinxupquote{, }}\sphinxstyleliteralemphasis{\sphinxupquote{optional}}) \textendash{} title. Defaults to ‘’.

\item {} 
\sphinxAtStartPar
\sphinxstyleliteralstrong{\sphinxupquote{ppos}} (\sphinxstyleliteralemphasis{\sphinxupquote{TYPE}}\sphinxstyleliteralemphasis{\sphinxupquote{, }}\sphinxstyleliteralemphasis{\sphinxupquote{optional}}) \textendash{} \# of position to use if split. Defaults to \sphinxhyphen{}1.

\item {} 
\sphinxAtStartPar
\sphinxstyleliteralstrong{\sphinxupquote{kind}} (\sphinxstyleliteralemphasis{\sphinxupquote{TYPE}}\sphinxstyleliteralemphasis{\sphinxupquote{, }}\sphinxstyleliteralemphasis{\sphinxupquote{optional}}) \textendash{} matplotlib kind . Defaults to ‘line’.

\item {} 
\sphinxAtStartPar
\sphinxstyleliteralstrong{\sphinxupquote{colrow}} (\sphinxstyleliteralemphasis{\sphinxupquote{TYPE}}\sphinxstyleliteralemphasis{\sphinxupquote{, }}\sphinxstyleliteralemphasis{\sphinxupquote{optional}}) \textendash{} columns per row . Defaults to 6.

\item {} 
\sphinxAtStartPar
\sphinxstyleliteralstrong{\sphinxupquote{sharey}} (\sphinxstyleliteralemphasis{\sphinxupquote{TYPE}}\sphinxstyleliteralemphasis{\sphinxupquote{, }}\sphinxstyleliteralemphasis{\sphinxupquote{optional}}) \textendash{} Share y axis between plots. Defaults to True.

\item {} 
\sphinxAtStartPar
\sphinxstyleliteralstrong{\sphinxupquote{top}} (\sphinxstyleliteralemphasis{\sphinxupquote{TYPE}}\sphinxstyleliteralemphasis{\sphinxupquote{, }}\sphinxstyleliteralemphasis{\sphinxupquote{optional}}) \textendash{} relative position of the title. Defaults to 0.90.

\item {} 
\sphinxAtStartPar
\sphinxstyleliteralstrong{\sphinxupquote{splitchar}} (\sphinxstyleliteralemphasis{\sphinxupquote{TYPE}}\sphinxstyleliteralemphasis{\sphinxupquote{, }}\sphinxstyleliteralemphasis{\sphinxupquote{optional}}) \textendash{} if the name should be split . Defaults to ‘\_\_’.

\item {} 
\sphinxAtStartPar
\sphinxstyleliteralstrong{\sphinxupquote{savefig}} (\sphinxstyleliteralemphasis{\sphinxupquote{TYPE}}\sphinxstyleliteralemphasis{\sphinxupquote{, }}\sphinxstyleliteralemphasis{\sphinxupquote{optional}}) \textendash{} save figure. Defaults to ‘’.

\item {} 
\sphinxAtStartPar
\sphinxstyleliteralstrong{\sphinxupquote{xsize}} (\sphinxstyleliteralemphasis{\sphinxupquote{TYPE}}\sphinxstyleliteralemphasis{\sphinxupquote{, }}\sphinxstyleliteralemphasis{\sphinxupquote{optional}}) \textendash{} x size default to 10

\item {} 
\sphinxAtStartPar
\sphinxstyleliteralstrong{\sphinxupquote{ysize}} (\sphinxstyleliteralemphasis{\sphinxupquote{TYPE}}\sphinxstyleliteralemphasis{\sphinxupquote{, }}\sphinxstyleliteralemphasis{\sphinxupquote{optional}}) \textendash{} y size per row, defaults to 2

\end{itemize}

\item[{Returns}] \leavevmode
\sphinxAtStartPar
a matplotlib fig.

\end{description}\end{quote}

\end{fulllineitems}

\index{melt() (in module modelvis)@\spxentry{melt()}\spxextra{in module modelvis}}

\begin{fulllineitems}
\phantomsection\label{\detokenize{index:modelvis.melt}}
\pysigstartsignatures
\pysiglinewithargsret{\sphinxcode{\sphinxupquote{modelvis.}}\sphinxbfcode{\sphinxupquote{melt}}}{\emph{\DUrole{n}{df}}, \emph{\DUrole{n}{source}\DUrole{o}{=}\DUrole{default_value}{\textquotesingle{}Latest\textquotesingle{}}}}{}
\pysigstopsignatures
\sphinxAtStartPar
melts a wide dataframe to a tall dataframe , appends a soruce column

\end{fulllineitems}

\index{heatshow() (in module modelvis)@\spxentry{heatshow()}\spxextra{in module modelvis}}

\begin{fulllineitems}
\phantomsection\label{\detokenize{index:modelvis.heatshow}}
\pysigstartsignatures
\pysiglinewithargsret{\sphinxcode{\sphinxupquote{modelvis.}}\sphinxbfcode{\sphinxupquote{heatshow}}}{\emph{\DUrole{n}{df}}, \emph{\DUrole{n}{name}\DUrole{o}{=}\DUrole{default_value}{\textquotesingle{}\textquotesingle{}}}, \emph{\DUrole{n}{cmap}\DUrole{o}{=}\DUrole{default_value}{\textquotesingle{}Reds\textquotesingle{}}}, \emph{\DUrole{n}{mul}\DUrole{o}{=}\DUrole{default_value}{1.0}}, \emph{\DUrole{n}{annot}\DUrole{o}{=}\DUrole{default_value}{False}}, \emph{\DUrole{n}{size}\DUrole{o}{=}\DUrole{default_value}{(11.69, 8.27)}}, \emph{\DUrole{n}{dec}\DUrole{o}{=}\DUrole{default_value}{0}}, \emph{\DUrole{n}{cbar}\DUrole{o}{=}\DUrole{default_value}{True}}, \emph{\DUrole{n}{linewidths}\DUrole{o}{=}\DUrole{default_value}{0.5}}}{}
\pysigstopsignatures
\sphinxAtStartPar
A heatmap of a dataframe

\end{fulllineitems}

\index{attshow() (in module modelvis)@\spxentry{attshow()}\spxextra{in module modelvis}}

\begin{fulllineitems}
\phantomsection\label{\detokenize{index:modelvis.attshow}}
\pysigstartsignatures
\pysiglinewithargsret{\sphinxcode{\sphinxupquote{modelvis.}}\sphinxbfcode{\sphinxupquote{attshow}}}{\emph{\DUrole{n}{df}}, \emph{\DUrole{n}{treshold}\DUrole{o}{=}\DUrole{default_value}{False}}, \emph{\DUrole{n}{head}\DUrole{o}{=}\DUrole{default_value}{5000}}, \emph{\DUrole{n}{tail}\DUrole{o}{=}\DUrole{default_value}{0}}, \emph{\DUrole{n}{t}\DUrole{o}{=}\DUrole{default_value}{True}}, \emph{\DUrole{n}{annot}\DUrole{o}{=}\DUrole{default_value}{False}}, \emph{\DUrole{n}{showsum}\DUrole{o}{=}\DUrole{default_value}{False}}, \emph{\DUrole{n}{sort}\DUrole{o}{=}\DUrole{default_value}{True}}, \emph{\DUrole{n}{size}\DUrole{o}{=}\DUrole{default_value}{(11.69, 8.27)}}, \emph{\DUrole{n}{title}\DUrole{o}{=}\DUrole{default_value}{\textquotesingle{}\textquotesingle{}}}, \emph{\DUrole{n}{tshow}\DUrole{o}{=}\DUrole{default_value}{True}}, \emph{\DUrole{n}{dec}\DUrole{o}{=}\DUrole{default_value}{0}}, \emph{\DUrole{n}{cbar}\DUrole{o}{=}\DUrole{default_value}{True}}, \emph{\DUrole{n}{cmap}\DUrole{o}{=}\DUrole{default_value}{\textquotesingle{}jet\textquotesingle{}}}, \emph{\DUrole{n}{savefig}\DUrole{o}{=}\DUrole{default_value}{\textquotesingle{}\textquotesingle{}}}}{}
\pysigstopsignatures
\sphinxAtStartPar
Shows heatmap of impacts of exogeneous variables
:df: Dataframe with impact
:treshold: Take exogeneous variables with max impact of treshold or larger
:numhigh: take the numhigh largest impacts
:t: transpose the heatmap
:annot: Annotate the heatmap
:head: take the head largest
.tail: take the tail smalest
:showsum: Add a column with the sum
:sort: Sort the data
.tshow: Show a longer title
:cbar:  if a colorbar shoud be displayes
:cmap: the colormap
:save: Save the chart (in png format)

\end{fulllineitems}

\index{attshowone() (in module modelvis)@\spxentry{attshowone()}\spxextra{in module modelvis}}

\begin{fulllineitems}
\phantomsection\label{\detokenize{index:modelvis.attshowone}}
\pysigstartsignatures
\pysiglinewithargsret{\sphinxcode{\sphinxupquote{modelvis.}}\sphinxbfcode{\sphinxupquote{attshowone}}}{\emph{\DUrole{n}{df}}, \emph{\DUrole{n}{name}}, \emph{\DUrole{n}{pre}\DUrole{o}{=}\DUrole{default_value}{\textquotesingle{}\textquotesingle{}}}, \emph{\DUrole{n}{head}\DUrole{o}{=}\DUrole{default_value}{5}}, \emph{\DUrole{n}{tail}\DUrole{o}{=}\DUrole{default_value}{5}}}{}
\pysigstopsignatures
\sphinxAtStartPar
shows the contribution to row=name from each column
the coulumns can optional be selected as starting with pre

\end{fulllineitems}

\index{water() (in module modelvis)@\spxentry{water()}\spxextra{in module modelvis}}

\begin{fulllineitems}
\phantomsection\label{\detokenize{index:modelvis.water}}
\pysigstartsignatures
\pysiglinewithargsret{\sphinxcode{\sphinxupquote{modelvis.}}\sphinxbfcode{\sphinxupquote{water}}}{\emph{\DUrole{n}{serxinput}}, \emph{\DUrole{n}{sort}\DUrole{o}{=}\DUrole{default_value}{False}}, \emph{\DUrole{n}{ascending}\DUrole{o}{=}\DUrole{default_value}{True}}, \emph{\DUrole{n}{autosum}\DUrole{o}{=}\DUrole{default_value}{False}}, \emph{\DUrole{n}{allsort}\DUrole{o}{=}\DUrole{default_value}{False}}, \emph{\DUrole{n}{threshold}\DUrole{o}{=}\DUrole{default_value}{0.0}}}{}
\pysigstopsignatures
\sphinxAtStartPar
Creates a dataframe with information for a watrfall diagram
\begin{quote}\begin{description}
\item[{Serx}] \leavevmode
\sphinxAtStartPar
the input serie of values

\item[{Sort}] \leavevmode
\sphinxAtStartPar
True if the bars except the first and last should be sorted (default = False)

\item[{Allsort}] \leavevmode
\sphinxAtStartPar
True if all bars should be sorted (default = False)

\item[{Autosum}] \leavevmode
\sphinxAtStartPar
True if a Total bar are added in the end

\item[{Ascending}] \leavevmode
\sphinxAtStartPar
True if sortorder = ascending

\end{description}\end{quote}

\sphinxAtStartPar
Returns a dataframe with theese columns:
\begin{quote}\begin{description}
\item[{Hbegin}] \leavevmode
\sphinxAtStartPar
Height of the first bar

\item[{Hend}] \leavevmode
\sphinxAtStartPar
Height of the last bar

\item[{Hpos}] \leavevmode
\sphinxAtStartPar
Height of positive bars

\item[{Hneg}] \leavevmode
\sphinxAtStartPar
Height of negative bars

\item[{Start}] \leavevmode
\sphinxAtStartPar
Ofset at which each bar starts

\item[{Height}] \leavevmode
\sphinxAtStartPar
Height of each bar (just for information)

\end{description}\end{quote}

\end{fulllineitems}



\chapter{Attribution}
\label{\detokenize{index:attribution}}

\section{Equation level}
\label{\detokenize{index:equation-level}}
\sphinxAtStartPar
Attribution can be performed on the equation level and on the model level

\sphinxAtStartPar
Equation level attribution is done in the modelclass module here  {\hyperref[\detokenize{index:modelclass.Dekomp_Mixin}]{\sphinxcrossref{\sphinxcode{\sphinxupquote{Dekomp\_Mixin}}}}}

\sphinxAtStartPar
The class {\hyperref[\detokenize{index:modelclass.Dekomp_Mixin}]{\sphinxcrossref{\sphinxcode{\sphinxupquote{Dekomp\_Mixin}}}}} also defines a a number of front end functions both for equation and model attribution


\section{Model level}
\label{\detokenize{index:module-modeldekom}}\label{\detokenize{index:model-level}}\index{module@\spxentry{module}!modeldekom@\spxentry{modeldekom}}\index{modeldekom@\spxentry{modeldekom}!module@\spxentry{module}}
\sphinxAtStartPar
Module for making attribution analysis of a model.

\sphinxAtStartPar
The main function is attribution

\sphinxAtStartPar
Created on Wed May 31 08:50:51 2017

\sphinxAtStartPar
@author: hanseni
\index{attribution() (in module modeldekom)@\spxentry{attribution()}\spxextra{in module modeldekom}}

\begin{fulllineitems}
\phantomsection\label{\detokenize{index:modeldekom.attribution}}
\pysigstartsignatures
\pysiglinewithargsret{\sphinxcode{\sphinxupquote{modeldekom.}}\sphinxbfcode{\sphinxupquote{attribution}}}{\emph{\DUrole{n}{model}}, \emph{\DUrole{n}{experiments}}, \emph{\DUrole{n}{start}\DUrole{o}{=}\DUrole{default_value}{\textquotesingle{}\textquotesingle{}}}, \emph{\DUrole{n}{end}\DUrole{o}{=}\DUrole{default_value}{\textquotesingle{}\textquotesingle{}}}, \emph{\DUrole{n}{save}\DUrole{o}{=}\DUrole{default_value}{\textquotesingle{}\textquotesingle{}}}, \emph{\DUrole{n}{maxexp}\DUrole{o}{=}\DUrole{default_value}{10000}}, \emph{\DUrole{n}{showtime}\DUrole{o}{=}\DUrole{default_value}{False}}, \emph{\DUrole{n}{summaryvar}\DUrole{o}{=}\DUrole{default_value}{{[}\textquotesingle{}*\textquotesingle{}{]}}}, \emph{\DUrole{n}{silent}\DUrole{o}{=}\DUrole{default_value}{False}}, \emph{\DUrole{n}{msilent}\DUrole{o}{=}\DUrole{default_value}{True}}, \emph{\DUrole{n}{type}\DUrole{o}{=}\DUrole{default_value}{\textquotesingle{}level\textquotesingle{}}}}{}
\pysigstopsignatures
\sphinxAtStartPar
Calculates an attribution analysis on a model
accepts a dictionary with experiments. the key is experiment name, the value is a list
of variables which has to be reset to the values in the baseline dataframe.

\end{fulllineitems}

\index{attribution\_new() (in module modeldekom)@\spxentry{attribution\_new()}\spxextra{in module modeldekom}}

\begin{fulllineitems}
\phantomsection\label{\detokenize{index:modeldekom.attribution_new}}
\pysigstartsignatures
\pysiglinewithargsret{\sphinxcode{\sphinxupquote{modeldekom.}}\sphinxbfcode{\sphinxupquote{attribution\_new}}}{\emph{\DUrole{n}{model}}, \emph{\DUrole{n}{experiments}}, \emph{\DUrole{n}{start}\DUrole{o}{=}\DUrole{default_value}{\textquotesingle{}\textquotesingle{}}}, \emph{\DUrole{n}{end}\DUrole{o}{=}\DUrole{default_value}{\textquotesingle{}\textquotesingle{}}}, \emph{\DUrole{n}{save}\DUrole{o}{=}\DUrole{default_value}{\textquotesingle{}\textquotesingle{}}}, \emph{\DUrole{n}{maxexp}\DUrole{o}{=}\DUrole{default_value}{10000}}, \emph{\DUrole{n}{showtime}\DUrole{o}{=}\DUrole{default_value}{False}}, \emph{\DUrole{n}{summaryvar}\DUrole{o}{=}\DUrole{default_value}{{[}\textquotesingle{}*\textquotesingle{}{]}}}, \emph{\DUrole{n}{silent}\DUrole{o}{=}\DUrole{default_value}{False}}, \emph{\DUrole{n}{msilent}\DUrole{o}{=}\DUrole{default_value}{True}}, \emph{\DUrole{n}{type}\DUrole{o}{=}\DUrole{default_value}{\textquotesingle{}level\textquotesingle{}}}}{}
\pysigstopsignatures
\sphinxAtStartPar
Calculates an attribution analysis on a model
accepts a dictionary with experiments. the key is experiment name, the value is a list
of variables which has to be reset to the values in the baseline dataframe.

\end{fulllineitems}

\index{ilist() (in module modeldekom)@\spxentry{ilist()}\spxextra{in module modeldekom}}

\begin{fulllineitems}
\phantomsection\label{\detokenize{index:modeldekom.ilist}}
\pysigstartsignatures
\pysiglinewithargsret{\sphinxcode{\sphinxupquote{modeldekom.}}\sphinxbfcode{\sphinxupquote{ilist}}}{\emph{\DUrole{n}{df}}, \emph{\DUrole{n}{pat}}}{}
\pysigstopsignatures
\sphinxAtStartPar
returns a list of variable in the model matching the pattern,
the pattern can be a list of patterns of a sting with patterns seperated by
blanks

\sphinxAtStartPar
This function operates on the index names of a dataframe. Relevant for attribution analysis

\end{fulllineitems}

\index{GetSumImpact() (in module modeldekom)@\spxentry{GetSumImpact()}\spxextra{in module modeldekom}}

\begin{fulllineitems}
\phantomsection\label{\detokenize{index:modeldekom.GetSumImpact}}
\pysigstartsignatures
\pysiglinewithargsret{\sphinxcode{\sphinxupquote{modeldekom.}}\sphinxbfcode{\sphinxupquote{GetSumImpact}}}{\emph{\DUrole{n}{impact}}, \emph{\DUrole{n}{pat}\DUrole{o}{=}\DUrole{default_value}{\textquotesingle{}PD\_\_*\textquotesingle{}}}}{}
\pysigstopsignatures
\sphinxAtStartPar
Gets the accumulated differences attributet to each impact group

\end{fulllineitems}

\index{GetLastImpact() (in module modeldekom)@\spxentry{GetLastImpact()}\spxextra{in module modeldekom}}

\begin{fulllineitems}
\phantomsection\label{\detokenize{index:modeldekom.GetLastImpact}}
\pysigstartsignatures
\pysiglinewithargsret{\sphinxcode{\sphinxupquote{modeldekom.}}\sphinxbfcode{\sphinxupquote{GetLastImpact}}}{\emph{\DUrole{n}{impact}}, \emph{\DUrole{n}{pat}\DUrole{o}{=}\DUrole{default_value}{\textquotesingle{}RCET1\_\_*\textquotesingle{}}}}{}
\pysigstopsignatures
\sphinxAtStartPar
Gets the last differences attributet to each impact group

\end{fulllineitems}

\index{GetAllImpact() (in module modeldekom)@\spxentry{GetAllImpact()}\spxextra{in module modeldekom}}

\begin{fulllineitems}
\phantomsection\label{\detokenize{index:modeldekom.GetAllImpact}}
\pysigstartsignatures
\pysiglinewithargsret{\sphinxcode{\sphinxupquote{modeldekom.}}\sphinxbfcode{\sphinxupquote{GetAllImpact}}}{\emph{\DUrole{n}{impact}}, \emph{\DUrole{n}{pat}\DUrole{o}{=}\DUrole{default_value}{\textquotesingle{}RCET1\_\_*\textquotesingle{}}}}{}
\pysigstopsignatures
\sphinxAtStartPar
Gets the last differences attributet to each impact group

\end{fulllineitems}

\index{GetOneImpact() (in module modeldekom)@\spxentry{GetOneImpact()}\spxextra{in module modeldekom}}

\begin{fulllineitems}
\phantomsection\label{\detokenize{index:modeldekom.GetOneImpact}}
\pysigstartsignatures
\pysiglinewithargsret{\sphinxcode{\sphinxupquote{modeldekom.}}\sphinxbfcode{\sphinxupquote{GetOneImpact}}}{\emph{\DUrole{n}{impact}}, \emph{\DUrole{n}{pat}\DUrole{o}{=}\DUrole{default_value}{\textquotesingle{}RCET1\_\_*\textquotesingle{}}}, \emph{\DUrole{n}{per}\DUrole{o}{=}\DUrole{default_value}{\textquotesingle{}\textquotesingle{}}}}{}
\pysigstopsignatures
\sphinxAtStartPar
Gets differences attributet to each impact group in period:per

\end{fulllineitems}

\index{AggImpact() (in module modeldekom)@\spxentry{AggImpact()}\spxextra{in module modeldekom}}

\begin{fulllineitems}
\phantomsection\label{\detokenize{index:modeldekom.AggImpact}}
\pysigstartsignatures
\pysiglinewithargsret{\sphinxcode{\sphinxupquote{modeldekom.}}\sphinxbfcode{\sphinxupquote{AggImpact}}}{\emph{\DUrole{n}{impact}}}{}
\pysigstopsignatures
\sphinxAtStartPar
Calculates the sum of impacts and place in the last column

\sphinxAtStartPar
This function is applied to the result iof a Get* function

\end{fulllineitems}

\index{totdif (class in modeldekom)@\spxentry{totdif}\spxextra{class in modeldekom}}

\begin{fulllineitems}
\phantomsection\label{\detokenize{index:modeldekom.totdif}}
\pysigstartsignatures
\pysiglinewithargsret{\sphinxbfcode{\sphinxupquote{class\DUrole{w}{  }}}\sphinxcode{\sphinxupquote{modeldekom.}}\sphinxbfcode{\sphinxupquote{totdif}}}{\emph{\DUrole{n}{model}}, \emph{\DUrole{n}{summaryvar}\DUrole{o}{=}\DUrole{default_value}{\textquotesingle{}*\textquotesingle{}}}, \emph{\DUrole{n}{desdic}\DUrole{o}{=}\DUrole{default_value}{\{\}}}, \emph{\DUrole{n}{experiments}\DUrole{o}{=}\DUrole{default_value}{None}}}{}
\pysigstopsignatures
\sphinxAtStartPar
Class to make modelvide attribution analysis
\index{explain\_last() (modeldekom.totdif method)@\spxentry{explain\_last()}\spxextra{modeldekom.totdif method}}

\begin{fulllineitems}
\phantomsection\label{\detokenize{index:modeldekom.totdif.explain_last}}
\pysigstartsignatures
\pysiglinewithargsret{\sphinxbfcode{\sphinxupquote{explain\_last}}}{\emph{\DUrole{n}{pat}\DUrole{o}{=}\DUrole{default_value}{\textquotesingle{}\textquotesingle{}}}, \emph{\DUrole{n}{top}\DUrole{o}{=}\DUrole{default_value}{0.9}}, \emph{\DUrole{n}{title}\DUrole{o}{=}\DUrole{default_value}{\textquotesingle{}\textquotesingle{}}}, \emph{\DUrole{n}{use}\DUrole{o}{=}\DUrole{default_value}{\textquotesingle{}level\textquotesingle{}}}, \emph{\DUrole{n}{threshold}\DUrole{o}{=}\DUrole{default_value}{0.0}}}{}
\pysigstopsignatures
\sphinxAtStartPar
Explains last period
\begin{quote}\begin{description}
\item[{Parameters}] \leavevmode\begin{itemize}
\item {} 
\sphinxAtStartPar
\sphinxstyleliteralstrong{\sphinxupquote{pat}} (\sphinxstyleliteralemphasis{\sphinxupquote{TYPE}}\sphinxstyleliteralemphasis{\sphinxupquote{, }}\sphinxstyleliteralemphasis{\sphinxupquote{optional}}) \textendash{} DESCRIPTION. Defaults to ‘’.

\item {} 
\sphinxAtStartPar
\sphinxstyleliteralstrong{\sphinxupquote{top}} (\sphinxstyleliteralemphasis{\sphinxupquote{TYPE}}\sphinxstyleliteralemphasis{\sphinxupquote{, }}\sphinxstyleliteralemphasis{\sphinxupquote{optional}}) \textendash{} DESCRIPTION. Defaults to 0.9.

\item {} 
\sphinxAtStartPar
\sphinxstyleliteralstrong{\sphinxupquote{title}} (\sphinxstyleliteralemphasis{\sphinxupquote{TYPE}}\sphinxstyleliteralemphasis{\sphinxupquote{, }}\sphinxstyleliteralemphasis{\sphinxupquote{optional}}) \textendash{} DESCRIPTION. Defaults to ‘’.

\item {} 
\sphinxAtStartPar
\sphinxstyleliteralstrong{\sphinxupquote{use}} (\sphinxstyleliteralemphasis{\sphinxupquote{TYPE}}\sphinxstyleliteralemphasis{\sphinxupquote{, }}\sphinxstyleliteralemphasis{\sphinxupquote{optional}}) \textendash{} DESCRIPTION. Defaults to ‘level’.

\item {} 
\sphinxAtStartPar
\sphinxstyleliteralstrong{\sphinxupquote{threshold}} (\sphinxstyleliteralemphasis{\sphinxupquote{TYPE}}\sphinxstyleliteralemphasis{\sphinxupquote{, }}\sphinxstyleliteralemphasis{\sphinxupquote{optional}}) \textendash{} DESCRIPTION. Defaults to 0.0.

\end{itemize}

\item[{Returns}] \leavevmode
\sphinxAtStartPar
DESCRIPTION.

\item[{Return type}] \leavevmode
\sphinxAtStartPar
fig (TYPE)

\end{description}\end{quote}

\end{fulllineitems}

\index{explain\_sum() (modeldekom.totdif method)@\spxentry{explain\_sum()}\spxextra{modeldekom.totdif method}}

\begin{fulllineitems}
\phantomsection\label{\detokenize{index:modeldekom.totdif.explain_sum}}
\pysigstartsignatures
\pysiglinewithargsret{\sphinxbfcode{\sphinxupquote{explain\_sum}}}{\emph{\DUrole{n}{pat}\DUrole{o}{=}\DUrole{default_value}{\textquotesingle{}\textquotesingle{}}}, \emph{\DUrole{n}{top}\DUrole{o}{=}\DUrole{default_value}{0.9}}, \emph{\DUrole{n}{title}\DUrole{o}{=}\DUrole{default_value}{\textquotesingle{}\textquotesingle{}}}, \emph{\DUrole{n}{use}\DUrole{o}{=}\DUrole{default_value}{\textquotesingle{}level\textquotesingle{}}}, \emph{\DUrole{n}{threshold}\DUrole{o}{=}\DUrole{default_value}{0.0}}}{}
\pysigstopsignatures
\sphinxAtStartPar
Explains the sum
\begin{quote}\begin{description}
\item[{Parameters}] \leavevmode\begin{itemize}
\item {} 
\sphinxAtStartPar
\sphinxstyleliteralstrong{\sphinxupquote{pat}} (\sphinxstyleliteralemphasis{\sphinxupquote{TYPE}}\sphinxstyleliteralemphasis{\sphinxupquote{, }}\sphinxstyleliteralemphasis{\sphinxupquote{optional}}) \textendash{} DESCRIPTION. Defaults to ‘’.

\item {} 
\sphinxAtStartPar
\sphinxstyleliteralstrong{\sphinxupquote{top}} (\sphinxstyleliteralemphasis{\sphinxupquote{TYPE}}\sphinxstyleliteralemphasis{\sphinxupquote{, }}\sphinxstyleliteralemphasis{\sphinxupquote{optional}}) \textendash{} DESCRIPTION. Defaults to 0.9.

\item {} 
\sphinxAtStartPar
\sphinxstyleliteralstrong{\sphinxupquote{title}} (\sphinxstyleliteralemphasis{\sphinxupquote{TYPE}}\sphinxstyleliteralemphasis{\sphinxupquote{, }}\sphinxstyleliteralemphasis{\sphinxupquote{optional}}) \textendash{} DESCRIPTION. Defaults to ‘’.

\item {} 
\sphinxAtStartPar
\sphinxstyleliteralstrong{\sphinxupquote{use}} (\sphinxstyleliteralemphasis{\sphinxupquote{TYPE}}\sphinxstyleliteralemphasis{\sphinxupquote{, }}\sphinxstyleliteralemphasis{\sphinxupquote{optional}}) \textendash{} DESCRIPTION. Defaults to ‘level’.

\item {} 
\sphinxAtStartPar
\sphinxstyleliteralstrong{\sphinxupquote{threshold}} (\sphinxstyleliteralemphasis{\sphinxupquote{TYPE}}\sphinxstyleliteralemphasis{\sphinxupquote{, }}\sphinxstyleliteralemphasis{\sphinxupquote{optional}}) \textendash{} DESCRIPTION. Defaults to 0.0.

\end{itemize}

\item[{Returns}] \leavevmode
\sphinxAtStartPar
DESCRIPTION.

\item[{Return type}] \leavevmode
\sphinxAtStartPar
fig (TYPE)

\end{description}\end{quote}

\end{fulllineitems}

\index{explain\_per() (modeldekom.totdif method)@\spxentry{explain\_per()}\spxextra{modeldekom.totdif method}}

\begin{fulllineitems}
\phantomsection\label{\detokenize{index:modeldekom.totdif.explain_per}}
\pysigstartsignatures
\pysiglinewithargsret{\sphinxbfcode{\sphinxupquote{explain\_per}}}{\emph{\DUrole{n}{pat}\DUrole{o}{=}\DUrole{default_value}{\textquotesingle{}\textquotesingle{}}}, \emph{\DUrole{n}{per}\DUrole{o}{=}\DUrole{default_value}{\textquotesingle{}\textquotesingle{}}}, \emph{\DUrole{n}{top}\DUrole{o}{=}\DUrole{default_value}{0.9}}, \emph{\DUrole{n}{title}\DUrole{o}{=}\DUrole{default_value}{\textquotesingle{}\textquotesingle{}}}, \emph{\DUrole{n}{use}\DUrole{o}{=}\DUrole{default_value}{\textquotesingle{}level\textquotesingle{}}}, \emph{\DUrole{n}{threshold}\DUrole{o}{=}\DUrole{default_value}{0.0}}, \emph{\DUrole{n}{ysize}\DUrole{o}{=}\DUrole{default_value}{5}}}{}
\pysigstopsignatures
\sphinxAtStartPar
Explains a periode
\begin{quote}\begin{description}
\item[{Parameters}] \leavevmode\begin{itemize}
\item {} 
\sphinxAtStartPar
\sphinxstyleliteralstrong{\sphinxupquote{pat}} (\sphinxstyleliteralemphasis{\sphinxupquote{TYPE}}\sphinxstyleliteralemphasis{\sphinxupquote{, }}\sphinxstyleliteralemphasis{\sphinxupquote{optional}}) \textendash{} DESCRIPTION. Defaults to ‘’.

\item {} 
\sphinxAtStartPar
\sphinxstyleliteralstrong{\sphinxupquote{per}} (\sphinxstyleliteralemphasis{\sphinxupquote{TYPE}}\sphinxstyleliteralemphasis{\sphinxupquote{, }}\sphinxstyleliteralemphasis{\sphinxupquote{optional}}) \textendash{} DESCRIPTION. Defaults to ‘’.

\item {} 
\sphinxAtStartPar
\sphinxstyleliteralstrong{\sphinxupquote{top}} (\sphinxstyleliteralemphasis{\sphinxupquote{TYPE}}\sphinxstyleliteralemphasis{\sphinxupquote{, }}\sphinxstyleliteralemphasis{\sphinxupquote{optional}}) \textendash{} DESCRIPTION. Defaults to 0.9.

\item {} 
\sphinxAtStartPar
\sphinxstyleliteralstrong{\sphinxupquote{title}} (\sphinxstyleliteralemphasis{\sphinxupquote{TYPE}}\sphinxstyleliteralemphasis{\sphinxupquote{, }}\sphinxstyleliteralemphasis{\sphinxupquote{optional}}) \textendash{} DESCRIPTION. Defaults to ‘’.

\item {} 
\sphinxAtStartPar
\sphinxstyleliteralstrong{\sphinxupquote{use}} (\sphinxstyleliteralemphasis{\sphinxupquote{TYPE}}\sphinxstyleliteralemphasis{\sphinxupquote{, }}\sphinxstyleliteralemphasis{\sphinxupquote{optional}}) \textendash{} DESCRIPTION. Defaults to ‘level’.

\item {} 
\sphinxAtStartPar
\sphinxstyleliteralstrong{\sphinxupquote{threshold}} (\sphinxstyleliteralemphasis{\sphinxupquote{TYPE}}\sphinxstyleliteralemphasis{\sphinxupquote{, }}\sphinxstyleliteralemphasis{\sphinxupquote{optional}}) \textendash{} DESCRIPTION. Defaults to 0.0.

\item {} 
\sphinxAtStartPar
\sphinxstyleliteralstrong{\sphinxupquote{ysize}} (\sphinxstyleliteralemphasis{\sphinxupquote{TYPE}}\sphinxstyleliteralemphasis{\sphinxupquote{, }}\sphinxstyleliteralemphasis{\sphinxupquote{optional}}) \textendash{} DESCRIPTION. Defaults to 5.

\end{itemize}

\item[{Returns}] \leavevmode
\sphinxAtStartPar
DESCRIPTION.

\item[{Return type}] \leavevmode
\sphinxAtStartPar
fig (TYPE)

\end{description}\end{quote}

\end{fulllineitems}

\index{explain\_all() (modeldekom.totdif method)@\spxentry{explain\_all()}\spxextra{modeldekom.totdif method}}

\begin{fulllineitems}
\phantomsection\label{\detokenize{index:modeldekom.totdif.explain_all}}
\pysigstartsignatures
\pysiglinewithargsret{\sphinxbfcode{\sphinxupquote{explain\_all}}}{\emph{\DUrole{n}{pat}\DUrole{o}{=}\DUrole{default_value}{\textquotesingle{}\textquotesingle{}}}, \emph{\DUrole{n}{stacked}\DUrole{o}{=}\DUrole{default_value}{True}}, \emph{\DUrole{n}{kind}\DUrole{o}{=}\DUrole{default_value}{\textquotesingle{}bar\textquotesingle{}}}, \emph{\DUrole{n}{top}\DUrole{o}{=}\DUrole{default_value}{0.9}}, \emph{\DUrole{n}{title}\DUrole{o}{=}\DUrole{default_value}{\textquotesingle{}\textquotesingle{}}}, \emph{\DUrole{n}{use}\DUrole{o}{=}\DUrole{default_value}{\textquotesingle{}level\textquotesingle{}}}, \emph{\DUrole{n}{threshold}\DUrole{o}{=}\DUrole{default_value}{0.0}}, \emph{\DUrole{n}{resample}\DUrole{o}{=}\DUrole{default_value}{\textquotesingle{}\textquotesingle{}}}, \emph{\DUrole{n}{axvline}\DUrole{o}{=}\DUrole{default_value}{None}}}{}
\pysigstopsignatures
\sphinxAtStartPar
Explains all
\begin{quote}\begin{description}
\item[{Parameters}] \leavevmode\begin{itemize}
\item {} 
\sphinxAtStartPar
\sphinxstyleliteralstrong{\sphinxupquote{pat}} (\sphinxstyleliteralemphasis{\sphinxupquote{TYPE}}\sphinxstyleliteralemphasis{\sphinxupquote{, }}\sphinxstyleliteralemphasis{\sphinxupquote{optional}}) \textendash{} DESCRIPTION. Defaults to ‘’.

\item {} 
\sphinxAtStartPar
\sphinxstyleliteralstrong{\sphinxupquote{stacked}} (\sphinxstyleliteralemphasis{\sphinxupquote{TYPE}}\sphinxstyleliteralemphasis{\sphinxupquote{, }}\sphinxstyleliteralemphasis{\sphinxupquote{optional}}) \textendash{} DESCRIPTION. Defaults to True.

\item {} 
\sphinxAtStartPar
\sphinxstyleliteralstrong{\sphinxupquote{kind}} (\sphinxstyleliteralemphasis{\sphinxupquote{TYPE}}\sphinxstyleliteralemphasis{\sphinxupquote{, }}\sphinxstyleliteralemphasis{\sphinxupquote{optional}}) \textendash{} DESCRIPTION. Defaults to ‘bar’.

\item {} 
\sphinxAtStartPar
\sphinxstyleliteralstrong{\sphinxupquote{top}} (\sphinxstyleliteralemphasis{\sphinxupquote{TYPE}}\sphinxstyleliteralemphasis{\sphinxupquote{, }}\sphinxstyleliteralemphasis{\sphinxupquote{optional}}) \textendash{} DESCRIPTION. Defaults to 0.9.

\item {} 
\sphinxAtStartPar
\sphinxstyleliteralstrong{\sphinxupquote{title}} (\sphinxstyleliteralemphasis{\sphinxupquote{TYPE}}\sphinxstyleliteralemphasis{\sphinxupquote{, }}\sphinxstyleliteralemphasis{\sphinxupquote{optional}}) \textendash{} DESCRIPTION. Defaults to ‘’.

\item {} 
\sphinxAtStartPar
\sphinxstyleliteralstrong{\sphinxupquote{use}} (\sphinxstyleliteralemphasis{\sphinxupquote{TYPE}}\sphinxstyleliteralemphasis{\sphinxupquote{, }}\sphinxstyleliteralemphasis{\sphinxupquote{optional}}) \textendash{} DESCRIPTION. Defaults to ‘level’.

\item {} 
\sphinxAtStartPar
\sphinxstyleliteralstrong{\sphinxupquote{threshold}} (\sphinxstyleliteralemphasis{\sphinxupquote{TYPE}}\sphinxstyleliteralemphasis{\sphinxupquote{, }}\sphinxstyleliteralemphasis{\sphinxupquote{optional}}) \textendash{} DESCRIPTION. Defaults to 0.0.

\item {} 
\sphinxAtStartPar
\sphinxstyleliteralstrong{\sphinxupquote{resample}} (\sphinxstyleliteralemphasis{\sphinxupquote{TYPE}}\sphinxstyleliteralemphasis{\sphinxupquote{, }}\sphinxstyleliteralemphasis{\sphinxupquote{optional}}) \textendash{} DESCRIPTION. Defaults to ‘’.

\item {} 
\sphinxAtStartPar
\sphinxstyleliteralstrong{\sphinxupquote{axvline}} (\sphinxstyleliteralemphasis{\sphinxupquote{TYPE}}\sphinxstyleliteralemphasis{\sphinxupquote{, }}\sphinxstyleliteralemphasis{\sphinxupquote{optional}}) \textendash{} DESCRIPTION. Defaults to None.

\end{itemize}

\item[{Returns}] \leavevmode
\sphinxAtStartPar
None.

\end{description}\end{quote}

\end{fulllineitems}

\index{totexplain() (modeldekom.totdif method)@\spxentry{totexplain()}\spxextra{modeldekom.totdif method}}

\begin{fulllineitems}
\phantomsection\label{\detokenize{index:modeldekom.totdif.totexplain}}
\pysigstartsignatures
\pysiglinewithargsret{\sphinxbfcode{\sphinxupquote{totexplain}}}{\emph{\DUrole{n}{pat}\DUrole{o}{=}\DUrole{default_value}{\textquotesingle{}*\textquotesingle{}}}, \emph{\DUrole{n}{vtype}\DUrole{o}{=}\DUrole{default_value}{\textquotesingle{}all\textquotesingle{}}}, \emph{\DUrole{n}{stacked}\DUrole{o}{=}\DUrole{default_value}{True}}, \emph{\DUrole{n}{kind}\DUrole{o}{=}\DUrole{default_value}{\textquotesingle{}bar\textquotesingle{}}}, \emph{\DUrole{n}{per}\DUrole{o}{=}\DUrole{default_value}{\textquotesingle{}\textquotesingle{}}}, \emph{\DUrole{n}{top}\DUrole{o}{=}\DUrole{default_value}{0.9}}, \emph{\DUrole{n}{title}\DUrole{o}{=}\DUrole{default_value}{\textquotesingle{}\textquotesingle{}}}, \emph{\DUrole{n}{use}\DUrole{o}{=}\DUrole{default_value}{\textquotesingle{}level\textquotesingle{}}}, \emph{\DUrole{n}{threshold}\DUrole{o}{=}\DUrole{default_value}{0.0}}, \emph{\DUrole{n}{ysize}\DUrole{o}{=}\DUrole{default_value}{10}}, \emph{\DUrole{o}{**}\DUrole{n}{kwargs}}}{}
\pysigstopsignatures\begin{description}
\item[{Wrapper for different explanations}] \leavevmode\begin{itemize}
\item {} 
\sphinxAtStartPar
{\hyperref[\detokenize{index:modeldekom.totdif.explain_last}]{\sphinxcrossref{\sphinxcode{\sphinxupquote{explain\_last}}}}}

\item {} 
\sphinxAtStartPar
{\hyperref[\detokenize{index:modeldekom.totdif.explain_per}]{\sphinxcrossref{\sphinxcode{\sphinxupquote{explain\_per}}}}}

\item {} 
\sphinxAtStartPar
{\hyperref[\detokenize{index:modeldekom.totdif.explain_sum}]{\sphinxcrossref{\sphinxcode{\sphinxupquote{explain\_sum}}}}}

\item {} 
\sphinxAtStartPar
{\hyperref[\detokenize{index:modeldekom.totdif.explain_all}]{\sphinxcrossref{\sphinxcode{\sphinxupquote{explain\_all}}}}}

\end{itemize}

\end{description}
\begin{quote}\begin{description}
\item[{Parameters}] \leavevmode\begin{itemize}
\item {} 
\sphinxAtStartPar
\sphinxstyleliteralstrong{\sphinxupquote{pat}} (\sphinxstyleliteralemphasis{\sphinxupquote{TYPE}}\sphinxstyleliteralemphasis{\sphinxupquote{, }}\sphinxstyleliteralemphasis{\sphinxupquote{optional}}) \textendash{} DESCRIPTION. Defaults to ‘*’.

\item {} 
\sphinxAtStartPar
\sphinxstyleliteralstrong{\sphinxupquote{vtype}} (\sphinxstyleliteralemphasis{\sphinxupquote{per}}\sphinxstyleliteralemphasis{\sphinxupquote{|}}\sphinxstyleliteralemphasis{\sphinxupquote{all}}\sphinxstyleliteralemphasis{\sphinxupquote{|}}\sphinxstyleliteralemphasis{\sphinxupquote{last}}\sphinxstyleliteralemphasis{\sphinxupquote{|}}\sphinxstyleliteralemphasis{\sphinxupquote{sum}}\sphinxstyleliteralemphasis{\sphinxupquote{, }}\sphinxstyleliteralemphasis{\sphinxupquote{optional}}) \textendash{} what data to attribute. Defaults to ‘all’.

\item {} 
\sphinxAtStartPar
\sphinxstyleliteralstrong{\sphinxupquote{stacked}} (\sphinxstyleliteralemphasis{\sphinxupquote{TYPE}}\sphinxstyleliteralemphasis{\sphinxupquote{, }}\sphinxstyleliteralemphasis{\sphinxupquote{optional}}) \textendash{} DESCRIPTION. Defaults to True.

\item {} 
\sphinxAtStartPar
\sphinxstyleliteralstrong{\sphinxupquote{kind}} (\sphinxstyleliteralemphasis{\sphinxupquote{TYPE}}\sphinxstyleliteralemphasis{\sphinxupquote{, }}\sphinxstyleliteralemphasis{\sphinxupquote{optional}}) \textendash{} DESCRIPTION. Defaults to ‘bar’.

\item {} 
\sphinxAtStartPar
\sphinxstyleliteralstrong{\sphinxupquote{per}} (\sphinxstyleliteralemphasis{\sphinxupquote{TYPE}}\sphinxstyleliteralemphasis{\sphinxupquote{, }}\sphinxstyleliteralemphasis{\sphinxupquote{optional}}) \textendash{} DESCRIPTION. Defaults to ‘’.

\item {} 
\sphinxAtStartPar
\sphinxstyleliteralstrong{\sphinxupquote{top}} (\sphinxstyleliteralemphasis{\sphinxupquote{TYPE}}\sphinxstyleliteralemphasis{\sphinxupquote{, }}\sphinxstyleliteralemphasis{\sphinxupquote{optional}}) \textendash{} DESCRIPTION. Defaults to 0.9.

\item {} 
\sphinxAtStartPar
\sphinxstyleliteralstrong{\sphinxupquote{title}} (\sphinxstyleliteralemphasis{\sphinxupquote{TYPE}}\sphinxstyleliteralemphasis{\sphinxupquote{, }}\sphinxstyleliteralemphasis{\sphinxupquote{optional}}) \textendash{} DESCRIPTION. Defaults to ‘’.

\item {} 
\sphinxAtStartPar
\sphinxstyleliteralstrong{\sphinxupquote{use}} (\sphinxstyleliteralemphasis{\sphinxupquote{TYPE}}\sphinxstyleliteralemphasis{\sphinxupquote{, }}\sphinxstyleliteralemphasis{\sphinxupquote{optional}}) \textendash{} DESCRIPTION. Defaults to ‘level’.

\item {} 
\sphinxAtStartPar
\sphinxstyleliteralstrong{\sphinxupquote{threshold}} (\sphinxstyleliteralemphasis{\sphinxupquote{TYPE}}\sphinxstyleliteralemphasis{\sphinxupquote{, }}\sphinxstyleliteralemphasis{\sphinxupquote{optional}}) \textendash{} DESCRIPTION. Defaults to 0.0.

\item {} 
\sphinxAtStartPar
\sphinxstyleliteralstrong{\sphinxupquote{ysize}} (\sphinxstyleliteralemphasis{\sphinxupquote{TYPE}}\sphinxstyleliteralemphasis{\sphinxupquote{, }}\sphinxstyleliteralemphasis{\sphinxupquote{optional}}) \textendash{} DESCRIPTION. Defaults to 10.

\item {} 
\sphinxAtStartPar
\sphinxstyleliteralstrong{\sphinxupquote{**kwargs}} (\sphinxstyleliteralemphasis{\sphinxupquote{TYPE}}) \textendash{} DESCRIPTION.

\end{itemize}

\item[{Returns}] \leavevmode
\sphinxAtStartPar
DESCRIPTION.

\item[{Return type}] \leavevmode
\sphinxAtStartPar
fig (TYPE)

\end{description}\end{quote}

\end{fulllineitems}


\end{fulllineitems}



\chapter{Targets and instruments}
\label{\detokenize{index:targets-and-instruments}}
\sphinxAtStartPar
Used from the model class here  {\hyperref[\detokenize{index:modelclass.Solver_Mixin.invert}]{\sphinxcrossref{\sphinxcode{\sphinxupquote{invert}}}}}

\phantomsection\label{\detokenize{index:module-modelinvert}}\index{module@\spxentry{module}!modelinvert@\spxentry{modelinvert}}\index{modelinvert@\spxentry{modelinvert}!module@\spxentry{module}}
\sphinxAtStartPar
Created on Thu Sep 21 12:41:10 2017

\sphinxAtStartPar
@author: IBH
\begin{quote}

\sphinxAtStartPar
Class to handle general target/instrument problems.

\sphinxAtStartPar
Number of targets should be equal to number of instruments

\sphinxAtStartPar
An instrument can comprice of severeral variables
instruments are inputtet as a list of instruments
\end{quote}
\index{targets\_instruments (class in modelinvert)@\spxentry{targets\_instruments}\spxextra{class in modelinvert}}

\begin{fulllineitems}
\phantomsection\label{\detokenize{index:modelinvert.targets_instruments}}
\pysigstartsignatures
\pysiglinewithargsret{\sphinxbfcode{\sphinxupquote{class\DUrole{w}{  }}}\sphinxcode{\sphinxupquote{modelinvert.}}\sphinxbfcode{\sphinxupquote{targets\_instruments}}}{\emph{\DUrole{n}{databank}}, \emph{\DUrole{n}{targets}}, \emph{\DUrole{n}{instruments}}, \emph{\DUrole{n}{model}}, \emph{\DUrole{n}{DefaultImpuls}\DUrole{o}{=}\DUrole{default_value}{0.01}}, \emph{\DUrole{n}{defaultconv}\DUrole{o}{=}\DUrole{default_value}{0.01}}, \emph{\DUrole{n}{nonlin}\DUrole{o}{=}\DUrole{default_value}{False}}, \emph{\DUrole{n}{silent}\DUrole{o}{=}\DUrole{default_value}{True}}, \emph{\DUrole{n}{maxiter}\DUrole{o}{=}\DUrole{default_value}{30}}, \emph{\DUrole{n}{solveopt}\DUrole{o}{=}\DUrole{default_value}{\{\}}}, \emph{\DUrole{n}{varimpulse}\DUrole{o}{=}\DUrole{default_value}{False}}}{}
\pysigstopsignatures
\sphinxAtStartPar
Class to handle general target/instrument problems.
Where the response is delayed specify this with delay.

\sphinxAtStartPar
Number of targets should be equal to number of instruments

\sphinxAtStartPar
An instrument can comprice of severeral variables

\sphinxAtStartPar
\sphinxstylestrong{Instruments} are inputtet as a list of instruments

\sphinxAtStartPar
To calculate the jacobian each instrument variable has a impuls,
which is used as delta when evaluating the jacobi matrix:

\begin{sphinxVerbatim}[commandchars=\\\{\}]
\PYG{p}{[} \PYG{l+s+s1}{\PYGZsq{}}\PYG{l+s+s1}{QO\PYGZus{}J}\PYG{l+s+s1}{\PYGZsq{}}\PYG{p}{,}\PYG{l+s+s1}{\PYGZsq{}}\PYG{l+s+s1}{TG}\PYG{l+s+s1}{\PYGZsq{}}\PYG{p}{]}   \PYG{n}{Simple} \PYG{n+nb}{list} \PYG{n}{each} \PYG{n}{variable} \PYG{n}{are} \PYG{n}{shocked} \PYG{n}{by} \PYG{n}{the} \PYG{n}{default} \PYG{n}{impulse}
\PYG{p}{[} \PYG{p}{(}\PYG{l+s+s1}{\PYGZsq{}}\PYG{l+s+s1}{QO\PYGZus{}J}\PYG{l+s+s1}{\PYGZsq{}}\PYG{p}{,}\PYG{l+m+mf}{0.5}\PYG{p}{)}\PYG{p}{,} \PYG{l+s+s1}{\PYGZsq{}}\PYG{l+s+s1}{TG}\PYG{l+s+s1}{\PYGZsq{}}\PYG{p}{]}  \PYG{n}{Here} \PYG{n}{QO\PYGZus{}J} \PYG{o+ow}{is} \PYG{n}{getting} \PYG{n}{its} \PYG{n}{own} \PYG{n}{impuls} \PYG{p}{(}\PYG{l+m+mf}{0.5}\PYG{p}{)}
\PYG{p}{[} \PYG{p}{[}\PYG{p}{(}\PYG{l+s+s1}{\PYGZsq{}}\PYG{l+s+s1}{QO\PYGZus{}J}\PYG{l+s+s1}{\PYGZsq{}}\PYG{p}{,}\PYG{l+m+mf}{0.5}\PYG{p}{)}\PYG{p}{,}\PYG{p}{(}\PYG{l+s+s1}{\PYGZsq{}}\PYG{l+s+s1}{ORLOV}\PYG{l+s+s1}{\PYGZsq{}}\PYG{p}{,}\PYG{l+m+mf}{1.}\PYG{p}{)}\PYG{p}{]} \PYG{p}{,} \PYG{p}{(}\PYG{l+s+s1}{\PYGZsq{}}\PYG{l+s+s1}{TG}\PYG{l+s+s1}{\PYGZsq{}}\PYG{p}{,}\PYG{l+m+mf}{0.01}\PYG{p}{)}\PYG{p}{]} \PYG{n}{here} \PYG{n}{an} \PYG{n}{impuls} \PYG{o+ow}{is} \PYG{n}{given} \PYG{k}{for} \PYG{n}{each} \PYG{n}{variable}\PYG{p}{,} \PYG{o+ow}{and} \PYG{n}{the} \PYG{n}{first} \PYG{n}{instrument} \PYG{n}{consiste} \PYG{n}{of} \PYG{n}{two} \PYG{n}{variables}
\end{sphinxVerbatim}

\sphinxAtStartPar
\sphinxstylestrong{Targets} are list of variables

\sphinxAtStartPar
Convergence is achieved when all targets are within convergens distance from the target value

\sphinxAtStartPar
Convergencedistance can be set individual for a target variable by setting a value in \textless{}modelinstance\textgreater{}.targetconv

\sphinxAtStartPar
Targets and target values are provided by a dataframe.
\index{jacobi() (modelinvert.targets\_instruments method)@\spxentry{jacobi()}\spxextra{modelinvert.targets\_instruments method}}

\begin{fulllineitems}
\phantomsection\label{\detokenize{index:modelinvert.targets_instruments.jacobi}}
\pysigstartsignatures
\pysiglinewithargsret{\sphinxbfcode{\sphinxupquote{jacobi}}}{\emph{\DUrole{n}{per}}, \emph{\DUrole{n}{delay}\DUrole{o}{=}\DUrole{default_value}{None}}}{}
\pysigstopsignatures
\sphinxAtStartPar
Calculates a jecobi matrix of derivatives based on the instruments and targets

\sphinxAtStartPar
returns a dataframe

\end{fulllineitems}

\index{invjacobi() (modelinvert.targets\_instruments method)@\spxentry{invjacobi()}\spxextra{modelinvert.targets\_instruments method}}

\begin{fulllineitems}
\phantomsection\label{\detokenize{index:modelinvert.targets_instruments.invjacobi}}
\pysigstartsignatures
\pysiglinewithargsret{\sphinxbfcode{\sphinxupquote{invjacobi}}}{\emph{\DUrole{n}{per}}, \emph{\DUrole{n}{diag}\DUrole{o}{=}\DUrole{default_value}{False}}, \emph{\DUrole{n}{delay}\DUrole{o}{=}\DUrole{default_value}{0}}}{}
\pysigstopsignatures
\sphinxAtStartPar
Calculates the inverted jacobi matrix

\sphinxAtStartPar
returns a dataframe

\end{fulllineitems}

\index{targetseek() (modelinvert.targets\_instruments method)@\spxentry{targetseek()}\spxextra{modelinvert.targets\_instruments method}}

\begin{fulllineitems}
\phantomsection\label{\detokenize{index:modelinvert.targets_instruments.targetseek}}
\pysigstartsignatures
\pysiglinewithargsret{\sphinxbfcode{\sphinxupquote{targetseek}}}{\emph{\DUrole{n}{databank}\DUrole{o}{=}\DUrole{default_value}{None}}, \emph{\DUrole{n}{shortfall}\DUrole{o}{=}\DUrole{default_value}{False}}, \emph{\DUrole{n}{ti\_damp}\DUrole{o}{=}\DUrole{default_value}{1.0}}, \emph{\DUrole{n}{delay}\DUrole{o}{=}\DUrole{default_value}{0}}, \emph{\DUrole{n}{progressbar}\DUrole{o}{=}\DUrole{default_value}{True}}, \emph{\DUrole{o}{**}\DUrole{n}{kwargs}}}{}
\pysigstopsignatures
\sphinxAtStartPar
Calculates the instruments as a function of targets

\end{fulllineitems}


\end{fulllineitems}



\chapter{Enriching Pandas DataFrames}
\label{\detokenize{index:enriching-pandas-dataframes}}
\sphinxAtStartPar
Pandas dataframes can be enriched and this is done in two instances.
\sphinxhyphen{} to make it convinient to update variables
\sphinxhyphen{} to embed modelflow into dataframes


\section{When modelclass is imported upd is embedded}
\label{\detokenize{index:when-modelclass-is-imported-upd-is-embedded}}
\sphinxAtStartPar
dataframes are equiped with the upd method. This allows convinient
updating of variables using the {\hyperref[\detokenize{index:modelclass.Model_help_Mixin.update}]{\sphinxcrossref{\sphinxcode{\sphinxupquote{update}}}}} method.


\section{When modelmf is imported, modelflow is imbedded dataframes}
\label{\detokenize{index:module-modelmf}}\label{\detokenize{index:when-modelmf-is-imported-modelflow-is-imbedded-dataframes}}\index{module@\spxentry{module}!modelmf@\spxentry{modelmf}}\index{modelmf@\spxentry{modelmf}!module@\spxentry{module}}
\sphinxAtStartPar
This is a module for extending pandas dataframes with the modelflow toolbox

\sphinxAtStartPar
Created on Sat March 2019

\sphinxAtStartPar
@author: hanseni
\index{mf (class in modelmf)@\spxentry{mf}\spxextra{class in modelmf}}

\begin{fulllineitems}
\phantomsection\label{\detokenize{index:modelmf.mf}}
\pysigstartsignatures
\pysiglinewithargsret{\sphinxbfcode{\sphinxupquote{class\DUrole{w}{  }}}\sphinxcode{\sphinxupquote{modelmf.}}\sphinxbfcode{\sphinxupquote{mf}}}{\emph{\DUrole{n}{pandas\_obj}}}{}
\pysigstopsignatures
\sphinxAtStartPar
A class to extend Pandas Dataframes with ModelFlow functionalities

\sphinxAtStartPar
Not to be used on its own
\index{copy() (modelmf.mf method)@\spxentry{copy()}\spxextra{modelmf.mf method}}

\begin{fulllineitems}
\phantomsection\label{\detokenize{index:modelmf.mf.copy}}
\pysigstartsignatures
\pysiglinewithargsret{\sphinxbfcode{\sphinxupquote{copy}}}{}{}
\pysigstopsignatures
\sphinxAtStartPar
copy a modelflow extended dataframe, so it remember its model and options

\end{fulllineitems}

\index{solve() (modelmf.mf method)@\spxentry{solve()}\spxextra{modelmf.mf method}}

\begin{fulllineitems}
\phantomsection\label{\detokenize{index:modelmf.mf.solve}}
\pysigstartsignatures
\pysiglinewithargsret{\sphinxbfcode{\sphinxupquote{solve}}}{\emph{\DUrole{n}{start}\DUrole{o}{=}\DUrole{default_value}{\textquotesingle{}\textquotesingle{}}}, \emph{\DUrole{n}{slut}\DUrole{o}{=}\DUrole{default_value}{\textquotesingle{}\textquotesingle{}}}, \emph{\DUrole{o}{**}\DUrole{n}{kwargs}}}{}
\pysigstopsignatures
\sphinxAtStartPar
Solves a model

\end{fulllineitems}

\index{makemodel() (modelmf.mf method)@\spxentry{makemodel()}\spxextra{modelmf.mf method}}

\begin{fulllineitems}
\phantomsection\label{\detokenize{index:modelmf.mf.makemodel}}
\pysigstartsignatures
\pysiglinewithargsret{\sphinxbfcode{\sphinxupquote{makemodel}}}{\emph{\DUrole{n}{eq}}, \emph{\DUrole{o}{**}\DUrole{n}{kwargs}}}{}
\pysigstopsignatures
\sphinxAtStartPar
Makes a model from equations

\end{fulllineitems}

\index{\_\_call\_\_() (modelmf.mf method)@\spxentry{\_\_call\_\_()}\spxextra{modelmf.mf method}}

\begin{fulllineitems}
\phantomsection\label{\detokenize{index:modelmf.mf.__call__}}
\pysigstartsignatures
\pysiglinewithargsret{\sphinxbfcode{\sphinxupquote{\_\_call\_\_}}}{\emph{\DUrole{n}{eq}\DUrole{o}{=}\DUrole{default_value}{\textquotesingle{}\textquotesingle{}}}, \emph{\DUrole{o}{**}\DUrole{n}{kwargs}}}{}
\pysigstopsignatures
\sphinxAtStartPar
returns a model instace
\begin{quote}\begin{description}
\item[{Parameters}] \leavevmode\begin{itemize}
\item {} 
\sphinxAtStartPar
\sphinxstyleliteralstrong{\sphinxupquote{eq}} (\sphinxstyleliteralemphasis{\sphinxupquote{TYPE}}\sphinxstyleliteralemphasis{\sphinxupquote{, }}\sphinxstyleliteralemphasis{\sphinxupquote{optional}}) \textendash{} DESCRIPTION. Defaults to ‘’.

\item {} 
\sphinxAtStartPar
\sphinxstyleliteralstrong{\sphinxupquote{**kwargs}} (\sphinxstyleliteralemphasis{\sphinxupquote{TYPE}}) \textendash{} DESCRIPTION.

\end{itemize}

\item[{Returns}] \leavevmode
\sphinxAtStartPar
a instance .

\item[{Return type}] \leavevmode
\sphinxAtStartPar
{\hyperref[\detokenize{index:modelclass.model}]{\sphinxcrossref{model}}}

\end{description}\end{quote}

\end{fulllineitems}

\index{\_\_getitem\_\_() (modelmf.mf method)@\spxentry{\_\_getitem\_\_()}\spxextra{modelmf.mf method}}

\begin{fulllineitems}
\phantomsection\label{\detokenize{index:modelmf.mf.__getitem__}}
\pysigstartsignatures
\pysiglinewithargsret{\sphinxbfcode{\sphinxupquote{\_\_getitem\_\_}}}{\emph{\DUrole{n}{name}}}{}
\pysigstopsignatures
\sphinxAtStartPar
Retrieves  \_\_getitem\_\_ from model instance

\end{fulllineitems}


\end{fulllineitems}

\index{mfcalc (class in modelmf)@\spxentry{mfcalc}\spxextra{class in modelmf}}

\begin{fulllineitems}
\phantomsection\label{\detokenize{index:modelmf.mfcalc}}
\pysigstartsignatures
\pysiglinewithargsret{\sphinxbfcode{\sphinxupquote{class\DUrole{w}{  }}}\sphinxcode{\sphinxupquote{modelmf.}}\sphinxbfcode{\sphinxupquote{mfcalc}}}{\emph{\DUrole{n}{pandas\_obj}}}{}
\pysigstopsignatures
\sphinxAtStartPar
Used to carry out calculation specified as equations
\begin{quote}\begin{description}
\item[{Parameters}] \leavevmode\begin{itemize}
\item {} 
\sphinxAtStartPar
\sphinxstyleliteralstrong{\sphinxupquote{eq}} (\sphinxstyleliteralemphasis{\sphinxupquote{TYPE}}) \textendash{} Equations one on each line. can be started with \textless{}start end\textgreater{} to control calculation sample .

\item {} 
\sphinxAtStartPar
\sphinxstyleliteralstrong{\sphinxupquote{start}} (\sphinxstyleliteralemphasis{\sphinxupquote{TYPE}}\sphinxstyleliteralemphasis{\sphinxupquote{, }}\sphinxstyleliteralemphasis{\sphinxupquote{optional}}) \textendash{} DESCRIPTION. Defaults to ‘’.

\item {} 
\sphinxAtStartPar
\sphinxstyleliteralstrong{\sphinxupquote{slut}} (\sphinxstyleliteralemphasis{\sphinxupquote{TYPE}}\sphinxstyleliteralemphasis{\sphinxupquote{, }}\sphinxstyleliteralemphasis{\sphinxupquote{optional}}) \textendash{} DESCRIPTION. Defaults to ‘’.

\item {} 
\sphinxAtStartPar
\sphinxstyleliteralstrong{\sphinxupquote{showeq}} (\sphinxstyleliteralemphasis{\sphinxupquote{TYPE}}\sphinxstyleliteralemphasis{\sphinxupquote{, }}\sphinxstyleliteralemphasis{\sphinxupquote{optional}}) \textendash{} If True the equations will be printed. Defaults to False.

\item {} 
\sphinxAtStartPar
\sphinxstyleliteralstrong{\sphinxupquote{**kwargs}} (\sphinxstyleliteralemphasis{\sphinxupquote{TYPE}}) \textendash{} Here all solve options can be provided.

\end{itemize}

\item[{Returns}] \leavevmode
\sphinxAtStartPar
Dataframe.

\end{description}\end{quote}
\index{\_\_call\_\_() (modelmf.mfcalc method)@\spxentry{\_\_call\_\_()}\spxextra{modelmf.mfcalc method}}

\begin{fulllineitems}
\phantomsection\label{\detokenize{index:modelmf.mfcalc.__call__}}
\pysigstartsignatures
\pysiglinewithargsret{\sphinxbfcode{\sphinxupquote{\_\_call\_\_}}}{\emph{\DUrole{n}{eq}}, \emph{\DUrole{n}{start}\DUrole{o}{=}\DUrole{default_value}{\textquotesingle{}\textquotesingle{}}}, \emph{\DUrole{n}{slut}\DUrole{o}{=}\DUrole{default_value}{\textquotesingle{}\textquotesingle{}}}, \emph{\DUrole{n}{showeq}\DUrole{o}{=}\DUrole{default_value}{False}}, \emph{\DUrole{o}{**}\DUrole{n}{kwargs}}}{}
\pysigstopsignatures
\sphinxAtStartPar
This call performs the calculation
\begin{quote}\begin{description}
\item[{Parameters}] \leavevmode\begin{itemize}
\item {} 
\sphinxAtStartPar
\sphinxstyleliteralstrong{\sphinxupquote{eq}} (\sphinxstyleliteralemphasis{\sphinxupquote{TYPE}}) \textendash{} Equations one on each line. can be started with \textless{}start end\textgreater{} to control calculation sample .

\item {} 
\sphinxAtStartPar
\sphinxstyleliteralstrong{\sphinxupquote{start}} (\sphinxstyleliteralemphasis{\sphinxupquote{TYPE}}\sphinxstyleliteralemphasis{\sphinxupquote{, }}\sphinxstyleliteralemphasis{\sphinxupquote{optional}}) \textendash{} DESCRIPTION. Defaults to ‘’.

\item {} 
\sphinxAtStartPar
\sphinxstyleliteralstrong{\sphinxupquote{slut}} (\sphinxstyleliteralemphasis{\sphinxupquote{TYPE}}\sphinxstyleliteralemphasis{\sphinxupquote{, }}\sphinxstyleliteralemphasis{\sphinxupquote{optional}}) \textendash{} DESCRIPTION. Defaults to ‘’.

\item {} 
\sphinxAtStartPar
\sphinxstyleliteralstrong{\sphinxupquote{showeq}} (\sphinxstyleliteralemphasis{\sphinxupquote{TYPE}}\sphinxstyleliteralemphasis{\sphinxupquote{, }}\sphinxstyleliteralemphasis{\sphinxupquote{optional}}) \textendash{} If True the equations will be printed. Defaults to False.

\item {} 
\sphinxAtStartPar
\sphinxstyleliteralstrong{\sphinxupquote{**kwargs}} (\sphinxstyleliteralemphasis{\sphinxupquote{TYPE}}) \textendash{} Here all solve options can be provided.

\end{itemize}

\item[{Returns}] \leavevmode
\sphinxAtStartPar
Dataframe.

\end{description}\end{quote}

\end{fulllineitems}


\end{fulllineitems}

\index{mfupdate (class in modelmf)@\spxentry{mfupdate}\spxextra{class in modelmf}}

\begin{fulllineitems}
\phantomsection\label{\detokenize{index:modelmf.mfupdate}}
\pysigstartsignatures
\pysiglinewithargsret{\sphinxbfcode{\sphinxupquote{class\DUrole{w}{  }}}\sphinxcode{\sphinxupquote{modelmf.}}\sphinxbfcode{\sphinxupquote{mfupdate}}}{\emph{\DUrole{n}{pandas\_obj}}}{}
\pysigstopsignatures
\sphinxAtStartPar
Extend a dataframe to update with values from another dataframe
\index{\_\_call\_\_() (modelmf.mfupdate method)@\spxentry{\_\_call\_\_()}\spxextra{modelmf.mfupdate method}}

\begin{fulllineitems}
\phantomsection\label{\detokenize{index:modelmf.mfupdate.__call__}}
\pysigstartsignatures
\pysiglinewithargsret{\sphinxbfcode{\sphinxupquote{\_\_call\_\_}}}{\emph{\DUrole{n}{df}}}{}
\pysigstopsignatures
\sphinxAtStartPar
Call self as a function.

\end{fulllineitems}


\end{fulllineitems}

\index{ibloc (class in modelmf)@\spxentry{ibloc}\spxextra{class in modelmf}}

\begin{fulllineitems}
\phantomsection\label{\detokenize{index:modelmf.ibloc}}
\pysigstartsignatures
\pysiglinewithargsret{\sphinxbfcode{\sphinxupquote{class\DUrole{w}{  }}}\sphinxcode{\sphinxupquote{modelmf.}}\sphinxbfcode{\sphinxupquote{ibloc}}}{\emph{\DUrole{n}{pandas\_obj}}}{}
\pysigstopsignatures
\sphinxAtStartPar
Extend a dataframe with a slice method which accepot wildcards in column selection.

\sphinxAtStartPar
The method just juse the method vlist from modelclass.model class
\index{\_\_call\_\_() (modelmf.ibloc method)@\spxentry{\_\_call\_\_()}\spxextra{modelmf.ibloc method}}

\begin{fulllineitems}
\phantomsection\label{\detokenize{index:modelmf.ibloc.__call__}}
\pysigstartsignatures
\pysiglinewithargsret{\sphinxbfcode{\sphinxupquote{\_\_call\_\_}}}{}{}
\pysigstopsignatures
\sphinxAtStartPar
Not in use

\end{fulllineitems}


\end{fulllineitems}



\chapter{Logical Structure}
\label{\detokenize{index:logical-structure}}

\section{modelnet}
\label{\detokenize{index:module-modelnet}}\label{\detokenize{index:modelnet}}\index{module@\spxentry{module}!modelnet@\spxentry{modelnet}}\index{modelnet@\spxentry{modelnet}!module@\spxentry{module}}
\sphinxAtStartPar
Created on Wed Oct 15 14:30:44 2014

\sphinxAtStartPar
Displays an adjencacy matrix
.
@author: ibh
\index{draw\_adjacency\_matrix() (in module modelnet)@\spxentry{draw\_adjacency\_matrix()}\spxextra{in module modelnet}}

\begin{fulllineitems}
\phantomsection\label{\detokenize{index:modelnet.draw_adjacency_matrix}}
\pysigstartsignatures
\pysiglinewithargsret{\sphinxcode{\sphinxupquote{modelnet.}}\sphinxbfcode{\sphinxupquote{draw\_adjacency\_matrix}}}{\emph{\DUrole{n}{G}}, \emph{\DUrole{n}{node\_order}\DUrole{o}{=}\DUrole{default_value}{None}}, \emph{\DUrole{n}{partitions}\DUrole{o}{=}\DUrole{default_value}{None}}, \emph{\DUrole{n}{type}\DUrole{o}{=}\DUrole{default_value}{False}}, \emph{\DUrole{n}{title}\DUrole{o}{=}\DUrole{default_value}{\textquotesingle{}Structure\textquotesingle{}}}, \emph{\DUrole{n}{size}\DUrole{o}{=}\DUrole{default_value}{(10, 10)}}}{}
\pysigstopsignatures\begin{itemize}
\item {} 
\sphinxAtStartPar
G is a netorkx graph

\item {} \begin{description}
\item[{node\_order (optional) is a list of nodes, where each node in G}] \leavevmode
\sphinxAtStartPar
appears exactly once

\end{description}

\item {} \begin{description}
\item[{partitions is a list of node lists, where each node in G appears}] \leavevmode
\sphinxAtStartPar
in exactly one node list

\end{description}

\item {} 
\sphinxAtStartPar
type is a list of saying “simultaneous” or something else, has to have same length as partitions

\end{itemize}

\end{fulllineitems}

\index{drawendoexo() (in module modelnet)@\spxentry{drawendoexo()}\spxextra{in module modelnet}}

\begin{fulllineitems}
\phantomsection\label{\detokenize{index:modelnet.drawendoexo}}
\pysigstartsignatures
\pysiglinewithargsret{\sphinxcode{\sphinxupquote{modelnet.}}\sphinxbfcode{\sphinxupquote{drawendoexo}}}{\emph{\DUrole{n}{model}}, \emph{\DUrole{n}{size}\DUrole{o}{=}\DUrole{default_value}{(6.0, 6.0)}}}{}
\pysigstopsignatures
\sphinxAtStartPar
Draw dependency including exogeneous. Used for illustrating for small models

\end{fulllineitems}



\chapter{Jupyter Stuff}
\label{\detokenize{index:jupyter-stuff}}

\section{update widgets}
\label{\detokenize{index:module-modelwidget}}\label{\detokenize{index:update-widgets}}\index{module@\spxentry{module}!modelwidget@\spxentry{modelwidget}}\index{modelwidget@\spxentry{modelwidget}!module@\spxentry{module}}
\sphinxAtStartPar
Created on Mon Aug  9 14:46:11 2021

\sphinxAtStartPar
To define Jupyter widgets to update and show variables.
@author: Ib
\index{basewidget (class in modelwidget)@\spxentry{basewidget}\spxextra{class in modelwidget}}

\begin{fulllineitems}
\phantomsection\label{\detokenize{index:modelwidget.basewidget}}
\pysigstartsignatures
\pysiglinewithargsret{\sphinxbfcode{\sphinxupquote{class\DUrole{w}{  }}}\sphinxcode{\sphinxupquote{modelwidget.}}\sphinxbfcode{\sphinxupquote{basewidget}}}{\emph{\DUrole{n}{datachildren: list = \textless{}factory\textgreater{}}}}{}
\pysigstopsignatures
\sphinxAtStartPar
basis for widget updating in jupyter
\index{update\_df() (modelwidget.basewidget method)@\spxentry{update\_df()}\spxextra{modelwidget.basewidget method}}

\begin{fulllineitems}
\phantomsection\label{\detokenize{index:modelwidget.basewidget.update_df}}
\pysigstartsignatures
\pysiglinewithargsret{\sphinxbfcode{\sphinxupquote{update\_df}}}{\emph{\DUrole{n}{df}}, \emph{\DUrole{n}{current\_per}}}{}
\pysigstopsignatures
\sphinxAtStartPar
will update container widgets

\end{fulllineitems}


\end{fulllineitems}

\index{tabwidget (class in modelwidget)@\spxentry{tabwidget}\spxextra{class in modelwidget}}

\begin{fulllineitems}
\phantomsection\label{\detokenize{index:modelwidget.tabwidget}}
\pysigstartsignatures
\pysiglinewithargsret{\sphinxbfcode{\sphinxupquote{class\DUrole{w}{  }}}\sphinxcode{\sphinxupquote{modelwidget.}}\sphinxbfcode{\sphinxupquote{tabwidget}}}{\emph{\DUrole{n}{tabdefdict}\DUrole{p}{:}\DUrole{w}{  }\DUrole{n}{dict}}, \emph{\DUrole{n}{tab}\DUrole{p}{:}\DUrole{w}{  }\DUrole{n}{bool}\DUrole{w}{  }\DUrole{o}{=}\DUrole{w}{  }\DUrole{default_value}{True}}, \emph{\DUrole{n}{selected\_index}\DUrole{p}{:}\DUrole{w}{  }\DUrole{n}{Optional\DUrole{p}{{[}}any\DUrole{p}{{]}}}\DUrole{w}{  }\DUrole{o}{=}\DUrole{w}{  }\DUrole{default_value}{None}}}{}
\pysigstopsignatures
\sphinxAtStartPar
A widget to create tab or acordium contaners
\index{update\_df() (modelwidget.tabwidget method)@\spxentry{update\_df()}\spxextra{modelwidget.tabwidget method}}

\begin{fulllineitems}
\phantomsection\label{\detokenize{index:modelwidget.tabwidget.update_df}}
\pysigstartsignatures
\pysiglinewithargsret{\sphinxbfcode{\sphinxupquote{update\_df}}}{\emph{\DUrole{n}{df}}, \emph{\DUrole{n}{current\_per}}}{}
\pysigstopsignatures
\sphinxAtStartPar
will update container widgets

\end{fulllineitems}

\index{reset() (modelwidget.tabwidget method)@\spxentry{reset()}\spxextra{modelwidget.tabwidget method}}

\begin{fulllineitems}
\phantomsection\label{\detokenize{index:modelwidget.tabwidget.reset}}
\pysigstartsignatures
\pysiglinewithargsret{\sphinxbfcode{\sphinxupquote{reset}}}{\emph{\DUrole{n}{g}}}{}
\pysigstopsignatures
\sphinxAtStartPar
will reset  container widgets

\end{fulllineitems}


\end{fulllineitems}

\index{sheetwidget (class in modelwidget)@\spxentry{sheetwidget}\spxextra{class in modelwidget}}

\begin{fulllineitems}
\phantomsection\label{\detokenize{index:modelwidget.sheetwidget}}
\pysigstartsignatures
\pysiglinewithargsret{\sphinxbfcode{\sphinxupquote{class\DUrole{w}{  }}}\sphinxcode{\sphinxupquote{modelwidget.}}\sphinxbfcode{\sphinxupquote{sheetwidget}}}{\emph{\DUrole{n}{df\_var: any = Empty DataFrame Columns: {[}{]} Index: {[}{]}}}, \emph{\DUrole{n}{trans: any = \textless{}function sheetwidget.\textless{}lambda\textgreater{}\textgreater{}}}, \emph{\DUrole{n}{transpose: bool = False}}, \emph{\DUrole{n}{expname: str = \textquotesingle{}Carbon tax rate}}, \emph{\DUrole{n}{US\$ per tonn \textquotesingle{}}}}{}
\pysigstopsignatures
\sphinxAtStartPar
class defining a widget which updates from a sheet

\end{fulllineitems}

\index{slidewidget (class in modelwidget)@\spxentry{slidewidget}\spxextra{class in modelwidget}}

\begin{fulllineitems}
\phantomsection\label{\detokenize{index:modelwidget.slidewidget}}
\pysigstartsignatures
\pysiglinewithargsret{\sphinxbfcode{\sphinxupquote{class\DUrole{w}{  }}}\sphinxcode{\sphinxupquote{modelwidget.}}\sphinxbfcode{\sphinxupquote{slidewidget}}}{\emph{\DUrole{n}{slidedef}\DUrole{p}{:}\DUrole{w}{  }\DUrole{n}{dict}}, \emph{\DUrole{n}{altname}\DUrole{p}{:}\DUrole{w}{  }\DUrole{n}{str}\DUrole{w}{  }\DUrole{o}{=}\DUrole{w}{  }\DUrole{default_value}{\textquotesingle{}Alternative\textquotesingle{}}}, \emph{\DUrole{n}{basename}\DUrole{p}{:}\DUrole{w}{  }\DUrole{n}{str}\DUrole{w}{  }\DUrole{o}{=}\DUrole{w}{  }\DUrole{default_value}{\textquotesingle{}Baseline\textquotesingle{}}}, \emph{\DUrole{n}{expname}\DUrole{p}{:}\DUrole{w}{  }\DUrole{n}{str}\DUrole{w}{  }\DUrole{o}{=}\DUrole{w}{  }\DUrole{default_value}{\textquotesingle{}Carbon tax rate, US\$ per tonn \textquotesingle{}}}}{}
\pysigstopsignatures
\sphinxAtStartPar
class defefining a widget with lines of slides
\index{update\_df() (modelwidget.slidewidget method)@\spxentry{update\_df()}\spxextra{modelwidget.slidewidget method}}

\begin{fulllineitems}
\phantomsection\label{\detokenize{index:modelwidget.slidewidget.update_df}}
\pysigstartsignatures
\pysiglinewithargsret{\sphinxbfcode{\sphinxupquote{update\_df}}}{\emph{\DUrole{n}{df}}, \emph{\DUrole{n}{current\_per}}}{}
\pysigstopsignatures
\sphinxAtStartPar
updates a dataframe with the values from the widget

\end{fulllineitems}

\index{set\_slide\_value() (modelwidget.slidewidget method)@\spxentry{set\_slide\_value()}\spxextra{modelwidget.slidewidget method}}

\begin{fulllineitems}
\phantomsection\label{\detokenize{index:modelwidget.slidewidget.set_slide_value}}
\pysigstartsignatures
\pysiglinewithargsret{\sphinxbfcode{\sphinxupquote{set\_slide\_value}}}{\emph{\DUrole{n}{g}}}{}
\pysigstopsignatures
\sphinxAtStartPar
updates the new values to the self.current\_vlues
will be used in update\_df

\end{fulllineitems}


\end{fulllineitems}

\index{sumslidewidget (class in modelwidget)@\spxentry{sumslidewidget}\spxextra{class in modelwidget}}

\begin{fulllineitems}
\phantomsection\label{\detokenize{index:modelwidget.sumslidewidget}}
\pysigstartsignatures
\pysiglinewithargsret{\sphinxbfcode{\sphinxupquote{class\DUrole{w}{  }}}\sphinxcode{\sphinxupquote{modelwidget.}}\sphinxbfcode{\sphinxupquote{sumslidewidget}}}{\emph{\DUrole{n}{slidedef}\DUrole{p}{:}\DUrole{w}{  }\DUrole{n}{dict}}, \emph{\DUrole{n}{maxsum}\DUrole{p}{:}\DUrole{w}{  }\DUrole{n}{Optional\DUrole{p}{{[}}any\DUrole{p}{{]}}}\DUrole{w}{  }\DUrole{o}{=}\DUrole{w}{  }\DUrole{default_value}{None}}, \emph{\DUrole{n}{altname}\DUrole{p}{:}\DUrole{w}{  }\DUrole{n}{str}\DUrole{w}{  }\DUrole{o}{=}\DUrole{w}{  }\DUrole{default_value}{\textquotesingle{}Alternative\textquotesingle{}}}, \emph{\DUrole{n}{basename}\DUrole{p}{:}\DUrole{w}{  }\DUrole{n}{str}\DUrole{w}{  }\DUrole{o}{=}\DUrole{w}{  }\DUrole{default_value}{\textquotesingle{}Baseline\textquotesingle{}}}, \emph{\DUrole{n}{expname}\DUrole{p}{:}\DUrole{w}{  }\DUrole{n}{str}\DUrole{w}{  }\DUrole{o}{=}\DUrole{w}{  }\DUrole{default_value}{\textquotesingle{}Carbon tax rate, US\$ per tonn \textquotesingle{}}}}{}
\pysigstopsignatures
\sphinxAtStartPar
class defefining a widget with lines of slides
\index{update\_df() (modelwidget.sumslidewidget method)@\spxentry{update\_df()}\spxextra{modelwidget.sumslidewidget method}}

\begin{fulllineitems}
\phantomsection\label{\detokenize{index:modelwidget.sumslidewidget.update_df}}
\pysigstartsignatures
\pysiglinewithargsret{\sphinxbfcode{\sphinxupquote{update\_df}}}{\emph{\DUrole{n}{df}}, \emph{\DUrole{n}{current\_per}}}{}
\pysigstopsignatures
\sphinxAtStartPar
updates a dataframe with the values from the widget

\end{fulllineitems}

\index{set\_slide\_value() (modelwidget.sumslidewidget method)@\spxentry{set\_slide\_value()}\spxextra{modelwidget.sumslidewidget method}}

\begin{fulllineitems}
\phantomsection\label{\detokenize{index:modelwidget.sumslidewidget.set_slide_value}}
\pysigstartsignatures
\pysiglinewithargsret{\sphinxbfcode{\sphinxupquote{set\_slide\_value}}}{\emph{\DUrole{n}{g}}}{}
\pysigstopsignatures
\sphinxAtStartPar
updates the new values to the self.current\_vlues
will be used in update\_df

\end{fulllineitems}


\end{fulllineitems}

\index{updatewidget (class in modelwidget)@\spxentry{updatewidget}\spxextra{class in modelwidget}}

\begin{fulllineitems}
\phantomsection\label{\detokenize{index:modelwidget.updatewidget}}
\pysigstartsignatures
\pysiglinewithargsret{\sphinxbfcode{\sphinxupquote{class\DUrole{w}{  }}}\sphinxcode{\sphinxupquote{modelwidget.}}\sphinxbfcode{\sphinxupquote{updatewidget}}}{\emph{\DUrole{n}{mmodel: any}}, \emph{\DUrole{n}{a\_datawidget: any}}, \emph{\DUrole{n}{basename: str = \textquotesingle{}Business as usual\textquotesingle{}}}, \emph{\DUrole{n}{keeppat: str = \textquotesingle{}*\textquotesingle{}}}, \emph{\DUrole{n}{varpat: str = \textquotesingle{}*\textquotesingle{}}}, \emph{\DUrole{n}{showvarpat: bool = True}}, \emph{\DUrole{n}{exodif: any = Empty DataFrame Columns: {[}{]} Index: {[}{]}}}, \emph{\DUrole{n}{lwrun: bool = True}}, \emph{\DUrole{n}{lwupdate: bool = False}}, \emph{\DUrole{n}{lwreset: bool = True}}, \emph{\DUrole{n}{lwsetbas: bool = True}}, \emph{\DUrole{n}{lwshow: bool = True}}, \emph{\DUrole{n}{outputwidget: str = \textquotesingle{}jupviz\textquotesingle{}}}, \emph{\DUrole{n}{prefix\_dict: dict = \textless{}factory\textgreater{}}}, \emph{\DUrole{n}{display\_first: typing.Optional{[}any{]} = None}}, \emph{\DUrole{n}{vline: list = \textless{}factory\textgreater{}}}, \emph{\DUrole{n}{relativ\_start: int = 0}}, \emph{\DUrole{n}{short: bool = False}}, \emph{\DUrole{n}{legend: bool = False}}}{}
\pysigstopsignatures
\sphinxAtStartPar
class to input and run a model

\end{fulllineitems}

\index{htmlwidget\_df (class in modelwidget)@\spxentry{htmlwidget\_df}\spxextra{class in modelwidget}}

\begin{fulllineitems}
\phantomsection\label{\detokenize{index:modelwidget.htmlwidget_df}}
\pysigstartsignatures
\pysiglinewithargsret{\sphinxbfcode{\sphinxupquote{class\DUrole{w}{  }}}\sphinxcode{\sphinxupquote{modelwidget.}}\sphinxbfcode{\sphinxupquote{htmlwidget\_df}}}{\emph{\DUrole{n}{mmodel: any}}, \emph{\DUrole{n}{df\_var: any = Empty DataFrame Columns: {[}{]} Index: {[}{]}}}, \emph{\DUrole{n}{trans: any = \textless{}function htmlwidget\_df.\textless{}lambda\textgreater{}\textgreater{}}}, \emph{\DUrole{n}{transpose: bool = False}}, \emph{\DUrole{n}{expname: str = \textquotesingle{}\textquotesingle{}}}, \emph{\DUrole{n}{percent: bool = False}}}{}
\pysigstopsignatures
\sphinxAtStartPar
class displays a dataframe in a html widget

\end{fulllineitems}

\index{htmlwidget\_fig (class in modelwidget)@\spxentry{htmlwidget\_fig}\spxextra{class in modelwidget}}

\begin{fulllineitems}
\phantomsection\label{\detokenize{index:modelwidget.htmlwidget_fig}}
\pysigstartsignatures
\pysiglinewithargsret{\sphinxbfcode{\sphinxupquote{class\DUrole{w}{  }}}\sphinxcode{\sphinxupquote{modelwidget.}}\sphinxbfcode{\sphinxupquote{htmlwidget\_fig}}}{\emph{\DUrole{n}{figs}\DUrole{p}{:}\DUrole{w}{  }\DUrole{n}{any}}, \emph{\DUrole{n}{expname}\DUrole{p}{:}\DUrole{w}{  }\DUrole{n}{str}\DUrole{w}{  }\DUrole{o}{=}\DUrole{w}{  }\DUrole{default_value}{\textquotesingle{}\textquotesingle{}}}, \emph{\DUrole{n}{format}\DUrole{p}{:}\DUrole{w}{  }\DUrole{n}{str}\DUrole{w}{  }\DUrole{o}{=}\DUrole{w}{  }\DUrole{default_value}{\textquotesingle{}svg\textquotesingle{}}}}{}
\pysigstopsignatures
\sphinxAtStartPar
class displays a dataframe in a html widget

\end{fulllineitems}

\index{htmlwidget\_label (class in modelwidget)@\spxentry{htmlwidget\_label}\spxextra{class in modelwidget}}

\begin{fulllineitems}
\phantomsection\label{\detokenize{index:modelwidget.htmlwidget_label}}
\pysigstartsignatures
\pysiglinewithargsret{\sphinxbfcode{\sphinxupquote{class\DUrole{w}{  }}}\sphinxcode{\sphinxupquote{modelwidget.}}\sphinxbfcode{\sphinxupquote{htmlwidget\_label}}}{\emph{\DUrole{n}{expname}\DUrole{p}{:}\DUrole{w}{  }\DUrole{n}{str}\DUrole{w}{  }\DUrole{o}{=}\DUrole{w}{  }\DUrole{default_value}{\textquotesingle{}\textquotesingle{}}}, \emph{\DUrole{n}{format}\DUrole{p}{:}\DUrole{w}{  }\DUrole{n}{str}\DUrole{w}{  }\DUrole{o}{=}\DUrole{w}{  }\DUrole{default_value}{\textquotesingle{}svg\textquotesingle{}}}}{}
\pysigstopsignatures
\sphinxAtStartPar
class displays a dataframe in a html widget

\end{fulllineitems}

\index{visshow (class in modelwidget)@\spxentry{visshow}\spxextra{class in modelwidget}}

\begin{fulllineitems}
\phantomsection\label{\detokenize{index:modelwidget.visshow}}
\pysigstartsignatures
\pysiglinewithargsret{\sphinxbfcode{\sphinxupquote{class\DUrole{w}{  }}}\sphinxcode{\sphinxupquote{modelwidget.}}\sphinxbfcode{\sphinxupquote{visshow}}}{\emph{\DUrole{n}{mmodel: \textless{}built\sphinxhyphen{}in function any\textgreater{}}}, \emph{\DUrole{n}{varpat: str = \textquotesingle{}*\textquotesingle{}}}, \emph{\DUrole{n}{showvarpat: bool = True}}, \emph{\DUrole{n}{show\_on: bool = True}}}{}
\pysigstopsignatures
\end{fulllineitems}



\section{modeljupytermagic, Defines magic functions to define models, data and graphs in jupyter cells}
\label{\detokenize{index:module-modeljupytermagic}}\label{\detokenize{index:modeljupytermagic-defines-magic-functions-to-define-models-data-and-graphs-in-jupyter-cells}}\index{module@\spxentry{module}!modeljupytermagic@\spxentry{modeljupytermagic}}\index{modeljupytermagic@\spxentry{modeljupytermagic}!module@\spxentry{module}}
\sphinxAtStartPar
This module defines several magic jupyter functions:
\begin{quote}\begin{description}
\item[{graphviz}] \leavevmode
\sphinxAtStartPar
Draw Graphviz graph

\item[{dataframe}] \leavevmode
\sphinxAtStartPar
Create Pandas Dataframe

\item[{latexflow}] \leavevmode
\sphinxAtStartPar
Create a modelflow modelinstance from latex script

\end{description}\end{quote}

\sphinxAtStartPar
To display the doc strings use the functions in jupyter.
\index{get\_options() (in module modeljupytermagic)@\spxentry{get\_options()}\spxextra{in module modeljupytermagic}}

\begin{fulllineitems}
\phantomsection\label{\detokenize{index:modeljupytermagic.get_options}}
\pysigstartsignatures
\pysiglinewithargsret{\sphinxcode{\sphinxupquote{modeljupytermagic.}}\sphinxbfcode{\sphinxupquote{get\_options}}}{\emph{\DUrole{n}{line}}, \emph{\DUrole{n}{defaultname}\DUrole{o}{=}\DUrole{default_value}{\textquotesingle{}test\textquotesingle{}}}}{}
\pysigstopsignatures
\sphinxAtStartPar
Retrives options from the first line
\begin{quote}\begin{description}
\item[{Parameters}] \leavevmode\begin{itemize}
\item {} 
\sphinxAtStartPar
\sphinxstyleliteralstrong{\sphinxupquote{line}} (\sphinxstyleliteralemphasis{\sphinxupquote{TYPE}}) \textendash{} DESCRIPTION.

\item {} 
\sphinxAtStartPar
\sphinxstyleliteralstrong{\sphinxupquote{defaultname}} (\sphinxstyleliteralemphasis{\sphinxupquote{TYPE}}\sphinxstyleliteralemphasis{\sphinxupquote{, }}\sphinxstyleliteralemphasis{\sphinxupquote{optional}}) \textendash{} DESCRIPTION. Defaults to ‘test’.

\end{itemize}

\item[{Returns}] \leavevmode
\sphinxAtStartPar
name .
opt (dict): options.

\item[{Return type}] \leavevmode
\sphinxAtStartPar
name (string)

\end{description}\end{quote}

\end{fulllineitems}



\chapter{Optimizaton}
\label{\detokenize{index:optimizaton}}

\section{model\_cvx}
\label{\detokenize{index:module-model_cvx}}\label{\detokenize{index:model-cvx}}\index{module@\spxentry{module}!model\_cvx@\spxentry{model\_cvx}}\index{model\_cvx@\spxentry{model\_cvx}!module@\spxentry{module}}
\sphinxAtStartPar
Created on Mon May 26 21:11:18 2014

\sphinxAtStartPar
@author: Ib Hansen

\sphinxAtStartPar
A good explanation of quadradic programming in cvxopt is in
\sphinxurl{http://courses.csail.mit.edu/6.867/wiki/images/a/a7/Qp-cvxopt.pdf}

\sphinxAtStartPar
This exampel calculates the efficient forntier in a small example
the example is based on a mean variance model for Indonesian Rupia running in Excel
\index{mv\_opt() (in module model\_cvx)@\spxentry{mv\_opt()}\spxextra{in module model\_cvx}}

\begin{fulllineitems}
\phantomsection\label{\detokenize{index:model_cvx.mv_opt}}
\pysigstartsignatures
\pysiglinewithargsret{\sphinxcode{\sphinxupquote{model\_cvx.}}\sphinxbfcode{\sphinxupquote{mv\_opt}}}{\emph{\DUrole{n}{PP}}, \emph{\DUrole{n}{qq}}, \emph{\DUrole{n}{riskaversion}}, \emph{\DUrole{n}{bsum}}, \emph{\DUrole{n}{weights}}, \emph{\DUrole{n}{weigthtedsum}}, \emph{\DUrole{n}{boundsmin}}, \emph{\DUrole{n}{boundsmax}}, \emph{\DUrole{n}{lprint}\DUrole{o}{=}\DUrole{default_value}{False}}, \emph{\DUrole{n}{solget}\DUrole{o}{=}\DUrole{default_value}{None}}}{}
\pysigstopsignatures
\sphinxAtStartPar
Performs mean variance optimization by calling a quadratic optimization function from the cvxopt

\sphinxAtStartPar
library

\end{fulllineitems}

\index{mv\_opt\_bs() (in module model\_cvx)@\spxentry{mv\_opt\_bs()}\spxextra{in module model\_cvx}}

\begin{fulllineitems}
\phantomsection\label{\detokenize{index:model_cvx.mv_opt_bs}}
\pysigstartsignatures
\pysiglinewithargsret{\sphinxcode{\sphinxupquote{model\_cvx.}}\sphinxbfcode{\sphinxupquote{mv\_opt\_bs}}}{\emph{\DUrole{n}{msigma}}, \emph{\DUrole{n}{vreturn}}, \emph{\DUrole{n}{riskaversion}}, \emph{\DUrole{n}{budget}}, \emph{\DUrole{n}{risk\_weights}}, \emph{\DUrole{n}{capital}}, \emph{\DUrole{n}{lcr\_weights}}, \emph{\DUrole{n}{lcr}}, \emph{\DUrole{n}{leverage\_weights}}, \emph{\DUrole{n}{equity}}, \emph{\DUrole{n}{boundsmin}}, \emph{\DUrole{n}{boundsmax}}, \emph{\DUrole{n}{lprint}\DUrole{o}{=}\DUrole{default_value}{False}}, \emph{\DUrole{n}{solget}\DUrole{o}{=}\DUrole{default_value}{None}}}{}
\pysigstopsignatures
\sphinxAtStartPar
performs balance sheet optimization

\end{fulllineitems}

\index{mv\_opt\_prop() (in module model\_cvx)@\spxentry{mv\_opt\_prop()}\spxextra{in module model\_cvx}}

\begin{fulllineitems}
\phantomsection\label{\detokenize{index:model_cvx.mv_opt_prop}}
\pysigstartsignatures
\pysiglinewithargsret{\sphinxcode{\sphinxupquote{model\_cvx.}}\sphinxbfcode{\sphinxupquote{mv\_opt\_prop}}}{\emph{\DUrole{n}{PP}}, \emph{\DUrole{n}{qq}}, \emph{\DUrole{n}{riskaversion}}, \emph{\DUrole{n}{bsum}}, \emph{\DUrole{n}{weights}}, \emph{\DUrole{n}{weigthtedsum}}, \emph{\DUrole{n}{boundsmin}}, \emph{\DUrole{n}{boundsmax}}, \emph{\DUrole{n}{probability}\DUrole{o}{=}\DUrole{default_value}{None}}, \emph{\DUrole{n}{lprint}\DUrole{o}{=}\DUrole{default_value}{False}}}{}
\pysigstopsignatures
\sphinxAtStartPar
select a numner of assets/liabilities which. when the selection is feasible an Mean variance optimazation is performed

\sphinxAtStartPar
the selection is based on probabilities

\end{fulllineitems}



\renewcommand{\indexname}{Python Module Index}
\begin{sphinxtheindex}
\let\bigletter\sphinxstyleindexlettergroup
\bigletter{m}
\item\relax\sphinxstyleindexentry{model\_cvx}\sphinxstyleindexpageref{index:\detokenize{module-model_cvx}}
\item\relax\sphinxstyleindexentry{modelclass}\sphinxstyleindexpageref{index:\detokenize{module-modelclass}}
\item\relax\sphinxstyleindexentry{modeldekom}\sphinxstyleindexpageref{index:\detokenize{module-modeldekom}}
\item\relax\sphinxstyleindexentry{modelgrabwf2}\sphinxstyleindexpageref{index:\detokenize{module-modelgrabwf2}}
\item\relax\sphinxstyleindexentry{modelinvert}\sphinxstyleindexpageref{index:\detokenize{module-modelinvert}}
\item\relax\sphinxstyleindexentry{modeljupytermagic}\sphinxstyleindexpageref{index:\detokenize{module-modeljupytermagic}}
\item\relax\sphinxstyleindexentry{modelmanipulation}\sphinxstyleindexpageref{index:\detokenize{module-modelmanipulation}}
\item\relax\sphinxstyleindexentry{modelmf}\sphinxstyleindexpageref{index:\detokenize{module-modelmf}}
\item\relax\sphinxstyleindexentry{modelnet}\sphinxstyleindexpageref{index:\detokenize{module-modelnet}}
\item\relax\sphinxstyleindexentry{modelnewton}\sphinxstyleindexpageref{index:\detokenize{module-modelnewton}}
\item\relax\sphinxstyleindexentry{modelnormalize}\sphinxstyleindexpageref{index:\detokenize{module-modelnormalize}}
\item\relax\sphinxstyleindexentry{modelpattern}\sphinxstyleindexpageref{index:\detokenize{module-modelpattern}}
\item\relax\sphinxstyleindexentry{modelvis}\sphinxstyleindexpageref{index:\detokenize{module-modelvis}}
\item\relax\sphinxstyleindexentry{modelwidget}\sphinxstyleindexpageref{index:\detokenize{module-modelwidget}}
\end{sphinxtheindex}

\renewcommand{\indexname}{Index}
\printindex
\end{document}